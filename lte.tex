\textit{Lifting The Exponent Lemma} is a powerful method for solving exponential Diophantine equations. It is pretty well-known in the literature though its origins are hard to trace. Mathematically, it is a close relative of \textit{Hensel's lemma} in number theory (in both the statement and the idea of the proof). This is a technique that has been used a lot in recent Olympiad problems.

One can use the Lifting The Exponent Lemma (this is a long name, let's call it \textbf{LTE}!) in problems involving exponential equations, especially when there are some prime numbers (and is actually an overkill for many problems). This lemma shows how to find the greatest power of a prime $p$ -- which is often $\geq 3$ -- that divides $a^n \pm b^n$ for some positive integers $a$ and $b$. The advantage of this lemma is that, it is quite simple to understand and if in some contest, it is refrained from being used, the proof is not hard as well.


In section \eqref{sec:powerofprimes} of chapter \eqref{ch:arithfunc}, we defined $v_p(n)$. Recall that $v_p(n)$ is the highest power of a prime $p$ which divides a positive integer $n$. Here, we will make use of this function to solve Diophantine equations.

Here is a problem which will explain the main idea behind LTE.

    \begin{problem}
    	Show that there exist no positive integers $x$ and $y$ such that
    	\begin{align*}
    		2^{6x+1} + 1 = 3^{2y}
    	\end{align*}
    \end{problem}

    \begin{solution}
    	The idea is that the largest power of $3$ which divides the right side of the given equation, should be the same as that of the left side. Clearly, $v_3(3^{2y})=2y$. Let's find $v_3\left(2^{6x+1} + 1\right)$. Since $6x+1= 2 \cdot 3 \cdot x +1$ is odd, we can write
    	\begin{align*}
    		2^{6x+1} + 1 = \left(2+1\right) \left(2^{6x} - 2^{6x-1} + 2^{6x-2} - \cdots - 2 + 1\right)
    	\end{align*}
    	$\left(2^{6x} - 2^{6x-1} + 2^{6x-2} - \cdots - 2 + 1\right)$ is not divisible by $3$ (try to figure out why, using induction on $x$). Therefore
    	\begin{align*}
    		v_3\left(2^{6x+1} + 1\right) &= v_3\left(2+1\right) + v_3\left(2^{6x} - 2^{6x-1} + 2^{6x-2} - \cdots - 2 + 1\right)\\
    		& = 1 + 0\\
    		& = 1
    	\end{align*}
    	This means that $2y=1$, which is impossible since $y$ is an integer.
    \end{solution}

\subsection{Two Important and Useful Lemmas}

    \begin{lemma}\label{lem:lte-firstlemma}
         Let $x$ and $y$ be (not necessarily positive) integers and let $n$ be a positive integer. Given an arbitrary prime $p$ (in particular, we can have $p=2$) such that $ \gcd(n,p) = 1$, $p \mid  x - y$ and neither $x$, nor $y$ is divisible by $p$ (i.e., $p \nmid x$ and $p \nmid y$).  We have
           \begin{align*}
	            v_p\left(  x^n - y^n \right)
	            	& = v_p(  x - y )
           \end{align*}
    \end{lemma}

    \begin{proof}
        We use the fact that
	        \begin{align*}
	      	  x^n - y^n
	      	  	& = (x - y)\left(x^{n - 1} + x^{n - 2}y + x^{n - 3}y^2 + \cdots + y^{n - 1}\right)
	        \end{align*}
        Now if we show that $p \nmid x^{n - 1} + x^{n - 2}y + x^{n - 3}y^2 + \cdots + y^{n - 1}$, then we are done.
        In order to show this, we use the assumption $p \mid x-y.$  So we have $x-y \equiv 0 \pmod p,$
        or $x \equiv y \pmod p.$  Thus
        \begin{align*}
        x^{n - 1} + & x^{n - 2}y + x^{n - 3}y^2 + \cdots + y^{n - 1} \\
        & \equiv x^{n-1} +x^{n-2}  \cdot x +x^{n - 3} \cdot x^2 +\cdots +x \cdot x^{n-2} + x^{n-1} \\
        & \equiv n x^{n-1} \\
        & \not\equiv 0 \pmod p
        \end{align*}
        This completes the proof.
       \end{proof}


    \begin{lemma}\label{secondlemma}
        Let $ x$ and $y$ be (not necessarily positive) integers and let $n$ be an odd positive integer. Given an arbitrary prime $p$ (in particular, we can have $p=2$) such that $ \gcd(n,p) = 1$, $p\mid x + y$ and neither $x$, nor $y$ is divisible by $p$, we have
        \begin{align*}
	        v_p\left(  x^n + y^n \right) = v_p ( x + y )
        \end{align*}
    \end{lemma}

    \begin{proof}
        Since $x$ and $y$ can be negative in lemma \eqref{lem:lte-firstlemma} we only need to put $(-y)^n$ instead of $y^n$ in the formula to obtain
	         \begin{align*}
		         v_p\left( x^n - (-y)^n \right)
		         	& = v_p ( x - (-y))\\
		         \implies v_p\left(x^n + y^n\right)
		         	& = v_p(x+y)
	         \end{align*}
        Note that since $n$ is an odd positive integer we can replace $(-y)^n$ with $-y^n$.
    \end{proof}

\subsection{Main Result}

    \begin{theorem}[First Form of LTE]\label{theorem1}
        Let $ x$ and $y$ be (not necessarily positive) integers, let $n$ be a positive integer, and let $p$ be an odd prime such that $ p \mid x - y$ and none of
        $x$ and $y$ is divisible by $p$ (i.e., $p \nmid x$ and $p \nmid y$).  We have
        \[ v_p(  x^n - y^n ) = v_p(  x - y ) + v_p  (n )\]
    \end{theorem}


    \begin{proof}
        We may use induction on $v_p(n).$ First, let us prove the following statement:
            \begin{equation}\label{equation1}
                v_p(  x^p - y^p ) = v_p  ( x - y ) + 1
            \end{equation}
        In order to prove this, we will show that
            \begin{equation}\label{equation2}
                p \mid x^{p-1}+x^{p-2}y+\cdots+xy^{p-2}+y^{p-1}
            \end{equation}
        and
            \begin{equation}\label{equation3}
                p^2 \nmid x^{p-1}+x^{p-2}y+\cdots+xy^{p-2}+y^{p-1}
            \end{equation}
        For \eqref{equation2}, we note that
        \begin{align*}
        x^{p-1}+x^{p-2}y+\cdots+xy^{p-2}+y^{p-1} \equiv px^{p-1} \equiv 0 \pmod p
        \end{align*}
        Now, let $y=x+kp$ where $k$ is an integer. For an integer $ 1  \leq t <p$ we have
            \begin{align*}
                y^t x^{p-1-t}
                	& \equiv (x+kp)^t x^{p-1-t} \\
	                & \equiv x^{p-1-t} \left( x^t + t(kp)(x^{t-1})+ \frac{t(t-1)}{2} (kp)^2 (x^{t-2}) +\cdots \right) \\
	                & \equiv x^{p-1-t} \left( x^t + t(kp)(x^{t-1}) \right) \\
	                & \equiv x^{p-1}+tkpx^{p-2} \pmod{p^2}
            \end{align*}
        This means
            \begin{align*}
            	y^t x^{p-1-t} \equiv x^{p-1}+tkpx^{p-2} \pmod{p^2}
            \end{align*}
        Using this fact, we have
            \begin{align*}
                x^{p-1}& +x^{p-2}y+\cdots+xy^{p-2}+y^{p-1} \\
                & \equiv x^{p-1}+\left(x^{p-1}+kpx^{p-2}\right)+\left(x^{p-1}+2kpx^{p-2}\right)+\cdots+\left(x^{p-1} +(p-1)kpx^{p-2}\right) \\
                & \equiv px^{p-1} + \left(1+2+\cdots+p-1\right)kpx^{p-2} \\
                & \equiv px^{p-1} + \left( \frac{p(p-1)}{2} \right) kpx^{p-2} \\
                & \equiv px^{p-1} + \left( \frac{p-1}{2} \right) kp^2x^{p-1} \\
                & \equiv px^{p-1}\\
                & \not \equiv 0 \pmod{p^2}
            \end{align*}
        So we proved the relation \eqref{equation3} and the proof of equation \eqref{equation1} is complete.  Now let us return to our problem. We want to show that
            \begin{align*}
            	v_p\left(  x^n - y^n \right) = v_p(  x - y ) + v_p(  n )
            \end{align*}
        Suppose that $n=p^{\alpha}b$ where $\gcd(p,b)=1.$ Then
            \begin{align*}
                v_p(x^n-y^n) & =  v_p\left((x^{p^{\alpha}})^b - (y^{p^{\alpha}})^b \right)  \\
                & =   v_p\left(  x^{p^{\alpha}} -y^{p^{\alpha}}\right) =v_p\left(  (x^{p^{\alpha-1}})^p - (y^{p^{\alpha-1}})^p \right) \\
                & = v_p\left(  x^{p^{\alpha-1}} - y^{p^{\alpha-1}} \right) + 1 =v_p\left(  (x^{p^{\alpha-2}})^p - (y^{p^{\alpha-2}})^p\right) +1  \\
                & = v_p\left(  x^{p^{\alpha-2}} - y^{p^{\alpha-2}} \right) + 2 \\
                & \phantom{=} \vdots \\
                & = v_p\left(  (x^{p^{1}})^1 - (y^{p^{1}})^1  \right) + \alpha -1  = v_p\left(  x-y \right) + \alpha \\
                & =v_p\left(  x-y \right) + v_p\left(  n \right)
            \end{align*}

        Note that we used the fact that if $p \mid x-y,$ then we have $p \mid x^k -y^k,$ because we have $x-y \mid x^k-y^k$ for all positive integers $k.$ The proof is complete.
    \end{proof}



    \begin{theorem}[Second Form of LTE]\label{theorem2}
        Let $ x,y$ be two integers, $n$ be an odd positive integer, and $p$ be an odd prime such that $ p\mid x + y$ and none of
        $x$ and $y$ is divisible by $p.$  We have
        	\begin{align*}
        		v_p(  x^n + y^n ) = v_p(  x + y ) + v_p(  n )
        	\end{align*}
    \end{theorem}

    \begin{proof}
        This is obvious using theorem \eqref{theorem1}. See the trick we used in proof of lemma \eqref{secondlemma}.
	\end{proof}

    The following theorem is a  special case of Zsigmondy's theorem (discussed later in \eqref{thm:zsigmondy}), which can be proved using LTE and EGL (theorem \eqref{thm:egl}). And probably it is the most important case of Zsigmondy's theorem  we use in problems. In case someone considers the original theorem to be a sledgehammer, in that case this theorem should work fine. We leave the proof as an exercise.

    \begin{theorem}
    	For a prime $p>3$ and coprime integers $x,y$, $x^{p^k}-y^{p^k}$ has a prime factor $q$ such that $q|x^{p^k}-y^{p^k}$ but $q\nmid x^{p^i}-y^{p^i}$ for $0\le i<k$.
    \end{theorem}


\subsection{The Case \texorpdfstring{${p=2}$}{p = 2}}

    \begin{question}
        Why did we assume that $p$ is an odd prime, i.e., $p \neq 2 ?$ Why can't we assume that $p=2$ in our proofs?
    \end{question}

    \begin{hint}
        Note that $\frac{p-1}{2}$ is an integer only for $p>2.$
    \end{hint}


    \begin{theorem}[LTE for $p = 2$]
        Let $x$ and $y$ be two odd integers such that $4 \mid x-y.$ Then
         \begin{align*}
         	v_2  \left(x^n - y^n \right) = v_2  (x-y ) + v_2  (n )
         \end{align*}
    \end{theorem}

    \begin{proof}
        We showed that for any prime $p$ such that $\gcd(p,n)=1, p \mid x-y$ and  none of $x$ and $y$ is divisible by $p,$ we have
            \[v_p(  x^n - y^n ) = v_p(  x - y )\]
        So it suffices to show that
            \[v_2(  x^{2^{n}} - y^{2^{n}} ) = v_2(  x-y ) + n\]
        Factorization gives
            \[x^{2^{n}} - y^{2^{n}} = (x^{2^{n-1}} + y^{2^{n-1}})(x^{2^{n-2}} + y^{2^{n-2}}) \cdots (x^2 + y^2 )(x + y)(x - y) \]
        Now since $x \equiv y \equiv \pm 1\pmod 4$ then we have $x^{2^{k}}  \equiv  y^{2^{k}} \equiv 1 \pmod 4$ for all positive integers $k$ and so $x^{2^{k}} + y^{2^{k}} \equiv 2 \pmod 4, k=1,2,3,\ldots.$ Also, since $x$ and $y$ are odd and $4\mid x-y$, we have $x+y \equiv 2\pmod 4$.
        This means the power of $2$ in all of the factors in the above product (except $x-y$) is one. We are done.
    \end{proof}


    \begin{theorem}
        Let $x$ and $y$ be two odd integers and let $n$ be an even positive integer. Then
        \begin{align*}
        v_2\left(  x^n - y^n \right) = v_2(  x - y )+v_2(  x + y )+v_2(  n )-1
        \end{align*}
    \end{theorem}

    \begin{proof}
        We know that the square of an odd integer is of the form $4k+1.$ So for odd $x$ and $y$ we have $4 \mid x^2-y^2.$ Now let $m$ be an odd integer and $k$ be a positive integer such that $n=m \cdot 2^k.$ Then

            \begin{align*}
                v_2 \left( x^{n} - y^{n} \right)  & = v_2\left( x^{m \cdot 2^{k}} - y^{m \cdot 2^{k}} \right)   \\
                & = v_2\left((x^2)^{2^{k-1}}-(y^2)^{2^{k-1}}\right) \\
                & \phantom{=} \vdots \\
                & =  v_2 \left( x^{2} - y^{2} \right) + k-1  \\
                & = v_2 ( x - y )+v_2 ( x + y )+v_2 ( n )-1
            \end{align*}
    \end{proof}

\subsection{Summary}

    Let $p$ be a prime number and let $x$ and $y$ be two (not necessarily positive) integers that are not divisible by $p.$ Then:

    \begin{enumerate}[(a)]
        \item  For a positive integer $n$
            \begin{itemize}
                \item if $p \neq 2$ and $p \mid x-y,$ then
                 \begin{align*}
                 v_p\left(  x^n - y^n \right) = v_p(  x - y ) + v_p(  n )
                 \end{align*}

                \item if $p=2$ and $4 \mid x-y,$ then
                \begin{align*}
                 v_2\left(  x^n - y^n \right) = v_2(  x-y ) + v_2(  n )
                \end{align*}

                \item if $p=2$, $n$ is even, and $2\mid x-y,$ then
                \begin{align*}
                v_2\left(  x^n - y^n \right) = v_2(  x - y )+v_2(  x + y )+v_2(  n )-1
                \end{align*}
            \end{itemize}

        \item For an odd positive integer $n,$ if $p \mid x+y,$ then
	            \begin{align*}
                 v_p\left(  x^n + y^n \right) = v_p(  x + y ) + v_p(  n )
                 \end{align*}

        \item For a positive integer $n$ with $\gcd(p,n)=1,$ if $p\mid x-y$, we have
         \begin{align*}
         v_p\left(  x^n - y^n \right) = v_p(  x - y )
         \end{align*}
        \par If $n$ is odd, $\gcd(p,n)=1,$ and $p\mid x+y$, then we have
		\begin{align*}
		v_p\left(  x^n + y^n \right) = v_p(  x + y )
		\end{align*}

    \end{enumerate}

    \begin{note}
        The most common mistake in using LTE is when you do not check the $p \mid x \pm y$ condition, so always remember to check it. Otherwise your solution will be completely wrong.
    \end{note}

\subsection{Solved Problems}


    \begin{problem}[Russia 1996]
        Find all positive integers $n$ for which there exist positive integers $x ,y$ and $k$ such that $\gcd(x,y)=1, k>1$ and $3^n = x^k + y^k.$
    \end{problem}

    \begin{solution}
        $k$ should be an odd integer (otherwise, if $k$ is even, then $x^k$ and $y^k$ are perfect squares, and it is well known that for integers $a,b$ we have $3 \mid a^2+b^2$ if and only if $3 \mid a$ and $3 \mid b,$ which is in contradiction with $\gcd(x,y)=1.$). Suppose that there exists a prime $p$ such that $p \mid x+y$. This prime should be odd. So $v_p(3^n)=v_p(x^k+y^k)$, and using \eqref{theorem2} we have $$v_p(3^n)=v_p(x^k+y^k)=v_p(k)+v_p(x+y).$$ But $p \mid x+y$ means that $v_p(x+y) \geq 1 >0$ and so $v_p(3^n)=v_p(k)+v_p(x+y) >0$ and so $p \mid 3^n$. Thus $p=3.$ This means $x+y=3^m$ for some positive integer $m$. Note that $n=v_3(k)+m$. There are two cases:
            \begin{enumerate}
                \item $m>1$. We can prove by induction that $3^a \ge a+2$ for all integers $a\ge 1$, and so we have $v_3(k) \leq k-2$ (why?). Let $M= \max(x,y)$. Since $x+y=3^m\ge 9$, we have $M \geq 5$. Then
                \begin{align*}
	                x^k+ y^k \geq M^k  & =  \underbrace{M}_{\geq \frac{x+y}{2} = \frac{1}{2} \cdot 3^m} \cdot \underbrace{M^{k-1}}_{\geq 5^{k-1}} \\
	                & > \frac{1}{2} 3^m \cdot 5^{k-1} \\
	                &  >3^m \cdot 5^{k-2}\\
	                & \geq 3^{m+k-2} \\
	                & \geq 3^{m + v_3(k)}\\
	                & = 3^n
                \end{align*}
                which is a contradiction.

                \item $m=1$. Then $x+y=3$, so $x=1, y=2$ (or $x=2, y=1$). Thus $3^{1+v_3(k)}= 1+2^k$. But note that
                $3^{v_3(k)} \mid k$ so $3^{v_3(k)} \leq k$. Thus
                	\begin{align*}
                		1+2^k
                			& = 3^{v_3(k)+1}\\
                			& = 3 \cdot \underbrace{3^{v_3(k)}}_{\leq k}\\
                			& \leq 3k\\
                		\implies 2^k +1
                			& \leq 3k
                	\end{align*}
                And one can check that the only odd value of $k>1$ that satisfies the above inequality is $k=3$. So $(x,y,n,k)=(1,2,2,3), (2,1,2,3)$ in this case.
            \end{enumerate}
    Thus, the final answer is $n=2.$

    \end{solution}

    \begin{problem}[Balkan 1993]
        Let $p$ be a prime number and $m>1$ be a positive integer. Show that if for some positive integers $x>1, y>1$ we have
        \begin{align*}
	        \frac{x^p+y^p}{2}= \func{}{\frac{x+y}{2}}^m
        \end{align*}
        then $m=p.$
    \end{problem}

    \begin{solution}
        One can prove by induction on $p$ that
        	\begin{align*}
        		\frac{x^p+y^p}2\ge \func{}{\frac{x+y}2}^p
        	\end{align*}
        for all positive integers $p$. Now since
        	\begin{align*}
        		\frac{x^p+y^p}{2}= \func{}{\frac{x+y}{2}}^m
        	\end{align*}
        we should have $ m \geq p$. Let $d=\gcd(x,y)$, so there exist positive integers
        $x_1$ and $y_1$ with $\gcd(x_1,y_1)=1$ such that $x=dx_1$, $y=dy_1$, and
        	\begin{align*}
        		2^{m-1}(x_1^p+y_1^p)=d^{m-p}(x_1+y_1)^m
        	\end{align*}
        There are two cases:

        \begin{enumerate}
        	\item Assume that $p$ is odd. Take any prime divisor $q$ of $x_1+y_1$ and let $v=v_q(x_1+y_1)$. If $q$ is odd, we see that
        		\begin{align*}
        			v_q(x_1^p +y_1^p)
        				& =v+v_q(p)\\
        			v_q(d^{m-p}(x_1+y_1)^m)
        				& \geq mv
        		\end{align*}
        	(because $q$  may also be a factor of $d$).
        	Thus $m\le 2$ and $p\le 2$, giving an immediate contradiction. If $q=2$, then $m-1+v\ge mv$, so $v\le 1$ and $x_1+y_1=2$, i.e., $x=y$, which immediately implies $m=p$.

        	\item Assume that $p=2$. We notice that for $x+y \geq 4$ we have
        		\begin{align*}
        			\frac{x^2+y^2}2< \func{2}{\frac{x+y}2}^2\le \func{}{\frac{x+y}2}^3
        		\end{align*}
        	so $m=2$. It remains to check that the remaining cases $(x,y)=(1,2),(2,1)$ are impossible.
        \end{enumerate}

    \end{solution}

    \begin{problem}
        Find all positive integers $a,b$ that are greater than $1$ and satisfy
        	\begin{align*}
	        	b^a
	        		& \mid a^b-1
        	\end{align*}
    \end{problem}

    \begin{solution}
        Let $p$ be the least prime divisor of $b$. Let $m$ be the least positive integer for which $p|a^m-1$. Then $m|b$ and $m\mid p-1$, so any prime divisor of $m$ divides $b$ and is less than $p$. Thus, not to run into a contradiction, we must have $m=1$. Now, if $p$ is odd, we have $av_p(b)\le v_p(a-1)+v_p(b)$, so
        	\begin{align*}
	        	a-1
	        		& \le (a-1)v_p(b)\\
	        		& \le v_p(a-1)
        	\end{align*}
        which is impossible. Thus $p=2$, $b$ is even, $a$ is odd, and
        	\begin{align*}
        		av_2(b)
        			& \le v_2(a-1)+v_2(a+1)+v_2(b)-1
        	\end{align*}
        whence
        	\begin{align*}
        		a
        			& \le (a-1)v_2(b)+1\\
        			& \le v_2(a-1)+v_2(a+1)
        	\end{align*}
        which is possible only if $a=3$ and $v_2(b)=1$. Put $b=2B$ with odd $B$ and rewrite the condition as $2^3B^3\mid 3^{2B}-1$. Let $q$ be the least prime divisor of $B$ (now, surely, odd). Let $n$ be the least positive integer such that $q\mid 3^n-1$. Then $n\mid 2B$ and $n\mid q-1$ whence $n$ must be $1$ or $2$ (or $B$ has a smaller prime divisor), so $q\mid 3-1=2$ or $q\mid 3^2-1=8$, which is impossible. Thus $B=1$ and $b=2$.
    \end{solution}

    \begin{problem}
        Find all positive integer solutions of the equation $ x^{2009} + y^{2009} = 7^z$
    \end{problem}

    \begin{solution}
        Factor 2009. We have $2009 = 7^2\cdot 41$. Since $x+y\mid x^{2009}+y^{2009}$ and $x+y>1$, we must have $7 \mid x+ y$. Removing the highest possible power of $7$ from $x, y$, we get
        	\begin{align*}
        		v_7(x^{2009} + y^{2009}) = v_7(x + y)+ v_7(2009) = v_7(x +y) +2
        	\end{align*}
        so $x^{2009} + y^{2009} = 49\cdot k\cdot (x + y)$ where $7\nmid k$. But we have $x^{2009} + y^{2009} = 7^z$, which means the only prime factor of $x^{2009} + y^{2009}$ is $7$, so $k=1$. Thus $x^{2009} + y^{2009} = 49(x + y)$.  But in this equation the left hand side is much larger than the right hand one if $\max(x, y) > 1$, and, clearly, $(x, y) = (1, 1)$ is not a solution. Thus the given equation does not have any solutions in the set of positive integers.
    \end{solution}
