\chapter{Modular Arithmetic}\label{ch:congruence}

\documentclass{subfile}

\begin{document}
	\section{Basic Modular Arithmetic}
	Consider the timestamp we use in our daily life. Certainly, there was a point when people started counting time. Then, why is it not something like, $2147483647$? Rather we say something like $12.09$ am (and there is a date of course, that separates two $12.09$ am). The reason is, each time the hour hand in a clock crosses $12$, it starts from $1$ again, not $13$. If the numbers kept going large, we would have a hard time realizing what time we are living in. Similarly, when the second hand ticks $60$ times, it starts from $1$ again (meaning it has been $1$ minute, letting the minute hand tick once). Here, intentionally or inadvertently, we have been using what number theorists call \textbf{modular arithmetic}. The idea is, we keep the integers that leave the same remainder (when divided by a certain integer) in the same \textit{class}. It will be clear afterwards what exactly we mean by class here when we discuss complete set of residue class\watermark.
	\begin{definition}
		For a non-zero integer $m$, integers $a$ and $b$ are \textit{congruent modulo} $m$ if and only if $m\mid a-b$. We show this by the notation 
			\begin{align*}
				a \equiv b \pmod m
			\end{align*}
		If $m$ does not divide $a-b$, we say that $a$ and $b$ are not congruent modulo $m$ and denote it by $a \not \equiv b \pmod m$.
	\end{definition}
	
	\begin{note}
		$ $
		\begin{enumerate}
			\item It is clear that if $m\mid a-b$, then $-m\mid a-b$. So from now on, we assume that $m$ is a \textit{positive} integer.
			\item If $a$ is divisible by $m$, then $a \equiv 0 \pmod m$. So, for example, an integer $a$ is even if and only if $a \equiv 0 \pmod 2$.
		\end{enumerate}
	\end{note}

	
	\begin{example}
		$3 \equiv 7 \pmod 4$,	$5^2 \equiv -1 \pmod {13}$, $n^2-1 \equiv 0 \pmod {n+1}$.
	\end{example}
	
	\begin{proposition}
		Assume that $a$ and $b$ are two integers and $m$ is a positive integer. Then the following propositions are correct.
			\begin{enumerate}[i.]
				\item If $a$ is divided by $b$ with remainder $r$, then $a$ is congruent to $r$ modulo $b$.
				\item If $a\equiv b\pmod m$, then for any divisor $d$ of $m$, $a\equiv b\pmod d$.
				\item $a \equiv a \pmod m$. We call this the \textit{reflexivity} property of modular congruences.
				\item If $a \equiv b \pmod m$, then $b \equiv a \pmod m$. We call this the \textit{symmetry} property.
				\item If $a \equiv b \pmod m$ and $b \equiv c \pmod m$, then $a \equiv c \pmod m$. We call this the \textit{transitivity} property.
				\item If $a \equiv b \pmod m$ and $c \equiv d \pmod m$, then $a\pm c \equiv b \pm d \pmod m$ and $ac \equiv bd \pmod m$.
				\item If $a \equiv b \pmod m$, then for any integer $k$, $ka \equiv kb \pmod m$.
			\end{enumerate}
	\end{proposition}
	
	\begin{proposition}\slshape\label{prop:powercong}
		If $n$ is a positive integer and $a \equiv b \pmod m$, then $a^n \equiv b^n \pmod m$.
	\end{proposition}
	
	\begin{proof}
		From the definition, $a \equiv b \pmod m$ means $m\mid a-b$. We know from Theorem \ref{id:fatandthin} that
		\begin{align*}
			a^n-b^n & = (a-b)\Big(a^{n-1}+a^{n-2}b+\cdots+b^{n-1}\Big).
		\end{align*}
		This gives $a-b\mid a^n-b^n$. So $m\mid a^n - b^n$, or $a^n \equiv b^n \pmod m$.
	\end{proof}	
	
	\begin{proposition}\slshape
		If $f(x)$ is a polynomial with integer coefficients and $a \equiv b \pmod m$, then $f(a) \equiv f(b) \pmod m$.
	\end{proposition}
	
	\begin{proof}
		Assume $f(x)=a_n x^n + a_{n-1} x^{n-1} + \cdots + a_1 x + a_0$. Use Proposition \eqref{prop:powercong} to get $a_i a^i \equiv a_i b^i \pmod m$ and add up all the terms.
	\end{proof}	
	
	
	\begin{proposition}\slshape
		If $a$ is an integer and $n$ is a positive integer, then exactly one of the following relations holds.
			\begin{align*}
				a &\equiv 0 \pmod n,\\
				a &\equiv 1 \pmod n,\\
				  &\vdots\\
				a &\equiv n-1 \pmod{n}.
			\end{align*}
	\end{proposition}
	
	
	\begin{theorem}\slshape
		Let $m$ be a positive integer and $a, b$, and $c$ be integers. Then
		\begin{enumerate}[(a)]
			\item If $ac \equiv bc \pmod m$ and $\gcd(c,m)=d$, then 
				\begin{align*}
					 a & \equiv b \mod{\Big(\dfrac{m}{d}\Big)}. 
				 \end{align*}
			We will call this the \textit{cancellation rule} for congruence.
			\item If $b \equiv c \pmod m$, then $\gcd(b,m)=\gcd(c,m)$.
		\end{enumerate}
	\end{theorem}
Before we prove it, let us see some examples. Usually, $ac=bc$ implies $a=c$ in equations and so far we have seen congruences maintain equation relations. However, is this any different in division? This is another trap you may fall into.

Since $15$ divides $35-20$, $5\cdot7\equiv5\cdot4\pmod{15}$. If we could just do division, this would give us
	\begin{align*}
		7
			& \equiv4\pmod{15},
	\end{align*}
which is clearly false. But, why? Here is the reason: $15=5\cdot3$. And when we canceled $5$ without thinking where that $5$ came from in $15$, we accidentally took out the only portion where $5$ came from. So we can not do that recklessly. However, this also means that if we took out $5$ from all sides, it would be true:
	\begin{align*}
		7\equiv4\pmod{3}.
	\end{align*}

	\begin{proof}
		$ $
		\begin{enumerate}[(a)]
			\item The greatest common factor of $c$ and $m$ is $d$, so  there exist integers $c_1$ and $m_1$ such that
				\begin{align*}
				 	c=c_1d, m = m_1 d & \mbox{and } \gcd(c_1,m_1) = 1.
				\end{align*}
			Since $ac \equiv bc \pmod m$, we have $m=m_1d\mid (a-b)c=(a-b)c_1d$. Canceling $d$ from both sides, we get $m_1\mid (a-b)c_1$. But $\gcd(c_1,m_1)=1$, and so by Proposition \ref{prop:cpdiv}, we get $m_1\mid a-b$. Thus,
				\begin{align*} 
					a 
						& \equiv b \pmod{m_1},
				\end{align*}
			as desired.
			
			\item Because $b \equiv c \pmod m$, there exists an integer $k$ for which $b-c=mk$. So $\gcd(b,m)\mid c$. On the other hand, from definition of $\gcd$, it is clear that $\gcd(b,m)\mid m$. Now by proposition \ref{prop:dividegcd} we have $\gcd(b,m)\mid \gcd(c,m)$. Similarly, one can show that $\gcd(c,m)\mid \gcd(b,m)$. Using Proposition \ref{prop:bothdivide}, we get $\gcd(c,m)=\gcd(b,m)$. 
		\end{enumerate}
	\end{proof}
	
	\begin{definition}[Arithmetic Progression]\label{def:AP}
		A sequence $a_1, a_2, a_3, \cdots$ of real numbers is called an \textit{arithmetic sequence}, \textit{arithmetic progression}, or \textit{AP} if each new term of the sequence is obtained by adding a constant real number $d$, called the \textit{common difference}, to the preceding term. In other words, the terms of an arithmetic progression are of the form
			\begin{align*}
				a, a+d, a+2d, a+3d, \cdots
			\end{align*}
		where $a$ is the \textit{initial term} of the sequence.
	\end{definition}
	
	\begin{example}
		The sequence of odd numbers is an AP. The following sequence
			\begin{align*}
				-3, 2, 7, 12, 17, \cdots
			\end{align*}
		which includes numbers of the form $5k+2$ for $k=-1,0,1,\cdots$, is also an arithmetic sequence with initial term $-3$ and common difference $5$.
	\end{example}
	
	\begin{corollary}
		All terms of an arithmetic progression are equivalent modulo the common difference.
	\end{corollary}
What is the sum of the terms of an arithmetic progression? Obviously, if the sequence has \textit{infinite} number of terms, that is, if it has infinitely many terms, then the sum is not a finite number as the common difference is constant\footnote{we say that it \textit{diverges}.}. However, when the arithmetic progression is finite (such as, a portion of the sequence), the sum of all its elements is finite as well. Often we consider partial sum of such a series.
	\begin{theorem}\slshape
		Let $(a_1,a_2,\cdots,a_n)$ be a finite arithmetic sequence such that
			\begin{align*}
				a_1 & = a,\\
				a_2 & = a+d,\\
					& \vdots\\
				a_n & = a+ (n-1)d,
			\end{align*}
		where $n$ is a positive integer and $a$ and $d$ are reals. The sum of all elements of this AP is
			\begin{align*}
				\sum_{i=1}^{n} a_i
					& = \frac{n}{2} \left(a_1+a_n\right) = \frac{n}{2} \big(2a_1 + (n-1)d\big).
			\end{align*}
	\end{theorem}
	
	\begin{proof}
		We have
			\begin{align*}
				\sum_{i=1}^{n} a_i
					&= a_1 + a_2 + \ldots + a_n \\
					&= a + (a+d) + \ldots + \Big(a+ (n-1)d\Big)\\
					&= na + d\Big(1+2+\ldots+ (n-1)\Big).
			\end{align*}
		From Identity \ref{id:sumofpowers} of Appendix \eqref{ch:appendices}, $1+2+\ldots+n = n(n-1)/2$. Therefore,
			\begin{align*}
				\sum_{i=1}^{n} a_i = \dfrac{n}{2} \big(2a_1 + (n-1)d\big).
			\end{align*}
	\end{proof}
	
	\begin{definition}[Geometric Progression]
		A sequence $a_1, a_2, a_3, \cdots$ of real numbers is called an \textit{geometric sequence}, \textit{geometric progression}, or \textit{GP} if each new term of the sequence is obtained by multiplying the previous term by a constant real number $r$, called the \textit{common ratio}. In other words, the terms of an arithmetic progression are of the form
			\begin{align*}
				a, ar, ar^2, ar^3, \cdots
			\end{align*}
		where $a$ is the \textit{initial term} of the sequence.
	\end{definition}
	
	\begin{example}
		The sequence of powers of $2$ is a geometric progression. The sequence
			\begin{align*}
				\dfrac{1}{2}, \dfrac{1}{6}, \dfrac{1}{18}, \dfrac{1}{54}, \cdots
			\end{align*}
		is a GP with initial term $1/2$ and common ratio $1/3$.
	\end{example}
	
	Similar to arithmetic progressions, the sum of terms of a finite geometric progression is always possible to find. An interesting question would be what happens if we add all the terms of an \textit{infinite} geometric sequence? For example, what is the value of the following sum?
		\begin{align*}
			1 + \dfrac{1}{2}+ \dfrac{1}{4}+\dfrac{1}{8}+ \ldots
		\end{align*}
	This is not a finite sequence. But is the sum \textbf{divergent} or \textbf{convergent}? The terms of the above sequence are gradually decreasing and approach zero. To see this, notice that the ninth term is
		\begin{align*}
			\frac{1}{256} = 0.00390625,
		\end{align*}
	which is very close to zero. So, on a second thought, we can guess that the given sum has a finite value. In general, when the absolute value of common ratio of a geometric progression is less than one, that is, when the absolute value of each term of the sequence is smaller that its preceding term, then the \textit{geometric series} (either finite or infinite) has a finite value\footnote{to put it differently, it \textit{converges} to a fixed value.}. We will see this from a different point of view. This is due to \textit{Chamok Hasan}, a teacher of the first author.
	
	Consider a pumpkin. Let us assume that it is totally symmetrical. Now, divide it in half and put aside half of it. You have half of the pumpkin to yourself. Divide it in half again. Keep one to yourself and discard the other half. So now you have one fourth of the pumpkin. Again, cut it in half. Keep one, discard one. Now you have one eighth. See that if you keep going this way, you end up getting $\frac{1}{2},\frac{1}{4},\frac{1}{8},\frac{1}{16},\cdots$ and dividing them. And the fun fact is, you can keep doing this for as many times as you want. Obviously, if we put together all the parts again, we get the whole pumpkin. That is, if we take all the discarded portions and put them back, the pumpkin becomes whole again. This shows us without any rigorous proof that
		\begin{align*}
			\dfrac{1}{2}+\dfrac{1}{4}+\dfrac{1}{8}+\dfrac{1}{16}+\ldots=1.
		\end{align*}
	Now you should be able to make sense how a sequence with infinite terms can have a finite sum.
		\begin{theorem}\slshape\label{thm:GP}
			Let be given the finite geometric sequence
				\begin{align*}
					a_1 & = a,\\
					a_2 & = ar,\\
					& \vdots\\
					a_n & = ar^{n-1},
				\end{align*}
			where $n$ is a positive integer and $a$ and $r \neq 1$ are reals. The sum of all elements of this GP is
				\begin{align*}
					\sum_{i=1}^{n} a_i &= \dfrac{a \left(r^n-1\right)}{r-1}.
				\end{align*}
		\end{theorem}
		
		\begin{proof}
			Multiply the sum by $(r-1)$ to obtain
				\begin{align*}
					(r-1)\sum_{i=1}^{n} a_i &= (r-1)(ar^0 + ar^1 + \cdots + ar^{n-1})\\
											&= a(r-1)(1+r+ \cdots + r^{n-1})\\
											&= a(r^n -1).
				\end{align*}
			We have used Theorem \ref{id:fatandthin} to write the last line. Since $r \neq 1$, we can divide both sides by $r-1$ to get the desired result.
		\end{proof}
	Did you notice anything? In fact, this is a special case of what we encountered in divisibility. Recall the expansion of $a^n-b^n$ and try to find a correlation between the two.
		\begin{corollary}
			Take the geometric progression in Theorem \ref{thm:GP}. If $|r|<1$, then the sum converges. More precisely,
				\begin{align*}
					\sum_{i\geq1} a_i &= \frac{a}{1-r}.
				\end{align*}
		\end{corollary}
		
		\begin{proof}
			If $|r|<1$, then $|r^n|$ decreases as we increase $n$. Therefore, when $n$ is very large, $|r^n|$ is almost zero. Here, we should borrow the idea of \textit{limit} from calculus but for now, let us convince ourselves\footnote{We are trying to avoid situations such as $r^\infty=0$ since that is a wrong concept. Because infinity is not a number.} that as $n\to\infty$, $r^n\to0$. This gives us
				\begin{align*}
					\sum_{i\geq1} a_i = \frac{a \left(0-1\right)}{r-1}= \frac{a}{1-r},
				\end{align*}
			which is what we wanted.
		\end{proof}
		
		\begin{note}
			In the footnote below this page, we used $r^{\infty}$ to mean that $r$ is raised to a very large power and thus is almost zero. The use of notation should not be misleading.
		\end{note}
		
		\begin{example}
		$ $
			\begin{itemize}
				\item We can now compute $\displaystyle 	1 + \frac{1}{2}+ \frac{1}{4}+\frac{1}{8}+ \cdots$. In fact, this is an infinite geometric series with initial term $a=1$ and common ratio $r=\frac{1}{2}<1$. Thus,
					\begin{align*}
						\sum_{i\geq0} \left(\frac{1}{2}\right)^i = \dfrac{1}{1-\frac{1}{2}} = 2.
					\end{align*}
				\item Suppose that we want to find
					\begin{align*}
						2 + (-6) + 18 + (-54) + \cdots + (-39366) + 118098
					\end{align*}
				This is a geometric sequence with common ratio $-3$ and initial term $2$. The last term equals $2(-3)^{10}$. So,
					\begin{align*}
						\sum_{i=0}^{10} 2(-3)^{i} = \frac{2 \left((-3)^{11} -1\right)}{(-3) - 1} = 88,573.
					\end{align*}
			\end{itemize}
		\end{example}
	\section{Modular Exponentiation} \label{modexponent}
	In the early stage of problem solving, we all calculate big integers modulo an integer. For example, consider the next problem.
		\begin{problem}
			Define $a_n=6^n+8^n$. Find the remainder of $a_{49}$ when divided by $49$.
		\end{problem}
	The first idea that crosses your mind might be calculating $6^{49}$ and finding the remainder when divided by $49$. This would be a large integer and the calculation is really tedious, not to mention, pointless. A slight improvement would be multiplying $6$ with $6$ and taking modulo $49$ each time. We need to do this for $49$ times but at least, now we do not have to deal with that large numbers anymore. Let us call this \textit{iterative exponentiation method}.
	
	Suppose we want to find $c \equiv a^k \mod n$. The iterative exponentiation method computes the values $1=a^0, a^1, a^2, \cdots, a^k=c$ modulo $n$ instead of directly calculating $a^k$ modulo $n$. Suppose that we have computed $a^i$ modulo $n$ for some $i<k$ and the result is $b$. According to the above theorem, to calculate $a^{i+1}$, all we need to do is to compute $a\cdot b \mod n$. Obviously, $a \cdot b$ is much smaller than $a^i$ when $i$ is large. This is why this method takes less time for computations. Iterative exponentiation may be expressed as an algorithm as shown below.
	
	\paragraph{Iterative Exponentiation Algorithm}
	\begin{enumerate}[1.]
		\item Set $k_1 \longleftarrow 0$ and $c \longleftarrow 1$.
		\item Increase $k_1$ by $1$.
		\item Set $c \longleftarrow a \cdot c \pmod n$.
		\item If $k_1<k$, go to step $2$. Otherwise return $c$.
	\end{enumerate}
	
	\begin{example}
		Let's compute $5^{20}$ modulo $751$ by iterative exponentiation algorithm. Table \ref{table:modmult} shows $5^i \mod{751}$ for $i=1$ to $20$. As obtained from the table, $5^{20} \equiv 200 \pmod{751}$, which is in agreement with what we previously found.
		\begin{table}
			\centering
			\begin{tabular}{|c|c|c|c|} 
				\hline 
				$k_1$ & $5^{k_1} \pmod{751}$ & $k_1$ & $5^{k_1} \pmod{751}$ \\ 
				\hline 
				1 & 5 & 11 & 358  \\ 
				\hline 
				2 & 25 & 12 & 288 \\ 
				\hline 
				3 & 125 & 13 & 689 \\ 
				\hline 
				4 & 625 & 14 & 441  \\ 
				\hline 
				5 & 121 & 15 & 703 \\ 
				\hline 
				6 & 605 & 16 & 511 \\ 
				\hline 
				7 & 21 & 17 & 302 \\ 
				\hline 
				8 & 105 & 18 & 8 \\ 
				\hline 
				9 & 525 & 19 & 40 \\ 
				\hline 
				10 & 372 & 20 & 200 \\ 
				\hline 
			\end{tabular} 
			\caption{Applying iterative exponentiation method to find $5^{20}$ modulo $751$.}
			\label{table:modmult}
		\end{table}
	\end{example}
	
	\begin{remark}
		In the above example where $a$ is small (compared to modulus $n$), we can increase $k_1$ more than one unit in each iteration of the algorithm. For example, in above example, we could increase $k_1$ two units each time to compute $5^2, 5^4,\cdots, 5^{20}$. In this case, the number of calculations is divided by two and therefore there will be less time needed to find the result. This is done in general case but one must notice that when one increases $k_1$, say, two units at each step, he is in fact computing $a^2 \cdot c \mod n$ instead of $a \cdot c \mod n$ in step $3$ of the algorithm to reduce the number of iterations of the algorithm. If $a$ is small, there will be no difference in computation time. But if $a$ is (too) large, computing $a^2 \cdot c \mod n$ may will be more time consuming and it may reduce the time efficiency of algorithm.
	\end{remark}
	
	A more efficient method to do this is \textit{modular exponentiation algorithm}. The beauty of this idea is that you can use it to compute big integers modulo $n$ by hand. The idea actually inherits from binary representation. Consider the binary number $(101101)_2$. We discussed how to convert it to a decimal integer in base conversion. However, Masum uses a variation for faster mental calculation. Start from the left most digit (which always will be $1$ if there is no leading $0$). Initially, the decimal integer is $1$. Now, go to the next digit. If it is $0$, double the current value. If it is $1$, double and add $1$. Since the next digit is $0$, we have $2$. Next digit is $1$. So it will become $2\cdot2+1=5$. Next digit is $1$ as well. It will be $5\cdot2+1=11$. Next digit is $0$, so we have $11\cdot2=22$. Next digit is $1$, so it will be $22\cdot2+1=45$. There is no more digits left, so this is the desired value in decimal. You can verify that this indeed is the intended result. And more importantly, think why this works if you have not figured it out already!
	
	We just saw a way of converting binary numbers into decimal. How does that help us in modular exponentiation? Assume that we want $a^k\pmod n$. We will not compute it directly or iteratively. Instead, represent $k$ in binary. Then, initially, the result is $1$. Divide $k$ by $2$ and keep the remainder. If it is $1$, multiply the current result by $a$ and do the modulo operation, that is, $r\to ra\pmod n$. Also, set $a\to a^2\pmod n$. Keep doing this until $k=0$. In the end we will see $r\equiv a^k\pmod n$. Again, make sense why this works. Do the example above this way and see if the result matches. Algorithm to find $a^k\pmod n$.
	\paragraph{Modular Exponentiation Algorithm ($a^k\pmod n$)}
		\begin{enumerate}[1.]
			\item Set $R \longleftarrow 1$.
			\item If $k=0$, stop and return the value $R$. Otherwise, continue.\label{alg:stopme}
			\item Divide $k$ by $2$, take the remainder $r$. That is, set $k\longleftarrow \lfloor k/2\rfloor$.
			\item Set $a\longleftarrow a^2\pmod n$.
			\item If $r=1$, set $R\longleftarrow Ra\pmod n$.
			\item Go to step $2$.
		\end{enumerate}
	However, we face another concern here. What if $Ra$ is very large? We can take care of it the same way. Express $a$ in binary and take modulo from there. Algorithm to find $ab\pmod n$ for large $b$.
		\begin{enumerate}[1.]
			\item Set $R=1$.
			\item If $b=0$, return $R$, otherwise continue.
			\item Set $b\longleftarrow \lfloor b/2\rfloor$ and $r=b\pmod2$.
			\item If $r=1$, set $R\longleftarrow R+a\pmod n$.
			\item Set $a = (2\cdot a)\pmod n$.
			\item Go to step $2$.
		\end{enumerate}
	We can call this \textit{modular multiplication}. This way, we will not have to actually multiply two numbers to get the remainder. The proofs for the last two ideas were not shown deliberately. We expect that you can do it easily. By the way, did you notice something else too? In modular exponentiation, we do not have to iterate $k$ times. The number of times we need to iterate is actually $\lfloor \log_2(k)\rfloor+1$ (again, why?). Same goes for modular multiplication. Therefore, it is a very desirable improvement. In fact, these methods are highly used in primality tests or similar fields (we will discuss about primes in Chapter \ref{ch:primes}).
	
	Notice that, we can write modular exponentiation algorithm in a better fashion.
	\paragraph{Modular Exponentiation Algorithm - Cleaner Version ($a^k\pmod n$)}
		\begin{enumerate}[1.]
			\item Set $R=1$.
			\item Represent $k$ in binary. Assume $k=(x_0x_1\cdots x_l)_2$.
			\item If $k=0$, return $R$.
			\item Find $r = (k\pmod2)$.
			\item If $r=1$, set $R\longleftarrow Ra\pmod n$.
			\item Set $k\longleftarrow \lfloor k/2\rfloor$.
			\item Set $a\longleftarrow a^2\pmod n$.
			\item Go to step $3$.
		\end{enumerate}
	
	\begin{example}
		Let us calculate $5^{20} \pmod{751}$ this way. First, we need to find the binary representation of $20$, which is $(10100)_2$. Then, we can write
			\begin{align*}
				5^{20} \equiv  \underbrace{\Bigg(\underbrace{\bigg(\overbrace{\overbrace{\big(\underbrace{5^2}_{R_1}\big)^2}^{R_2} \cdot 5}^{R_3}\bigg)^2}_{R_4}\Bigg)^2}_{R_5} \pmod{751}.
			\end{align*}
		This is how we proceed: we want to construct $5^{20}$. The rightmost digit is zero. What happens if we remove this digit? The number gets divided by $2$. This is identical to writing $5^{20} = 5^2 \cdot 5^{10}$. Therefore, we first compute $R_1=5^2$. We now need to construct $5^{10}$. The binary representation of $10$ is $(1010)_2$. Again, divide it by two to remove the rightmost zero. This time, we are doing this operation:
			\begin{align*}
				5^{20} = R_1 \cdot 5^2 \cdot 5^5.
			\end{align*}
		We must compute $R_2 = R_1 \cdot 5^2$ at this stage. Now, how do we construct $5^5$ given its binary representation $(101)_2$, which does not end in zero? It's easy. We just have to write it as $1+(100)_2$. Now, we have $(100)_2$ which ends in zero. In other words,
			\begin{align*}
				5^{20} = R_2 \cdot 5 \cdot 5^4
			\end{align*}
		So, we calculate $R_3 = R_2 \cdot 5$ at this stage and try to construct $5^4$. The rest of the solution is similar and we expect the reader to finish it. Try to find $R_4$ and $R_5$ for yourself. In case you want to check your answers, you can consult table \ref{table:modexp}.
%		Initially, $R_0=1$ and $a_0=5$ (step $1$ in the algorithm). In the first iteration of the algorithm, we divide $20$ by $2$ and find the quotient of $10$ and remainder of zero. So, $a_1=a_0^2 \pmod {751}$ and $R_1=R_0$. In the second iteration, the quotient is $5$ and remainder is still zero. So, $a_2=a_1^2 \pmod {751}$ and $R_2=R_1$. In the third iteration, we have a quotient of $2$ and remainder of $1$. In this case, $a_3=a_2^2 \pmod {751}$ and $R_3=R_2 a_3$. The fourth iteration gives the quotient $1$ and the remainder is zero: $a_4=a_3^2 \pmod {751}$ and $R_4=R_3$. Finally, we find  $a_5=a_4^2 \pmod {751}$ and $R_5=R_4a_5$.
		
		\begin{table}[ht]
			\centering
			\begin{tabular}{|c|c|c|c|c|c|} 
				\hline 
				$i$ & 1 & 2 & 3 & 4 & 5  \\ 
				\hline 
				$R_i$ & $25$ & $625$ &  $121$ & $372$ & $200$\\ 
				\hline
			\end{tabular} 
			\caption{Applying modular exponentiation method to find $5^{20}$ modulo $751$.}
			\label{table:modexp}
		\end{table}
	\end{example}
	\section{Residue Systems}
	Residue systems are very simple definitions which will help you make a good sense of some later-explained theorems such as Fermat's and Euler's.
	\subsubsection{Complete Residue Systems}
	The definitions and theorems below assume that $m$ is a positive integer.
		\begin{definition}
			Two integers $a$ and $b$ are said to be members of the same \slshape{residue class} modulo $m$, if and only if $a \equiv b \pmod m$.
		\end{definition}
		
		Clearly, there are $m$ distinct residues modulo $m$.
		
		\begin{definition}\label{def:completeresiduesystem}
			Let $m$ be a positive integer. The set $A$ is called a \slshape{complete residue system modulo $m$} if and only if every number is congruent to a unique element of $A$ modulo $m$. In other words, $A$ should be representing all the residue classes modulo $m$.
		\end{definition}
		
		\begin{example}
			$A = \{0,1, \cdots, m-1\}$ is a complete residue system modulo $m$. So is $B=\{15, 36, -7, 27, 94\}$ modulo $5$.
		\end{example}
	We will state two simple propositions without proof. The reader should be able to prove them on their own.
	
	\begin{proposition}
		The set $A=\{a_1, a_2, \cdots, a_k\}$ is a complete residue set (or system) modulo $m$ if and only if $k=m$ and $a_i \not\equiv a_j \pmod m$ for $i \neq j$.
	\end{proposition}
	
	\begin{proposition} \label{prop:generalcompletesystem}
		Let $A=\{a_1, a_2, \cdots, a_m\}$ be a complete residue set modulo $m$ and let $a,b$ be integers such that $a \bot m$. Then the set
		\begin{align*}
			B=\{aa_1+b, aa_2+b, \cdots, aa_m+b\}
		\end{align*}
		is also a complete residue set modulo $m$.
	\end{proposition}
	
	\subsubsection{Reduced Residue Systems and Euler's Totient Function} 
	We really wish that you have a firm grasp of \textit{function}. However, if you are in $10$th grade or below, there is a good chance, you are not familiar with the concept of functions very well. Since that is entirely a different topic, we restrain ourselves from discussing it. Make sure you at least realize what function actually is. Here, we will say a thing about function or two but it is not nearly enough for covering the fundamentals.
	
	A \textit{function} is like a machine. It takes a number as its input, \textit{functions on the number}, and gives another number as its output with the property that each input is related to exactly one output. This property seems logical. Consider a weighing scale designed to measure the weight of people. Obviously, a person cannot be both $70$ and $75$ kilograms at the same time. The weight of a person (in a specific time) is a constant number, and hence the weighing scale actually works as a function: it takes a person as its input, measures his weight, and then shows the person's weight as its input. 
	
	Another example would be a function that takes a real number $x$ as its input and gives $x^2$ as its output. For convenience, we can call this function $f: \mathbb R \to \mathbb R$ and write its \textit{relation} as $f(x) = x^2$ for all $x \in \mathbb R$. The notation $f: \mathbb S \to \mathbb T$ means that the function $f$ takes its inputs from the \textit{domain} $S$ (the set of inputs) and assigns them an output from the \textit{codomain} $T$ (the set of outputs and maybe some additional elements). For the previous example, we see that both domain and codomain of $f$ are $\mathbb R$. However, the \textit{range} (or \textit{image}) of $f$, which is the set containing only outputs of $f$, is $\mathbb R^{+}$, the set of all positive real numbers.
	
	\begin{definition}[Euler's Totient Function]\label{def:totient} 
		For every positive integer $n>1$, $\varphi(n)$ is the number of positive integers less than or equal to $n$ which are relatively prime to $n$. We call this function Euler's phi function (or totient\footnote{You might be wondering what \slshape{totient} means. In Latin, \slshape{tot} means so many. The suffix of \slshape{iens} is probably from the Sanskrit.} function).
	\end{definition}
	
	\begin{example}
		$\varphi(5)=4$ and $\varphi(10)=4$. 
	\end{example}
	
	We will investigate properties of this function in details in Chapter \ref{ch:arithfunc}. For now let's just assume the following claims are true.
	\begin{proposition}[Properties of Euler's Totient Function]\label{prop:phiproperties}\slshape
		Let $m$ and $n$ be two positive integers.
		\begin{enumerate}[(a)]
			\item $\varphi$ is a multiplicative function. That is, if $m \bot n$, then
			\begin{align*}
				\varphi(mn)=\varphi(m) \cdot \varphi (n).
			\end{align*}
			\item For all $n \geq 3$, $\varphi(n)$ is even.
			\item $\varphi$ is neither increasing\footnote{The function $f$ is increasing if for $a_1 >a_2$, we have  $f(a_1) > f(a_2)$.}, injective\footnote{The function $f$ is injective if for $a_1 \neq a_2$, we have $f(a_1) \neq f(a_2)$.} nor surjective\footnote{The function $f:X \to Y$ is surjective if for every $y \in Y$, there exists $x \in X$ such that $f(x)=y$.}. 
			\item If $n$ is factorized as $n= p_1^{\alpha_1} p_2^{\alpha_2} \cdots p_k^{\alpha_k}$, then
			
			\begin{align*}
				\varphi(n) & =n \left( 1 - \frac{1}{p_1} \right)  \left( 1 - \frac{1}{p_2} \right)  \cdots \left( 1 - \frac{1}{p_k} \right)  \\
				& = p_1^{\alpha_1-1} p_2^{\alpha_2-1} \cdots p_k^{\alpha_k-1} \left( p_1 -1 \right) \cdots \left( p_k -1 \right) .
			\end{align*}					
			
		\end{enumerate}
	\end{proposition}
	
	
	Why do we require function in congruence? Moreover, Euler's totient function? Before you decide it sounds irrelevant, take a look at the following example.
	
	Consider the integers $\{1,2,3,4,5,6\}$ (complete set of residue class modulo $7$ except $0$). Now, take an integer, say $3$. Multiply the whole set with $3$ and find the residues again:
		\begin{align*}
			\{3,6,9,12,15,18\} & \equiv\{3,6,2,5,1,4\}\pmod7.
		\end{align*}
	Does this look interesting? If not, take a look again. And try to understand what happened and why. Firstly, the products forms a residue class as well. Alternatively, it is a permutation of the residue class. Why? What if we multiplied by $5$? Check it out yourself and see if the conclusion holds. Check for some more integers like $10, 13,14$ etc. You will see the same is true for all integers except $0,7,14,\cdots$ i.e. multiples of $7$. Again, why? $7$ is a prime. So we know if an integer is not divisible by $7$, it is co-prime to $7$. Take $a$ such that $a\bot7$. Now, what does it mean that the set of products is a permutation of the original? We could state it this way: no two products leave the same remainder when divided by $7$. And you can see, if this is true, everything makes sense. If we can show that for $0<i<j<7$, $ia$ and $ja$ are not congruent modulo $7$, we are done! That is indeed the case. For the sake of contradiction, assume that,
		\begin{align*}
			ia  \equiv &ja\pmod7\\
			\iff 7\mid ia-ja&=a(i-j)
		\end{align*}
	Here $a\bot7$, so we have $7\mid i-j$. But remember that, $i<j<7$ so $|i-j|<7$. This yields the contradiction we were looking for. This claim was true mainly because $a\bot7$. What if we did not take a prime $7$? Well, we could still do something similar. And that is why Euler's totient function comes to the play. This is more valuable than you may realize.
	
	\begin{definition}
		Let $m$ be a positive integer. The set $A$ is called a \slshape{reduced residue system modulo $m$} if all elements of $A$ are coprime to $m$, and also every integer which is coprime to $m$ is congruent to a unique element of $A$ modulo $m$.
	\end{definition}
	
	\begin{example}
		The set $A=\{ 1, 2, \cdots, p-1 \}$ is a reduced residue system modulo a prime $p$. The set $\{7, 17\}$ is also a reduced residue system mod $6$.
	\end{example}
	
	You can clearly sense how the Euler's phi function is related to reduced residue systems: the number of elements of $A$ is $\phi(m)$. So we can express the above definition in a better way:
	
	
	\begin{proposition}
		The set $A=\{a_1, a_2, \cdots, a_k\}$ is a reduced residue set (or system) modulo $m$ if and only if
		\begin{itemize}
			\item $a_i \perp m$ for all $i$,
			\item $k=\varphi(m)$, and
			\item $a_i \not \equiv a_j \pmod m$ for $i \neq j$.
		\end{itemize}
	\end{proposition}
	
	An important aspect of the reduced systems is stated in the next proposition. It will help us prove the Euler's theorem later.	
	
	\begin{proposition}\label{prop:generalreducedsystem} 
		Let $A=\{a_1, a_2, \cdots, a_{\varphi(m)}\}$ be a reduced residue set modulo $m$ and let $a$ be an integer such that $a \bot m$. Then the set
		\begin{align*}
			B=\{aa_1, aa_2, \cdots, aa_{\varphi(m)}\}
		\end{align*}
		is also a reduced residue set modulo $m$.
	\end{proposition}
	
	The proof of this theorem is pretty easy, try it for yourself. Pay attention to the difference between this proposition and the similar Proposition \ref{prop:generalcompletesystem} for complete systems. 
	

	The latest theorem says that there are infinitely many reduced residue systems for any $m$. So, it makes sense to define a set as the original reduced residue system for any positive integer $m$. We call this set $\mathbb U_m$.
	
	\begin{definition}\label{def:setofunits}
		Let $m$ be a positive integer. The \textit{set of units} modulo $m$, $\mathbb U_m$, is the set of positive integers $g_1,\cdots,g_{\varphi(m)}$ less than $m$ which are coprime to $m$. 
	\end{definition}
	
	\begin{example}
		$\mathbb U_8=\{1,3,5,7\}$, and $\mathbb U_{15}=\{1,2,4,7,8,11,13,14\}$. If $p$ is a prime, then $\mathbb U_p=\{1,2,\cdots, p-1\}$.
	\end{example}

	
	You might be wondering why we call $\mathbb U_m$ the set of \textit{units}. In algebraic structures, a unit is an element $a$ for which there exists some element $b$ such that $ab=1$. In our case, the number $a$ is a unit if there exists some $b$ such that $ab \equiv 1 \pmod m$. As proved before, $a$ is a unit if and only if it is coprime to $m$, and this shows us why $\mathbb U_m$ is called the set of units.
	
\end{document}
\documentclass{subfile}

\begin{document}
	\section{B\'{e}zout's Lemma}
	
	In this section, we are going to explain the simple but useful B\'{e}zout's lemma and then introduce modular multipliccative inverses.
	
	\subsection{B\'{e}zout's Identity and Its Generalization}
	Before representing this lemma, we would like you define the \textit{linear combination} of two integers.
	
	\begin{definition}\label{def:linearcombination}
		For two integers $a$ and $b$, every number of the form
		\begin{align*} ax+by\end{align*}
		is called a \itshape{linear combination} of $a$ and $b$, where $x$ and $y$ are integers.
	\end{definition}
	
	For example, $2a+3b$ and $-4a$ are both linear combinations of $a$ and $b$, but $a^2-b$ is not. The \textit{B\'{e}zout's lemma} (sometimes called \textit{B\'{e}zout's identity} states that for every two integers $a$ and $b$, there exists a linear combination of $a$ and $b$ which is equal to $(a,b)$. For example if $a=18$ and $b=27$, then $18 \cdot (-1) + 27 \cdot 1 = 9 =(18,27)$.
	
	\begin{theorem} [B\'{e}zout's Identity] \slshape
		For two nonzero integers $a$ and $b$, there exists $x, y \in \mathbb Z$ such that
		\begin{align*}
		ax+by = (a,b)
		\end{align*}
	\end{theorem}
	
	You are probably familiar with this theorem. A simple proof uses Euclidean division, but it doesn't show you where exactly to use this identity. So, we prove a stronger theorem and the proof of B\'{e}zout's identity is immediately implied from it.
	
	\begin{theorem}\slshape \label{thm:equationgcd}
		Let $a,b,m$ be integers such that $a, b$ are not zero at the same time. Then the equation
		\begin{align*} ax + by = m\end{align*}
		has solutions for $x$ and $y$ in positive integers if and only if $(a,b) | m$.
	\end{theorem}
	
	\begin{proof}
		The first part is easy. Suppose that there exist integers $x_0$ and $y_0$ such that
		\begin{align*} ax_0 + by_0 = m.\end{align*}
		We know that $(a,b)|a$ and also $(a,b)|b$, thus $(a,b) | m$ and we are done.
		
		Conversely, if $(a,b)=d$ and $m$ is divisible by $d$, then we want to prove that there exist some positive integers $x$ and $y$ for which $ax+by=m$. First, we show that it's sufficient to show that there exist $x$ and $y$ such that
		\begin{align*} ax + by = d.\end{align*}
		The reason is simple: if there exist $x$ and $y$ such that $ax + by = d$, then
		\begin{align*} a\left( x \frac{m}{d} \right)  + b \left( y \frac{m}{d} \right) = m.\end{align*}
		Assume that $A$ is the set of all positive integer linear combinations of $a$ and $b$. $A$ is non-empty because if $a \neq 0$, then
		\begin{align*}	0<|a| = a\frac{|a|}{a} + b\cdot 0, \end{align*}
		and if $b \neq 0$, then
		\begin{align*}	0<|b| =  a\cdot 0 + b \frac{|b|}{b}. \end{align*}
		Because of well-ordering principle\footnote{The well-ordering principle states that every non-empty set of positive integers contains a least element.}, $A$ contains a least element. Let this smallest element be $t$. So there exist integers $x_0$ and $y_0$ such that
		\begin{align*} ax_0 + by_0 = t.\end{align*}
		We claim that $t|a$ and $t|b$. Using division theorem, divide $a$ by $t$:
		\begin{align*} a = tq+r, \quad 0 \leq r \leq t,\end{align*}
		and thus
		\begin{align*} a\underbrace{(1-qx_0)}_{=x_1}+b\underbrace{(-qy_0)}_{=y_1}=a-tq=r<t.	\end{align*}
		If $r \neq 0$, then $r$ is a positive integer written in the form $ax_1+by_1$, which is a positive integer linear combination of $a$ and $b$, so $r \in A$. But $r<t$, which is in contradiction with minimality of $t$. Therefore $r=0$ and $t|a$. We can prove that $t|b$ in a similar way. By Proposition \ref{prop:dividegcd}, we find that $t|d$. Also, according to the first part of the proof, we have $d|t$. Following Proposition \ref{prop:bothdivide}, $t=d$. This means that $d \in A$ and there exist integers $x$ and $y$ such that
		\begin{align*} ax + by = d.\end{align*}
	\end{proof}
	
	B\'{e}zout's Identity has many interesting applications. We will see one such application in Chapter \ref{ch:special}, to prove \textit{Chicken McNugget Theorem}. 
	
	We are now ready to represent a stronger version and also a generalization of B\'{e}zout's lemma.
	
	\begin{corollary} \slshape  [Stronger Form of B\'{e}zout's Identity]\label{cor:strongbezout}
		The smallest positive integer linear combination of $a$ and $b$ is $(a,b)$.
	\end{corollary}
	
	\begin{corollary} \slshape \label{cor:bezoutcoprime}
		If $a \perp b$ for non-zero integers $a$ and $b$, then there exist integers $x$ and $y$ such that
		\begin{align*}
		ax+by=1.
		\end{align*}
	\end{corollary}
	
	\begin{theorem} [Generalization of B\'{e}zout's Identity] \slshape 
		If $a_1, a_2, \cdots, a_n$ are integers with $(a_1, a_2, \cdots, a_n)=d$, then the equation
		\begin{align*}
		a_1x_1 + a_2x_2 + \cdots + a_n x_n = m
		\end{align*}
		has a solution $(x_1, x_2, \cdots, x_n)$ in integers if and only if $d|m$.
	\end{theorem}
	
	\begin{theorem}\slshape\label{thm:ax=b}
		Let $m$ be a positive integer and let $a$ and $b$ be positive integers. Then the modular arithmetic equation
		\begin{align*} ax \equiv b \pmod m\end{align*}
		has a solution for $x$ in integers if and only if $(m,a)|b$.
	\end{theorem}
	
	\begin{proof}
		Rewrite the congruence equation as $ax-my = b$ for some integer $y$. Now it is the same as Theorem \ref{thm:equationgcd}. The equation $ax-by=m$ has solutions if and only if $(m,a)|b$, which is what we want.
	\end{proof}
	
	\begin{problem}
		Let $a,b,$ and $c$ be non-zero integers such that $(a, c)=(b,c)=1$. Prove that $(ab,c)=1$.
	\end{problem}
	
	\begin{solution}
		By Corollary \ref{cor:bezoutcoprime}, there exist integers $x,y,z,$ and $t$ such that
			\begin{align*}
				ax+cy&=1, \text{ and}\\
				bz+ct&=1.
			\end{align*}
		Multiply these two equations to get
			\begin{align*}
				1 &= (ax+cy)(bz+ct)\\
				  &= ab(xz)+c(axt+byz+cyt).
			\end{align*}
		This means that we have found a linear combination of $c$ and $ab$ which is equal to $1$. From Corollary \ref{cor:strongbezout} it follows that $(ab,c)=1$ (why?).
	\end{solution}
	
	Let's prove the second part of proposition \eqref{prop:cpdiv} in section \eqref{sec:gcd-lcm}.
	
	\begin{problem}\label{prob:a|bc}
		Let $a,b,$ and $c$ be integers. If $a|bc$ and $(a,b)=1$, prove that $a|c$.
	\end{problem}
	
	\begin{solution}
		The problem is obvious for $c=0$. Assume that $c \neq 0$. Since $(a,b)=1$, there exist integers $x$ and $y$ such that $ax+by=1$. Multiply both sides of this equation by $c$ to obtain $acx+bcy=c$. Because $a$ divides both $acx$ and $bcy$, it must also divide their sum, which is equal to $c$.
		 
	\end{solution}
	
	\subsection{Modular Arithmetic Multiplicative Inverse}\label{sec:arithinverse}
	
	When speaking of real numbers, the multiplicative inverse of $x$ -- usually named reciprocal of $x$ -- is $\frac{1}{x}$. This is because $x \cdot \frac{1}{x} = 1$ for non-zero $x$.
	
	The definition of a multiplicative inverse in modular arithmetic must be more clear for you now.
	
	\begin{definition}
		Let $a$ be an integer and let $m$ be a positive integer. The \textit{modular multiplicative inverse} of $a$ modulo $m$ is an integer $x$ such that
		\begin{align*}
		ax \equiv 1 \pmod m.
		\end{align*}
		Once defined, $x$ may be denoted by $a^{-1}$ and simply called \textit{inverse of $a$}.
	\end{definition}
	
	\begin{note}
		Unlike real numbers which have a unique reciprocal, an integer $a$ has either no inverse, or infinitely many inverses modulo $m$.
	\end{note}
	
	\begin{example}
		An inverse for $3$ modulo $7$ is $5$:
			\begin{align*}
				3 \cdot 5 \equiv 1 \pmod 7.
			\end{align*}
		We can easily generate other inverses of $3$ modulo $7$ by adding various multiples of $7$ to $5$. Thus, the numbers in the set $\{\cdots, -2, 5, 12, 19, \cdots \}$ are all inverses of $3$ modulo $7$.
	\end{example}
	
	\begin{example}
		An inverse for $2^{16}+1$ modulo $2^{31}-1$ is $2^{16}-1$. In fact,
			\begin{align*}
				(2^{16} - 1)(2^{16} + 1) = 2^{32} -1 = 2(2^{31} - 1) + 1 \equiv 1 \pmod{2^{31} - 1}.
			\end{align*}
	\end{example}
	
	
	\begin{theorem} \label{thm:arithinverse} \slshape
		Let $a$ be an integer and let $m$ be a positive integer such that $a \perp m$. Then $a$ has an inverse modulo $m$. Also, every two inverses of $a$ are congruent modulo $m$.
	\end{theorem}
	
	\begin{proof}
		The proof is a straightforward result of corollary \eqref{cor:bezoutcoprime}. Since $a \perp m$, the equation $ax+my=1$ has solutions. Now take modulo $m$ from both sides to complete the proof of the first part. For the second part, assume that $x_1$ and $x_2$ are inverses of $a$ modulo $m$. Then,
		\begin{align*}
		ax_1 \equiv ax_2 \equiv 1 \pmod m \stackrel{(a,m)=1}{\implies} x_1 \equiv x_2 \pmod m,
		\end{align*}
		as desired.
	\end{proof}
	
The uniqueness of inverse of an integer $a$ modulo $m$ gives us the following corollary.

	\begin{corollary}
		For a positive integer $m$, let $\{a_{1}, a_{2}, \cdots, a_{\varphi(m)}\}$ be a reduced residue system modulo $m$. Then $\{a_{1}^{-1}, a_{2}^{-1}, \cdots, a_{\varphi(m)}^{-1}\}$ is also a reduced residue system modulo $m$.
	\end{corollary}
	
	
	\begin{problem}
		Find the unique odd integer $t$ such that $0<t<23$ and $t+2$ is the modular inverse of $t$ modulo $23$.
	\end{problem}
	
	\begin{solution}
		This means that $t(t+2)\equiv 1 \pmod{23}$. Add $1$ to both sides of this congruence relation to get $(t+1)^2 \equiv 2 \equiv 25\pmod{23}$. Therefore, $23|(t+1)^2-25$ or $23|(t-4)(t+6)$. By Euclid's lemma (Proposition \ref{prop:euclidslemma}), $23|t-4$ or $23|t+6$, which give $t=4$ and $t=17$ as solutions. Since we want $t$ to be odd, the answer is $t=17$.
	\end{solution}

We are going to prove a very simple fact which will be very useful later (for instance, in the next theorem or in the proof of Wolstenholme's theorem, where we re-state the same result as Lemma \ref{lem:wolstproof3}).

	\begin{proposition}\label{prop:inversepower}
		For a prime $p\geq 3$ and any positive integer $a$ coprime to $p$,
		\[ (a^{-1})^n \equiv (a^n)^{-1} \pmod p,\]
		for all positive integers $n$.
	\end{proposition}

	\begin{proof}
		Since $a$ is coprime to $p$, $a^{-1}$ exists. Therefore,
			\begin{align*}
				a \cdot a^{-1} \equiv 1 \pmod p &\implies a^n \cdot (a^{-1})^n \equiv 1 \pmod p\\
				&\implies (a^{-1})^n \equiv (a^n)^{-1} \pmod p,
			\end{align*}
		as desired.
	\end{proof}

	

	\begin{theorem}\label{thm:modgcd}
	Let $a,b$ be integers and $x,y,$ and $n$ be positive integers such that $(a,n)=(b,n)=1$,  $a^x\equiv b^x\pmod n$, and $a^y\equiv b^y\pmod n$. Then,
	\begin{align*}
		a^{(x,y)} & \equiv b^{(x,y)}\pmod n.
	\end{align*}
\end{theorem}

\begin{proof}
	By B\'{e}zout's identity, we know there are integers $u$ and $v$ so that $ux+vy=(x,y)$. Therefore,
	\begin{align}
		a^{(x,y)} &\equiv a^{ux+vy}\nonumber\\
		&\equiv \left(a^x\right)^u \cdot \left(a^y\right)^v \equiv \left(b^x\right)^u \cdot \left(b^y\right)^v\label{eq:modgcd}\\
		&\equiv b^{ux+vy} \equiv b^{(x,y)} \pmod n.\nonumber
	\end{align}
\end{proof}

\begin{remark}
		Thanks to Professor Greg Martin, we should point out a very important detail here. In the computations above, we used the fact that there exist integers $u$ and $v$ such that $ux+by = 1$. One must notice that these integers $u$ and $v$ need not be positive. In fact, if $x$ and $y$ are both positive, then $u$ and $v$ cannot be both positive (why?). But that doesn't make our calculations wrong, due to Proposition \ref{prop:inversepower}. If it's not clear to you yet, think of it in this way: suppose that, say, $u$ is negative. For instance, consider the example when $x=3$ and $y=15$. Then, since $3 \cdot (-4) + 15 \cdot 1 = (3,15)$, we have $u=-4$ and $v=1$ . Then, equation \eqref{eq:modgcd}, would look like $$(a^3)^{-4} \cdot (a^{15})^{1} \equiv (b^3)^{-4} \cdot (b^{15})^{1} \pmod n.$$
		This might not seem normal because we have a $-4$ in the exponents. So, using Proposition \ref{prop:inversepower}, we can write the above congruence equation as
		$$\left(\left(a^{-1}\right)^3\right)^{4} \cdot (a^{15})^{1} \equiv \left(\left(b^{-1}\right)^3\right)^{4} \cdot (b^{15})^{1} \pmod n.$$
		Notice that we need $(a,n)=1$ and $(b,n)=1$ to imply $a^{-1}$ and $b^{-1}$ exist modulo $n$.
\end{remark}
		
	\begin{problem}
		Prove that, $$\dfrac{(m,n)}{m}\binom{m}{n}$$ is an integer.
	\end{problem}
	Since this problem is juxtaposed with this section, it is obvious we are going to use this theorem. But in a real contest, that is not the case at all.
	\begin{solution}
		Since there are integers $x,y$ with $(m,n)=mx+ny$, it is easy to deduce that:
			\begin{align*}
			\dfrac{(m,n)}{m}\binom{m}{n} & = \dfrac{mx+ny}{m}\binom{m}{n}\\
			& = x\binom{m}{n}+\dfrac{ny}{m}\binom{m}{n}\\
			& = x\binom{m}{n}+\dfrac{ny}{m}\cdot\dfrac{m}{n}\binom{m-1}{n-1}\\
			& = x\binom{m}{n}+y\binom{m-1}{n-1}.
			\end{align*}
		Now we can say this is an integer. See how tactfully we tackled this problem.
	\end{solution}
	

\section{Chinese Remainder Theorem}	
	Chinese Remainder Theorem --usually called \textit{CRT}-- is a very old principle in mathematics. It was first introduced by a Chinese mathematician Sun Tzu almost $1700$ years ago. Consider the following example.
	
	\begin{problem}
		A positive integer $n$ leaves remainder $2$ when divided by $7$ but has a remainder $4$ when divided by $9$. Find the smallest value of $n$.
	\end{problem}
	You might have encountered similar problems when you were in $4$th or $5$th grade. May be even more basic ones. But the idea is essentially the same. If the problem was a bit different, like
	\begin{problem}
		A positive integer $n$ leaves remainder $2$ when divided by $7$ or $9$. Find the smallest value of $n$.
	\end{problem}
	Then it would be easier. Because then we would have that $n-2$ is 
	divisible by both $7$ and $9$. That means $n-2$ has to be divisible by their least common multiple, $63$. Obviously, the minimum such $n$ is $n=2$. Let's see what happens if we want $n>2$. Then all such positive integers would be $n=2+63k$. Now, as for this problem, we can't do this directly when the remainders are different. So we go back to the original problem and see how we can tackle the new one. Let's write them using congruence.
	\begin{align*}
	n & \equiv2\pmod{7},\\
	n & \equiv4\pmod{9}.
	\end{align*}
	In other words, using divisibility notation, $7|n-2$ and $9|n-4$. We can not do the same now. But \textit{if} these two remainders were same, we could do that. Probably we should focus on that. That is, we want it to be something like
		\begin{align*}
			n & \equiv a\pmod 7,\\
			n & \equiv a\pmod 9.
		\end{align*}
	The only thing we can do here is
		\begin{align*}
			n & \equiv2+7k\pmod7,\\
			n & \equiv4+9l\pmod9,
		\end{align*}
	for some \textit{suitable} integer $k$ and $l$. Our aim is to find their values. Since both $2+7k$ and $4+9l$ must be the same modulo $7$ and $9$, if we can find a way to keep $9$ in $2+7k$ and $7$ is $4+9l$, that could work! One way around it is to do the following:
		\begin{align*}
			n & \equiv2\cdot1+7\cdot4\pmod7,\\
			n & \equiv2\cdot9\cdot9^{-1}+7\cdot4\pmod{7}.
		\end{align*}
	Let's do the same for the other congruence.
		\begin{align*}
			n & \equiv4\cdot1+9\cdot2\pmod{9},\\
			n & \equiv4\cdot7\cdot7^{-1}+2\cdot9\pmod{9}.
		\end{align*}
	Now you should understand what we can do to make those two equal. In the first congruence, no matter what we multiply with $7\cdot4$, the remainder won't change modulo $7$. The same for $9$ in the second congruence. We will exploit this fact. We need to rearrange it just a bit more. But here is a warning. We wouldn't be able to do it if $7$ and $9$ were not coprime, since then they would not have any multiplicative inverse. We could do that trick writing $1=7\cdot7^{-1}$ only because $7^{-1}$ modulo $9$ exists. For simplicity, let's assume $7^{-1}\equiv u\pmod{9}$ and $9^{-1}\equiv v\pmod{7}$.
		\begin{align*}
			n & \equiv2\cdot9\cdot v+4\cdot7\cdot u\pmod{7}\\
			n & \equiv4\cdot7\cdot u+2\cdot9\cdot v\pmod{9}
		\end{align*}
	And now, we have what we want! We can say,
		\begin{align*}
			n & \equiv2\cdot9\cdot v+4\cdot7\cdot u\pmod{7\cdot9},
		\end{align*}
	since $7 \perp 9$. We have our solution! Think more on our approach and what led us to do this. Question is, is this $n$ the smallest solution? If we take $r$ with $0\leq r\leq mn$ so that
		\begin{align*}
			n & \equiv r\equiv18v+28u\pmod{63},
		\end{align*}
	where $u\equiv7^{-1}\equiv4\pmod9$ and $v\equiv9^{-1}\equiv4\pmod 7$. Therefore,
		\begin{align*}
			n = (18\cdot4+28\cdot4)\pmod{63}=184\pmod{63}=58.
		\end{align*}
	Since $58<63$, such a solution will be unique! Mathematically, we can write it this way. Let the inverse of $a$ modulo $n$ be $a^{-1}_n$.
		\begin{theorem}[Chinese Remainder Theorem for Two Integers]\slshape
			For two positive integers $a\bot b$,
				\begin{align*}
					x & \equiv m\pmod a,\\
					x & \equiv n\pmod b,
				\end{align*}
			has a solution 
				\begin{align*}
					x_0 \equiv (mbb^{-1}_a+naa^{-1}_b)\pmod{ab},
				\end{align*}
			and all the solutions are given by $x=x_0+abk$.
		\end{theorem}
	But this form is not that convenient. We will give it a better shape. Let $M=ab$, then $\frac{M}{a}\bot b$ and $\frac{M}{b}\bot b$. Rewrite the theorem in the following form.
		\begin{theorem}[Refined CRT]\slshape
			If $a_1\bot a_2$ and $M=a_1a_2$, then the congruences	
			\begin{align*}
			x & \equiv r_1\pmod{a_1},\\
			x & \equiv r_2\pmod{a_2},
			\end{align*}
			has the smallest solution
			\begin{align*}
			x_0 & \equiv 
			\left(r_1\left(\dfrac{M}{a_1}\right)\left(\dfrac{M}{a_1}\right)^{-1}_{a_2}+r_2\left(\dfrac{M}{a_2}\right)\left(\dfrac{M}{a_2}\right)^{-1}_{a_1}\right)\pmod{M}.
			\end{align*}
		\end{theorem}
	If we take $n$ relatively prime integers instead of two, the same process will work! So we can generalize this for $n$ variables.
		\begin{theorem}[CRT]\slshape
			For $n$ pairwise coprime integers $a_1,a_2,\cdots,a_n$ there exists a solution to the congruences
			\begin{eqnarray*}
			x & \equiv& r_1\pmod{a_1},\\
			x & \equiv& r_2\pmod{a_2},\\
			   &\vdots&\\
			x & \equiv& r_n\pmod{a_n}.
			\end{eqnarray*}
			If $M=a_1a_2\cdots a_n$ and $M_i=\dfrac{M}{a_i}$ and $M_ie_i\equiv1\pmod{a_i}$, then the smallest  modulo $M$ is given by
			\begin{align*}
			x_0  \equiv \left(r_1 M_1e_i+\cdots+r_n M_ne_n\right)\equiv \left(\sum_{i=1}^{n} r_i M_ie_i\right)\pmod M
			\end{align*}
		\end{theorem}
	
		\begin{proof}
			Note that, for a fixed $i$, $M_j$ is divisible by $a_i$ if $i\neq j$. Therefore,
			\begin{align*}
			x_0 = \sum_{i=1}^{n} r_i M_ie_i \equiv r_iM_ie_i\equiv r_i\pmod{a_i}
			\end{align*}
			So $x_0$ is a solution to those congruences. Since $M_i\bot a_i$, there is a multiplicative inverse of $M_i$ modulo $a_i$ due to B\'{e}zout's identity. We leave it to the reader to prove that if $x,y$ are two solutions, then $x\equiv y\pmod M$. That would prove its uniqueness modulo $M$.
		\end{proof}
	We want to mention a particular use of CRT. When you are facing some problems related to congruence equation, if you can not solve for some $n$, instead show a solution for $p_i^{e_i}$ where $n=p_1^{e_1}\cdots p_k^{e_k}$. Then you can say that such a solution modulo $n$ exists as well. In short, we could reduce the congruences to prime powers because $p_1,\cdots,p_k$ are pairwise coprime integers. By the way, we could generalize CRT the following way.
		\begin{theorem}[General CRT]\slshape
			For $n$ integers $a_1,\cdots,a_n$ the system of congruences
			\begin{eqnarray*}
				x & \equiv& r_1\pmod{a_1},\\
				x & \equiv& r_2\pmod{a_2},\\
				&\vdots&\\
				x & \equiv& r_n\pmod{a_n}.
			\end{eqnarray*}
			has a solution if and only if
			\begin{align*}
			r_i & \equiv r_j\pmod{(a_i,a_j)},
			\end{align*}
			for all $i$ and $j$. Any two solutions $x,y$ are congruent modulo the least common multiple of all $a_i$. That is, if $M=[a_1,\cdots,a_n]$ and $x,y$ are two solutions, then $x\equiv y\pmod M$.
		\end{theorem}
	
		\begin{problem}
			Prove that, for any $n$ there are $n$ consecutive integers such that all of them are composite.
		\end{problem}
		
		\begin{solution}
			We will use CRT here forcibly, even though it has a much easier solution. Consider the following congruences:
				\begin{eqnarray*}
					x & \equiv& -1\pmod{p_1p_2},\\
					x & \equiv& -2\pmod{p_3p_4},\\
					  & \vdots& \\
					x & \equiv& -n\pmod{p_{2n-1}p_{2n}}.
				\end{eqnarray*}
			Here, $p_1,\cdots,p_{2n}$ are distinct primes. Therefore, $M_1=p_1p_2,\cdots,M_n=p_{2n-1}p_{2n}$ are pair-wisely co-prime. So, by CRT, there is indeed such an $x$ which satisfies all of the congruences above. And our problem is solved. Notice that, $x+1$ is divisible by at least two primes $p_1,p_2$. Similarly, $x+i$ is divisible by $p_{2i-1}p_{2i}$.
		\end{solution}
	
		\begin{note}
			A common idea in such problems is to bring factorial into the play. Here, $(n+1)!+2,(n+1)!+3,\cdots,(n+1)!+(n+1)$ are such $n$ consecutive integers. But the motivation behind the solution above is that, we making use of the fact: primes are co-prime to each other. And to make an integer composite, we can just use two or more primes instead of one.
		\end{note}
		
		\begin{problem}
			Suppose that $ \{s_1,s_2\cdots , s_{\phi(m)}\} $ is a reduced residue set modulo $m$. Find all positive integers $a$ for which $ \{s_1+a,s_2+a\cdots , s_{\phi(m)}+a\} $ is also a reduced residue set modulo $m$.
		\end{problem}
		
		\begin{solution}
			We claim that the given set is a reduced residue system modulo $m$ if and only if $a$ is divisible by each prime factor of $m$. 
			
			First, suppose that $m$ has a factor $p$ and $a$ is not divisible by $p$. Let $m=p^{\alpha}n$ for some positive integer $n$ coprime to $p$. Since $n \bot p$, by CRT, there exists some integer $k$ such that
				\begin{eqnarray*}
				k &\equiv& -a \pmod p,\\
				k &\equiv& \phantom{-}1  \pmod n.
				\end{eqnarray*}
			Since $k \equiv -a \not \equiv 0 \pmod p$, we have $k \bot p$. Also, let $(k,n)=d$. Then $d\mid n\mid k-1$ and $d|k$, meaning $d|1$ and so $d=1$. It follows that $k \bot m$. So, $k \in \{s_1,s_2\cdots , s_{\varphi(m)}\} $. But $k+a$ is divisible by $p$, and therefore not coprime to $n$, forcing $ \{s_1+a,s_2+a\cdots , s_{\varphi(m)}+a\} $ not a reduced residue system.
			
			For the converse, suppose that $a$ is an integer which is divisible by all prime factors of $m$. Obviously, $s_1+a,s_2+a\cdots , s_{\varphi(m)}+a$ are all distinct modulo $m$. We just need to show that if $s$ is coprime to $m$, then so is $s+a$. For any prime $p$ which divides $m$, we have $s+a \equiv s \pmod p$ because as assumed, $a$ is divisible by $p$. Since $s$ is co-prime to $p$, so is $s+a$. Thus $s+a$ is co-prime to all prime factors of $m$, making it relatively prime to $m$ as well. 
		\end{solution}
		
		\begin{problem}[1997 Czech and Slovak Mathematical Olympiad]
			Show that there exists an increasing sequence $\{a_{n}\}_{n=1}^{\infty}$ of natural numbers such that for any $k \geq  0$, the sequence $\{k+a_{n}\}$ contains only finitely many primes.
		\end{problem}
	It is a standard example of CRT because it is not obvious how CRT comes into the play here.
		\begin{solution}
			Let $p_{k}$ be the $k$th prime number. Set ${a_1}= 2$. For	$n \geq  1$, let $a_{n+1}$ be the least integer greater than $a_{n}$ that is congruent to $-k$ modulo $p_{k+1}$ for all $k \leq  n$. Such an integer exists by the Chinese Remainder
			Theorem. Thus, for all $k \geq 0$, $k+a_{n}\equiv 0\pmod{p_{k+1}}$ for $n \geq  k + 1$. Then at most $k+1$ values in the sequence $\{k+a_{n}\}$ can be prime since the $i$th term onward for $i\geq k+2$, the values are nontrivial multiples of $p_{k+1}$ and must be composite. This completes the proof.
		\end{solution}
	
	\begin{note}
		We could deal with this using $a_n=(p_n-1)!$ as well, combining with Wilson's theorem. Because if $k>1$ then $p_n-1>k$ for sufficiently large $n$ so it will be composite from that $n$. Otherwise $(p-1)!+1$ is divisible by $p$, so it is composite as well.
	\end{note}
	
	
\end{document}

\section{Wilson's Theorem}
We have probably discussed that if $n>4$ is a composite integer, then $(n-1)!$ is divisible by $n$. What if $n$ is a prime? Take $n=3$, then $(n-1)!=2$, not divisible by $3$. Take $n=5$. $(n-1)!=24$, which is not divisible by $5$. Take $n=7$, then $(n-1)!=120$ which is not divisible by $7$. Take $n=11$, $(n-1)!=3\, 628\, 800$ and this is not divisible by $11$ either. Since they are not divisible by the primes (as expected), we should check for remainders. $2!$ leaves a remainder $2$ when divided by $3$. $4!$ leaves $4$ when divided by $5$, $6!$ leaves $6$ when divided by $7$. If you calculate further, you will see the pattern goes on. So it suggests us to conjecture that the remainder of $(p-1)!$ when divided by the prime $p$ is $p-1$. In fact, this is what we call \textit{Wilson's theorem}.
	\begin{theorem} \label{thm:selfinverse}
		Let $p$ be a prime number and $a$ is a positive integer. Show that if the inverse of $a$ modulo $p$ is equal to $a$, then $a \equiv 1$ or $p-1 \pmod p$.
	\end{theorem}

	\begin{proof}
		The proof for $p=2$ is obvious, so assume that $p>2$. The inverse of $a$ is itself, so $a^2 \equiv 1 \pmod p$. This means that $p\mid a^2-1$. So, $p$ divides $(a-1)(a+1)$. We know that $2\mid (a-1, a+1)$, so $p$ divides either $a+1$ or $a-1$, which results in $a \equiv 1$ or $p-1 \pmod p$.
	\end{proof}
Now, why are we concerned about such a situation at all? The inverse of $a$ being $a$ is not something of interest, at least not obviously. In fact, there is a reason behind it. And you should have thought of it before reaching this point. Notice that, in the congruence
	\begin{align*}
		(p-1)! & \equiv(p-1)\pmod p
	\end{align*}
$(p-1)$ is co-prime to $p$. So we can cancel it from both sides and get $(p-2)!\equiv1\pmod p$. Now, this is interesting and it asks us to reach $1$ from a product. Notice the following rearrangement for $p=11$.
	\begin{align*}
		9! & = 1\cdot2\cdots9\\
			&= \left(2\cdot6\right)\cdot\left(3\cdot4\right)\cdot\left(5\cdot9\right)\cdot\left(7\cdot8\right)\\
			& = 12\cdot12\cdot45\cdot56\\
			& \equiv1\cdot1\cdot1\cdot1\pmod{15}
	\end{align*}
Does it make sense now why the theorem above is necessary? If not, think a little bit more. Then proceed and you will realize we have actually found the crucial step to prove Wilson's theorem.

According to this theorem, the only two numbers in the set $\{1, 2, \ldots, p-1\}$ which have their inverse equal themselves are $1$ and $p-1$. This will help us to prove the Wilson's theorem.
	\begin{theorem}[Wilson's Theorem]
		The positive integer $p>1$ is a prime if and only if $(p-1)! \equiv -1 \pmod p$.
	\end{theorem}

	\begin{proof}
		We divide the proof of this theorem into two parts. First, we show that if $p$ is a prime, then $(p-1)! \equiv -1 \pmod p$.
		The result is obvious for $p=2$ and $3$. So we assume that $p \geq 5$. According to the \autoref{thm:selfinverse}, the only two numbers among  $\{1, 2, \ldots, p-1\}$ which have their inverse equal themselves are $1$ and $p-1$. Put them away and consider the set  $A=\{2, 3 \ldots, p-2\}$. There are $p-3$ elements in this set, and each of them has an inverse modulo $p$, as proved in \autoref{thm:arithinverse}. Furthermore, inverse of each number is congruent to $p$. This means that the inverses of elements of $A$ are distinct. So we can divide the elements of $A$ into $(p-3)/2$ inverse pairs. Thus
		\begin{align*}
		2 \cdot 3  \cdots  (p-2) \equiv 1 \pmod p
		\end{align*}
		Multiply $1$ and $p-1$ to both sides of the above equation and the proof is complete.

		Now, we should show that if $n$ is not a prime number, then $(n-1)! \not \equiv -1 \pmod n$.
		The theorem is obviously true for $n=3$ and $n=4$. So assume that $n \geq 5$ is a composite number. We can write $n=pq$ where $p$ and $q$ are integers greater than $1$. If $p \neq q$, then both $p$ and $q$ appear in $(n-1)!$, which means $(n-1)! \equiv 0 \pmod n$. In case $p=q$, we have $n=p^2$. Note that $n>2p$ and so both $p$ and $2p$ appear in $(n-1)!$, which again yields to $(n-1)! \equiv 0 \pmod n$.
	\end{proof}

	\begin{corollary}
		For a prime $p$, $(p-2)! \equiv 1 \pmod p$.
	\end{corollary}

	\begin{problem}
		What is the remainder of $24!$ when divided by $29$?
	\end{problem}

	\begin{solution}
		By Wilson's theorem, $28! \equiv -1 \pmod{29}$. Also,
			\begin{align*}
				-1 \equiv 28! &\equiv 24! \cdot 25 \cdot 26 \cdot 27 \cdot 28\\
	    &\equiv 24! \cdot (-4) \cdot (-3) \cdot (-2) \cdot (-1)\\
	    &\equiv 24! \cdot 24\\
	    &\equiv 24! \cdot (-5)\pmod{29}
			\end{align*}
		The last congruence equation can be written as $24! \cdot 5 \equiv 1 \pmod{29}$. In other words, $24!$ is the modular inverse of $5$ modulo $29$. The problem now reduces to finding the inverse of $5$ modulo $29$, which is $6$.

	\end{solution}

	\begin{problem}
		Let $n$ be a positive integer such that
			\begin{align*}
				1 + \dfrac{1}{2} + \dfrac{1}{3} + \dfrac{1}{4} +\ldots + \dfrac{1}{23} = \frac{n}{23!}
			\end{align*}
		Find the remainder of $n$ modulo $13$.
	\end{problem}

	\begin{solution}
		Multiply both sides by $23!$. The right side will be $n$ and the left side includes $23$ terms all of which are divisible by $13$ except $\frac{23!}{13}$. Thus, we must find the remainder of this term modulo $13$. Note that
			\begin{align*}
				\frac{23!}{13} &= 12! \cdot 14 \cdot 15 \cdots 23\\
				   &\equiv -1 \cdot 1 \cdot 2 \cdots 10\\
				   &\equiv -10! \pmod{13}
			\end{align*}
		This means that we only need to find the remainder of $-10!$ modulo $13$. We will use the same trick as in the previous problem:
			\begin{align*}
				-1 \equiv 12! &\equiv 10! \cdot 11 \cdot 12\\
					&\equiv 10! \cdot (-2) \cdot (-1)\\
					&\equiv 10! \cdot 2\pmod{13}
			\end{align*}
		Rewriting the last equation, we find that $2$ is the modular inverse of $-10!$ modulo $13$. Therefore, the answer is the modular inverse of $2$ mod $13$, which is $7$.
	\end{solution}
Try the next problem yourself!
	\begin{problem}
		Let $p$ be a prime such that $p \equiv 1 \pmod 4$. Prove that
			\begin{align*}
				\left(\left(\frac{p-1}{2}\right)!\right)^2 \equiv -1 \pmod p
			\end{align*}
	\end{problem}

\section{Euler and Fermat's Theorem}
Find the remainder of $2016^{2016}$ when divided by $2017$. You might think we are joking but we are not. Without explicitly calculating this \textit{BIG} integer, number theorists will tell you the remainder is $1$. They will even consider this trivial. This demonstrates another important aspect of numbers. If you know how numbers dance, you know numbers. Anyway, as you may have already guessed, there is a theorem for it. But we want to focus on the intuition part. What could lead you to find this remainder without actually calculating it? If you have been attentive so far, you should already understand that you should focus on finding a $1$ when you are multiplying (obviously, since we want to reduce the work we do!). We can try this in couple of ways, but let us start with the most obvious one. Even before that, have you noticed that $2017$ is a prime? This may not be of much importance right now, but keep going. Instead of such big numbers, try a smaller example first. Find $6^6$ modulo $7$
	\begin{align*}
		6^1 & \equiv -1\pmod7\\
		6^2 & \equiv \phantom{-}1\pmod7
	\end{align*}
We already have $1$. Is it clear to you that we will reach $1$ in $6^4$ and $6^6$ as well? If not, just calculate them by hand and see if this is true or false. We will come back to this topic later. But it seems we have the result $1$. Now, do this for $4^4\pmod5$ and $10^{10}\pmod{11}$. After you have done all the work, you should realize, like in Wilson's theorem, we are getting $1$ again. This should encourage you to experiment with some further values such as $2^4\pmod5$, $3^6\pmod7$ etc. Surprisingly, the result is always $1$ when the exponent is $1$. \textit{Pierre De Fermat} was the first one to observe and propose this.
	\begin{theorem}[Fermat's Little \footnote{Fermat proposed (but did not prove) another theorem in number theory which is much more difficult than this one. So they call this theorem the ``little" one. The other theorem is called {Fermat's Last Theorem}.} Theorem]
		If $p$ is a prime and $a$ is a positive integer such that $a \bot p$. Then
		\begin{align*}
			a^{p-1} \equiv 1 \pmod p
		\end{align*}
	\end{theorem}
% example with a=3, p=7. Then show the proof.
Again, let's see an example. Take $a=3$ and $p=7$. And consider the numbers $3\cdot1,3\cdot2,3\cdot3,3\cdot4,3\cdot5,3\cdot6$ modulo $7$. They are respectively $3,6,2,5,1,4$. Notice anything? It's just a rearrangement of $1,2,\ldots,6$. In fact we already proved it before! Now we will just multiply them all to get
	\begin{align*}
		3\cdot1\times3\cdot2\times3\cdot3\times3\cdot4\times3\cdot5\times3\cdot6  \equiv  1\cdot2\cdot3\cdot4\cdot5\cdot6 \pmod7
	\end{align*}
Collecting all $3$'s in the left-hand side, we will have
	\begin{align*}
		3^6\times1\cdot2\cdot3\cdot4\cdot5\cdot6  \equiv 1\cdot2\cdot3\cdot4\cdot5\cdot6 \pmod 7
	\end{align*}
Since $1,2,\ldots, 6$ are all co-prime to $7$, we can divide both sides of the above equation by $1\cdot 2 \cdots 6$ to obtain
	\begin{align*}
		3^6 \equiv 1 \pmod 7
	\end{align*}
It is now clear that the same argument works for the general case.
	\begin{proof}
		From the Definition \ref{def:completeresiduesystem}, it is clear that the set $A=\{0, 1, \ldots, p-1\}$ is a complete residue system modulo $p$. We know that $a \bot p$, so from Proposition \ref{prop:generalcompletesystem} the set $A'=\{0 \cdot a, 1 \cdot a, \ldots, (p-1) \cdot a\}$ is also a complete residue system modulo $p$. Putting aside the first element, $0$, it is clear that the product of the elements of $A$ and $A'$ are congruent modulo $p$:
			\begin{align*}
				1 \times a \cdot 2 \times a \cdots (p-1) \times a  \equiv 1 \cdot 2 \cdots (p-1) \pmod py
			\end{align*}
		So, we find that
			\begin{align}
				a^{p-1} \cdot (p-1)! \equiv (p-1)! \pmod p. \label{eq:factorialrelatively prime}
			\end{align}
		In congruence equation \eqref{eq:factorialrelatively prime}, we can use the fact that $(p, (p-1)!)=1$ to divide both sides by $(p-1)!$ and obtain $a^{p-1} \equiv 1 \pmod p$.
	\end{proof}

You can find some other proofs for Fermat's Little Theorem that use either number theoretic techniques or even combinatorial approaches. For a very strange yet interesting proof of this theorem, read the proof by counting necklaces in \textcite{engel_1998}.

 \begin{corollary}
	 	If $p$ is a prime and $a$ is an arbitrary positive integer (not necessarily co-prime with $p$), then
	 	\begin{align*}
	 	a^p \equiv a \pmod p
	 	\end{align*}
 \end{corollary}
Fermat's little theorem comes in handy in so many situations, but it only handles prime numbers. So we present the Euler's theorem which is a more general form of Fermat's little theorem. The proof is similar as well. Let's take the following example.

% take n=14 and all co-prime numbers less than 14
	\begin{problem}
		Show that $4^{20} + 6^{40} + 12^{60}$ is divisible by $13$.
	\end{problem}

	\begin{solution}
		Obviously, $12^{60}\equiv (-1)^{60} \equiv 1 \pmod {13}$. By Fermat's little theorem, $6^{12} \equiv 1 \pmod{13}$, hence
			\begin{align*}
				6^{40}
					& \equiv 6^{36} \cdot 6^4\\
					& \equiv \left(6^{12}\right)^3 \cdot 6^4\\
					& \equiv 6^4\\
					& \equiv 9 \pmod{13}
			\end{align*}
		We can apply the same method to find $4^{20} \equiv 3 \pmod{13}$. Finally,
			\begin{align*}
				4^{20} + 6^{40} + 12^{60}
					& \equiv 3+9+1\\
					& \equiv 0 \pmod{13}
			\end{align*}
	\end{solution}
Here \textcite[Page $29$, Example $1.29$]{andreescuandricafeng2007} is an easy consequence of Fermat's little theorem.
	\begin{problem}\label{e2}
		For integers $a,b$, prove that $a^pb-ab^p$ is divisible by $p$.
	\end{problem}

	\begin{solution}
		This problem is kind of a direct consequence of Fermat's little theorem. Write $a^pb-ab^p=ab(a^{p-1}-b^{p-1})$. If one of $a$ or $b$ is divisible by $p$, we are done. If neither of them is divisible by $p$, then $a^{p-1}\equiv1\equiv b^{p-1}\pmod p$. So, $p$ divides $a^{p-1}-b^{p-1}$.
	\end{solution}
The next problem is taken from \textcite[Problem $124$]{WaclawSierpinski1964}.
	\begin{problem}
		Prove that there exist infinitely many composite numbers of the form $(2^{2n}+1 )^2+4$, where $n$ is a positive integer.
	\end{problem}

	\begin{solution}
		A common approach for this kind of problems is to take different moduli (the first ones would be primes, obviously). Here, we will make use of modulo $29$. We will show that for any $n$ of the form $28k+1$ (for $k \geq 1$), the number $(2^{2n}+1 )^2+4$ will be divisible by $29$. By Fermat's theorem, $2^{2 \cdot 28k} \equiv 1 \pmod{29}$. Therefore, for $n=28k+1$,
			\begin{align*}
				(2^{2n}+1 )^2+4 &\equiv (2^{2\cdot (28k+1)}+1 )^2+4\\
								&\equiv (2^{2}+1 )^2+4\\
								&\equiv 0 \pmod{29}
			\end{align*}

	\end{solution}

	\begin{theorem}[Euler's \footnote{Sometimes called Fermat-Euler theorem or Euler's totient theorem, proposed by Euler in $1763$.} Theorem]
		If $a$ and $n$ are positive integers such that $a \bot n$. Then
		\begin{align*}
			a^{\varphi(n)} \equiv 1 \pmod n
		\end{align*}
		where $\varphi$ is the Euler's totient function.
	\end{theorem}

	\begin{proof}
		Before we start the proof, remember Definition \ref{def:totient} where we defined Euler's totient function. The proof is very similar to the proof of Fermat's theorem and you only need to apply Proposition \ref{prop:generalreducedsystem} once. Let $A=\{a_1, a_2, \cdots, a_{\varphi(m)}\}$ be a reduced residue set mod $m$. Then so is $B=\{aa_1, aa_2, \cdots, aa_{\varphi(m)}\}$. From the definition of reduced systems, any number which is relatively prime to $m$ is congruent to exactly one element of $A$ and exactly one element of $B$. Thus, the product of all elements of $A$ must be congruent to that of $B$, modulo $B$. Therefore
		\begin{align}
			(aa_1) (aa_2) \cdots (aa_{\varphi(m)})
				&\equiv a_1 a_2 \cdots a_{\varphi(m)}\pmod m\nonumber\\
		\implies a^{\varphi(m)} \cdot \left( a_1 a_2 \cdots a_{\varphi(m)}\right)
			& \equiv a_1 a_2 \cdots a_{\varphi(m)} \pmod m\nonumber\\\label{eq:productrelatively prime}
		\implies a^{\varphi(m)}
			&\equiv 1\pmod m
		\end{align}
		Note that in equation \eqref{eq:productrelatively prime} we have used the fact that $a_i \perp m$ for $1 \leq i \leq \varphi(m)$, which results in $a_1 a_2 \cdots a_{\varphi(m)} \perp m$.
	\end{proof}
We can easily conclude the Fermat's little theorem from Euler's theorem: for a prime $p$, we have $\varphi(p)=p-1$ and so $a^{\varphi(p)} \equiv a^{p-1} \equiv 1 \pmod p$ for any integer $a$ not divisible by $p$.
	\begin{problem}
		Let $a$ and $b$ be positive integers. Prove that in the arithmetic progression $ak+b$ (for $k \geq 0$ an integer), there exist infinitely many terms with the same prime divisors.
	\end{problem}

	\begin{solution}
		It's not obvious at all how we should approach this problem. First, let us discard the common factor between $a$ and $b$ so they do not have any common factor. Then we see that $ak+b=d(uk+v)$ for some relatively prime $u,v$. Since $d$ is a fixed positive integer, we now have to worry about $uk+v$ only. We want to show that there are many $k$ such that $uk+v$ has a fixed set of prime divisors. So, if we could show anyhow that $uk+v$ is the power of the same number for infinitely many $k$, that would give us a solution (note that the converse does not have to be true).

		Let $d=(a, a+b)$. There exist positive integers $a_1$ and $c$ such that
			\begin{align*}
				a
					& =da_1\\
				a+b
					& =dc
			\end{align*}
		Also, $(a_1,c)=1$ and $c >1$ (why?). Using Euler's theorem, one can write $c^{n\varphi(a_1)} \equiv 1 \pmod {a_1}$ for any positive integer $n$. This means that there exists a positive integer $t_n$ such that $c^{n\varphi(a_1)}-1 = t_na_1$. Now,
			\begin{align*}
				a(ct_n+1) + b &= da_1(ct_n+1) + (dc-da_1)\\
					  &= dc(t_na_1 + 1)\\
					  &= dc \left(c^{n\varphi(a_1)}\right)\\
					  &= dc^{n\varphi(a_1)+1}
			\end{align*}
		Therefore, the only prime divisors of the term $a(ct_n+1) + b$ in the progression are prime divisors of $dc$, which are fixed (because $d$ and $c$ depend only on $a$ and $b$, which are fixed). This means that there exist infinitely many terms in the sequence which have the same prime divisors and we are done.
	\end{solution}
Here is an exercise for you.
	\begin{problem}
		Find all primes $p$ such that $p^2$ divides $5^{p^2}+1$
	\end{problem}


\section{Quadratic Residues}\label{sec:qr}
	\documentclass[main.tex]{subfile}

\begin{document}
	Let $n$ be a fixed positive integer. There are many cases when we are interested in integers $a$ relatively prime to $n$ for which there exists another integer $x$ such that $a \equiv x^2 \pmod n$. As an example, assume that we want to solve the quadratic congruence relation
	\begin{align*}
		ax^2 + bx + c \equiv 0 \pmod n
	\end{align*}
	for $x$. Multiply both sides of the above relation by $4a$ to obtain
	\begin{align*}
		4a^2x^2+4abx+c \equiv 0 \pmod n
	\end{align*}
	Rewriting the left side of the last relation as $(2ax+b)^2 -b^2+c$, we have
	\begin{align*}
		(2ax+b)^2 \equiv b^2-c \pmod n
	\end{align*}
	which is of the form $y^2 \equiv z \pmod n$. Therefore, solving any quadratic congruence relation is equivalent to solving $x^2 \equiv a \pmod n$ for some $a$. We call such an $a$ a \textbf{quadratic residue} modulo $n$. Quadratic residues play an important role in cryptography. They are even used in acoustical engineering.

	In this section, we will discuss different aspects of quadratic residues in number theory.

	\begin{definition}[Quadratic Residue]
		Let $m>1$ be a positive integer and let $a$ be an integer such that $(a,n)=1$. Then $a$ is a \textit{quadratic residue} of $n$ if there exists an integer $x$ such that
		\[x^2\equiv a\pmod n\]
		If there is no such $x$, then $a$ is a \textit{quadratic non-residue} of $n$.
	\end{definition}

	\begin{example}
		$2$ is a quadratic residue modulo $7$ because $3^2 \equiv 2 \pmod 7$. However, $3$ is a non-residue modulo $7$. In fact, modulo $7$,
		\begin{align*}
			1^2 \equiv 1, \quad 2^2 \equiv 4\\
			3^2 \equiv 2, \quad 4^2 \equiv 2\\
			5^2 \equiv 4, \quad 6^2 \equiv 1
		\end{align*}
		This means that the only quadratic residues modulo $7$ are $1, 2$, and $4$.
	\end{example}

	\begin{corollary}\label{cor:qrequiv}
		If $a \equiv b \pmod n$, then $a$ is a quadratic residue (non-residue) modulo $n$, then so is $b$.
	\end{corollary}

	\begin{note}
		Whenever we say that an integer $a$ is a quadratic residue (non-residue) modulo $n$, it is clear that $a+kn$ is also a quadratic residue(non-residue) modulo $n$. Therefore, in order to find which numbers are quadratic residues modulo $n$, we only need to check the numbers $1, 2, \ldots, n-1$. Obviously, $a=0$ is a quadratic residue modulo any $n$, and we omit this case in our calculations.
	\end{note}

	\begin{theorem}\label{thm:primeresidue}
		Let $p$ be an odd prime number. There are exactly $\displaystyle \frac{p-1}{2}$ quadratic residues modulo $p$ (excluding zero). Furthermore, the residues come from the numbers $1^2, 2^2, \ldots, \displaystyle \left(\frac{p-1}{2}\right)^2$.
	\end{theorem}

	\begin{proof}
		Clearly, the quadratic residues modulo $p$ are
		\begin{align*}
			1^2, 2^2, \ldots, (p-1)^2 \pmod p
		\end{align*}
		Note that $x^2 \equiv (p-x)^2 \pmod p$ for $x=1,2,\ldots,p-1$. So we only need to go halfway, i.e., we should only consider the numbers
		\begin{align*}
			1^2, 2^2, \ldots, \displaystyle \left(\frac{p-1}{2}\right)^2 \pmod p
		\end{align*}
		These numbers are distinct modulo $p$, because otherwise if $x^2 \equiv y^2 \pmod p$ for some $x,y \in \{1,2,\ldots,\frac{p-1}{2}\}$, then
		\begin{align*}
			p\mid x^2 -y^2\\
			\implies p\mid (x-y)(x+y)
		\end{align*}
		But note that $x+y< \frac{p-1}{2}+\frac{p-1}{2}=p-1$, and so $p \nmid x+y$, which means $p\mid x-y$. Finally, since $x$ and $y$ are less than $p$, we should have $x=y$.
		So we have proved that there are exactly $(p-1)/2$ quadratic residues and they are
		\begin{align*}
			1^2, 2^2, \ldots, \displaystyle \left(\frac{p-1}{2}\right)^2
		\end{align*}
	\end{proof}

	\begin{theorem}\label{thm:qrnr}
		Let $p$ be an odd prime. Then,
		\begin{enumerate}[(i)]
			\item the product of two quadratic residues is also a quadratic residue,
			\item the product of two quadratic non-residues is also a quadratic residue, and
			\item the product of a quadratic residue and a quadratic non-residue is a quadratic non-residue.
		\end{enumerate}
	\end{theorem}

	\begin{proof}
		The first one is obvious. If $a \equiv x^2 \pmod p$ and $b\equiv y^2 \pmod p$, then $ab \equiv (xy)^2 \pmod p$.
		Let's prove $(iii)$ now. Assume that $a \equiv x^2$ is a residue and $b$ is a non-residue modulo $p$ and suppose to the contrary that $ab$ is a residue and $ab \equiv y^2 \pmod p$. Then
		\begin{align*}
			ab
				& \equiv x^2b\\
				& \equiv y^2 \pmod p
		\end{align*}
		Note that since $(x^2,p)=(a,p)=1$, the multiplicative inverse of $x^2$ exists. Therefore
		\begin{align*}
			b
				& \equiv (x^2)^{-1} \cdot y^2\\
				& \equiv (x^{-1})^2 \cdot y^2\\
				& \equiv (x^{-1} \cdot y)^2 \pmod p
		\end{align*}
		which contradicts the assumption that $b$ is a non-residue. So $ab$ is a non-residue.
		In order to prove $(ii)$, we use the fact that if $a$ is an integer relatively prime to $p$, then
		\begin{align*}
			\{a,2a,\ldots,(p-1)a\} = \{ 1,2,\ldots,p-1\}
		\end{align*}
		(The proof is easy, try it yourself). From \autoref{thm:primeresidue}, we see that there are exactly   $(p-1)/2$ quadratic residues among $\{a,2a,\ldots,(p-1)a\}$. Assume that $a$ is a fixed quadratic non-residue modulo $p$. From the proof of $(iii)$, we can say that whenever $a$ is multiplied by one of $\displaystyle \frac{p-1}{2}$ quadratic residues of the set $\{ 1,2,\ldots,p-1\}$, the result is a non-residue. Therefore, each non-residue element of $\{a,2a,\ldots,(p-1)a\}$ is multiplication of $a$ by a residue in the same set. This means that the multiplication of $a$ by any non-residue element of the set $\{a,2a,\ldots,(p-1)a\}$ is a residue, and we are done.
	\end{proof}

	\autoref{thm:qrnr} gives us a nice result. Quadratic residues and quadratic non-residues act just like $1$ and $-1$. How? Notice that $1 \times 1 =1, (-1) \times 1 =-1,$ and $(-1) \times (-1)=1$, and this guides us to a point that quadratic residues behave like $1$, and quadratic non-residues behave like $-1$. %Euler noticed this behavior and found a criterion to see whether a number is residue or non-residue. We will explain Euler's criterion
	We can represent this result using Legendre's notation.

	\begin{definition}[Legendre Symbol]
		We call $\left(\frac{a}{p}\right)$ the {\it Legendre symbol} for a prime $p$. It is defined by:
		\begin{align*}
			\left(\dfrac{a}{p}\right)
			& =
			\begin{cases}
				0 & \mbox{if }p\mid a\\
				1 &\mbox{if }a\mbox{ is a quadratic residue of }p\\
				-1 &\mbox{otherwise}
			\end{cases}
		\end{align*}

	\end{definition}

	Using this notation, \autoref{thm:qrnr} becomes
	\begin{theorem}\label{thm:qrproduct}
		Let $p$ be an odd prime and let $a,b$ be two integers. Then
		\begin{align*}
			\left(\dfrac{ab}{p}\right) = \left(\dfrac{a}{p}\right) \left(\dfrac{b}{p}\right)
		\end{align*}
	\end{theorem}

	\begin{remark}
		Clearly, the same relation holds for the product of any $n$ integers, that is,
		\begin{align*}
			\left(\dfrac{a_1a_2\cdots a_n}{p}\right) = \left(\dfrac{a_1}{p}\right) \left(\dfrac{a_2}{p}\right) \cdots \left(\dfrac{a_n}{p}\right)
		\end{align*}
	\end{remark}

	\begin{example}
		Theorem \eqref{thm:qrproduct} is helpful specially when dealing with big numbers. For instance, let's see if $18$ is a quadratic residue modulo $73$. According to the theorem,
		\begin{align*}
			\left(\dfrac{18}{73}\right) = \left(\dfrac{3}{73}\right) \left(\dfrac{3}{73}\right) \left(\dfrac{2}{73}\right)
		\end{align*}
		Note that we don't need to calculate $ \left(\frac{3}{73}\right)$ because whatever it is ($-1$ or $1$), it has appeared twice and $1^2=(-1)^2=1$. So
		\begin{align*}
			\left(\dfrac{18}{73}\right) = \left(\dfrac{2}{73}\right)
		\end{align*}
		Now, how can we find $ \left(\frac{2}{73}\right)$? It would be a pain to check all the values $1,2,\ldots, \frac{73-1}{2}$ to see if square of any of them is equal to $2$ modulo $73$. There are two ways to trick this. One is to use the general formula for $ \left(\frac{2}{p}\right)$ which will be discussed next. The second (and better) idea is to \textit{construct} the solution. Assume that you have a prime $p$ and an integer $a$ relatively prime to $p$. We know from Corollary \ref{cor:qrequiv} that if $a$ is a residue (non-residue), then $a+kp$ is also a residue (non-residue). So we will add multiples of $p$ to $a$ and check if we have reached a perfect square. If yes, then $a$ is a residue. Otherwise, we factorize the new number into perfect squares times some other number $b$. Continue this process until you reach either a non-residue $b$ (which means $a$ was a non-residue) or a perfect square factorization (which means $a$ was a residue). This whole process might seem a little confusing to you, but applying it to our case ($a=2, p=73$) will make it clear:
		\begin{align*}
			2 &\equiv 75 \equiv 148\equiv 2^2 \cdot 37\\
			& \equiv 2^2\cdot (37+73) \equiv 2^2 \cdot 110\\
			&\equiv 2^2\cdot (110+73) \equiv 2^2 \cdot 183\\
			&\equiv 2^2\cdot (183+73) \equiv 2^2 \cdot 256\\
			&\equiv 2^2 \cdot 16^2\\
			&\equiv 32^2
		\end{align*}
		So $2$ is a quadratic residue modulo $73$ and finally
		\begin{align*}
			\left(\dfrac{18}{73}\right) = \left(\dfrac{2}{73}\right)
				& =1
		\end{align*}

	\end{example}


	\subsection{Euler's Criterion}
	In the last example of previous section, we provided a method for computing $ \left(\frac{a}{p}\right)$. However, this method only works when $p$ and $a$ are small enough to make the calculations. Fortunately, Euler developed a criteria  to find out whether an integer is a quadratic residue or a quadratic non-residue. After we explain and prove Euler's criterion, we will explore some special cases, e.g., we will find the value of $ \left(\frac{2}{p}\right)$ and $ \left(\frac{-1}{p}\right)$ for all primes $p$.

	\begin{theorem}[Euler's Criterion]
		\label{thm:eulerscriterion}
		Let $p$ be an odd prime and let $a$ be an integer relatively prime to $p$. Then
		\begin{align*}
			a^{\frac{p-1}{2}} \equiv \left(\dfrac{a}{p}\right) \pmod p
		\end{align*}
	\end{theorem}

	\begin{proof}
		First notice that from Fermat's theorem, $a^{p-1} \equiv 1 \pmod p$. Since $p-1$ is even, we can write this as
		\begin{align*}
			a^{p-1} - 1
				& = \left(a^{\frac{p-1}{2}} - 1\right)\left(a^{\frac{p-1}{2}} + 1\right)\\
				& \equiv 0 \pmod p
		\end{align*}
		So either $a^{\frac{p-1}{2}} \equiv 1$ or $a^{\frac{p-1}{2}} \equiv -1 \pmod p$.
		Assume that $a$ is a quadratic residue modulo $p$. That is, $ \left(\frac{a}{p}\right)=1$. We should prove that $a^{\frac{p-1}{2}} \equiv 1$. Since $a$ is a residue, there exists some integer $x$ for which $a \equiv x^2 \pmod p$. So, from Fermat's theorem,
		\begin{align*}
			a^{\frac{p-1}{2}} \equiv x^{p-1} \equiv 1 \pmod p
		\end{align*}
		Now assume the case where $a$ is a quadratic non-residue modulo $p$. Then we should prove that $a^{\frac{p-1}{2}} \equiv -1$. We will use an interesting approach here. Let $b \in \{1,2,\ldots,p-1\}$. Since $(b,p)=1$, the congruence equation $bx \equiv a$ has a unique solution $x \equiv a \cdot b^{-1} \pmod p$. Also, $x \not \equiv b \pmod p$ because otherwise $b^2 \equiv a \pmod p$ which is in contradiction with $a$ being a non-residue. This means that the set $\{1,2,\ldots,p-1\}$ can be divided into $\displaystyle \frac{p-1}{2}$ pairs $(b,x)$ such that $bx \equiv a \pmod p$. So
		\begin{align*}
			(p-1)!
				& = 1 \times 2 \times \cdots \times (p-1)\\
				& \equiv \underbrace{a \times a \times \cdots \times a}_{\frac{p-1}{2}\text{ times}}\\
				& \equiv a^{\frac{p-1}{2}} \pmod p
		\end{align*}
		By Wilson's theorem, $(p-1)! \equiv -1 \pmod p$ and therefore $a^{\frac{p-1}{2}} \equiv -1 \pmod p$, as desired. The proof is complete.
	\end{proof}
Let's find $\left(\frac{a}{p}\right)$ for some small values of $a$.
	\begin{problem}
		What is $\left(\frac{-1}{p}\right)$ for a prime $p$?
	\end{problem}

	\begin{solution}
		This is a simple case. By Euler's criterion, we get
		\begin{align*}
			(-1)^{\frac{p-1}{2}} \equiv \left(\dfrac{-1}{p}\right) \pmod p
		\end{align*}
		If $p \equiv 1 \pmod 4$, that is, if $p=4k+1$, then $\frac{p-1}{2}$ is even and so $(-1)^{\frac{p-1}{2}} =1$. On the other side, we know that $\left(\frac{-1}{p}\right)$ is either $1$ or $-1$. So in this case it must equal one. This means that
		\begin{align*}
			p
				& \equiv 1 \pmod 4\\
			\iff \left(\dfrac{-1}{p}\right)
				& =1
		\end{align*}
		Note that the \textit{only if} part of the above statement is true because if $(-1)^{\frac{p-1}{2}}=1$, then $\frac{p-1}{2} = 2k$ for some integer $k$. So $p=4k+1$, or $p \equiv 1 \pmod 4$. You can easily check that the following statement is also true
		\begin{align*}
			p
				& \equiv 3 \pmod 4\\
			\iff \left(\dfrac{-1}{p}\right)
				& =-1
		\end{align*}
		Each prime has either the form $p \equiv 1 \pmod 4$ or $p \equiv 3 \pmod 4$, so these primes together make all primes. All in all,
	\end{solution}

	\begin{theorem}\label{thm:-1qr} For all primes $p$,
		\begin{align*}
		\left(\dfrac{-1}{p}\right)
		& =
		\begin{cases}
		1,&\mbox{ if } p \equiv 1 \pmod 4\text{ or }p=2\\
		-1, &\mbox{ if } p \equiv 3 \pmod 4
		\end{cases}
		\end{align*}
	\end{theorem}
This infers the following theorem and the next.
	\begin{theorem}
		$-1$ is a quadratic residue of a prime $p$ if and only if $p\equiv1\pmod4$.
	\end{theorem}

	\begin{theorem} \label{thm:a^2+b^2}
		Let $a$ and $b$ be relatively prime positive integers. Then every prime divisor of $a^2+b^2$ is either $2$ or of the form $4k+1$.
	\end{theorem}

	\begin{proof}
		Let $p$ be a prime divisor of $a^2+b^2$. If $a$ and $b$ both are odd then $p$ can be $2$. Now assume $p$ is larger than $2$. Then
			\begin{align*}
				a^2
					& \equiv-b^2 \pmod p\\
				\implies \left(a^2\right)^{\frac{p-1}{2}}
					& \equiv \left(-b^2\right)^{\frac{p-1}{2}}\pmod p
			\end{align*}
		If $p$ is of the form $4k+3$, then $\frac{p-1}{2}=2k+1$ is odd, hence,
			\begin{align*}
			\left(a^2\right)^{\frac{p-1}{2}}
				& \equiv \left(-b^2\right)^{\frac{p-1}{2}}\pmod p\\
			\implies a^{p-1}
				& \equiv -b^{p-1} \pmod p
			\end{align*}
		Clearly, since $p\mid a^2+b^2$, if $p$ divides one of $a$ or $b$, it should divide the other one. But this is impossible because $a \bot b$. So $p \bot a$ and $p \bot b$. By Fermat's little theorem, $a^{p-1} \equiv b^{p-1} \equiv 1 \pmod p$, which is in contradiction with the above equation since $p$ is odd.
		So, $p$ cannot be of the form $4k+3$ and therefore every odd prime divisor of $a^2+b^2$ is of the form $4k+1$.
	\end{proof}

	\begin{note}
	We could just say,
			\begin{align*}
				a^2
					& \equiv-b^2\pmod p\\
				\implies (ab^{-1})^2
					& \equiv-1\pmod p
			\end{align*}
		which means that $-1$ is a quadratic residue of $p$, and therefore $p\equiv1\pmod4$.
	\end{note}
We get the following corollary, which can directly solve an IMO problem.
	\begin{corollary}\label{cor:4n+1}
		Let $k$ be a positive integer. Every divisor of $4k^2+1$ is of the form $4n+1$ for some integer $n$.
	\end{corollary}

	\begin{proof}
		Thanks to the previous theorem, we know that all the prime divisors of $4k^2+1$ are of the form $4t+1$. Every divisor of $4k^2+1$ is a multiplication of its prime divisors. And if we multiply two numbers of the form $4t+1$, then the number is again of the same form (multiply $4x+1$ and $4y+1$ and see the result yourself). Hence, conclusion.
	\end{proof}

	\begin{problem}[Iran, Third Round Olympiad, 2007]
		Can $4xy-x-y$ be a square for integers $x$ and $y$?
	\end{problem}

	\begin{solution}
		Assume that $4xy-x-y=t^2$. We can rearrange it as $x(4y-1)  = t^2+y$, so
			\begin{align*}
				&4y-1  \mid t^2+y\\
				\implies &4y-1  \mid 4t^2+4y\\
				\implies &4y-1  \mid 4t^2+4y-(4y-1)\\
				\implies &4y-1  \mid 4t^2+1
			\end{align*}
		Since $4y-1$ is of the form $4k+3$, it must have at least one prime factor of the form $4j+3$. Then we have a prime factor of $4t^2+1$ which is of this form, a contradiction. Thus, it can't be a square.
	\end{solution}
Back to quadratic residues, the next step is to determine if $2$ is a quadratic residue modulo a prime. Unfortunately, we cannot apply Euler's criterion directly in this case because we do not know what $2^\frac{p-1}{2} \pmod p$ would be for different values of $p$. So our challenge is to find $2^\frac{p-1}{2}$ modulo $p$.

	\begin{problem}\label{pr:qr2p}
		What is $ \left(\frac{2}{p}\right)$ for a prime $p$?
	\end{problem}

	\begin{solution}
		As explained just above, we only need to find a way to calculate $2^\frac{p-1}{2} \pmod p$. The idea is similar to what we did in the proof of Fermat's little theorem. Remember that in Fermat's theorem, we needed to somehow construct $a^{p-1}$ and we did that by multiplying elements of the set $\{a,2a,\ldots,(p-1)a\}$. Now how can we construct $2^\frac{p-1}{2}$? The idea is to find a set with  $\frac{p-1}{2}$ elements such that the product of all elements has the factor $2^\frac{p-1}{2}$. One possibility is to consider the set $A=\{2, 4, \ldots, p-1\}$. Then the product of elements of $A$ is
		\begin{align}\label{eq:factorialqr1}
			2 \times 4 \times \cdots \times (p-1) &= 2^\frac{p-1}{2} \times 1 \times 2 \times \cdots \times \frac{p-1}{2} \nonumber\\
			& = 2^\frac{p-1}{2} \times \left(\frac{p-1}{2}\right)!
		\end{align}
		In order to get rid of the term $ \left(\frac{p-1}{2}\right)!$, we have to compute the product of elements of $A$ in some other way. Notice that we are looking for $2 \times 4 \times \cdots \times (p-1)$ and we want to make it as close as possible to $ \left(\frac{p-1}{2}\right)!$ so that we can cancel out this term and find the value of $2^\frac{p-1}{2} \pmod p$. To construct this factorial, we need all the numbers in the set
			\begin{align*}
				B=\left\{1, 2, \ldots, \frac{p-1}{2}\right\}
			\end{align*}
		However, we only have even integers in $A$. For example, take $p=11$. Then $A=\{2,4,6,8,10\}$ and $B=\{1,2,3,4,5\}$. Now, we want to construct $5!$ using the product of elements of $A$. Clearly, the elements $2$ and $4$ are directly chosen from $A$. Now it remains to somehow construct the product $1\times 3 \times 5$ with the elements $6,8,10$. The trick is pretty simple: just notice that $10 \equiv -1, 8 \equiv -3$, and $6 \equiv -5 \pmod{11}$. This means that $ 6 \times 8 \times 10 \equiv (-1)^3  \cdot 1 \times 3 \times 5$, and so
		\begin{align}\label{eq:factorialqr2}
		2 \times 4 \times 6 \times 8 \times 10 \equiv (-1)^3 \cdot 5!
		\end{align}
		Comparing equations \eqref{eq:factorialqr1} (for $p=11$) and \eqref{eq:factorialqr2}, we find that
		\begin{align*}
		2^5 \cdot 5!
			& \equiv (-1)^3 \cdot 5! \pmod{11}\\
		\implies 2^5
			& \equiv (-1)^3\\
			& \equiv -1 \pmod{11}
		\end{align*}
		Let's go back to the solution of the general problem. As in the example of $p=11$, we are searching for the power of $(-1)$ appeared in the congruence relation. In fact, this power equals the number of even elements bigger than $ \frac{p-1}{2}$ and less than or equal to $p-1$. Depending on the remainder of $p$ modulo $8$, this power of $(-1)$ can be even or odd. Consider the case when $p \equiv 1 \pmod 8$. Then $p-1 = 8k$ for some positive integer $k$ and so the even numbers less than or equal to $\frac{p-1}{2} = 4k$ in the set $A=\{2,4,\ldots,8k\}$ are $2,4,\ldots,4k$, which are $2k+1$ numbers. Therefore
			\begin{align*}
				2 \times 4 \times \cdots \times (p-1)
					&= \overbrace{\left(2 \times 4 \times \cdots \times 4k\right)}^{2k+1 \text{ items}} \cdot \overbrace{\left((4k+2) \times (4k+4) \times \cdots \times 8k\right)}^{2k \text{ items}}\\
					& \equiv \left(2 \times 4 \times \cdots \times 4k\right) \cdot \left((-(4k-1)) \times (-(4k-3)) \times \cdots \times (-1)\right) \\
					& \equiv (-1)^{2k} \cdot (4k)!\\
					& \equiv (-1)^{2k} \cdot \left(\frac{p-1}{2}\right)! \pmod p
			\end{align*}
		Compare this result with \eqref{eq:factorialqr1}, you see that
			\begin{align*}
				2^\frac{p-1}{2} \times \left(\frac{p-1}{2}\right)!
					& \equiv (-1)^{2k} \cdot \left(\frac{p-1}{2}\right)! \pmod p\\
				\implies 2^\frac{p-1}{2}
					& \equiv (-1)^{2k} \equiv 1 \pmod p
			\end{align*}
		So if $p \equiv 1 \pmod 8$, then $ \left(\frac{2}{p}\right)=1$.

		The process is similar for $p \equiv 3, 5, 7 \pmod 8$ and we put it as an exercise for the reader.

		After all this work, we are finally done computing $ \left(\frac{2}{p}\right)$. The final result is stated in the following theorem.

		\begin{theorem}\label{thm:2qr}
			\begin{align*}
			\left(\dfrac{2}{p}\right)
			& =
			\begin{cases}
			1,&\mbox{ if } p \equiv 1 \mbox{ or } 7\pmod 8\\
			-1,&\mbox{ if } p \equiv 3 \mbox{ or } 5\pmod 8
			\end{cases}
			\end{align*}
		\end{theorem}

	\end{solution}
We can generalize the method used in the solution of Problem \ref{pr:qr2p} to find $ \left(\frac{a}{p}\right)$ for all integers $a$ and primes $p$.
	\begin{theorem}[Gauss' Criterion]\label{thm:gausscriterion}
		Let $p$ be a prime number and let $a$ be an integer relatively prime to $p$. Let $\mu(a,p)$ denote the number of integers $x$ among
		\begin{align*}
		a, 2a, \ldots, \frac{p-1}{2}a
		\end{align*}
		such that $\displaystyle x > p/2 \pmod p$. Then
		\begin{align*}
		\left(\dfrac{a}{p}\right) = (-1)^{\mu(a,p)}
		\end{align*}
	\end{theorem}

The proof is left as an exercise for the reader.
		\begin{theorem}
			The smallest quadratic non-residue of a prime $p$ is a prime less than $\sqrt{p}+1$.
		\end{theorem}

		\begin{proof}
			Let $r$ be the smallest quadratic non-residue of prime $p$. Then, any $i<r$ is a quadratic residue of $p$. If $r=kl$ for $k,l>1$, using Legendre's symbol, we have
			\begin{align*}
				\left(\dfrac{r}{p}\right) & = \left(\dfrac{k}{p}\right)\cdot\left(\dfrac{l}{p}\right)
			\end{align*}
			which gives $-1=1\cdot 1$ (since $r$ is a quadratic non-residue modulo $p$) and we get a contradiction. So, $r$ can't be a prime. Let's move on to the next part of the theorem.

			We have that $r,\cdots,(r-1)r$ are quadratic non-residues mosulo $p$. If any of them is greater than $p$, say $ri$, we have $r(i-1)<p<ri$ for some $i$. But then,
			\begin{align*}
				ri & < p+r
			\end{align*}
			so that $ri=p+s$ with $s<r$. Thus,
			\begin{align*}
				ri & = p+s\\
				& \equiv s\pmod p
			\end{align*}
			Since $s<r$, $s$ is a quadratic residue of $p$, which in turn means $ri$ is a quadratic residue of $p$ as well. Another contradiction, so $r(r-1)<p$. If $r\geq\sqrt{p}+1$, we have $r(r-1)\geq\sqrt{p}(\sqrt{p}+1)=p+\sqrt{p}>p$, yet another contradiction. The claim is therefore true.
		\end{proof}

	\subsection{Quadratic Reciprocity}
	Assume we want to compute $ \left(\frac{a}{p}\right)$ for some integer $a$ and a prime $p$. The remark after \autoref{thm:qrproduct} says that it's enough to find
		\begin{align*}
			\left(\dfrac{a_1}{p}\right), \left(\dfrac{a_2}{p}\right), \ldots, \left(\dfrac{a_n}{p}\right)
		\end{align*}
	where $a_1,a_2,\ldots,a_n$ are divisors of $a$. This means that if we know the value of $\left(\frac{q}{p}\right)$ for a prime $q$, we can find the values of $\left(\frac{a}{p}\right)$ for any $a$.

	So let's discuss on the value of $\left(\frac{q}{p}\right)$. If $q$ is a big prime number, then by Corollary \ref{cor:qrequiv}, we can reduce $q$ modulo $p$ until we reach some $c<p$ and find $\left(\dfrac{q}{p}\right)$ by some method (Euler's or Gauss's criteria). So we can handle the case when $q$ is a big prime.

	Now, what about the case when $p$ is a big prime? In this case, finding $\left(\frac{q}{p}\right)$ would be very hard with theorems and methods stated by now. There is a very nice property of prime numbers which helps us to handle this case. This property is called the \textit{Law of Quadratic Reciprocity} which relates $\left(\frac{q}{p}\right)$ and $\left(\frac{p}{q}\right)$. So, in case $q$ is big, we can first calculate $\left(\frac{q}{p}\right)$ and then use this law to find $\left(\frac{p}{q}\right)$.

	\begin{theorem}[Law of Quadratic Reciprocity]\label{thm:lawofqr}
		Let $p$ and $q$ be different odd primes. Then
		\begin{align*}
			\left(\dfrac{p}{q}\right)\left(\dfrac{q}{p}\right)=(-1)^{\frac{p-1}{2}\cdot \frac{q-1}{2}}
		\end{align*}
	\end{theorem}

	\begin{example}
		Let's find $\left(\frac{11}{6661}\right)$. We have
		\begin{align}\label{eq:qrex}
			\left(\dfrac{11}{6661}\right) \left(\dfrac{6661}{11}\right)=(-1)^{5\cdot 3330}=-1
		\end{align}
		Now, since $6661 \equiv 6 \pmod{11}$, we obtain
		\begin{align*}
		\left(\dfrac{6661}{11}\right)
			& = \left(\dfrac{6}{11}\right)\\
			& = \left(\dfrac{3}{11}\right)\left(\dfrac{2}{11}\right)\\
			& =1 \times (-1)\\
			& = -1
		\end{align*}
		Replacing in equation \eqref{eq:qrex}, we finally find
		\begin{align*}
		\left(\dfrac{11}{6661}\right) = 1
		\end{align*}
	\end{example}

	The proof of this theorem is a bit complicated and it would make you lose the continuity of the context. For this reason, we will provide the proof in section \eqref{sec:qrlawproof}.

	\subsection{Jacobi Symbol}

	In previous sections, whenever we used the Legendre symbol $\left(\frac{p}{q}\right)$, we needed $p$ to be a prime number. We are now interested in cases where $p$ can be a composite number. In $1837$, Jacobi generalized the symbol used by Legendre in this way:

	\begin{definition}[Jacobi Symbol]\label{def:jacobi}
		Let $a$ be an integer and $n=p_1^{\alpha_1}p_2^{\alpha_2}\cdots p_k^{\alpha_k}$, where $p_i$ are odd primes and $\alpha_i$ are non-negative integers ($1 \leq i \leq k$). The \textit{Jacobi symbol} is defined as the product of the Legendre symbols corresponding to the prime factors of $n$:
		\begin{align*}
		\left(\frac{a}{n}\right) = \left(\frac{a}{p_1}\right)^{\alpha_1}\left(\frac{a}{p_2}\right)^{\alpha_2}\cdots \left(\frac{a}{p_k}\right)^{\alpha_k}
		\end{align*}
		Also, we define $\left(\frac{a}{1}\right)$ to be $1$.
	\end{definition}

	\begin{remark}
			The immediate result of the above definition is that if $\gcd(a,n) = 1$, then $\left(\frac{a}{n}\right)$ is either $+1$ or $-1$. Otherwise, it equals zero.
	\end{remark}

	\begin{example}
		\begin{align*}
		\left(\frac{14}{2535}\right) &= \left(\frac{14}{3}\right)\left(\frac{14}{5}\right) \left(\frac{14}{13}\right)^{2}\\
		&= \left(\frac{2}{3}\right)\left(\frac{4}{5}\right) \left(\frac{1}{13}\right)^{2}\\
		&= (-1) \cdot 1 \cdot 1\\
		&= -1
		\end{align*}
	\end{example}

	\begin{example}
		\begin{align*}
			\left(\frac{2}{15}\right)
				& =\left(\frac{2}{3}\right)\left(\frac{2}{5}\right)\\
				& = (-1)\cdot (-1)\\
				& =1
		\end{align*}

	\end{example}

	\begin{note}
		If $\left(\frac{a}{n}\right)=-1$ for some $a$ and $n$, then $a$ is a quadratic non-residue modulo $n$. However, the converse is not necessarily true. The above example shows you a simple case when $a$ is a quadratic non-residue modulo $n$ but $\left(\frac{a}{n}\right)=1$.
	\end{note}

	\begin{theorem}
		Let $a$ be an integer and $n=p_1^{\alpha_1}p_2^{\alpha_2}\cdots p_k^{\alpha_k}$, where $p_i$ are odd primes and $\alpha_i$ are non-negative integers ($1 \leq i \leq k$). Then $a$ is a quadratic residue modulo $n$ if and only if $a$ is a quadratic residue modulo every $p_i^{\alpha_i}$  ($1 \leq i \leq k$).
	\end{theorem}

	\begin{proof}
		The \textit{if} part is easy to prove. Assume that $a$ is a quadratic residue modulo $n$ and $a \equiv x^2 \pmod n$ for some integer $x$. Then $a \equiv x^2 \pmod{p_i^{\alpha_i}}$ since $p_i$s are relatively prime to each other. Now the \textit{only if} part: assume that $a \equiv x_i^2 \pmod{p_i^{\alpha_i}}$ for all $i$, where $x_i$ are integers. According to Chinese Remainder Theorem, since the numbers ${p_i^{\alpha_i}}$ are pairwise relatively prime, the system of congruence equations
		\begin{align*}
			x & \equiv x_1\pmod{p_1^{\alpha_1}}\\
			x & \equiv x_2\pmod{p_2^{\alpha_2}}\\
			& \vdots\\
			x & \equiv x_k\pmod{p_k^{\alpha_k}}
		\end{align*}
		has a solution for $x$. Now $x^2 \equiv x_i^2 \equiv a \pmod{p_i^{\alpha_i}}$, and therefore $x^2 \equiv a \pmod n$, which means that $a$ is a quadratic residue modulo $n$.
	\end{proof}

	The following theorem sums up almost everything explained in quadratic residues. The proofs are simple and straightforward, so we leave them as exercises for the reader.
		\begin{theorem}
			Let $a$ and $b$ be any two integers. Then for every two odd integers $m$ and $n$, we have
			\begin{enumerate}[i.]
				\item $\displaystyle \left(\frac{ab}{n}\right) = \left(\frac{a}{n}\right) \left(\frac{b}{n}\right)$
				\item $\displaystyle \left(\frac{a+bn}{n}\right) = \left(\frac{a}{n}\right)$
				\item $\displaystyle \left(\dfrac{m}{n}\right)\left(\dfrac{n}{m}\right)=(-1)^{\frac{m-1}{2}\cdot \frac{n-1}{2}}$
				\item $\displaystyle \left(\frac{a}{mn}\right) = \left(\frac{a}{n}\right) \left(\frac{a}{m}\right)$
				\item $\displaystyle \left(\frac{-1}{n}\right) = (-1)^{\frac{n-1}{2}}$
				\item $\displaystyle \left(\frac{2}{n}\right) = (-1)^{\frac{n^2-1}{8}}$
			\end{enumerate}
		\end{theorem}

		\begin{theorem}
			If a positive integer is a quadratic residue modulo every prime, then it is a perfect square.
		\end{theorem}
	This is a really cool theorem. However, proving this might be challenging!\footnote{A popular proof for this uses Dirichlet's theorem: For two co-prime positive integers $a$ and $b$, there are infinitely many primes in the sequence $\{an+b\}_{n\geq1}$. This theorem is very famous for being difficult to prove and it is well beyond our scope.} Give it a try.
		\begin{theorem}
			Let $b,n > 1$ be integers. Suppose that for each $k > 1$ there exists an integer $a_k$ such that $b - a^n_k$ is divisible by $k$. Prove that $b = A^n$ for some integer $A$.
		\end{theorem}

		\begin{proof}
			Assume that $ b$ has a prime factor, $ p$, so that $ p^{xn + r}\|b$ with $ 0 < r < n$. Then, we can let $ k = p^{xn + n}$. It follows that $ b\equiv a_k^n\bmod p^{xn + n}$. Since $ p^{xn + r}\|b$, we see that $ p^{xn + r}\|a_k^n$. Then, $ p^{x + \frac {r}{n}}\|a_k^n$, which is a contradiction since $ \frac {r}{n}$ is not an integer. Thus, we have a contradiction, so $ r = 0$, which means that only $ n$th powers of primes fully divide $ b$, so $ b$ is an $ n$th power.
		\end{proof}



\end{document}
\section{Wolstenholme's Theorem}
	\documentclass[12pt]{subfile}
\begin{document}
	The purpose of this section is to discuss the sum
		\begin{align*}
			\sum_{k=1}^{p-1} \frac{1}{k} = 1+\frac{1}{2}+ \frac{1}{3}+\cdots+ \frac{1}{p-1}
		\end{align*}

	Well, not exactly. We are more interested in this sum modulo $p$ where $p$ is a prime. But how do we calculate fractions modulo $p$? The answer should be obvious by now. $\frac{a}{b}\pmod p$ is actually $ab^{-1}\pmod p$ where $b^{-1}\pmod p$ is the inverse of $b$ modulo $p$. So if $be\equiv1\pmod p$, then
		\begin{align*}
			\frac{a}{b}\equiv ae\pmod p
		\end{align*}
	However, from modular cancellation property, we can take fractions modulo $p$ if $p\nmid b$. Moreover, we can introduce some sort of divisibility here.

	Let $ \frac{a}{b}$ be a fraction and let $n$ be an integer such that $(n,b)=1$. If $a$ is divisible by $n$, we say that $ \frac{a}{b}$ is divisible by $n$. Following this convention, the congruence $\frac{a}{b} \equiv 0 \pmod n$ makes sense.
		\begin{example}
			Since $25 \mid 100$ and $(3,25)=1$, we have $\frac{100}{3} \equiv 0 \pmod 3$. We can also calculate it this way:
				\begin{align*}
					\frac{100}{3}   &\equiv 100 \cdot (3)^{-1} \\
					&\equiv 100 \cdot 17\\
					&\equiv 0 \pmod{25}
				\end{align*}
			Now, let's compute a non-zero fraction modulo $7$:
				\begin{align*}
					\frac{840}{77} &= \frac{120}{11} \equiv 120 \cdot (11)^{-1}\\
					&\equiv 120 \cdot 2 \\
					& \equiv 2 \pmod{7}
				\end{align*}
		\end{example}


		\begin{theorem}[Wolstenholme's Theorem]\label{thm:wolst}
			Let $p>3$ be a prime. Then
			\begin{align*}
				S
					& = \sum_{k=1}^{p-1} \frac{1}{k}\\
					& = 1+\frac{1}{2}+ \frac{1}{3}+\cdots+ \frac{1}{p-1}\\
					& \equiv 0 \pmod{p^2}
			\end{align*}
		\end{theorem}

		\begin{note}
			According to our assumption, the sum $ \sum_{k=1}^{p-1} 1/k$ has been written in lowest terms, that is, as a fraction $a/b$ such that $(a,b)=1$.
		\end{note}

		\begin{remark}
			Theorem \eqref{thm:wolst} is not the original theorem stated by Wolstenholme. Actually the theorem was as stated below.
		\end{remark}

		\begin{theorem}\label{thm:origwolst}
			If $p$ is a prime bigger than $3$, then
			\begin{align*}
				\binom{2p}{p} & \equiv2\pmod{p^3}\\
				\binom{2p-1}{p-1} & \equiv 1 \pmod{p^3}
			\end{align*}
			This theorem is equivalent to \autoref{thm:wolst}.
		\end{theorem}
	This seems to be a very interesting theorem, however the proof is not straightforward. Let us tackle this theorem step by step (these steps are really intuitive and very useful in olympiad problems). But first we will show a weaker version of the \autoref{thm:origwolst}.
		\begin{theorem}
			For any prime $p$,
				\begin{align*}
					\binom{2p}{p} & \equiv2\pmod{p^2}
				\end{align*}
		\end{theorem}

		\begin{proof}
			We make use of an identity in \gls{binomialidentities} to write
				\begin{align*}
					\binom{2p}p & = \binom{p}0^2+\binom{p}{1}^2+\cdots+\binom{p}{p-1}^2+\binom{p}{p}^2\\
								& = 2+\binom{p}{1}^2+\cdots+\binom{p}{p-1}^2\\
								& \equiv2\pmod{p^2}
				\end{align*}
			The last line is true because from \autoref{thm:binpdiv}, for $0<i<p$, we have
				\begin{align*}
					\binom{p}{i}
						& \equiv 0\pmod p\\
					\binom{p}{i}^2
						& \equiv 0\pmod{p^2}
				\end{align*}
		\end{proof}

		\begin{lemma}\label{lem:wolstproof1}
			Let $p>3$ be a prime and $S$ be defined as in \autoref{thm:wolst}. Then,
				\begin{align*}
					S \equiv 0 \pmod p
				\end{align*}
		\end{lemma}

		\begin{proof}
			The proof is straightforward. There are $p-1$ terms in the sum and since $p>3$ is an odd prime, the number of terms is even. So we can write $S$ as sum of pairs of the form $ \frac{1}{k} +‌\frac{1}{p-k}$, for $k=1, 2, \ldots, \frac{p-1}{2}$. Thus
			\begin{align*}
				S
					&= 1+\frac{1}{2}+ \frac{1}{3}+\cdots+ \frac{1}{p-1}\\
					& = \left(1 + \frac{1}{p-1}\right) +‌\left(\frac{1}{2} + \frac{1}{p-2}\right) + \cdots + \left(\frac{1}{\frac{p-1}{2}} + \frac{1}{\frac{p-1}{2}+1}\right)\\
					& = \sum_{k=1}^{\frac{p-1}{2}} \left(\frac{1}{k}+\frac{1}{p-k} \right)\\
					& = \sum_{k=1}^{\frac{p-1}{2}} \frac{(k) + (p-k)}{k(p-k)}\\
					& = \sum_{k=1}^{\frac{p-1}{2}} \frac{p}{k(p-k)}\\
					& =p \cdot \sum_{k=1}^{\frac{p-1}{2}} \frac{1}{k(p-k)}\\
					& \equiv 0 \pmod p
			\end{align*}
			In the last line of above equations, the sum can be written as $\frac{a}{(p-1)!}$, where $a$ is some integer. Note that $(p, (p-1)!)=1$ and that's why we can conclude
				\begin{align*}
					p \cdot \sum_{k=1}^{\frac{p-1}{2}} \frac{1}{k(p-k)} \equiv 0 \pmod p
				\end{align*}
		\end{proof}

		\begin{lemma}\label{lem:wolstproof2} For a prime $p>3$,
			\[(1^{-1})^2+(2^{-1})^2+\cdots+((p-1)^{-1})^2 \equiv 0 \pmod p\] where $i^{-1}$ is the multiplicative inverse of $i$ modulo $p$ for $i=1,2,\ldots,p-1$.
		\end{lemma}

		\begin{proof}
			We recommend you re-read section \eqref{sec:arithinverse} if you have forgotten the definition of multiplicative inverse. We already know that
				\begin{align*}
					1^2+2^2+\cdots+(p-1)^2 = \frac{(p-1)(p)(2p-1)}{6}
				\end{align*}
			Clearly, the sum is an integer. Therefore $(p-1)(p)(2p-1)$ is divisible by $6$. Now since $p>3$, we have $(p,6)=1$ and thus $p$ divides $(p-1)(p)(2p-1)/6$. Therefore,
				\begin{align*}
					1^2+2^2+\cdots+(p-1)^2 \equiv 0 \pmod p
				\end{align*}
			In order to prove the lemma we should show that
				\begin{align*}
					(1^{-1})^2+(2^{-1})^2+\cdots+((p-1)^{-1})^2 \equiv 1^2+2^2+\cdots+(p-1)^2 \pmod p
				\end{align*}
			We shall show that the sets $A=\{1,2,\ldots,p-1\}$ and $B=\{1^{-1}, 2^{-1},\ldots,(p-1)^{-1}\}$ are equal. A proof is as follows: from \autoref{thm:arithinverse}, for any $x \in A$, there exists some $y \in B$ such that $xy \equiv 1 \pmod p$. This $y$ is unique, because if there exists some other $z \in B$ for which $xz \equiv 1 \pmod p$, then $xy \equiv xz \pmod p$, and since $(x,p)=1$, we have $y \equiv z \pmod p$ which means $y=z$ (why?). So there exists a unique $y\in B$ for each $x \in A$, and thus $A=B$ since $A$ and $B$ have equal number of elements. Finally,
				\begin{align*}
					(1^{-1})^2+(2^{-1})^2+\cdots+((p-1)^{-1})^2
						& \equiv 1^2+2^2+\cdots+(p-1)^2\\
						& \equiv 0 \pmod p
				\end{align*}
		\end{proof}
	We are going to re-state Proposition \ref{thm:modgcd} because, as we already mentioned, it is very useful:
		\begin{lemma}\label{lem:wolstproof3}
			For a prime $p\geq 3$ and any positive integer $a$ relatively prime to $p$,
			\[ (a^{-1})^n\equiv (a^n)^{-1} \pmod p\]
			for all positive integers $n$.
		\end{lemma}

		\begin{proof}
			\begin{align*}
				a
					& \cdot a^{-1} \equiv 1 \pmod p\\
				\implies a^n
					& \cdot (a^{-1})^n \equiv 1 \pmod p\\
				\implies(a^{-1})^n
					& \equiv (a^n)^{-1} \pmod p
			\end{align*}

		\end{proof}

		\begin{lemma}\label{lem:wolstproof4}
			For a prime $p>3$,
				\begin{align*}
					\sum_{i=1}^{\frac{p-1}{2}} \frac{1}{i(p-i)}
						& \equiv 0 \pmod p
				\end{align*}
		\end{lemma}

		\begin{proof}
			Let's write the sum as
				\begin{align*}
					\sum_{i=1}^{\frac{p-1}{2}} \frac{1}{i(p-i)}
						& = \sum_{i=1}^{\frac{p-1}{2}} \frac{\frac{(p-1)!}{i(p-i)}}{(p-1)!}\\
						& = \frac{1}{(p-1)!} \cdot \sum_{i=1}^{\frac{p-1}{2}} \frac{(p-1)!}{i(p-i)}
					%&= \frac{1}{1(p-1)}+\frac{1}{2(p-2)}+\cdots+\frac{1}{\frac{p-1}{2} \left(p - \frac{p-1}{2} \right)}\\
					%&=\frac{\frac{(p-1)!}{1(p-1)} +\frac{(p-1)!}{2(p-2)}+\cdots+\frac{(p-1)!}{\frac{p-1}{2} \left(p - \frac{p-1}{2} \right)}}{(p-1)!}.
				\end{align*}
			Since $(p-1)!$ is relatively prime to $p$, we only need to show that
				\begin{align*}
					\sum_{i=1}^{\frac{p-1}{2}} \frac{(p-1)!}{i(p-i)}
						& \equiv 0 \pmod p
				\end{align*}
			Define
				\begin{align*}
					a_i = \frac{(p-1)!}{i(p-i)}
				\end{align*}
			for $i=1,2,\ldots,\frac{p-1}{2}$. From Wilson's theorem, we know that $(p-1)! \equiv -1 \pmod p$. Observe that
				\begin{align*}
					i\cdot (p-i) \cdot a_i = (p-1)! \equiv -1 \pmod p
				\end{align*}
			Replacing $p-i \equiv -i \pmod p$ in the above equation, we have
				\begin{align}\label{eq:wolstproof1}
					-i^2
						& \cdot a_i = -1 \pmod p\\
					\implies i^2
						& \cdot a_i \equiv 1 \pmod p
				\end{align}
			Notice that the above equations are true for $i=1,2,\ldots,(p-1)/2$. Now, \eqref{eq:wolstproof1} means that $a_i$ is the multiplicative inverse of $i^2$ modulo $p$. So we have proved that
				\begin{align}\label{eq:wolstproof2}
					a_i
						& = \frac{(p-1)!}{i(p-i)}\\
						& \equiv (i^2)^{-1} \pmod p
				\end{align}
			for $i=1,2,\ldots,\frac{p-1}{2}$ where $(i^2)^{-1}$ means the multiplicative inverse of $i^2$ modulo $p$. We should now prove that the sum of all $a_i$s is divisible by $p$. Let $ a=\sum_{i=1}^{(p-1)/2} a_i$. According to \eqref{eq:wolstproof2},
				\begin{align*}
					a
						& = \sum_{i=1}^{\frac{p-1}{2}} a_i\\
						& = \sum_{i=1}^{\frac{p-1}{2}} (i^2)^{-1}
				\end{align*}
			From Lemma \ref{lem:wolstproof3},  $(i^2)^{-1} \equiv (i^{-1})^{2} \pmod p$, and so
				\begin{align}\label{eq:wolstproof3}
					a \equiv
						& \sum_{i=1}^{\frac{p-1}{2}} (i^{-1})^{2} \pmod p
				\end{align}
			We want to show that $a \equiv 0 \pmod p$. The trick is to convert \eqref{eq:wolstproof3} to what we proved in Lemma \ref{lem:wolstproof2}, using the fact that $-(a^{-1}) \equiv (-a)^{-1} \pmod p$:
				\begin{align*}
					2a
						&\equiv a+a \equiv \sum_{i=1}^{\frac{p-1}{2}} (i^{-1})^{2} + \sum_{i=1}^{\frac{p-1}{2}} (i^{-1})^{2}\\
						& \equiv \sum_{i=1}^{\frac{p-1}{2}} (i^{-1})^{2} + \sum_{i=1}^{\frac{p-1}{2}} (-(i)^{-1})^{2}  \\
						&\equiv \sum_{i=1}^{\frac{p-1}{2}} (i^{-1})^{2} + \sum_{i=1}^{\frac{p-1}{2}} ((-i)^{-1})^{2}\\
						& \equiv \sum_{i=1}^{\frac{p-1}{2}} (i^{-1})^{2} + \sum_{i=1}^{\frac{p-1}{2}} ((p-i)^{-1})^{2}\\
						& \equiv \sum_{i=1}^{\frac{p-1}{2}} (i^{-1})^{2} + \sum_{i=\frac{p+1}{2}}^{p-1} (i^{-1})^{2}\\
						& \equiv \sum_{i=1}^{p-1} (i^{-1})^{2}\\
						& \equiv 0 \pmod p
				\end{align*}
			Thus $a \equiv 0 \pmod p$ and we are done.
		\end{proof}
	We are ready to prove Wolstenholme's theorem now.
		\begin{proof}[Proof of Wolstenholme's Theorem]
			According to Lemma \ref{lem:wolstproof1}, we can write $S$ as
			\begin{align*}
				S
					& = p \cdot \sum_{i=1}^{\dfrac{p-1}{2}} \dfrac{1}{i(p-i)}
			\end{align*}
			From Lemma \ref{lem:wolstproof4}, we know that the above sum is divisible by $p$, so $S$ is divisible by $p^2$.
		\end{proof}

		\begin{problem}
			Let $p \geq 5$ be a prime number, and
				\begin{align*}
					1 + \dfrac{1}{2} + \ldots + \dfrac{1}{p}=\dfrac{a}{b}
				\end{align*}
			where $a$ and $b$ are two relatively prime integers. Show that $p^4\mid ap-b$.
		\end{problem}

		\begin{solution}
			From Wolstenholme's theorem, we have
				\begin{align*}
					1 + \dfrac{1}{2} + \cdots + \dfrac{1}{p-1}
						& = p^2 \cdot \dfrac{x}{y}
				\end{align*}
			for some integers $x$ and $y$ such that $y \bot p$. Replacing this in the given equation,
				\begin{align*}
					p^2 \cdot \dfrac{x}{y}+ \dfrac{1}{p}
						& = \frac{a}{b}\\
					ap-b
						& = p^3b \cdot \dfrac{x}{y}
				\end{align*}
			Since $b$ is divisible by $p$, we have $p^4\mid ap-b$.
		\end{solution}

		\begin{problem}
			Let $p \geq 5$ be a prime and
				\begin{align*}
					\dfrac{1}{p-1} + \dfrac{2}{p-2} + \ldots + \dfrac{p-1}{1}
						& =\dfrac{a}{b}
				\end{align*}
			where $a$ and $b$ are two relatively prime integers. Show that $p^3\mid a-b+bp$.
		\end{problem}

		\begin{solution}
			Note that
				\begin{align*}
					\dfrac{a}{b}
						& = \sum_{i=1}^{p-1} \frac{p-i}{i}\\
						& =\sum_{i=1}^{p-1} \left(\frac{p}{i} -1 \right)\\
						& =\sum_{i=1}^{p-1} \dfrac{p}{i} - (p-1)\\
						& = p\cdot\left(\sum_{i=1}^{p-1} \dfrac{1}{i}\right)-(p-1)\\
						& = p \cdot p^2 \dfrac{x}{y}-(p-1)
				\end{align*}
			where $x$ and $y$ are relatively prime integers with $y \bot p$ (we have used Wolstenholme's theorem in the last line). Now,
				\begin{align*}
					\dfrac a b+p-1
						& =p\cdot\dfrac{p^2x}y\\
					\implies (a-b+bp)y
						& =p^3xb
				\end{align*}
			and since $y\bot p$, we have $p^3\mid a-b+bp$.
		\end{solution}



		\begin{problem}\label{prob:binom(p-1)(k)}
			For any prime $p$ and a positive integer $k$ such that $1 \leq k \leq p-1$, prove that
				\begin{align*}
					\binom{p-1}{k} & \equiv(-1)^k\pmod p
				\end{align*}
		\end{problem}

		\begin{solution}
			We use the fact that $p-i\equiv-i\pmod p$ and that $(i,p)=1$ for $0<i<p$.
				\begin{align*}
					\binom{p-1}{k}
						& = \dfrac{(p-1)(p-2)\cdots(p-1-k+1)}{1\times 2 \times \cdots \times k}\\
						& \equiv \dfrac{(-1) \times (-2) \times \cdots \times (-k)}{1\times 2 \times \cdots \times k}\\
						& \equiv\dfrac{(-1)^k \cdot 1\times 2 \times \cdots \times k}{1\times 2 \times \cdots \times k}\\
						& \equiv(-1)^k\pmod p
				\end{align*}

		\end{solution}

		\begin{problem}
			For an odd prime $p$, show that
				\begin{align*}
					\dfrac{2^p-2}{p} & \equiv1-\dfrac{1}{2}+\cdots-\dfrac{1}{p-1}\pmod p
				\end{align*}
		\end{problem}

		\begin{solution}
			The approach is not obvious here unless one knows the above theorem. In problems like this, it is usually hard to pin down how to approach the problem. However, one should of course try to make use of the fact that
				\begin{align*}
					2^p
						& = (1+1)^p = 1+\binom{p}{1}+\cdots+\binom{p}{p-1}+1\\
						& = 1+\dfrac{p}{1}\binom{p-1}{0}+\cdots+\dfrac{p}{p-1}\binom{p-1}{p-2}+1
				\end{align*}
			So,
				\begin{align*}
					2^p-2
						& = p\left(\frac{1}{1}\binom{p-1}{0}+\dfrac{1}{2}\binom{p-1}{1}+\cdots+\dfrac{1}{p-1}\binom{p-1}{p-2}\right)
				\end{align*}
			Now, the problem is in a suitable shape and we can use the theorem above to write
				\begin{align*}
					\dfrac{2^p-2}{p}
						& = \dfrac{1}{1}\binom{p-1}{0}+\dfrac{1}{2}\binom{p-1}{1}+\cdots+\dfrac{1}{p-1}\binom{p-1}{p-2}\\
						& \equiv(-1)^0+\dfrac{1}{2}(-1)^1+\cdots+\dfrac{1}{p-1}(-1)^{p-3}\\
						& \equiv1-\dfrac{1}{2}+\cdots-\dfrac{1}{p-1}\pmod p
				\end{align*}

		\end{solution}

		\begin{corollary}
			For an odd prime $p$,
				\begin{align*}
					\dfrac{2^{p-1}-1}{p}
						& \equiv1-\dfrac{1}{2}+\cdots-\dfrac{1}{p-2}\pmod p
				\end{align*}
		\end{corollary}

		\begin{problem}
			Let $p \geq 5$ be a prime. Prove that
				\begin{align*}
					\binom{p^2}{p}
						& \equiv p \pmod{p^5}
				\end{align*}
		\end{problem}

		\begin{solution}
			Notice that
				\begin{align*}
					\binom{p^2}{p} - p
						& =\dfrac{p^2(p^2-1)(p^2-2)\cdots(p^2-(p-1))}{p!}-p\\
						& =\dfrac{p}{(p-1)!} \Big((p^2-1)(p^2-2)\cdots(p^2-(p-1)) - (p-1)!\Big)
				\end{align*}
			Since $(p, (p-1)!)=1$, it suffices to show that
				\begin{align*}
					(p^2-1)(p^2-2)\cdots(p^2-(p-1)) \equiv (p-1)! \pmod{p^4}
				\end{align*}
			Expand the left side to obtain
				\begin{align*}
					(p^2-1)(p^2-2)\cdots(p^2-(p-1))
						& = (p-1)! + p^2\Big(1+\frac{1}{2}+\cdots+\frac{1}{p-1}\Big)\left(p-1\right)!+p^4\cdot x
				\end{align*}
			where $x$ is some integer. By Wolstenholme's theorem, the second term in the above expansion is divisible by $p^4$ and we are done.
		\end{solution}

		\begin{corollary}
			Let $p \geq 5$ be a prime and $n\geq 1$ be an integer. Then
				\begin{align*}
					\binom{p^{n+1}}{p}
						& \equiv p^n \pmod{p^{2n+3}}
				\end{align*}
		\end{corollary}
	A result from \textcite{carlitz_1954}.
		\begin{problem}
			Let $k$ be a non-negative integer and $p\geq 5$ be a prime. Prove that
				\begin{align*}
					\dfrac{1}{kp+1}+ \dfrac{1}{kp+2}+ \ldots+\dfrac{1}{kp+(p-1)}\equiv 0\pmod{p^2}
				\end{align*}
		\end{problem}

		\begin{hint}
			Use the following:
				\begin{align*}
					\sum_{i=1}^{p-1}\dfrac{1}{kp+i}
						& = \dfrac{1}{2}\sum_{i=1}^{p-1}\Bigl(\dfrac{1}{kp+i}+\dfrac{1}{kp+p-i}\Bigr)
				\end{align*}
		\end{hint}

		\begin{problem}
			For a prime $p \geq 5$, show that
				\begin{align*}
					\binom{p^3}{p^2}
						& \equiv \binom{p^2}{p} \pmod{p^8}
				\end{align*}
		\end{problem}
	The following problem appears in \textcite[D23]{vandendriessche_lee_2007}.
		\begin{problem}
			Let $p$ be an odd prime of the form $p=4n+1$.
			\begin{itemize}
				\item Show that $n$ is a quadratic residue $\pmod{p}$.
				\item Calculate the value $n^{n}$  $\pmod{p}$.
			\end{itemize}
		\end{problem}

		\begin{problem} %[http://www.artofproblemsolving.com/community/c6h229692p1272166]
			Let $p \geq 7$ be a prime and let $s$ be a positive integer such that $p-1 \nmid s$. Prove that
			\begin{align*}
				1 + \dfrac {1}{2^s} + \dfrac {1}{3^s} + \ldots + \dfrac {1}{(p - 1)^s}
					& \equiv 0 \pmod p
			\end{align*}
		\end{problem}

		\begin{problem} %[Generalized Wolstenholme's Theorem]%[http://www.artofproblemsolving.com/community/q1h164729p916854]
			Let $n$ be a positive integer not divisible by $6$. Also, let $S$ be a reduced residue system modulo $n$ such that $1 \leq a <n$ for all $a \in S$. Prove that
				\begin{align*}
					\sum_{a \in S} \frac{1}{a}
						& \equiv 0 \pmod{n^2}
				\end{align*}
		\end{problem}



\end{document}
\section{Lucas' Theorem}
	\documentclass[12pt]{subfile}
\begin{document}
In $2010$, Masum encountered the following problem:
	\begin{problem}
		Find the number of odd binomial coefficients in the expansion of $(a+b)^{2010}$.
	\end{problem}
Try to solve it yourself. If you do not have any idea how to proceed, then here is a hint: you need to find the value of $\binom{2010}{i}\pmod2$ for $0\leq i\leq2010$. One idea for doing that is to count the exponent of $2$ in $\binom{2010}{i}$ using Legendre's theorem. If look for the condition when a coefficient can be odd.

Here, we will focus on a generalized version of such problems. In problems like this, it happens that we need to find the remainder of division of the binomial coefficient $\tbinom{m}{n}$ by a prime number $p$. \textit{Edouard Lucas} found patterns in Pascal triangle which resulted in the following theorem.
	\begin{theorem}[Lucas's Theorem]\slshape
		Let $p$ be a prime and let $m$ and $n$ be non-negative integers. Then
			\begin{align*}
				\binom{m}{n}\equiv\prod_{i=0}^k\binom{m_i}{n_i}\pmod p,
			\end{align*}
		where
			\begin{align*}
				m&=m_kp^k+m_{k-1}p^{k-1}+\cdots +m_1p+m_0, \text{ and}\\
				n&=n_kp^k+n_{k-1}p^{k-1}+\cdots +n_1p+n_0
			\end{align*}
		are the base $p$ expansions of $m$ and $n$ respectively. This uses the convention that $\binom{m}{n}=0$ if $m<n$.
	\end{theorem}
	
	\begin{example}
		For $p=7, m=67$, and $n=10$. Now
			\begin{align*}
				67 = 1 \cdot 7^2 + 2 \cdot 7 + 4, \quad 10 = 0 \cdot 7^2 + 1 \cdot 7 + 3,
			\end{align*}
		and therefore
			\begin{align*}
				\binom{67}{10}\equiv\binom{1}{0}\binom{2}{1}\binom{4}{3}\equiv 1 \cdot 2 \cdot 4 \equiv 1 \pmod 7.
			\end{align*}
		Note that $\binom{67}{10} = 247,994,680,648 \equiv 1 \pmod 7$, which is a huge number and it would be a tedious work to find the remainder modulo $7$ without Lucas's theorem.
	\end{example}
	
In order to prove Lucas's theorem, we need to state a lemma first.
	
	\begin{lemma}
		For a prime $p$, an integer $x$, and a positive integer $r$, we have
			\begin{align*}
				(1+x)^{p^r}\equiv 1+x^{p^r}\pmod{p}.
			\end{align*}
	\end{lemma}
	
	\begin{proof}
		We will use induction on $r$ to prove them lemma. The base case $r=1$ is easy: for any integer $k$ such that $1 \leq k \leq p-1$, we know that $\binom{p}{k} \equiv 0 \pmod p$. Now
			\begin{align*}
				(1+x)^p&\equiv 1+\binom{p}{1}x+\binom{p}{2}x^2+\cdots+\binom{p}{p-1}x^{p-1}+x^p\\ &\equiv 1+x^p\pmod p.
			\end{align*}
		Now suppose that $(1+x)^{p^r}\equiv 1+x^{p^r}\pmod{p}$ is true for some integer $r \geq 1$. Then
			\begin{align*}
				(1+x)^{p^{r+1}} &\equiv\left((1+x)^{p^r}\right)^p \equiv\left(1+x^{p^r}\right)^p\\ &\equiv\binom{p}{0}+\binom{p}{1}x^{p^r}+\binom{p}{2}x^{2p^r}+\cdots+\binom{p}{p-1}x^{(p-1)p^r}+\binom{p}{p}x^{p^{r+1}}\\ &\equiv 1+x^{p^{r+1}}\pmod{p}.
			\end{align*}
		So the congruence relation holds for all $r \geq 1$.
	\end{proof}
	
	\begin{proof}[Proof of Lucas's Theorem]
		The idea is to find the coefficient of $x^n$ in the expansion of $(1+x)^m$. We have
			\begin{align*}
				(1+x)^m&=(1+x)^{m_kp^k+m_{k-1}p^{k-1}+\cdots+m_1p+m_0}\\ &=[(1+x)^{p^k}]^{m_k}[(1+x)^{p^{k-1}}]^{m_{k-1}}\cdots[(1+x)^p]^{m_1}(1+x)^{m_0}\\ &\equiv(1+x^{p^k})^{m_k}(1+x^{p^{k-1}})^{m_{k-1}}\cdots(1+x^p)^{m_1}(1+x)^{m_0}\pmod{p}.
			\end{align*}
		We want the coefficient of $x^n$ in $(1+x)^m$. Since $n=n_kp^k+n_{k-1}p^{k-1}+\cdots +n_1p+n_0$, we want the coefficient of $(x^{p^{k}})^{n_{k}}(x^{p^{k-1}})^{n_{k-1}}\cdots (x^p)^{n_1}x^{n_0}$.
		The coefficient of each $(x^{p^{i}})^{n_{i}}$ comes from the binomial expansion of $(1+x^{p^i})^{m_i}$, which is $\binom{m_i}{n_i}$. Therefore we take the product of all such $\binom{m_i}{n_i}$, and thus we have
		\begin{equation*}
		\binom{m}{n}\equiv\prod_{i=0}^{k}\binom{m_i}{n_i}\pmod{p}. \qedhere
		\end{equation*}		
	\end{proof}
	
	\begin{corollary}\label{cor:lucas}
		Let $s,t,q,r$ be non-negative integers and $p$ $p$ be a prime such that $0 \leq q,r \leq p-1$. Then
		\begin{align*}
		\binom{sp+q}{tp+r} \equiv \binom{s}{t} \binom{q}{r} \pmod p.
		\end{align*}
	\end{corollary}
	
	\begin{problem}
		How many ordered triples $(a,b,c)$ of positive integers satisfy $a+b+c=94$ and $3$ does not divide
			\begin{align*}
				\frac{94!}{a!b!c!}?
			\end{align*}
	\end{problem}
	
	\begin{solution}
		Write $c=94-a-b$, and hence
			\begin{align*}
				\frac{94!}{a!b!(94 - a - b)!} = \binom{94}{a} \cdot \binom{94 - a}{b}.
			\end{align*}
		By Lucas' theorem, since $94=(10111)_3$, $3$ does not divide $\binom{94}{a}$ only when $a$ is an element of the set
			\begin{align*}
				 S= \{1, 3, 4, 9, 10, 12, 13, 81, 82, 84, 85, 90, 91, 93, 94\}.
			\end{align*}
		By symmetry, we only need to find $a,b,c$ which are elements of $S$. There exist six such triples $(a,b,c)$ which sum to $94$:
			\begin{align*}
				(1,3,90), (1, 9, 84), (1, 12, 81), (3, 9, 82), (3, 10, 81), (4,9,81).
			\end{align*}
	\end{solution}
	
	\begin{problem}
		Let $p$ be a prime. Prove that
			\begin{align*}
				\binom{p^n-1}{k}\equiv (-1)^{s_p(k)}\pmod p,
			\end{align*}
		where $s_p(k)$ is the sum of digits of $k$ when represented in base $p$.
	\end{problem}
	
	\begin{hint}
		Use Problem \ref{prob:binom(p-1)(k)} and apply Lucas' theorem.
	\end{hint}
	
	\begin{problem}[Taken from \cite{ch:congruence-cai}]
		Let $p$ and $q$ be distinct odd primes. Prove that
			\begin{align*}
				\binom{2pq-1}{pq-1}\equiv 1\pmod{pq}
			\end{align*}
		if and only if
			\begin{align*}
				\binom{2p-1}{p-1} &\equiv 1 \pmod q, \text{ and}\\
				\binom{2q-1}{q-1} &\equiv 1 \pmod p.
			\end{align*}
	\end{problem}
	
	\begin{solution}
		Since $2pq-1 = (2q-1)p+p-1$, the rightmost digit of $2pq-1$ when represented in base $p$ is $p-1$ and the other digits form $2q-1$. Analogously, the rightmost digit of $pq-1$ when represented in base $p$ is $p-1$ and the other digits form $q-1$. Applying corollary \eqref{cor:lucas}, we find
			\begin{align}
				\binom{2pq-1}{pq-1}\equiv \binom{2q-1}{q-1} \binom{p-1}{p-1}\equiv \binom{2q-1}{q-1} \pmod p.\label{eq:lucasproblem}
			\end{align}
		The if part is obvious since $p$ and $q$ are different primes. We will prove the only if part now. 
		
		Suppose  that $\binom{2pq-1}{pq-1}\equiv 1\pmod{pq}$. This means that $\binom{2pq-1}{pq-1}\equiv 1\pmod{p}$. By equation \eqref{eq:lucasproblem}, $\binom{2q-1}{q-1} \equiv 1 \pmod p$ as desired. Proving $\binom{2p-1}{p-1}\equiv 1 \pmod q$ is similar.
	\end{solution}
	
	\begin{problem}[Taken from \cite{ch:congruence-kokhas}]
		Let $n$ and $k$ be arbitrary positive integers and let $p$ be an odd prime $p$. Prove that 
			\begin{align*}
				p^2 \Big| \binom{pk}{pm} - \binom{k}{m}.
			\end{align*}
	\end{problem}
	
	\begin{hint}
		Induct on $n$ and equate the coefficients of $a^{pm}b^{p(n-m)}$ in both sides of
			\begin{align*}
				(a+b)^{pn}=(a+b)^{p(n-1)}(a+b)^{p}.
			\end{align*}
	\end{hint}
	
	
%	\begin{problem}
%		All the binomial coefficients $\binom{n}{k}$, where $0<k<n$, are divisible by $p$ if and only if $n$ is a power of $p$.
%		
%		All the binomial coefficients $\binom(n,k)$, where $0\leq k \leq n$, are not divisible by $p$ if and only if $n+1$ is divisible by $p^d$,
%		in other words, all the digits of $n$, except the leftmost, in base $p$ are equal to $p-1$.
%	\end{problem}
	
\end{document}
\section{Lagrange's Theorem}
	\documentclass[12pt]{subfile}

\begin{document}
	Lagrange's theorem in polynomial congruence is a really influential result in number theory. It has many implications and applications. And it is so important that we decided to keep it in this book even though we are not discussing polynomials or polynomial congruence in the current book. We just need the following simple definition.

		\begin{definition}
			A polynomial is an expression consisting of variables and coefficients which only employs the operations of addition, subtraction, multiplication, and non-negative integer exponents.
		\end{definition}

		\begin{example}
			An example of a polynomial of a single variable $x$ and with integer coefficients is $P(x)=x^4+3x^2 + x -8$. An example in three variables and rational coefficients is
				$$P(x,y,z)=2x^3 + \dfrac{4}{5}xy- 7xyz + 3zy^2 - 6$$
		\end{example}

		\begin{note}
			We only work with polynomials of a single variable and with integer coefficients in this book.
		\end{note}

		\begin{definition}
			Consider a polynomial $P(x)$ with integer coefficients. The \textit{degree} of $P(x)$ is the largest exponent of $x$ in $P(x)$. That is, if
				\begin{align*}
					P(x)=a_nx^n + a_{n-1}x^{n-1} + \cdots + a_1 x +a_0
				\end{align*}
			 where $a_i$, $0 \leq i \leq n$ are integers and $a_n \neq 0$, then the degree of $P(x)$ is $n$. We show this by $\deg P(x)=n$.
		\end{definition}

		\begin{definition}
			Let
				\begin{align*}
					P(x)=a_nx^n + a_{n-1}x^{n-1} + \cdots + a_1 x +a_0
				\end{align*}
			be a polynomial with integer coefficients. Assume that at least one coefficient of $P(x)$ is not divisible by $p$. For any prime $p$, the degree of $P(x)$ modulo $p$ is the largest integer $k$, $0 \leq k \leq n$,  for which $p \nmid a_k$. We denote this by $\deg_p P(x)=k$.
		\end{definition}

		\begin{example}
			The degree of $P(x)=7x^4 + 14x^3 - 5x^2 + 5x + 3$ is $4$. However, the degree of $P(x)$ is $2$ modulo $7$.
		\end{example}


		\begin{theorem}[Lagrange's Theorem]\label{thm:lagrange}
			Let $p$ be a prime and let $P(x)$ be a polynomial with integer coefficients not all divisible by $p$. Also, let $\deg_p P(x) = k$. The congruence equation $P(x)\equiv0\pmod p$ has at most $k$ incongruent solutions modulo $p$.
		\end{theorem}

		\begin{note}
			Assume that
				\begin{align*}
					P(x)
						& =a_nx^n + a_{n-1}x^{n-1} + \cdots + a_1 x +a_0
				\end{align*}
			If $P(x) \equiv 0 \pmod p$ for some $x$, then since
				\begin{align*}
					(x+kp)^i
						& \equiv x^i + (kp)^i\\
						& \equiv x^i \pmod p
				\end{align*}
			$\forall i,k \in \mathbb N$, we have $P(x+kp) \equiv 0 \pmod p$ as well. This means that we only need to search for solutions in the set $\{0, 1, \ldots, p-1\}$. The term \textit{incongruent solutions} in the above theorem is there just for the same reason.
		\end{note}


		\begin{example}
			Let $P(x)=10x^3+3x^2 + 12x+17$.  The degree of $P(x)$ modulo $5$ is $2$. According to Lagrange's theorem, the equation $P(x) \equiv 0 \pmod 5$ has at most $2$ solutions modulo $5$. To check this, note that
				\begin{align*}
					P(x)
						& = 10x^3+3x^2 + 12x+17\\
						& \equiv 3x^2 + 12x+12\\
						& \equiv 3(x+2)^2\\
						& \equiv 0 \pmod 5
				\end{align*}
			has only one solution $x \equiv -2$ modulo $5$.
		\end{example}

	We will prove Lagrange's theorem in the following.

		\begin{proof}
			We induct on $k$. Since $\deg_p P(x) = k$, we can write
				\begin{align*}
					P(x)
						& \equiv a_kx^k + a_{k-1}x^{k-1} + \cdots + a_1 x +a_0
				\end{align*}
			where $a_k, a_{k-1}, \ldots, a_0$ are coefficients of $P(x)$. It is clear that for $k=0$, the equation $f(x)=a_0$ has no solutions modulo $p$ because $p \nmid a_0$. Assume that the claim is true for all polynomials of degree up to $k-1$ modulo $p$. Assume that $P(x) \equiv 0 \pmod p$ has $d$ solutions. If $d < k$, we are done. Otherwise, if $d\geq k$, take $x_1, x_2, \ldots, x_k$ to be $k$ arbitrary incongruent solutions of $P(x) \equiv 0 \pmod p$. Define
				\begin{align*}
					Q(x)
						& = P(x) - a_k(x-x_1)(x-x_2)\cdots (x-x_k)
				\end{align*}
			Clearly, $\deg_p Q(x) < \deg_p P(x)=k$. However,
				\begin{align*}
					Q(x_1)
						& \equiv Q(x_2)\\
						& \equiv \cdots \\
						& \equiv Q(x_k)\\
						& \equiv 0 \pmod p
				\end{align*}
			which means $Q(x) \equiv 0 \pmod p$ has at least $k$ solutions. The induction hypothesis forces that $Q(x) \equiv 0 \pmod p$ for all $x$. It follows that
				\begin{align*}
					P(x)
						& \equiv a_k(x-x_1)(x-x_2)\cdots (x-x_k) \pmod p
				\end{align*}
			This means that $P(x) \equiv 0 \pmod p$ if and only if $x-x_i \equiv 0 \pmod p$ for some $i \in \{1,2,\ldots, k\}$. So, $x_1, x_2, \ldots, x_k$ are the only solutions to $P(x) \equiv 0 \pmod p$. The induction is complete.
		\end{proof}


	In this section, we will discuss only the following result and see how to apply it to prove some other theorems.
		\begin{theorem}[Lagrange]
			If $p$ is a prime and\label{thm:lag2}
				\begin{align}\label{eq:lagrangeproof0}
					(x+1)(x+2)\cdots(x+p-1) & = x^{p-1}+a_1x^{p-2}+\cdots+a_{p-2}x+(p-1)!
				\end{align}
			then the coefficients $a_1,a_2, \ldots,a_{p-2}$ are divisible by $p$ where $p$ is an odd prime.
		\end{theorem}

	The term $x^{p-1}$ is produced by multiplying all $x$ terms. Multiplying all the constant terms, we get $1\cdot2\cdots(p-1)=(p-1)!$, which explains the reasoning behind the terms on the right side of the equation. The proof is an intuitive one. Though there maybe other proofs, we prefer this one.

		\begin{proof}[Proof]
			Assume that $f(x)=(x+1)(x+2)\cdots(x+p-1)$. We start by noticing that $f(x+1)  = (x+2) (x+3) \cdots (x+p)$. We can write
				\begin{align*}
					(x+p)f(x) &=(x+1)f(x+1)
				\end{align*}
			or equivalently,
				\begin{align}\label{eq:lagrangeproof1}
					 pf(x) &=(x+1)f(x+1)-xf(x)
				\end{align}
			Substituting the expressions for $f(x)$ and $f(x+1)$ in equation \eqref{eq:lagrangeproof0}, we see that
				\begin{align*}
					pf(x) & = px^{p-1}+pa_1x^{p-2}+\cdots+pa_{p-2}x+p!\\
					(x+1)f(x+1) & = (x+1)^p+a_1(x+1)^{p-1}\\ & \quad +\cdots+a_{p-2}(x+1)^2+(x+1)(p-1)!\\
					xf(x) & = x^p+a_1x^{p-1}+\cdots+a_{p-2}x^2+x(p-1)!
				\end{align*}
			Replace these values into \eqref{eq:lagrangeproof1},
				\begin{align}\label{eq:lagrangeproof2}
					(x+1)f(x+1)-xf(x)
						& = (x+1)^p-x^p+a_1((x+1)^{p-1}-x^{p-1}) \nonumber\\
						& \quad +\cdots+a_{p-2}((x+1)^2-x^2)+(x+1-x)(p-1)!
				\end{align}
			We need to expand the terms $(x+1)^i - x^i$ (for $1 \leq i \leq p$) using binomial theorem so we can collect the terms with same degree (exponent).
				\begin{align}\label{eq:lagrangeproof3}
					(x+1)^i-x^i
						& = \left(x^i+\binom{i}{1}x^{i-1} + \binom{i}{2} x^{i-2}+\cdots+ \binom{i}{i-1} x + 1\right)-x^i \nonumber\\
						& = \binom{i}{1}x^{i-1} + \binom{i}{2} x^{i-2}+\cdots+ \binom{i}{i-1} x + 1
				\end{align}
			Since $pf(x) = (x+1)f(x+1)-xf(x)$, the coefficients of same exponents of $x$ should be the same for both sides. The coefficient of $x^{p-2}$ in $pf(x)$ is $pa_1$, while that of $(x+1)f(x+1)-xf(x)$ comes from the first two terms of \eqref{eq:lagrangeproof2} (that is, $(x+1)^p-x^p$ and $a_1((x+1)^{p-1}-x^{p-1})$). Using \eqref{eq:lagrangeproof3} to calculate these two terms, we get
				\begin{align*}
					pa_1 & = \binom{p}{2}+\binom{p-1}{1}a_1
				\end{align*}
			From \autoref{thm:binpdiv}, we know that $p$ divides $\binom{p}{k}$ for any $0<k<p$. So, $p$ divides $\binom{p}{2}$, therefore $p$ divides $a_1$.

			Equating coefficient of $x^{p-3}$, we find
				\begin{align*}
					pa_2
						& = \binom{p}{3}+\binom{p-1}{2}a_1+\binom{p-2}{1}a_2
				\end{align*}
			Here, $p$ divides $\binom{p}{3}$ and $a_1$, so $p$ divides $a_2$. Continuing this process in a similar way, we find that $a_1, a_2, \ldots, a_{p-2}$ are divisible by $p$. To check correctness of this, we can equate the coefficient of $x$ and find
				\begin{align*}
					pa_{p-2}
						& = \binom{p}{p-1}+\binom{p-1}{p-2}a_1+\cdots+\binom{2}{1}a_{p-2}
				\end{align*}
			This equation implies $p$ divides $a_{p-2}$, as claimed. The proof is complete.
		\end{proof}
	Before we describe some applications, let's try to understand the coefficients $a_1$, $a_2$, $\ldots,a_{p-2}$ in a better way. By investigating \eqref{eq:lagrangeproof0}, one can easily obtain
		\begin{align*}
			a_1 & = 1+2+\cdots+p-1\\
			a_2 & = 1\cdot2+\cdots+1\cdot(p-1)+2\cdot3+\cdots+2\cdot(p-1)+\cdots\\
				&  \vdots
		\end{align*}
	You should already guess what $a_1,\ldots,a_{p-2}$ are. $a_1$ is the sum of all $1,\cdots,p-1$. $a_2$ is the sum of products of two numbers from $1,\cdots,p-1$ (all possible $\binom{p-1}{2}$ combinations). Similarly, $a_{p-2}$ is the sum of products of $p-2$ numbers taken at a time. In general $a_i$ the sum of all possible products of $i$ numbers taken from $1,2,\cdots,p-1$. Therefore, we can state \autoref{thm:lag2} as
		\begin{theorem}
			If $p$ is an odd prime and $0<k<p-1$, then the sum of all possible products of $k$ numbers taken at a time from $1,2,\ldots,p-1$ is divisible by $p$.
		\end{theorem}
	Let's see just how powerful this theorem can be, if used properly. We can take advantage of the fact that the theorem is actually an identity, so we can choose $x$ freely as we wish.
		\begin{proof}[Proof of Wilson's Theorem]
			The theorem is true when $p=2$. Therefore, it is safe to assume that $p$ is odd. Put $x=1$ in \autoref{thm:lag2} to obtain
				\begin{align*}
					2\times 3 \times \cdots \times p & = 1+(a_1+\cdots+a_{p-2})+(p-1)!
				\end{align*}
			and so,
				\begin{align*}
					p! & = 1+a_1+\cdots+a_{p-2}+(p-1)!
				\end{align*}
			Clearly, $p!$ is divisible by $p$, and so are $a_1,\cdots,a_{p-2}$. Thus, $1+(p-1)!$ must be divisible by $p$ too, which is exactly what we want.
		\end{proof}
	We will use this as an intermediary to prove Fermat's theorem. We want to prove $x^{p-1}-1$ is divisible by $p$ when $x\bot p$.
		\begin{proof}[Proof of Fermat's Theorem]
			Since $x$ is co-prime to $p$, one of $x+1,\cdots,x+p-1$ is divisible by $p$ because they are $p-1$ consecutive integers. Therefore, their product is divisible by $p$ too. Thus,
				\begin{align*}
					(x+1)\cdots(x+p-1) & = x^{p-1}+a_1x^{p-2}+\cdots+a_{p-2}x+(p-1)!
				\end{align*}
			Here, left side is divisible by $p$ so must be right side. Again, since $a_1,\ldots,a_{p-2}$ are multiples of $p$, we have $x^{p-1}+(p-1)!$ is a multiple of $p$.
				\begin{align*}
					x^{p-1} & \equiv-(p-1)!\pmod p
				\end{align*}
			Hence, by Wilson's theorem, $x^{p-1} \equiv 1 \pmod p$, which finishes the proof.
		\end{proof}
	As for the last demonstration, we will use it to prove Wolstenholme's theorem, which we also proved before. The theorem requires us to show that for $p>3$ a prime, the numerator of $$1+\dfrac{1}{2}+\cdots+\dfrac{1}{p-1}$$ is divisible by $p^2$ in its reduced form.
		\begin{proof}[Proof of Wolstenholme's Theorem]
			The numerator is the sum of products of $p-2$ numbers taken from $1, 2, \ldots, p-1$. So, it is $a_{p-2}$. Since the denominator of the fraction is $(p-1)!$, which is not divisible by $p$, we only need to show that $p^2|a_{p-2}$.

			Set $x=-p$ in \autoref{thm:lag2} to obtain
				\begin{align*}
					(-p+1)\cdots(-p+p-1)
						& = p^{p-1}-a_1p^{p-2}+\cdots-a_{p-2}p+(p-1)!
				\end{align*}
			The left hand side of the above equation equals $(p-1)!$. So
				\begin{align*}
					p^{p-1}-a_1p^{p-2}+\cdots-a_{p-2}p
						& = 0
				\end{align*}
			which gives
				\begin{align*}
					a_{p-2}
						& = p^{p-2}+a_1p^{p-3}+\cdots+a_{p-3}p^2
				\end{align*}
			If $p>3$, then $p-2\geq 2$ and all the terms on the right side are divisible by $p^2$.
		\end{proof}

		\begin{note}
			You should try to guess what motivates us to set exactly those values of $x$ to get nice results.
		\end{note}
\end{document}
\section{Order, Primitive Roots} \label{sec:order}
	\documentclass{subfile}

\begin{document}
Recall the examples we took while discussing Fermat's little theorem. We were working with something like $2^6\pmod7$ or $6^6\pmod7$. While calculating, we found that $6^2\equiv1\pmod7$ or $2^3\equiv1\pmod7$ which eventually led to $6^6\equiv1\pmod7$ and $2^6\pmod7$. Along with the ideas we used there, did you conjecture anything else? We left a hint when we said that since $6^2\equiv1\pmod7$, $6^4\equiv1\pmod7$ and $6^6\equiv1\pmod7$ as well. We hope that this is sort of obvious by now. But it should also trigger you to think of something. If we can find the smallest exponent for which $2^x\equiv1\pmod7$, then we can say $2^y\equiv1\pmod7$ for all multiples of $x$ ($y$ here). We will shortly prove this formally. Moreover, it also encourages us to study these \textit{smallest }values for which we get $1$. The motivation is obvious. Whenever we get $1$, we get a cycle of remainders from which point, the remainders repeat. Just finish the examples above if you did not entirely understand what we meant. We call this smallest integer order. And it should be clear to you why the study of order is important.
	\begin{definition}[Order Modulo Integers]
		Let $a$ and $n$ be co-prime positive integers. If $x$ is the smallest positive integer such that  \[a^x\equiv1\pmod n,\] then $x$ is called the \textit{order} of $a$ modulo $n$. We denote this by $\ord_n(a)=x$.
	\end{definition}

	\begin{example}
		$\ord_8(3)=2$ i.e. $2$ is the smallest positive integer such that  $3^2\equiv1\pmod 8$.
	\end{example}


	\begin{theorem}\slshape\label{thm:ordDiv}
		Let $a$ and $n$ be positive integers. If $\ord_n(a)=d$ and $a^x\equiv1\pmod n$, then $d|x$.
	\end{theorem}

	\begin{proof}
		If $x<d$, it would contradict the fact that, $d$ is such smallest positive integer that $a^d\equiv1\pmod n$. We are left with the case $x>d$. Assume that $x=dq+r$ with $0\leq r<d$.
			\begin{align*}
				a^x
					& \equiv a^{dq}\cdot a^r\pmod{n}\\
					& \equiv (a^d)^q \cdot a^r\pmod{n}\\
					& \equiv 1 \cdot a^r\pmod{n}\\
					& \equiv a^r \pmod n
			\end{align*}
		So $a^r \equiv a^x \equiv 1 \pmod n$. Since $0\leq r<d$ and $d$ is the order of $a$, this is impossible unless $r=0$. Thus $x=dq$ and we are done.
	\end{proof}

	\begin{corollary}\label{cor:phiDiv}
		If $a\perp n$, then $\ord_n(a)|\varphi (n)$.
	\end{corollary}

	\begin{proof}
		If $d=\ord_n(a)$, then $a^d\equiv1\pmod n$. From Euler's theorem, $a^{\varphi (n)}\equiv1\pmod n$. Then using Theorem \ref{thm:ordDiv}, we can say that $d|\varphi (n)$.
	\end{proof}
We can use this result to find orders in practice. We only need to check for divisors of $\varphi(n)$ and find the smallest divisor for which the relation $a^d\equiv1\pmod n$ holds.
	\begin{corollary}
		$a^k\equiv a^l\pmod n$ if and only if $k\equiv l\pmod{\ord_n(a)}$.
	\end{corollary}

	\begin{proof}
		$a^k\equiv a^l\pmod n$ implies $a^{k-l} \equiv 1 \pmod n$. By Theorem \ref{thm:ordDiv}, we have $\ord_n(a) | k-l$. The reverse of this approach can be applied to prove the other part of the corollary.
	\end{proof}
One could ask if we know the order of $a$ modulo $n$, how do we find the order of other powers of $a$. Or, if we know order of $a$ modulo two positive integers $m$ and $n$, then what would be the order of $a$ modulo $mn$?
	\begin{theorem}\slshape
		If $m$ and $n$ are coprime positive integers such that $\ord_m(a)=d$ and $\ord_n(a)=e$, then $\ord_{mn}(a)=[d,e]$.\label{thm:ordL}
	\end{theorem}

	\begin{proof}
		Let $\ord_{mn}(a)=h$, so
			\begin{align*}
				a^h
					& \equiv1\pmod {mn},
			\end{align*}
		which gives $a^h\equiv1\pmod m$ and $a^h\equiv1\pmod n$ as well. By Theorem \ref{thm:ordDiv}, since $d$ and $e$ are order of $a$ modulo $m$ and $n$, respectively, we have $d|h$ and $e|h$. Therefore, for the minimum $h$, we must have $h=[d,e]$ to satisfy the conditions.
	\end{proof}

	\begin{theorem}\label{thm:prproduct}
		Let $a,b$, and $n$ be positive integers such that $\ord_n(a)=k$ and $\ord_n(b)=l$, where $k\bot l$. Then $\ord_n(ab)=kl$.
	\end{theorem}

	\begin{proof}
		Let $\ord_n(ab)=h$. First, note that
			\begin{align*}
				a^{lh}
					& \equiv a^{lh}\cdot b^{lh}\pmod{n}\\
					& \equiv (ab)^{lh}\pmod{n}\\
					& \equiv \left((ab)^h\right)^l\pmod{n}\\
					& \equiv 1 \pmod n
			\end{align*}
		So, by Theorem \ref{thm:ordDiv}, we have $k|lh$ and since $(k,l)=1$, it follows that $k|h$. We can similarly prove that $l|h$. So $kl|h$. On the other hand,
			\begin{align*}
				(ab)^{kl}
					& \equiv (a^k)^l \cdot (b^l)^k\pmod{n}\\
					& \equiv 1 \pmod n.
			\end{align*}
		Again, by Theorem \eqref{thm:ordDiv}, we have $h|kl$. This finishes the proof.
	\end{proof}

	\begin{theorem}\slshape
		If the order of $a$ modulo $n$ is $d$, then the order of $a^k$ modulo $n$ is $d/{(d,k)}$.
	\end{theorem}

	\begin{proof}
		Let the order of $a^k$ modulo $n$ be $h$. Then $\left(a^k\right)^h \equiv a^{kh} \equiv 1\pmod n$. Theorem \ref{thm:ordDiv} says that $d$ must divide $kh$. Assume that $(k,d)=g$, so there exist coprime positive integers $l$ and $e$ such that $k=gl$ and $d=ge$. Rewriting $d | kh$ implies
			\begin{align*}
				ge|glh \implies e|lh,
			\end{align*}
		and since $l\bot e$, $e$ must divide $h$. Since $dl=ke=gel$,
			\begin{align*}
				\left(a^k\right)^{e} & \equiv \left(a^d\right)^{l} \equiv 1\pmod n.
			\end{align*}
		This means that the order of $a^k$ modulo $n$ must divide $e$. So, $h$ divides $e$ as well. We get that $$h=e=\dfrac{d}{(d,k)}$$ must hold.
	\end{proof}
The previous theorem also implies the following one.
	\begin{theorem}\slshape
		The order of $a$ modulo $n$ is the same as the order of $a^k$ modulo $n$ if and only if $(k,n)=1$.
	\end{theorem}
Here is a very useful theorem, often used to solve Diophantine equations.
	\begin{theorem}\label{thm:cyclodiv}\slshape
		Let $q$ be a prime and $x$ be a positive integer. Every prime divisor of the number
			\begin{align*}
				1+x+\cdots+x^{q-1}
			\end{align*}
		is either $q$ or congruent to $1$ modulo $q$.
	\end{theorem}

	\begin{proof}
		The sum can be written as
			\begin{align*}
				S = 1+x+\cdots+x^{q-1} & = \dfrac{x^q-1}{x-1}.
			\end{align*}
		Let $p$ be any prime factor of $S$. Then
			\begin{align*}
				x^q & \equiv1\pmod p.
			\end{align*}
		If the order of $x$ modulo $p$ is $d$, we have $d|q$. Since $q$ is a prime, either $d=1$ or $d=q$. If $d=1$, then $x\equiv1\pmod p$. In that case,
			\begin{align*}
				S & \equiv 1+1+\cdots+1\pmod p,
			\end{align*}
		which gives $0  \equiv q\pmod p$. So, $p=q$. Now assume the case that $d=q$. Because of Fermat's little theorem,
			\begin{align*}
				x^{p-1} & \equiv1\pmod p.
			\end{align*}
		This implies that $d=q$ divides $p-1$. So $p\equiv1\pmod q$, as claimed.
	\end{proof}

	\begin{definition}[Primitive Root]
		Let $n$ be a given positive integer. An integer $g$ which is relatively prime to $n$  is called a \textit{primitive root} modulo $n$ if $\ord_n(g)=\varphi(n)$. That is, if $g^x\not\equiv1\pmod n$ for any positive integer $x<\varphi(n)$.
	\end{definition}

	\begin{note}
		Using this definition, we can say that:
			\begin{enumerate}
				\item Let $g$ be a positive integer coprime to $n$. It is clear that $g^m$ is also co-prime to $n$ for any $m \in \mathbb N$.
				\item If $g$ is a primitive root modulo $n$ and if $g^a \equiv g^b \pmod n$ for some positive integers $a$ and $b$ less than $\varphi(n)$, then $a=b$. The reason is simple: if $a \neq b$, then $g^{a-b} \equiv 1 \pmod n$, which is absurd since $a-b \leq \varphi(n)$ and $g$ is a primitive root.
			\end{enumerate}
	\end{note}

These two notes tell us that if $g$ is a primitive root of $n$, then the set $\{g,g^2,\cdots,g^{\varphi(n)}\}$ is equal to $\mathbb U_n$, where $\mathbb U_n$ is the set of units modulo $n$ (as defined in Definition \ref{def:setofunits}). Notice that equality of these two sets is considered modulo $n$. Actually, the set $\{g,g^2,\cdots,g^{\varphi(n)}\}$ may contain some elements larger than $n$. We reduce those elements modulo $n$ so that we have all elements less than $n$. This new set is now equal to $\mathbb U_n$. We may denote this by the notation $\{g,g^2,\cdots,g^{\varphi(n)}\} \equiv \mathbb U_n \pmod n$.

In algebraic words, $g$ is a \textit{generator} of $\mathbb U_n$. Moreover, the generators of $\mathbb U_n$ are exactly the primitive roots of $n$ (if there is any). We will summarize this result in the following theorem.

	\begin{theorem}\slshape\label{thm:prequalsU}
		A primitive root $g$ modulo $n$ (if existing) is a generator of $\mathbb U_n$. That is, for any $a\in\mathbb{U}$, there is a unique $k$ with $0< k\leq\varphi(n)$ such that $g^k \equiv a \pmod n$.
	\end{theorem}

	\begin{proof}
		Consider the powers $g,g^2,\ldots,g^{\varphi (n)}$ modulo $n$. Now assume  $g^u  \equiv g^v \pmod n$ for some $u, v \in \{1,2,\ldots, \varphi(n)\}$, then since $(g,n)=1$, we obtain
			\begin{align*}
				g^{u-v}&\equiv1\pmod n.
			\end{align*}
		This is not possible unless $u=v$. The reason is simple: if $u \neq v$, then we have found some $x=u-v$ such that $0 < x \leq \varphi(n)$ and $g^x \equiv 1 \pmod n$, which is in contradiction with $g$ being a primitive root modulo $n$.
	\end{proof}

		\begin{example}
		$ $
		\begin{enumerate}
			\item $3$ is a primitive root modulo $7$ since $\varphi(7)=6$ and $3^i\not\equiv1\pmod 7$ for $i\in\{1,2,3,4,5\}$. Notice that powers of $3$ create the whole set $\mathbb U_7$:
			\begin{align*}
				3^1 \equiv 3, \quad 3^2 \equiv 2, \quad 3^3 \equiv 6, 3^4 \equiv 4, \quad 3^5 \equiv 5, \quad 3^6 \equiv 1,
			\end{align*}
			where all the congruences are taken modulo $7$.
			\item Let's see if there exists a primitive root modulo $15$. To show this, a possible way is to start from $a=2$ and compute all the powers $a^i$ for $i=2,\ldots,\varphi(15)-1=7$ modulo $15$ one by one:
			$$ 2^2 \equiv 4, \quad 2^3 \equiv 8, \quad 2^4 \equiv 1.$$
			We stop at $2^4$ because we got $1$ mod $15$, and this shows that $2$ is not a primitive root modulo $15$. Then, we should do the same process, but this time for $a=4$ (we don't check $3$ because it's not coprime to $15$). Now you should be able to do the math much faster, and come up with $4^2 \equiv 1 \pmod{15}$, which shows $a=4$ is not a primitivie root modulo $15$. Fortunately, we don't need to check $a=5$ and $a=6$. For $a=7$, the computations are not as easy as $a=2$ and $a=4$, but still not hard
			$$7^2 \equiv 4, \quad 7^3 \equiv 13, \quad 7^4 \equiv 1.$$
			So, $7$ is not a primitive root mod $15$ either. Now, we don't need to do the computations for $a=8$ because in this case, $a^{-1}$ is $2$ and we showed that $2$ is not a primitive root (why is that enough?). The next values for $a$ to check are $11, 13$, and $14$. Since $13=7^{-1}$, we don't need to worry about $13$. Check $11$ and $14$ for yourself and verify that neither of them are primitive roots mod $15$ (we can't do the modular arithmetic invers trick here because $11$ and $14$ are the inverse of themselves modulo $15$). This shows that there is no primitive root mod $15$.

			\item Let's assume that elements $g_i$ of $\mathbb U_n$ are sorted in ascending order. That is, $g_1<\cdots<g_{\varphi(n)}$. Then we  have $g_1=1$ and $g_{\varphi(n)}=n-1$.
		\end{enumerate}
	\end{example}


	\begin{corollary}\label{cor:prres}
		If $g$ is a primitive root of $p$ then
		\[\mathbb G=\{g^1,g^2,\ldots,g^{p-1}\}\]
		forms a complete residue system modulo $p$.
	\end{corollary}

	\begin{theorem}\slshape\label{prd}
		Let $n$ be a positive integer and let $a$ be a quadratic non-residue modulo $n$ such that $a\bot n$. Assume that $\mathbb U_n = \{g_1, g_2, \ldots, g_{\varphi(n)}\}$. Then
		\[g_1g_2\cdots g_{\varphi(n)}\equiv a^{\frac{\varphi(n)}{2}}\pmod n.\]
	\end{theorem}

	\begin{proof}
		According to Theorem \ref{thm:ax=b}, for any $g_i$, there exists some $x$ such that
			\begin{align}
				g_i x\equiv a\pmod n.
			\end{align}
		It is clear that $x \bot n$ because if $(x,n)=d$, then $d\mid n$ and $n\mid g_{i}x-a$, so $d\mid g_ix-a$ implies $d\mid a$. Since $\gcd(a,d)=1$, we have $d=1$. So $x=g_j$ for some $j$. We have $g_i \neq g_j$ since $a$ is a quadratic non-residue. Moreover, $g_j$ is unique because if $g_ig_k \equiv a \pmod n$ for some $k$, then $g_k \equiv g_j \pmod n$ and since $g_k$ and $g_j$ are both less than $n$, this forces $g_k=g_j$. Thus, we can pair up the $\varphi(n)$ elements of $\mathbb{U}_n$ into $\varphi(n)/2$ pairs $(g_i, g_j)$, such that $g_ig_j \equiv a \pmod n$. Hence,
			\begin{equation*}
			g_1g_2\cdots g_{\varphi(n)}\equiv a^{\frac{\varphi(n)}{2}}\pmod n.
			\end{equation*}
	\end{proof}
Here is a nice theorem which relates primitive roots and quadratic residues modulo a prime $p$.
	\begin{theorem}\slshape\label{thm:pr+qr}
		If $g$ is a primitive root modulo a prime $p$, then the quadratic residues of $p$ are $g^2,g^4,\ldots,g^{p-1}$.
	\end{theorem}

	\begin{proof}
		By Euler's criterion, we know that if $a$ is a quadratic residue modulo $p$, then
			\begin{align*}
				a^{\frac{p-1}{2}} \equiv 1 \pmod p.
			\end{align*}
		Using Fermat's little theorem, it follows that $g^2,g^4,\ldots,g^{p-1}$ are all quadratic residues modulo $p$. Since the set $\{g^1,g^2,\ldots,g^{p-1}\}$ is a complete residue set modulo $p$ and we know by Theorem \ref{thm:primeresidue} that there are exactly $\frac{p-1}{2}$ incongruent quadratic residues modulo $p$, we find that $g^2,g^4,\ldots,g^{p-1}$ are the only quadratic residues.
	\end{proof}
It is a natural question whether there exists a primitive root $g$ modulo an arbitrary positive integer $n$. If the answer is negative, one might ask for which $n$ there exists a primitive root. We will answer these questions shortly.
	\begin{theorem}\slshape\label{thm:prTest}
		A positive integer $g$ is a primitive root modulo $n$ if and only if
			\begin{align*}
				g^{\frac{\varphi(n)}{p}}\not\equiv1\pmod n
			\end{align*}
		for any prime $p$ which divides $\varphi(n)$.
	\end{theorem}

	\begin{proof}
		It is straightforward to check the truth of the ``if'' part. For the sake of contradiction, assume $\varphi(n)=pk$ and
			\begin{align*}
				g^{k} & \equiv1\pmod n.
			\end{align*}
		But this would contradict the minimality of $\varphi(n)$ since $k$ is less than $\varphi(n)$, meaning that $g$ is not a primitive root.

		For the ``only if'' part, assume that for every prime divisor $p$ of $\varphi(n)$, we have
			\begin{align*}
				g^{\frac{\varphi(n)}{p}}\not\equiv1\pmod n.
			\end{align*}
		We want to show that $g$ is a primitive root modulo $n$. Let $d=\ord_n(g)$, so that $d \mid \varphi(n)$. If $d<\varphi(n)$, then we must have $d\mid {\varphi(n)}/{p}$ for some prime $p$ dividing $\varphi(n)$. Letting $\varphi(n)=pdl$,
			\begin{align*}
				g^{\frac{\varphi(n)}{p}}
					& \equiv g^{dl}\pmod{n}\\
					& \equiv \left(g^d\right)^l\pmod{n}\\
					& \equiv1\pmod n,
			\end{align*}
		which is a contradiction. Hence, $d=\varphi(n)$ must hold.
	\end{proof}

	\begin{note}
		In the proof above, we could just take $p$ to be the smallest prime divisor of $\varphi(n)$. Then we must have that $d$ is a divisor of ${\varphi(n)}/{p}$. This is because the greatest divisor of $n$ less than $n$ is ${n}/{p}$, where $p$ is the smallest prime divisor of $n$ (can you sense why?).
	\end{note}

	\begin{corollary}\label{cor:prpower}
		Let $m$ be a positive integer. If $g$ is a primitive root of $n$, then $g^m$ is also a primitive root modulo $n$ if and only if $m \bot \varphi(m)$.
	\end{corollary}

	\begin{proof}
		Let $(m,\varphi(n))=d$, so that $m\varphi(n)=d\cdot [m,\varphi(n)]$. According to Theorem \ref{thm:prTest}, $g^m$ is a primitive root modulo $n$ if and only if
			\begin{align}\label{eq:prpower1}
			g^{{d\cdot [m,\varphi(n)]}/{p}}\not\equiv1\pmod n,
			\end{align}
		for all prime divisors $p$ of $\varphi(n)$. Now, if $d \neq 1$, there exists a prime $q$ which divides $d$. In that case, write $d=qk$ for some integer $k$. But then,
			\begin{align*}
				g^{{d\cdot [m,\varphi(n)]}/{q}}
					& \equiv g^{k \cdot [m,\varphi(n)]}\pmod n
			\end{align*}
		and since $[m,\varphi(n)]$ is divisible by $\varphi(n)$, we have $g^{ [m,\varphi(n)]} \equiv 1\pmod n$. Thus,
			\begin{align*}
			g^{{d\cdot [m,\varphi(n)]}/{q}} \equiv \left(g^{[m,\varphi(n)]}\right)^k \equiv 1\pmod n,
			\end{align*}
		which is in contradiction with equation \eqref{eq:prpower1} since $q$ is a prime divisor of $\varphi(n)$ (why?). So, we must have $d=1$, and the proof is complete.
	\end{proof}

Assume that some positive integer $n$ has a primitive root. An interesting question is to find the number of primitive roots which are incongruent modulo $n$. The next theorem answers this question.

	\begin{theorem}\slshape\label{thm:npr}
		For any positive integer $n$, if there exists a primitive root modulo $n$, then there are exactly $\varphi(\varphi(n))$ incongruent primitive roots modulo $n$.
	\end{theorem}

	\begin{note}
		In case the word \textit{incongruent} is somewhat unclear to you: two integers $a$ and $b$ are called \textit{incongruent} modulo a natural number $n$ if and only if $a \not \equiv b \pmod n$.
	\end{note}

	\begin{proof}
		Assume that $g$ is a primitive root modulo $n$. We aim to find all primitive roots of $n$. Since we are looking for incongruent primitive roots modulo $n$, it suffices to search in the set $\mathbb U_n$. Theorem \ref{thm:prequalsU} tells us that $\{g,g^2,\cdots,g^{\varphi(n)}\} \equiv \mathbb U_n \pmod n$ and so we should search for primitive roots in the set $\{g,g^2,\cdots,g^{\varphi(n)}\}$. On the other hand, Corollary \ref{cor:prpower} implies tha we should only investigate powers $g^m$ of $g$ for which $(m, \varphi(n))=1$. The number of such elements is $\varphi(\varphi(n))$.
	\end{proof}

We are back to the first question: for which integers $n$ do we have a primitive root? The process of finding such $n$ is long, and we will break it into smaller parts. The first step is to see if there exist primitive roots modulo primes. We will soon prove that there always exists a primitive root modulo any prime. We need the following lemma to prove our claim.

	\begin{lemma}\label{lem:exactlyDsolutions}
		Let $p$ be a prime and $d$ be a positive integer such that $d \mid p-1$. Then $x^d -1 \equiv 0 \pmod p$ has exactly $d$ incongruent solutions modulo $p$.
	\end{lemma}

	\begin{proof}
		Let $p-1=dk$ for some integer $k$. Consider the polynomial
			\begin{align*}
				P(x)= 1+x^d + \left(x^d\right)^2 + \cdots + \left(x^d\right)^{k-1}.
			\end{align*}
		Then,
			\begin{align}\label{eq:exactlyDsolutions}
				x^{p-1} -1 = (x^d-1)P(x).
			\end{align}
		By Fermat's theorem, all integers $1, 2, \ldots, p-1$ are solutions to $x^{p-1}-1 \equiv 0 \pmod p$. So, this equation has exactly $p-1=dk$ solutions. From \eqref{eq:exactlyDsolutions}, each of these $dk$ solutions is either a solution of $P(x) \equiv 0 \pmod p$ or a solution of $x^d-1 \equiv 0 \pmod p$. However, Lagrange's theorem says that $P(x) \equiv 0 \pmod p$ has at most $d(k-1)$ solutions and that $x^d-1 \equiv 0 \pmod p$ has at most $d$ solutions. Since $dk=d(k-1)+d$, this is only possible when $x^d-1 \equiv 0 \pmod p$ has exactly $d$ solutions and also $P(x) \equiv 0 \pmod p$ has exactly $d(k-1)$ solutions.
	\end{proof}

	\begin{theorem}\slshape\label{thm:primeprimitive}
		Let $p$ be a prime. There are exactly $\varphi(p-1)$ incongruent primitive roots modulo $p$.
	\end{theorem}

	\begin{proof}
		The case $p=2$ is obvious. If there exists one primitive root of $p$, then by Theorem \ref{thm:npr} there are exactly $\varphi(\varphi(p))=\varphi(p-1)$ incongruent primitive roots of $p$.

		So we just need to construct a primitive root for $p$. The trick is to factorize $\varphi(p)=p-1$ into product of prime powers. Let $q$ be a prime such that $q^k \mid p-1$ for some integer $k \geq 1$. We want to show that there exists some integer $a$ for which $\ord_p(a)=q^k$. By previous lemma, the equation $x^{q^k} - 1 \equiv 0 \pmod p$ has exactly $q^k$ solutions. Take $a$ to be one of these solutions. Then $a^{q^k} \equiv 1 \pmod p$, and so by Theorem \ref{thm:ordDiv}, it follows that $\ord_p(a)\mid q^k$. So $\ord_p(a)=q^j$ for some integer $0 \leq j \leq k$. This means that $a$ is a solution to the equation $x^{q^j} - 1 \equiv 0 \pmod p$. If $j=k$, we have found such an $a$. Otherwise, suppose that $j<k$. Let $i=k-j\geq 1$. Note that if $x^{q^j} \equiv 1 \pmod p$, then,
			\begin{align*}
				x^{q^{k-1}}
					& \equiv \left(x^{q^j}\right)^{k-j-1}\pmod{p}\\
					& \equiv \left(x^{q^j}\right)^{i-1}\pmod{p}\\
					& \equiv 1 \pmod p
			\end{align*}
		That is, every solution to $x^{q^j} \equiv 1 \pmod p$ is also a solution to $x^{q^{k-1}} \equiv 1 \pmod p$. According to the preceding lemma, number of solutions of $x^{q^{k-1}} \equiv 1 \pmod p$ is exactly $q^{k-1}$. So there are exactly $q^k - q^{k-1}$ integers $x$ which satisfy $x^{q^k} \equiv 1 \pmod p$ but not $x^{q^{k-1}} \equiv 1 \pmod p$. If we select $a$ from these solutions, we will have $\ord_p(a)=q^k$.

		To finish the proof, let $$p-1 = q_1^{\alpha_1}q_2^{\alpha_2} \cdots q_t^{\alpha_t},$$ be the prime factorization of $p-1$. From what we have just proved, there exists some integer $a_i$ for each $q_i$ such that $\ord_p(a_i)=q_i^{\alpha_i}$. According to Theorem \eqref{thm:prproduct}, since $q_i$ are coprime,
			\begin{align*}
				\deg_p\left(\prod_{i=1}^{t}a_i\right)
					&=\deg_p(a_1) \deg_p(a_2) \cdots \deg_p(a_t)\\
					& =q_1^{\alpha_1}q_2^{\alpha_2} \cdots q_t^{\alpha_t}\\
					& = p-1 =\varphi(p),
			\end{align*}
		and so $\prod_{i=1}^{t}a_i$ is a primitive root modulo $p$.
	\end{proof}

The next step is to find what other numbers have a primitive root. We will show, by the help of the following lemma, that all powers of an odd prime number have a primitive root.

	\begin{lemma}
		Let $p$ be an odd prime and let $g$ be a primitive root modulo $p$ such that $g^{p-1} \not \equiv 1 \pmod{p^2}$. Then, $$g^{\varphi(p^k)} \not \equiv 1 \pmod{p^{k+1}}$$ for any integer $k \geq 1$.
	\end{lemma}

	\begin{proof}
		We will induct on $k$. The base case $k=1$ is immediately followed from the assumption that $g^{p-1} \not \equiv 1 \pmod{p^2}$. As the induction hypothesis, consider that $g^{\varphi(p^k)} \not \equiv 1 \pmod{p^{k+1}}$ for some $k\geq 1$. From Euler's theorem, $g^{\varphi(p^k)} \equiv 1 \pmod{p^k}$, which means $$g^{\varphi(p^k)} = 1+mp^k$$ for some $m$. The induction hypothesis implies that $p \nmid m$. By Proposition \ref{prop:phiproperties}, we know that $$\varphi(p^{k+1})=p^{k+1}-p^k=p\left(p^k - p^{k-1}\right)=p\cdot \varphi(p^k).$$
		Hence,
			\begin{align*}
				g^{\varphi(p^{k+1})}
					&= (1+mp^k)^p \\
					&= 1+ \binom{p}{1}mp^k + \underbrace{\binom{p}{2} (mp^k)^2 + \cdots +\binom{p}{p-1} (mp^k)^{p-1} +(mp^k)^p}_{\mbox{divisible by }p^{k+2}}\\
								&\equiv 1+mp^{k+1} \pmod {p^{k+2}}.
			\end{align*}
		As $m$ is not divisible by $p$, $mp^{k+1}$ is not divisible by $p^{k+2}$. So,
			\begin{align*}
				g^{\varphi(p^{k+1})} \not \equiv 1 \pmod {p^{k+2}},
			\end{align*}
		as desired.
	\end{proof}

	\begin{theorem}\slshape\label{thm:primepowerprimitive}
		Let $p$ be an odd prime and let $g$ be a primitive root modulo $p$ (as we know exists from Theorem \ref{thm:primeprimitive}). Then, either $g$ or $g+p$ is a primitive root modulo $p^k$ for any integer $k\geq 1$.
	\end{theorem}

	\begin{proof}
		We break the proof into two parts:
		\begin{enumerate}
			\item If $g^{p-1} \not \equiv 1 \pmod{p^2}$, then we will show that $g$ is a primitive root of $p^k$. That is, we will prove that
				\begin{align}\label{eq:ordermodprimepower}
					\ord_{p^k} (g)=\varphi(p^k)=p^{k-1}(p-1).
				\end{align}
			This is obviously true for $k=1$. Suppose that equation \eqref{eq:ordermodprimepower} holds for some $k\geq 1$. We will prove that it also holds for $k+1$. Let $\ord_{p^{k+1}} (g)=m$. Then
				\begin{align*}
					g^m \equiv 1 \pmod{p^{k+1}} \implies g^m \equiv 1 \pmod{p^{k}}.
				\end{align*}
			Since we know that order of $g$ modulo $p^k$ is $\varphi(p^k)$, we should have $\varphi(p^k)=p^{k-1}(p-1)|m$. On the other hand, since $m$ is the order of $g$ modulo $p^{k+1}$, by corollary \eqref{cor:phiDiv}, we get $m|\varphi(p^{k+1})=p^k(p-1)$. Therefore, $m$ equals either $\varphi(p^k)=p^{k-1}(p-1)$ or $\varphi(p^{k+1})=p^{k}(p-1)$. Previous lemma states that it is impossible to have $m=\ord_{p^{k+1}} (g)=\varphi(p^k)$. So, $m=\varphi(p^{k+1})$ and we are done.

			\item If $g^{p-1} \equiv 1 \pmod{p^2}$, then we will show that $g+p$ is a primitive root modulo $p^k$ for any integer $k\geq 1$. Note that
				\begin{multline*}
					(g+p)^{p-1} = g^{p-1} + \binom{p-1}{1} g^{p-2}p \\ + \underbrace{\binom{p-1}{2}g^{p-3}p^2+\cdots + \binom{p-1}{p-2} gp^{p-2} + p^{p-1}}_{\mbox{divisible by} p^2}
				\end{multline*}
			Taking modulo $p^2$, the above is
				\begin{align*}
					\equiv g^{p-1}+(p-1)g^{p-2}p \equiv 1 - g^{p-2}p \not \equiv 1 \pmod{p^2},
				\end{align*}
			because $p \nmid g$. We can now apply the same approach we followed in the first case, but now with $g+p$ instead of $g$. So $g+p$ is a primitive root modulo all powers of $p$ and the proof is complete.
		\end{enumerate}
	\end{proof}

Finally, we are ready to answer our question.

	\begin{theorem}[Primitive Root Theorem]\slshape
		Let $n>1$ be a positive integer. There exists a primitive root modulo $n$ if and only if $n\in\{2,4,p^k,2p^k\}$ for some odd prime $p$ and a positive integer $k$.
		\label{thm:pr:wh}
	\end{theorem}

	\begin{proof}
		Obviously, $g=1$ and $g=3$ are primitive roots modulo $2$ and $4$, respectively. So, $n=2$ and $n=4$ are off the list. Let's consider the ``if'' part first. If $n$ has a primitive root, we will prove $n$ must be of the form $p^k$ or $2p^k$, where $p$ is an odd prime.
		First, let us show that $2^k$ does not have a primitive root for $k >2$. It is obvious that if $a$ is a primitive root modulo $2^k$, then $a$ is odd. We leave it as an exercise for the reader to prove by induction that for any odd $a$ and $k>2$,
			\begin{align*}
				2^k & \mid a^{2^{k-2}}-1.
			\end{align*}
		Since $\varphi(2^k)=2^{k-1}$, $a$ is never a primitive root modulo $2^k$.

		Now, if $n$ is not of the form $p^k$ or $2p^k$, we can write $n=ab$ with $\gcd(a,b)=1$ and $a>b>2$. So, $\varphi(b)$ and $\varphi(a)$ are larger than $1$, and by Proposition \ref{prop:phiproperties}, they are both even. Let $g$ be a primitive root modulo $n$. This means that $\ord_{n}(g)=\varphi(ab)$. We will show that this cannot happen. Let $\ord_a(g)=d$ and $\ord_b(g)=e$. Since
			\begin{align*}
				g^{\varphi(a)} & \equiv1\pmod a\\
				g^{\varphi(b)} & \equiv1\pmod b,
			\end{align*}
		by Corollary \eqref{cor:phiDiv}, we find that $d \mid \varphi(a)$ and $e\mid \varphi(b)$.
		Hence, by Theorem \eqref{thm:ordL},
			\begin{align*}
				\ord_{ab}(g)
					& =  [d,e]\\
					& \leq [\varphi(a),\varphi(b)]\\
					& = \dfrac{\varphi(a) \varphi(b)}{(\varphi(a), \varphi(b))}\\
					& = \dfrac{\varphi(ab)}{(\varphi(a), \varphi(b))}\\
					& \leq  \dfrac{\varphi(ab)}{2}
			\end{align*}
		where we have used the fact that $(\varphi(a),\varphi(b))$ is at least $2$. This gives us the contradiction we were looking for. So, $n$ must be of the form $p^k$ or $2p^k$.

		The only remaining part is to prove that for an odd prime $p$ and $k \geq 1$, there exist primitive roots modulo the numbers $p^k$ and $2p^k$. According to Theorem \ref{thm:primeprimitive}, $p$ has a primitive root, say $g$. It now follows from Theorem \ref{thm:primepowerprimitive} that either $g$ or $g+p$ is a primitive root modulo $p^k$. Since Euler's totient function is multiplicative, we have
			\begin{align*}
				\varphi(2p^k)
					& = \varphi(2)\cdot \varphi(p^k)\\
					& = \varphi(p^k)
			\end{align*}
		Let $g$ be a primitive root modulo $p^k$.
		\begin{enumerate}
			\item If $g$ is odd, then
			\begin{align*}
				g^a
					& \equiv 1 \pmod{p^k}\\
				\iff g^a
					& \equiv 1 \pmod{2p^k}
			\end{align*}
			Let $m = \ord_{2p^k}(g)$. If $m<\varphi(2p^k)=\varphi(p^k)$, then $g^m \equiv 1\pmod{2p^k}$ implies $g^m \equiv 1\pmod{p^k}$, which contradicts the fact that $g$ is a primitive root modulo $p^k$. Therefore, $g$ is also a primitive root modulo $2p^k$.
			\item If $g$ is even, then $g'=g+p^k$ is an odd number and it is also a primitive root modulo $p^k$. Applying the same approach used in the first case, we find that $g'$ is a primitive root modulo $2p^k$.
		\end{enumerate}
		We have shown that $2p^k$ always has a primitive root and the proof is complete.
	\end{proof}

Here is a generalization of Wilson's theorem, though it can be generalized even further. We refer the reader to section \eqref{sec:wilsongeneral} of the book to see another generalization of Wilson's theorem.
	\begin{problem}\label{thm:genWilson}
		Let $n$ be a positive integer and let $\mathbb U_n = \{g_1, g_2, \ldots, g_{\varphi(n)}\}$. Prove that if there exists a primitive root modulo $n$, then
		\[g_1g_2\cdots g_{\varphi(n)}\equiv-1\pmod n\]
		Otherwise,
		\[g_1g_2\cdots g_{\varphi(n)}\equiv1\pmod n\]
	\end{problem}

	\begin{hint}
		Combine Theorems \ref{thm:pr:wh} and \ref{thm:prTest} along with the fact that if $p$ is an odd prime and $k$ is a positive integer, then $p^k \mid a^2-1$ implies $p^k \mid a+1$ or $p^k\mid a-1$.
	\end{hint}

	\begin{theorem}\slshape
		Let $g$ be a primitive root modulo $n$. Then $n-g$ is a primitive root modulo $n$ as well if $4$ divides $\varphi(n)$.
	\end{theorem}

	\begin{proof}
		We have a criteria to see if $x$ is a primitive root modulo $n$. We need to check if $x^{{\varphi(n)}/{p}}\not\equiv1\pmod n$ for any prime $p$ which divides $\varphi(n)$. Therefore, to check if $n-g$ is a primitive root of $n$, we just need to prove the following holds
		\begin{align*}
		(n-g)^{{\varphi(n)}/{p}} & \not\equiv1\pmod n
		\end{align*}
		for any prime divisor $p$ of $\varphi(n)$. Now, since $4 \mid \varphi(n)$, we have $2 \mid {\varphi(n)}/{2}$. So, ${\varphi(n)}/{p}$ is even for any proper $p$. Using the fact that $g^2\equiv(n-g)^2\pmod n$, we get
		\begin{align*}
		(n-g)^\frac{\varphi(n)}{p}   & \equiv\left((n-g)^2\right)^{\frac{\varphi(n)}{2p}} \equiv \left(g^2\right)^{\frac{\varphi(n)}{2p}}  \equiv g^{\frac{\varphi(n)}{p}} \not\equiv1\pmod n.
		\end{align*}
		Thus, $(n-g)$ is a primitive root modulo $n$ as well.
	\end{proof}
The use of primitive roots is usually not obvious in problems. There is hardly any hint on why you should use it. Best if you see its use through problems.
	\begin{problem}
		Let $p$ be odd prime number. Prove that equation  $x^{p-1}\equiv 1 \pmod{p^n}$ has exactly $p -1$ different solution modulo $p^{n}$.
	\end{problem}

	\begin{solution}[1]
		Let $g$ be a primitive root modulo $p^n$ (which exists by Theorem \ref{thm:pr:wh}). Now, take $x=g^k$, so every solution $x$ maps to a certain $k$. The number of different $k$ is the number of solutions of this congruence equation. Since
			\begin{align*}
				\ord_{p^n}(g)
					& =p^{n-1}(p-1)\\
				g^{k(p-1)}
					& \equiv1\pmod{p^n}
			\end{align*}
		we either have $$p^{n-1}(p-1) \mid k(p-1)$$ or $p^{n-1} \mid k$ for any such $k$. Take $k=p^{n-1}\ell$. If $\ell=sp+r$ with $1\leq r<p$, then we have that
			\begin{align*}
				g^k
					& =g^{p^{n-1}(p-1)(sp+r)}\pmod{p^{n}}\\
					& \equiv g^{p^{n-1}(p-1)r}\pmod{p^n}
			\end{align*}
		Therefore, for two incongruent solutions, we must have $1\leq r\leq p-1$, giving us exactly $p-1$ solutions.
	\end{solution}

	\begin{solution}[2]
		This is a special case of Lemma \ref{lem:exactlyDsolutions}, where $d=p-1$.
	\end{solution}

	\begin{problem}
		Prove that $3$ is a primitive root modulo $p$, where $p$ is any prime of the form $2^n+1$ for some integer $n>1$.
	\end{problem}

	\begin{solution}[1]
		$p=2^n+1$ in particular means $p \equiv 1 \pmod{4}$.
		According to Problem \ref{prob:prime=poweroftwoplusone}, we find that $p$ is of the form $2^{2^r} + 1$ for some positive integer $r$. Therefore
			\begin{align*}
				p
					& = 2^{2^{r}} +1\\
					& \equiv (-1)^{2^r} +1\\
					& \equiv 2 \pmod{3}
			\end{align*}
		which is not a quadratic residue modulo $3$.
		Using the law of quadratic reciprocity and the fact that $p \equiv 1 \pmod{4}$,
		\[\left(\dfrac{3}{p}\right) \left(\dfrac{p}{3}\right) = (-1)^{(p-1)/2} =1.\] From the above discussion, we know that $\left( \frac{p}{3} \right) = -1$. Therefore, $\left(\frac{3}{p}\right)=-1$ and $3$ is a quadratic non-residue modulo $p$.

		We will now prove that for a prime of the form $p=2^{2^{r}}+1$ every quadratic non-residue modulo $p$ is a primitive root modulo $p$. Since $p$ is a prime, we know that there exists a primitive root modulo $p$, say $g$. By Theorem \ref{thm:pr+qr}, we know that $g^2,g^4,\cdots,g^{p-1}$ are ${p-1}/{2}$ different nonzero residues modulo $p$ and they are all quadratic residues. Therefore, all the quadratic non-residues are given by $$g,g^3,g^5,\cdots,g^{p-2}.$$
		We will now take one of these residues, say $g^{2k+1}$, and show that it is a primitive root mod ${p}$. This means we want to show that $$g^{2k+1},g^{2(2k+1)},g^{3(2k+1)},\ldots,g^{(p-1)(2k+1)}$$ are incongruent modulo $p$, which  happens if and only if $$2k+1,2(2k+1),3(2k+1),\ldots,(p-1)(2k+1)$$ are all different modulo ${p-1}$. This happens if and only if $(2k+1,p-1)=1$, or $(2k+1,2^{2^r})=1$, which is clearly true since $2k+1$ is odd and $2^{2^{r}}$ is a power of $2$.

		Therefore, all quadratic-non residues are primitive roots modulo $p$, and as we have shown $3$ is among them, we are done.
	\end{solution}

	\begin{solution}[2]
		Just like the previous solution, we will use the fact that $3$ is not a quadratic residue modulo $p$. Therefore, by Euler's criterion,
			\begin{align}\label{eq:3^{2^{m-1}}}
				3^{\frac{p-1}{2}} \equiv  \left( \frac{3}{p} \right) = -1 \pmod p \implies 3^{2^{m-1}} \equiv -1 \pmod p.
			\end{align}
		Let $d$ be the order of $3$ modulo $p$. Since $d\mid p-1=2^n$, we must have $d=2^{\alpha}$ for some integer $\alpha$. If $\alpha<n$ then
			\begin{align*}
				3^{2^{\alpha}}\equiv 1\pmod{p}\implies 3^{2^{n-1}}\equiv 1\pmod{p},
			\end{align*}
		which is in contradiction with equation \eqref{eq:3^{2^{m-1}}}. So, $d=2^n$, and this means that $3$ is primitive root modulo $p=2^n+1$.
	\end{solution}

	\begin{problem}
		Let $p$ and $q$ be prime numbers such that $ p=2q+1$. Let $a$ be an integer coprime to $p$ and incongruent to $-1$, $0$, and $1$ modulo $p$. Show that $ -a^2$ is primitive root modulo $p$.
	\end{problem}

	\begin{solution}
		Check $ q=2$ for yourself. Assume $ q \ge 3$ is an odd prime, say $q=2k+1$. Hence, $ p=4k+3$, or $p \equiv 3 \pmod 4$. According to Theorem \ref{thm:a^2+b^2}, $ -a^2$ is not a quadratic residue modulo $ p$. Suppose that $ -a^2$ is not a primitive root modulo $p$. Let $ g$ be a primitive root modulo $p$. Theorem \ref{thm:pr+qr} states that there exists an $ l\ge 1$ such that $$g^{2l+1} \equiv -a^2 \pmod p.$$ Since $-a^2$ is not a primitive root, there exists an integer $k$ with $ k<p-1$ such that \[g^{(2l+1)k} \equiv (-a^2)^k \equiv 1 \pmod p .\] This, together with Fermat's little theorem, implies $ (2l+1)k \mid p-1=2q$ and hence $ k=2$. Therefore, $ a^4 \equiv 1 \pmod p$ and by Fermat's little theorem $ 4\mid 2q$, which leads to a contradiction as $q$ is prime. Hence, $ -a^2$ is a primitive root modulo $p$.
	\end{solution}

	\begin{problem}
		Let $q$ be a prime such that $q\equiv 1\pmod 4$ and that $p=2q+1$ is also prime. Prove that $2$ is a primitive root mod $p$.
	\end{problem}

	\begin{solution}
		By Euler's criterion, we have
		\[2^q \equiv 2^{\frac{p-1}{2}} \equiv \pm 1 \pmod{p}.\]
		 We analyze both cases now:
			\begin{itemize}
				\item Assume that $2^q \equiv -1 \pmod p$. Let $\ord_p(2)=d$. Then, $d\mid \varphi(p)=p-1=2q$. Since $q$ is a prime, we must have $$d \in \{1,2,q,2q\}.$$ Since $q \equiv 1 \pmod 4$, we have $q \geq 5$ and the cases $d=1$ and $d=2$ cannot happen. Also, if $d=q$, then $2^q \equiv 1 \pmod p$, which is in contradiction with $2^q \equiv -1 \pmod p$. Thus $d=2q=p-1$ and $2$ is a primitive root modulo $p$.
				\item Assume that $2^q \equiv 1 \pmod{p}$. Multiply both sides of this equation by $2$ to get $2^{q+1} \equiv 2 \pmod p$. Since $q+1$ is even, the latter equation means that $2$ is a quadratic residue modulo $p$. Therefore, by Theorem \ref{thm:2qr}, $p$ must be congruent to either $1$ or $7$ modulo $8$. However, problem says that $q \equiv 1 \pmod 4$ which results in $q \equiv 1$ or $5 \pmod 8$. Now,
				\[p \equiv 2q +1 \equiv 3 \text{ or } 5 \pmod 8,\]
				which is a quick contradiction. Hence, $2^q \equiv 1 \pmod{p}$ is not possible.
			\end{itemize}

	\end{solution}

	\begin{problem}
		Suppose that $p$ is an odd prime number. Prove that there exists a positive integer $x$ such that $x$ and $4x$ are both primitive roots modulo $p$.
	\end{problem}

	\begin{solution}
		We will prove a stronger claim: there exists some $x$ such that both $x$ and $d^2x$ are primitive roots mod $p$ for any integer $d$. Let $g$ be a primitive root modulo $p$. Since $d^2$ is a quadratic residue mod $p$, it follows by Theorem \ref{thm:pr+qr} that
			\begin{align*}
				d^2
					& \equiv g^{2k} \pmod{p}
			\end{align*}
		for some integer $k$. We then find by Corollary \ref{cor:prpower} that any power $g^n$ of $g$ is a primitive root modulo $p$ if and only if $(n, p-1)=1$.

		Now, it suffices to show there exist two integers $a$ and $b$ such that
			\begin{align*}
				b-a
					& = 2k\\
				\quad \gcd(b,p-1)\\
					& = \gcd(a,p-1) = 1
			\end{align*}
		because then $x=g^a$ would be a solution.
		This is luckily easy. Let $2,q_1,q_2,\ldots,q_z$ be the prime divisors of $p-1$. Suppose that $a_1,a_2,\ldots,a_z$ are integers such that $2k \equiv a_i \pmod{q_i}$ for each $1 \le i \le z$.
		By CRT, there exists an $a$ such that
			\begin{align*}
				a
					& \equiv 1 \pmod{2}\\
				a
					& \equiv -a_i + p_i \pmod{q_i}
			\end{align*}
		where $p_i$ is some prime, not equal to $q_i$, and $a_i \not \equiv p_i \pmod{q_i}$. It is easy to see $\gcd(a,p-1) = 1$ and $\gcd(a+2k, p-1) = 1$. Thus, $g^a$ and $g^{a+2k}$ are primitive roots modulo $p$ and $g^{a+2k} \equiv d^2g^a \pmod{p}$, done.
	\end{solution}

\end{document}
\section{Carmichael Function, Primitive \texorpdfstring{$\lambda$}{L}-roots}
	\documentclass{subfile}


\begin{document}
	\subsection{Carmichael $\lambda$ Function}
	In this section, we discuss a very important function in number theory. \textcite{carmichael1910} first introduced it for generalizing Euler's totient function. Consider that two relatively prime positive integers $a,n$ are given, and $\ord_n(a)=d$. Now, fix $n$. Consider the case when $a^d\equiv1\pmod n$ holds for any positive integer $a$ relatively prime to $n$. This brings up some questions.
	\begin{problem}\label{prob:CarmichaelQuestion1}
		Does there exists an $a$ such that $\ord_n(a)=d$?
	\end{problem}
	
	\begin{problem}\label{prob:CarmichaelQuestion2}
		How do we find the minimum $d$ such that $a^d\equiv1\pmod n$ holds for any $a$ relatively prime to $n$?
	\end{problem}
	Let's proceed slowly. We will develop the theories that can solve these problems. For doing that, we have to use properties of order and primitive roots we discussed in previous sections.
	\begin{definition}[Carmichael Function]
		For a positive integer $n$, $\lambda(n)$ is the smallest positive integer for which $a^{\lambda(n)}\equiv1\pmod n$ holds for any positive integer $a$ relatively prime to $n$. This number $\lambda(n)$ is called the \textit{Carmichael function} of $n$. Sometimes, it is also called the \textit{lambda function} of $n$ or the \textit{minimum universal exponent} $\pmod m$ \citep[Chapter $\S$VI, Section $4$, Page $265$]{sierpinski_schinzel_1988}.
	\end{definition}
	Note that Theorem \eqref{thm:ordDiv} implies the following theorem.
	\begin{theorem}\slshape\label{thm:carDiv}
		If $a^d\equiv1\pmod n$ holds for all $a$ relatively prime to $n$, then $\lambda(n)\mid d$.
	\end{theorem}
	
	\begin{corollary}\label{cor:LambdaDividesPhi}
		For any positive integer $n$, $\lambda(n)\mid \varphi(n)$.
	\end{corollary}
	The following theorem is self-implicating and solves the first problem, if we can prove that $\lambda(n)$ exists. For now, let's assume it does.
	\begin{theorem}\slshape
		Let $n$ be a positive integer. There exists a positive integer $a$ relatively prime to $n$ such that $\ord_n(a)=\lambda(n)$.
	\end{theorem}
	Let's focus on finding $\lambda(n)$. First, consider the case $n=2^k$.
	\begin{theorem}\slshape
		If $k>2$ then $\lambda(2^k)=2^{k-2}$.
	\end{theorem}
	
	\begin{proof}
		The integers relatively prime to $2^k$ are all odd numbers. We will prove by induction that $x^{2^{k-2}} \equiv1\pmod{2^k}$ holds for all odd positive integers $x$. The base case $k=3$ is obvious. Assume that for some $k\geq 3$, we have
			\begin{align*}
				x^{2^{k-2}} & \equiv1\pmod{2^k},
			\end{align*}
		or equivalently, $x^{2^{k-2}} -1=2^kt$ for some $t$. Using the identity $a^2-b^2=(a-b)(a+b)$, we can write
			\begin{align}
				x^{2^{k-1}} -1 = \left(x^{2^{k-2}} -1\right)\left(x^{2^{k-2}} +1\right) = 2^kt\left(2^kt +2\right) = 2^{k+1}t \left(2^{k-1}t+1\right). \label{eq:x^{2^{k-1}}-1}
			\end{align}
		This gives $x^{2^{k-1}} \equiv 1\pmod{2^{k+1}}$, and the induction is complete.
		
		Now, we should prove that $2^{k-2}$ indeed is the smallest such integer. Again, by induction, the base case is to find an $x$ for which $\ord_8(x)=2$. Obviously, any $x=8j\pm3$ satisfies this condition. Assume that for all numbers $t$ from $1$ up to $k$, we have $\lambda(2^l)=2^{l-2}$. Let $\lambda(2^{k+1})=\lambda$. Since we proved that $x^{2^{k-1}} \equiv 1\pmod{2^{k+1}}$ for all odd $x$, it follows from Theorem \ref{thm:carDiv} that $\lambda \mid 2^{k-1}$. So $\lambda$ is a power of $2$. If $\lambda = 2^{k-1}$, we are done. Otherwise, let $\lambda=2^\alpha$, where $1 \leq \alpha <k-1$. Then for every $x$, one can write
			\begin{align}\label{eq:x^2^k+1}
				x^{2^\alpha} \equiv 1 \pmod{2^{k+1}}.
			\end{align}
		However, similarly as in \eqref{eq:x^{2^{k-1}}-1}, for some $t$,
			\begin{align}\label{eq:x^2^a}
				x^{2^{\alpha}} -1 &= 2^{\alpha+2}t \left(2^{\alpha}t+1\right).
			\end{align}
		In \eqref{eq:x^2^a}, the highest power of $2$ which divides $x^{2^{\alpha}} -1$ is $2^{\alpha+2}$ (since $2^{\alpha}t+1$ is odd). But $$\alpha+2 <(k-1)+2=k+1$$, which contradicts \eqref{eq:x^2^k+1}. The induction is complete.
	\end{proof}
	
	\begin{theorem}\slshape
		For any prime $p$ and any positive integer $k$, 
		\[\lambda(p^k)=\lambda(2p^k)=\varphi(p^k).\]
	\end{theorem}
	
	\begin{proof}
		Consider the congruence equation $x^d\equiv1\pmod{p^k}$ and let $d=\lambda(p^k)$. By Corollary \ref{cor:LambdaDividesPhi}, $d \mid \varphi(p^k)$. Take $x=g$ where $g$ is a primitive root modulo $p^k$. Then, $\ord_{p^k}(g)=\varphi(p^k)$ and we immediately have $\varphi(p^k)|d$. Thus, $d=\varphi(p^k)$. A very similar proof can be stated to show that $\lambda (2p^k)=\varphi (2p^k) = \varphi(p^k)$.
	\end{proof}
	
	\begin{theorem}\slshape
		Let $a$ and $b$ be relatively prime positive integers. Then
		 \[\lambda(ab)=\lcm(\lambda(a),\lambda(b)).\]
	\end{theorem}
	
	\begin{proof}
		Suppose that $$\lambda(a)=d, \quad \lambda(b)=e, \quad \text{and} \quad \lambda(ab)=h.$$ Then, $$x^d\equiv1\pmod a, \quad x^e\equiv1\pmod b, \quad \text{and} \quad x^{h}\equiv1\pmod{ab}.$$ We also have $x^h\equiv1\pmod a$ and $x^h\equiv1\pmod b$ as well. Hence, $d \mid h$ and $e \mid h$. This means that $[d,e]=h$ since $[d,e]$ is the smallest positive integer that is divisible by both $d$ and $e$.
	\end{proof}
	Generalization of this theorem is as follows.
	\begin{theorem}\slshape
		For any two positive integers $a$ and $b$,
		\begin{align*}
		\lcm(\lambda(a),\lambda(b)) & = \lambda(\lcm(a,b)).
		\end{align*}
	\end{theorem}
	The next theorem combines the above results and finds $\lambda(n)$ for all $n$.
	\begin{theorem}\slshape\label{thm:CarmichaelFormula}
		Let $n$ be a positive integer with prime factorization $n=p_1^{e_1}p_2^{e_2}\cdots p_r^{e_r}$. Also, let $p$ be a prime and $k$ be a positive integer. Then
		\begin{align*}
			\lambda(n) & = 
			\begin{cases}
				\varphi(n),\mbox{ if } n = 2,4,p^k, \mbox{ or } 2p^k,\\
				\dfrac{\varphi(n)}{2},\mbox{ if }n=2^k\mbox{ with }k>2,\\
				\lcm(\lambda(p_1^{e_1}),\cdots,\lambda(p_r^{e_r})) \mbox{ otherwise.}
			\end{cases}
		\end{align*}
	\end{theorem}
	
	\begin{theorem}\slshape
		For positive integers $a$ and $b$, if $a \mid b$, then $\lambda(a) \mid \lambda(b)$.
	\end{theorem}
	The proof is left as an exercise for the reader. We are now ready to fully solve Problem \ref{prob:CarmichaelQuestion1}.
	\begin{theorem}\slshape
		For fixed positive integers $n$ and $d$, there exists a positive integer $a$  relatively prime to $n$ so that $\ord_n(a)=d$ if and only if $d \mid \lambda(n)$.
	\end{theorem}
	
	\begin{proof}
		The ``if'' part is true by Theorem \ref{thm:ordDiv}. For the ``only if'' part, assume that $g$ is an integer with $\ord_n(g)=\lambda(n)$ and $\lambda(n)=de$. Then $\ord_n(g^e)=d$, as desired.
	\end{proof}
	
	We finish this section by proposing a theorem. We will leave the proof for the reader as an exercise.
		\begin{theorem}\slshape
			If $\lambda(n)$ is relatively prime to $n$, then $n$ is square-free.
		\end{theorem}
	Recall that $n$ is square-free if it is not divisible by any perfect square other than $1$.
	
	\subsection{Primitive $\lambda$-roots}
	\textcite[Page $232-233$, Result II]{carmichael1910} defined a generalization of primitive roots as follows using his function. As you will see, this section generalizes everything related to primitive roots.
	\begin{definition}[Primitive $\lambda$-root]
		Let $a$ and $n$ be relatively prime positive integers. If $\ord_n(a)=\lambda(n)$, then $a$ is a primitive $\lambda$-root modulo $n$. That is, $a^{\lambda(n)}$ is the smallest power of $a$ which is congruent to $1$ modulo $n$. 
	\end{definition}
\textcite{cameron_preece_2014} shows the following theorem.
	\begin{definition}
		Let $n$ be a positive integer. Define $\xi(n) = \frac{\varphi(n)}{\lambda(n)}$ (read $\xi$ as ``ksi''). According to Corollary \ref{cor:LambdaDividesPhi}, $\xi(n)$ is an integer.
	\end{definition}

	\begin{proposition}
		 There is a primitive root (defined in the previous section) modulo $n$ if and only if $\xi(n)=1$. Carmichael calls a primitive root a $\varphi$-primitive root, and they are in fact a special case of $\lambda$-primitive roots.
	\end{proposition}

	Now, the existence of primitive root is generalized to the following theorem from Carmichael's original paper.
	\begin{theorem}[Carmicahel]\slshape
		For any positive integer $n$, the congruence equation
			\begin{align*}
				x^{\lambda(n)} & \equiv1\pmod n
			\end{align*}
		has a solution $a$ which is a primitive $\lambda$-root, and for any such $a$, there are $\varphi(\lambda(n))$ primitive roots congruent to powers of $a$.
	\end{theorem}
	We can show that this theorem is true in a similar fashion to what we did in last section, and we leave it as an exercise.
	
	As we mentioned earlier in Proposition \ref{prop:phiproperties}, $\varphi(n)$ is always even for $n>2$. As it turns out, $\lambda$ and $\varphi$ share some common properties.
	\begin{problem}
		For any integer $n\geq 1$, either $\xi(n)=1$ or $\xi(n)$ is even.
	\end{problem}

	\begin{hint}
		Use the formula for $\lambda(n)$ in Theorem \ref{thm:CarmichaelFormula}.
	\end{hint}

	\begin{problem}
		If $\lambda(n)>2$, the number of primitive $\lambda$-roots modulo $n$ is even.
	\end{problem}

	The next theorem generalizes Theorem \ref{thm:genWilson}, which itself was a generalization to Wilson's theorem. 
	\begin{theorem}\slshape
		Let $n$ be a positive integer such that $\lambda(n)>2$. Also, suppose that $g$ is a primitive $\lambda$-root modulo $n$. The product of primitive $\lambda$-roots of $n$ is congruent to $1$ modulo $n$.
	\end{theorem}
	
	\begin{proof}
		Since $\lambda(n)>2$, we can easily argue that it must be even. If $g$ is a primitive $\lambda$-root modulo $n$, all the primitive $\lambda$-roots are $$\{g^{e_1}, g^{e_2}, \cdots, g^{e_{k}}\},$$ where $e_i$ (for $1 \leq i \leq k$) are all (distinct) positive integers with $(e_i, \lambda(n))=1$. Also, note that we can pair them up since $\lambda(n)$ is even if $n>2$. In fact, we can pair $g^{e_i}$ with $g^{\lambda(n)-e_i}$ for all $i$. Then,
		\begin{equation*}
		g^{e_1}\cdot g^{e_2}\cdots g^{e_k}  \equiv g^{\lambda(n)}\cdots g^{\lambda(n)}  \equiv 1\pmod n. \qedhere
		\end{equation*}
	\end{proof}
	
	\begin{corollary}
		For any $n$, there are $\varphi(\lambda(n))$ primitive $\lambda$-roots modulo $n$.
	\end{corollary}
	

\end{document}
\section{Pseudoprimes} \label{sec:pseudoprimes}
	\documentclass{subfile}

\begin{document}
In general, a \textit{pseudoprime} is an integer which shares a common property (also known as a probable prime, defined later in \autoref{sec:pseudoprimes}) with all prime numbers but is not actually a prime. Pseudoprimes are classified according to which property of primes they satisfy. We will investigate a few types of pseudoprimes in this section.

\subsection{Fermat Pseudoprimes,  Carmichael Numbers}
	The most important class of pseudoprimes are Fermat pseudoprimes which come from Fermat's little theorem.
		\begin{definition}[Fermat Pseudoprime to Base $a$]
			For an integer $a>1$, if a composite integer $n$ satisfies $a^{n-1}\equiv 1\pmod n$, then $n$ is said to be a \textit{Fermat pseudoprime to base $a$} and is denoted by $\psp(a)$.
		\end{definition}
	Suppose $a>1$ is an integer. It can be shown that the number of Fermat pseudoprimes to base $a$ is small compared to the number of primes. Therefore, any number $n$ that passes Fermat's little theorem (i.e., $a^{n-1} \equiv 1 \pmod n$) could be considered to be probably a prime and that is why it is called \textit{pseudoprime}.

		\begin{example}
			Fermat pseudoprime to base $2$ are called \textit{Poulet numbers}. $341=11\times 31$ is the smallest Poulet number. The reason is that
				\begin{align*}
					2^{340}
						& \equiv \left(2^5\right)^{68}\\
						& \equiv (32)^{68}\\
						& \equiv 1^{68}\\
						& \equiv 1 \pmod{31}
				\end{align*}
			and
				\begin{align*}
					2^{340}
						& \equiv \left(2^{10}\right)^{34}\\
						& \equiv (1024)^{34}\\
						& \equiv (1)^{34}\\
						& \equiv 1 \pmod{11}
				\end{align*}
			which yields $2^{340} \equiv 1 \pmod{341}$.
		\end{example}


		\begin{theorem}
			For any integer $a>1$, there are infinitely many Fermat pseudoprime to base $a$.
		\end{theorem}

		\begin{proof}
			Let $p\geq 3$ be any prime number such that $p \nmid a^2-1$. We show that
				\begin{align*}
					n
						& = \frac{a^{2p}-1}{a^2-1}
				\end{align*}
			is a Fermat pseudoprime to base $a$. First, $n$ is composite because
				\begin{align*}
					n
						& = \frac{a^p-1}{a-1}\cdot \frac{a^p+1}{a+1}
				\end{align*}
			By Fermat's little theorem, $a^{2p} \equiv a^2 \pmod p$ and therefore $p\mid a^{2p}-a^2$. Since $p$ does not divide $a^2-1$, it divides
				\begin{align*}
					n-1
						& = \frac{a^{2p}-a^2}{a^2-1}\\
						& = a^{2p-2}+a^{2p-4}+\cdots+a^4+a^2
				\end{align*}
			which is an even integer. We can now deduce that $2p \mid n-1$ because $p$ is odd. Now, $a^{2p}-1 = n\left(a^2-1\right)$ which means $a^{2p} \equiv 1 \pmod n$. Thus $a^{n-1}\equiv 1 \pmod n$ and $n$ is a Fermat pseudoprime to base $a$.
		\end{proof}
	When you first encountered Fermat's little theorem, you may have wondered if the reverse is true. That is, if $a^{n-1}\equiv 1\pmod n$ for all integers $a$ relatively prime to $n$, then $n$ is prime or not. If you try some examples by hand, you may convince yourself that $n$ must be a prime in order to hold the condition true. Unfortunately, that is not the case. There are infinitely many composite integers $n$ with the given property and the are called \textit{Carmichael numbers}.
	\begin{note}
		Do not be mistaken by this simple statement. It took a long time for number theorists to prove that there indeed exist infinitely many Carmichael numbers.
	\end{note}
	With the above definition of Fermat pseudoprimes, we may provide another definition for Carmichael numbers.

		\begin{definition}[Carmichael Number]
			Let $n$ be a positive integer. If $n$ is a Fermat pseudoprime for all values of $a$ that are relatively prime to $n$, then it is a \textit{Carmichael number} or \textit{Fermat pseudoprime} (and sometimes \textit{absolute Fermat pseudoprime}).
		\end{definition}

	The first few Carmichael numbers are $561, 1105, 1729, \dots$.

The following theorem shows us a way to determine if an integer is a Carmichael number.

	\begin{theorem}[Korselt's Criterion]\slshape
		A positive integer $n$ is a Carmichael number if and only if all of the following conditions meet.
		\begin{enumerate}[i.]
			\item $n$ is composite.
			\item $n$ is squarefree.
			\item For any prime $p|n$, we also have $p-1 \mid n-1$.
		\end{enumerate}
	\end{theorem}

	\begin{proof}
		Let's prove the second proposition first. For the sake of contradiction, let $p$ be a prime factor of $n$ such that $p^2$ divides $n$. Then for all $a$, $p^2 \mid n \mid a^n-a$. Choose $a=p$ and we have $p^2 \mid p^n-p$ or $p^2 \mid p$, which is impossible. So, $n$ is square-free.

		Now we will prove the third one. To prove this, we will use a classical technique. Let $p$ be a prime divisor of $n$. Since $a^n\equiv a\pmod n$, we can say $a^n\equiv a\pmod p$ for all $a$. Choose $a$ so that $a\bot p$. Then $p$ divides $a^n-a=a(a^{n-1}-1)$, thus $p \mid a^{n-1}-1$. Also from Fermat's little theorem, $p \mid a^{p-1}-1$.

		Here is the crucial part. From \autoref{thm:primeprimitive}, we know that \textit{there is a primitive root for all primes} $p$, i.e., there is a positive integer $g$ with $\ord_p(g)=p-1$. For that $g$,
		\begin{align*}
			g^{n-1}
				& \equiv1\pmod p\\
			g^{p-1}
				& \equiv1\pmod p
		\end{align*}
		Since $p-1$ is the order, by \autoref{thm:ordDiv}, $p-1 \mid n-1$ must hold.
	\end{proof}

	\begin{note}
		The connection of Carmichael numbers with Carmichael function is obvious. We could just do it in the following way:

		It is evident that we need $\lambda(n)\mid n-1$. For $n>2$, $\lambda(n)$ is even so $n-1$ is even too. This means $n$ is odd. Next, $\lambda(n)$ is co-prime to $n$, so $n$ is square-free.
	\end{note}

	\begin{definition}[Euler Pseudoprime to Base $a$]\label{def:eulerpseudoprime}
		For an integer $a>1$, if an odd composite integer $n$ which is relatively prime to $a$ satisfies the congruence relation
			\begin{align*}
				\parenthesis{\dfrac{a}{n}}
					& \equiv a^{(n-1)/2} \pmod n
			\end{align*}
		where $\left(\frac{a}{n}\right)$ is the Jacobi symbol, then $n$ is called an \textit{Euler pseudoprime to base $a$} and denoted by $\epsp(a)$.
	\end{definition}

	\begin{corollary}
		Let $a>1$ be an odd integer. Then every Euler pseudoprime to base $a$ is also a Fermat pseudoprime to base $a$.
	\end{corollary}

There are infinitely many $\epsp(a)$ for any integer $a>1$. Actually, even more is true: there exist infinitely many Euler pseudoprimes to base $a$ which are product of $k$ distinct primes and are congruent to $1$ modulo $d$, where $k, d \geq 2$ are arbitrary integers.

You may wonder if there exist \textit{absolute Euler pseudoprimes}, numbers which are Euler pseudoprimes to every base relatively prime to themselves. The answer is negative. In fact, it can be shown that an odd composite integer $n$ can be Euler pseudoprime for at most $\frac{1}{2}\varphi(n)$ bases $a$, where $1<a<n$ and $(a,n)=1$. The proof needs some algebraic background and we do not include it in this book.

	\begin{example}
		$121$ is an $\epsp(3)$. To see why, note that
			\begin{align*}
				\parenthesis{\dfrac{3}{121}}
					& = \parenthesis{\dfrac{3}{11}}^2 = 1
			\end{align*}
		by the definition of Jacobi symbol (Definition \ref{def:jacobi}). Now,
			\begin{align*}
				3^{60}
					& = \parenthesis{3^5}^{12}\\
					& =\left(243\right)^{12}\\
					& \equiv 1^{12} \equiv 1 \pmod{121}
			\end{align*}
	\end{example}

As the last class of pseudoprimes, we mention strong pseudoprimes.

	\begin{definition}[Strong Pseudoprime to Base $a$]
		Let $n=2^sd+1$ where $s$ and $d$ are positive integers and $d$ is odd. Also, let $a>1$ be a positive integer relatively prime to $n$ such that one of the following conditions holds:
			\begin{align*}
				a^d
					& \equiv \phantom{-} 1 \pmod n\\
				a^{2^rd} &\equiv -1 \pmod n
			\end{align*}
		for some integer $0 \leq r <s$. Then $n$ is called a \textit{strong pseudoprime to base $a$} and is denoted by $\spsp(a)$.
	\end{definition}
It can be proved that every $\spsp(a)$ is also a $\epsp(a)$ (and hence a $\psp(a)$). There exist infinitely many strong pseudoprimes to base $a$ for every integer $a \geq 1$. We show a special case of this where $a=2$ in the following proposition.

	\begin{proposition}
		There are infinitely many strong pseudoprimes to base $2$.
	\end{proposition}

	\begin{proof}
		If $n$ is a Fermat pseudoprime to base $2$, then $2^{n-1} \equiv 1 \pmod n$ and so $2^{n-1}-1=nk$ for some integer $k$. Choose $m=2^{n}-1$. We will show that $m$ is a strong pseudoprime to base $2$. To proceed, notice that $m-1=2^n-2=2\left(2^{n-1}-1\right)$ and $2^{n-1}-1$ is an odd integer. So it suffices to show that $2^{2^{n-1}-1} \equiv 1 \pmod m$. Now,
			\begin{align*}
				2^{2^{n-1}-1}
					& = 2^{nk}\\
					& = \left(2^n\right)^k\\
					& \equiv 1^k \equiv 1 \pmod{m}
			\end{align*}
		The proof is complete.
	\end{proof}
\end{document}
\section{Using Congruence in Diophantine Equations}
	\textit{Diophantine equations} are an especial kind of equations which allow solutions only in integers. They have been studied for a really long time. The name is taken after the mathematician \textit{Diophantus of Alexandria}. We have avoided discussing such equations in this book because this area is too huge for us to include right now, and for the same reason we had to ax a lot of topics. However, it is compulsory that we discuss how to use modular arithmetic to solve some particular Diophantine equations. And even if the whole equation can not be solved, we can say a lot about the solutions using modular properties.

\subsection{Some Useful Properties}
	There are some  modular arithmetic properties that usually come handy. But before showing them, we intend to pose a question.
	\begin{question}
		Find two positive integer whose sum of squares is $123$.
	\end{question}
	Since there does not exist many squares below $123$, you may try to do it by hand. And after exhausting all possible cases, you must conclude there are no such integers. But if you are clever, you don't have to go through trial and error. Let's write $a^2+b^2=123$ and notice the following. Exactly one of $a$ or $b$ must be odd since $123$ is odd. Without loss of generality, assume $a$ is even (you can take $b$ if you want). Then $b$ is odd, and we know $b^2\equiv1\pmod4$. Thus, $a^2+b^2\equiv1\pmod4$, whereas $123\equiv3\pmod4$. This is a straight contradiction implying there are no such positive integers $a$ and $b$. The idea seems simple enough, yet powerful to be of great use.

	For reaching such a contradiction (it is often the case, Diophantine equations usually do not have any solutions), we use some common facts. The main idea is the same: find a proper $n$ so that the two sides of the equation leave different remainders modulo $n$.

	You might ask what happens if the equation actually \textit{does} have a solution in integers? Let us explain this with an example. Suppose that you are given the simple linear Diophantine equation $6x+5y = 82$ and you want to solve it over non-negative integers. Let's solve this problem by trial and error. First, notice that $x\leq 13$ (otherwise $6x$ would exceed $82$). We can draw a table to find the solutions.

\begin{table}[h]
\centering
\begin{tabular}{|c|c|c|c|c|c|c|c|}
\hline
$x$ & 0 & 1 & 2 & 3 & 4 & 5 & 6 \\
\hline
$y$ & none & none & 14 & none & none & none & none \\
\hline
$x$ & 7 & 8 & 9 & 10 & 11 & 12 & 13 \\
\hline
$y$ & 8 & none & none & none & none & 2 & none \\
\hline
\end{tabular}
\caption{Solving $6x+5y=82$ by trial and error.}
\label{table:diophantine}
\end{table}

	As seen in Table \ref{table:diophantine}, we need to do $13$ calculations to find the solutions $$(x, y)=(2,14), (7,8), (12, 2)$$

	Now, consider the same linear equation $6x+5y = 82$ again. We are going to solve it using modular arithmetic this time. Take modulo $5$ from both sides of the equation. The left side would be $x$ while the right side is $2$, giving us the relation $x \equiv 2 \pmod 5$. Although this does not give us the solution directly, it helps us find the solutions much faster. Just notice that we already know $x$ must be less than or equal to $13$, and it must have a remainder of $2$ when divided by $5$. The only choices for $x$ then are $2, 7$, and $12$. We can now plug these values of $x$ into the equation and find the solutions with only three calculations (instead of thirteen).


	Sometimes we need to use some theorems such as Fermat's little theorem or Wilson's theorem and pair them up with some modular arithmetic. Here are some highly useful congruences:
		\begin{theorem}\label{thm:diophModulo}
			Let $x$ be an integer (not necessarily positive). Then
			\begin{align*}
			x^2 & \equiv 0,1 \pmod3\\
			x^2 & \equiv 0,1 \pmod4\\
			x^2 & \equiv 0,1,4\pmod8\\
			x^2 & \equiv 0,1,4,9 \pmod{16}\\
			x^3 & \equiv 0, \pm1 \pmod7\\
			x^3 & \equiv 0, \pm1 \pmod9\\
			x^4 & \equiv 0,1 \pmod{16}\\
			x^4 & \equiv 0, \pm 1, \pm 4 \pmod{17}\\
			x^5 & \equiv 0, \pm1 \pmod{11}\\
			x^6 & \equiv 0,1,4 \pmod{13}
			\end{align*}
		\end{theorem}
	Most of congruences above can be proved easily. Some are direct consequence of Fermat's or Euler's theorem. Or you can just consider the complete set of residue of the modulus and then investigate their powers. Whatever the case, we will leave the proofs as exercises. Sometimes you may notice that Fermat's little theorem or Euler's theorem is disguised in the equation.

		\begin{problem}
			The sum of two squares is divisible by $3$. Prove that both of them are divisible by $3$.
		\end{problem}

		\begin{solution}
			Assume that $a^2+b^2$ is divisible by $3$. If $a$ is divisible by $3$, so must be $b$. So, take $a$ not divisible by $3$. Then, from the properties above, we have $a^2\equiv1\pmod3$ and $b^2\equiv1\pmod3$. And this immediately gives us a contradiction that $a^2+b^2\equiv1+1\equiv2\pmod3$.
		\end{solution}

		\begin{remark}
			We could just use \autoref{thm:a^2+b^2} which shows that every prime factor of $a^2+b^2$ is of the form $4k+1$ if $a$ and $b$ are coprime.
		\end{remark}

		\begin{problem}
			Show that there are no integers $a,b,c$ for which $a^2+b^2-8c=6$.
		\end{problem}

		\begin{solution}
			The term $-8c$ guides us to choose the right modulo. Consider the equation modulo $8$. We have $a^2+b^2\equiv 6\pmod{8}$. By \autoref{thm:diophModulo}, $a^2\equiv 0, 1$, or  $4\pmod{8}$. Now you may check the possible combinations to see that $a^2+b^2\equiv 6\pmod{8}$ is impossible.
		\end{solution}

		\begin{problem}
			Solve the Diophantine equation $x^4-6x^2+1=7 \cdot 2^y$ in integers.
		\end{problem}

		\begin{solution}
			There are no solutions for $y<0$. So assume $y\geq 0$. Add $8$ to both sides of the equation to get
				\begin{align*}
			 (x^2-3)^2=7\cdot 2^y+8
				\end{align*}
			Note that if $y \geq 3$, the right hand side of above equation is divisible by $8$. So taking modulo $8$ may seem reasonable. However, it leads to $(x^2-3)^2 \equiv 0 \pmod 8$ and no further results are included. We should look for another modulo. If $y \geq 4$, then the right hand side is congruent to $8$ modulo $16$. However, the left hand side, $(x^2-3)^2$ is a square and so it's $0, 1, 4$, or $9$ modulo $16$. The only left cases are $y=0,1,2$, and $3$ which imply no solutions. Hence, no solutions at all.
		\end{solution}


	Let's see another problem in which we will also see an application of \textit{Fermat's method of infinite descent}. This is a technique for solving Diophantine equations but we briefly use the idea here.
		\begin{problem}
			Find all integer solutions to the equation
				\begin{align*}
					x^2+y^2 & = 7(z^2+t^2)
				\end{align*}
		\end{problem}

		\begin{solution}
			First of all, using the same approach as in previous problem, we can prove that $7$ divides both $x$ and $y$. Let $x=7a$ and $y=7b$ and substitute them in the equation. After dividing by $7$,
				\begin{align*}
					z^2+t^2 & = 7(a^2+b^2)
				\end{align*}
			Note that this equation looks like the original one. However, $z$ and $t$ in the latter equation are strictly smaller than $x$ and $y$ in the original equation. We can continue this process by noting the fact that $z$ and $t$ are divisible by $7$. So, assume that $z=7u$ and $t=7v$ and rewrite the equation as
				\begin{align*}
				u^2+v^2 & = 7(a^2+b^2)
				\end{align*}
			This process can be done infinitely many times. Thus, we get that $x$ and $y$ are divisible by $7^i$ for all positive integers $i$, which is not possible. So, the equation does not have any solutions. The process of finding new equations similar to the original one is called the method of \textit{infinite descent}.
		\end{solution}


		\begin{problem}
			Show that the following equation does not have any solutions in positive integers:
				\begin{align*}
					5^n-7^m = 1374
				\end{align*}
		\end{problem}

		\begin{solution}
			The most important thing in solving a Diophantine equation is to take the right modulo. In this case, it's obvious that the easiest mods to take are $5$ and $7$. Let's take modulo $5$ from both sides of the equation. Since $7^m \equiv 2^m \pmod 5$,
				\begin{align*}
					-2^m
						& \equiv -1 \pmod 5\\
					\implies 2^m
						& \equiv 1 \pmod 5
				\end{align*}
			Since $\ord_5(2)=4$, we have $4 \mid m$. Let $m=4k$ for some integer $k$. So $7^m = 7^{4k}=\left(7^k\right)^4$. This reminds us of the fact that $x^2$ (and thus $x^4$) is either $0$ or $1$ modulo $4$. So, $7^m \equiv 1 \pmod 4$. Taking modulo $4$ from the original equation, we get
				\begin{align*}
					5^n - 7^{4k}
						& \equiv 2 \pmod 4\\
					\implies 1-1
						& \equiv 2 \pmod 4
				\end{align*}
			which is a contradiction. Thus, there are no solutions.
		\end{solution}

		\begin{problem}[Kazakhstan 2016]
			Solve in positive integers the equation
			\[n!+10^{2014}=m^4\]
		\end{problem}

		\begin{solution}
			You can usually use modular arithmetic to solve the problem when there is a factorial term in the given equation. The interesting property of $n!$ is that it is divisible by all integers less than or equal to $n$. In this problem, if we find the right modulo $k$, we can assume $n\geq k$ and take modulo $k$ from the equation (we will check the cases when $n < k$ later). It will be $10^{2014} \equiv m^4 \pmod k$. As said before, we guess the equation does not have any solutions. So, we are searching for a modulo $k$ for which $m^4$ cannot be congruent to $10^{2014}$. We should first try the simplest values for $k$, i.e., values of $k$ for which $m^4$ can have a few values. For $k=16$, we have $m^4 \equiv 0 \pmod{16}$, no contradiction. For $k=17$, we have $m^4 \equiv 8 \pmod{17}$, which is impossible because $m^4$ can only have the values $0, \pm 1$, or $\pm 4$ modulo $17$. We have found our desired contradiction, and we just have to check the values of $n < 17$. This is easy. Obviously, $n! + 10^{2014}$ is bigger than $10^{2014}$. However, the smallest perfect square bigger than $10^{2014}$ is
				\begin{align*}
					\left(10^{1007}+1\right)^2 = 10^{2014} + 2 \cdot 10^{1007} + 1
				\end{align*}
			which is way bigger than $10^{2014} + 16!$. So, no solutions in this case as well.
		\end{solution}

		\begin{problem}
			Prove that the equation $x^2+5=y^3$ has no integer solutions.
		\end{problem}

		\begin{solution}
			Taking modulo $4$, since $x^2+5$ is congruent to either $1$ or $2$ modulo $5$, but $y^3$ is never congruent to $2$ modulo $4$, we have $x^2 + 5 \equiv y^3 \equiv 1 \pmod 4$, and so $x$ is even, $y \equiv 1 \pmod 4$. Rewrite the equation as
				\begin{align*}
					x^2+4 = (y-1)(y^2+y+1)
				\end{align*}
			Note that since $y \equiv 1 \pmod 4$, we have $y^2+y+1 \equiv 3 \pmod 4$. According to theorem \eqref{thm:4k+3prime}, we know that every number congruent to $3$ modulo $4$ has a prime divisor also congruent to $3$ modulo $4$. Let $p \equiv 3 \pmod 4$ be that prime divisor of $y^2+y+1$. Then
				\begin{align*}
					x^2 + 4 \equiv 0 \pmod p
				\end{align*}
			If we raise both sides of the congruence equation $x^2 \equiv -4 \pmod p$ to the power of $\frac{p-1}{2}$ (which is an odd integer since $p \equiv 3 \pmod 4$), we have
				\begin{align*}
					\left(x^2\right)^{\frac{p-1}{2}}
						& \equiv -\left(4\right)^{\frac{p-1}{2}} \pmod p
				\end{align*}
			or, by Fermat's little theorem,
				\begin{align*}
					1
						& \equiv x^{p-1}\\
						& \equiv -4^{p-1}\\
						& \equiv -1 \pmod p
				\end{align*}
			This is the contradiction we were looking for and the equation does not have integer solutions.
		\end{solution}

		\begin{note}
			The idea of taking a prime $p \equiv 3 \pmod 4$ of a number $n \equiv 3 \pmod 4$ comes handy in solving Diophantine equations pretty a lot. Keep it in mind.
		\end{note}

		\begin{problem}[Romania JBMO TST 2015]
			Solve in nonnegative integers the equation
			\[21^x+4^y=z^2\]
		\end{problem}

		\begin{solution}
			First, let us consider the case $x=0$. Then $z^2-1 =(z-1)(z+1)=4^y$. If $z-1=1$ and $z+1=4^y$, we have no solutions. Otherwise, both $z-1$ and $z+1$ should be perfect powers of $4$, which is impossible.

			We can show, in a similar way, that the case $y=0$ gives no solutions as well. So, suppose that $x$ and $y$ are positive integers.

			Rewrite the original equation as
				\begin{align*}
					3^x \cdot 7^x = (z-2^y)(z+2^y)
				\end{align*}
			There are a few cases to check:
				\begin{itemize}
					\item If $z-2^y=1$, then $z+2^y = z-2^y+2^{y+1}=1+2^{y+1}=21^x$. This implies $2^{y+1}=21^x-1$. But the right hand side of the latter equation is divisible by $20$, contradiction. So no solutions in this case.

					\item If both $z-2^y$ and $z+2^y$ are divisible by $21$, then $21|(z-2^y, z+2^y)$. This is impossible because if $d=(z-2^y, z+2^y)$, then $d\mid (z+2^y) -(z-2^y)=2^{y+1}$, which means $d$ is a power of $2$.

					\item If $z-2^y=3^x$ and $z+2^y=7^x$, then
						\begin{align}\label{eq:romaniajbmotst2015}
							7^x-3^x=2^{y+1}
						\end{align}
			 $y=1$ gives the solution $(x,y,z)=(1,1,5)$. Assume $y \geq 2$. Take modulo $8$ from equation \eqref{eq:romaniajbmotst2015}. $2^{y+1}$ is divisible by $8$ and so $7^x-3^x \equiv 0 \pmod 8$. But this does not happen for any $x$ (just consider two cases when $x$ is even or odd). So the only solution in this case is $(x,y,z)=(1,1,5)$.
				\end{itemize}
			Note that we have used the fact that $z+2^y > z-2^y$ to omit some cases (like when $z-2^y=21^x$ and $z+2^y = 1$). So, $(x,y,z)=(1,1,5)$ is the only solution to the given equation.
		\end{solution}


	\newpage
	\section{Exercises}

\begin{problem} %https://artofproblemsolving.com/community/c6h456895
	Consider the following progression:
		\begin{align*}
			u_0 &= \dfrac{1}{2}\\
			u_{n+1} &= \dfrac{u_n}{3-2u_n}
		\end{align*}
	for $n\in\mathbb{N}$. Let $a$ be a real number. We define the series $\{w_n\}$ as
		\begin{align*}
			w_n &= \dfrac{u_n}{u_n + a}
		\end{align*}
	Find all values of $a$ such that $w_n$ is a geometric progression.
\end{problem}

\begin{problem} %https://artofproblemsolving.com/community/c6h1350191
	Let $p$ be an odd prime number and consider the following sequence of integers: $a_1$, $a_2\ldots$ $a_{p-1}$, $a_p$. Prove that this sequence is an arithmetic progression if and only if there exists a partition of the set of natural numbers $\mathbb{N}$ into $p$ disjoint sets $A_1$, $A_2\ldots$, $A_{p-1}$, $A_p$ such that the sets $\left\{ a_i+n\mid n\in A_i\right\}$ (for $i=1, 2,\cdots, p$) are identical.
\end{problem}

\begin{problem} %https://artofproblemsolving.com/community/c6h1184607
	Assume we have 15 prime numbers which are elements of some arithmetic sequence with common difference $d$. Prove that $d>30000$.
\end{problem}

\begin{problem}[Vietnam Pre-Olympiad 2012] %https://artofproblemsolving.com/community/c6h448257
	Determine all values of $n$ for which there exists a permutation $(a_1,a_2,a_3,\cdots,a_n)$ of $(1,2,3,\cdots,n)$ such that $$\left\{ {{a_1},{a_1}{a_2},{a_1}{a_2}{a_3},\cdots,{a_1}{a_2}\cdots{a_n}} \right\}$$ is a complete residue system modulo $n$.
\end{problem}

\begin{problem} %https://artofproblemsolving.com/community/c6h1310199
	Prove that for any two positive integers $m$ and $n$, there exists a positive integer $x$, such that
		\begin{align*}
			2^x
				& \equiv 1999 \pmod{3^m}\\
			2^x
				& \equiv 2009\pmod{5^n}
		\end{align*}
\end{problem}

\begin{problem} %https://artofproblemsolving.com/community/c4h57345
	Let $f(x) = 5 x^{13} + 13 x^5 + 9ax$. Find the least positive integer $a$ such that $65$ divides $f(x)$ for every integer $x$.
\end{problem}

\begin{problem}[Romanian Mathematical Olympiad 1994] %https://artofproblemsolving.com/community/c6h56051
	Find the remainder when $2^{1990}$ is divided by $1990$.
\end{problem}

\begin{problem} %https://artofproblemsolving.com/community/c6h1401608
	Prove that $a^{2}+b^{5}=2015^{17}$ has no solutions in $\mathbb{Z}$.
\end{problem}

\begin{problem}[Middle European Mathematical Olympiad 2009] %https://artofproblemsolving.com/community/c6h303625
	Determine all integers $ k\ge 2$ such that for all pairs $ (m$, $ n)$ of different positive integers not greater than $ k$, the number $ n^{n-1}-m^{m-1}$ is not divisible by $ k$.
\end{problem}

\begin{problem}[ELMO 2000] %https://artofproblemsolving.com/community/c6h1344240
	Let $a$ be a positive integer and let $p$ be a prime. Prove that there exists an integer $m$ such that \[ m^{m^m} \equiv a \pmod p\]
\end{problem}

\begin{problem} %https://artofproblemsolving.com/community/c6h1450695
	Find all pairs of prime numbers $(p,q)$ for which
	\begin{align*}
	7pq^2 + p = q^3 + 43p^3 + 1
	\end{align*}
\end{problem}

\begin{problem} [IMO 1996] %https://artofproblemsolving.com/community/c6h60430p365167
	The positive integers $ a$ and $ b$ are such that the numbers $ 15a + 16b$ and $ 16a - 15b$ are both squares of positive integers. What is the least possible value that can be taken on by the smaller of these two squares?
\end{problem}

\begin{problem} %https://artofproblemsolving.com/community/c6h1370199
	2017 prime numbers $p_1,\ldots,p_{2017}$ are given. Prove that $$\prod_{i<j} (p_i^{p_j}-p_j^{p_i})$$ is divisible by $5777$.
\end{problem}

\begin{problem}[Ukraine 2014] %https://artofproblemsolving.com/community/c6h610667p36302840
	Find all pairs of prime numbers $(p,q)$ that satisfy the equation $$3p^{q}-2q^{p-1}=19$$
\end{problem}

\begin{problem} %https://artofproblemsolving.com/community/c6h1191622
	Let $p$ be an odd prime and let $\omega$ be the $p^{th}$ root of unity (that is, $\omega$ is some complex number such that $\omega^p = 1$). Let
		\begin{align*}
			X
				& =\sum \omega^i\\
			Y
				& =\sum \omega^j
		\end{align*}
	where $i$ in the first sum runs through quadratic residues and $j$ in the second sum runs over quadratic non-residues modulo $p$ and $0<i,j<p$. Prove that $XY$ is an integer.
\end{problem}

\begin{problem} %https://artofproblemsolving.com/community/c6h1194731
	Let $p>2$ be a prime number. Prove that in the set
		\begin{align*}
			1,2, \cdots ,\floor{\sqrt{p}}+1
		\end{align*}
	there exists an element which is not a quadratic residue mod $p$.
\end{problem}

%	\begin{solution}
%		Let $b$ be the smallest quadratic nonresidue modulo $p$. Assume that $p-1 \ge b \ge \left \lfloor \sqrt{p} \right \rfloor+2$. There exists $1 \le r \le b-1$ such that $b \mid p+r$. Hence, let $p+r=ab \; (1 \le a \le p-1)$. If $a \ge \left \lfloor \sqrt{p} \right \rfloor+2$ then $b(a-1) >p$, a contradiction since $p=ab-r>b(a-1)$. Thus, $a \le \left \lfloor \sqrt{p} \right \rfloor+1$. This follows that $\left( \frac ap \right)=1$. Hence, $$\left( \frac{p+r}{p} \right) = \left( \frac{ab}{p} \right) = -1.$$
%		Note that since $1 \le r \le b-1$ so $\left( \frac{r}{p} \right)= \left( \frac{r+p}{p} \right)=1$, a contradiction.
%		Thus, $b \le \left \lfloor \sqrt{p} \right \rfloor+1$, or we can say that in the set $\{ 1,2, \cdots , \left \lfloor \sqrt{p} \right \rfloor+1 \}$ there exists an element which is quadratic nonresidue modulo $p$.
%	\end{solution}

\begin{problem}[APMO 2014] %https://artofproblemsolving.com/community/c6h582821
	Find all positive integers $n$ such that for any integer $k$ there exists an integer $a$ for which $a^3+a-k$ is divisible by $n$.
\end{problem}

\begin{problem}
	Let $b,n > 1$ be integers. Suppose that for each $k > 1$ there exists an integer $a_k$ such that $b - a^n_k$ is divisible by $k$. Prove that $b = A^n$ for some integer $A$.
\end{problem}

%	\begin{solution}
%		\textit{First Solution.} Assume that $ b$ has a prime factor, $ p$, so that $ p^{xn + r}||b$ with $ 0 < r < n$. Then, we can let $ k = p^{xn + n}$. It follows that $ b\equiv a_k^n\bmod p^{xn + n}$. Since $ p^{xn + r}\|b$, we see that $ p^{xn + r}||a_k^n$. Then, $ p^{x + \frac {r}{n}}\|a_k^n$, which is a contradiction since $ \frac {r}{n}$ is not an integer. Thus, we have a contradiction, so $ r = 0$, which means that only $ n$th powers of primes fully divide $ b$, so $ b$ is an $ n$th power.\\
%
%		\textit{Second Solution.} Assume that there is no $A$ such that $b=A^n$. Then there must exist a prime number $p$ such that, if $p^a \| b$, then $n \not | a$. Assume now that $mn < a < (m+1)n$ for some $m \in \mathbb{N}$. Taking $k=p^{(m+1)n}$ yields that
%		\[b \equiv a_{k}^n \pmod{p^{(m+1)n}}\]
%		Which implies that $p^{a} |a_{k}^n$ and, since $a_{k}^n$ is a perfect $n$th power, that $p^{(m+1)n} | a_{k}^n$. Hence $b \equiv 0 \pmod{p^{(m+1)n}}$ and $n|a$ which is a contradiction.
%	\end{solution}

\begin{problem}
	$16$ is an eighth power modulo every prime.
\end{problem}

\begin{problem}
	Let $m$ and $n$ be integers greater than $1$ with $n$ odd. Suppose that $n$ is a quadratic residue modulo $p$ for any sufficiently large prime number $p \equiv -1 \pmod{2^m}$. Prove that $n$ is a perfect square.
\end{problem}

%	\begin{solution}
%		Let $n = p_1^{a_1}p_2^{a_2}...p_k^{a_k}$ and assume $n$ is not a perfect square. WLOG let $a_1$ be odd. Use Dirichlet to find a prime $p$ such that $p \equiv -1 \pmod {2^m}$ and $\left(\frac{p_i}{p}\right) = 1$ for $i > 1$ but $\left(\frac{p_1}{p}\right) = -1$ as for any prime $p > 2$ there exist $b,c$ such that if $p \equiv b \pmod q$ then $p$ is not a quadratic residue and if $p \equiv c \pmod q$ then $p$ is a quadratic residue for prime $q$. Then it follows that $n$ is a not a quadratic residue $\pmod p$ which is a contradiction.
%	\end{solution}


\begin{problem}
	Form the infinite graph $A$ by taking the set of primes $p$ congruent to $1\pmod{4}$, and connecting $p$ and $q$ if they are quadratic residues modulo each other. Do the same for a graph $B$ with the primes $1\pmod{8}$. Show $A$ and $B$ are isomorphic to each other.
\end{problem}

%	\begin{solution}
%		We will use the following lemma which is an easy corollary of the chinese remainder theorem and Dirichlet's theorem on primes in arithmetic progressions:
%		Lemma: Let $m_1,...,m_r,M$ be pairwise relatively prime integers, let $\varphi_1,...,\varphi_r \in \{ \pm 1 \}$ and let $a$ be an integer relatively prime to $M$. Then there exists a prime $p$ such that $\left( \frac{p}{m_i} \right)=\varphi_i$ for all $i$ and auch that $p \equiv a \mod M$.\\
%		We will now construct two sequences $a_n$ and $b_n$ such that $a_n$ contains every prime $\equiv 1 \mod 4$ exactly once and $b_n$ does the same for the primes $\equiv 1 \mod 8$ in such a way that mapping $a_i$ to $b_i$ induces an isomorphism of the graphs induced by being quadratic residues:\\
%		Assume we have found $a_1,...,a_n$ and $b_1,...,b_n$ such that mapping $a_i$ to $b_i$ induces an isomorphism of these finite graphs of $n$ vertices. Then continue in the following way:\\
%		-- If $n$ is even, take $a_{n+1}$ to be the smallest prime $\equiv 1 \mod 4$ not in the set $\{a_1,...,a_n\}$. Then the lemma allows us to find a prime $b_{n+1} \equiv 1 \mod 8$ such that $\left( \frac{b_{n+1}}{b_i} \right) = \left( \frac{a_{n+1}}{a_i} \right) $ for all $i$ (note that especially $b_{n+1} \neq b_i$). By construction, the graphs therefore remain isomorphic after adding these new vertices.\\
%		-- If $n$ is odd, do something similar, but this time choose the smallest prime $ b_{n+1} \equiv 1 \mod 8$ not contained in $\{b_1,...,b_n\}$ and then construct $a_{n+1}$ accordingly.\\
%		Now its easy to see that we finished our task as we can conclude that:\\
%		a) $a_n$ contains only primes $\equiv 1 \mod 4$ and $b_n$ contains only primes $\equiv 1 \mod 8$\\
%		b) each of $a_n$ and $b_n$ contains every prime at most once\\
%		c) The $n$-th prime $p \equiv 1 \mod 4$ (ordered by size) is contained in $\{a_1,...,a_{2n}\}$ and the $n$-th prime $p \equiv 1 \mod 8$ is contained in $\{b_1,...,b_{2n}\}$\\
%		d) For every $n$ the graphs on $\{a_1,...,a_n\}$ and $\{b_1,...,b_n\}$ are isomorphic with the isomorphism given by sending $a_i$ to $b_i$.\\
%		a) just says we didn't get any 'wrong' primes, b) and c) imply that every prime occurs exactly once, and d) implies that sending $a_i$ to $b_i$ induces an isomorphism between the graphs on $\{a_1,a_2,...\}$ and $\{b_1,b_2,...\}$, solving the problem.
%	\end{solution}

\begin{problem}
	Find all positive integers $n$ that are quadratic residues modulo all primes greater than $n$.
\end{problem}

%	\begin{solution}
%		We can avoid Dirichlet (the only analytical step) by using quadratic reciprocity for Jacobi symbols (the rest of the proof being the same).\\
%		Here an adapted copy of more or less the same from a post of mine on MathLinks:\\
%		Assume that $n$ isn't a square, but a square $\mod $ all primes $>n$.\\
%		Let $n=2^s b = 2^s p_1^{v_1} p_2^{v_k} ... p_k^{v_k}$ be the prime factorisation and $b$ it's greatest odd factor. Since $n$ is not a square, either (w.l.o.g.) $v_1$ is odd or $s$ is odd.\\
%		1. case: all $v_i$ are even, so $s$ is odd.
%		Then simply taking an integer $q$ (relatively prime to all primes $\leq n$) not dividing $n$ and $\equiv 3\mod 8$ gives $\left( \frac{n}{q} \right) = \left( \frac{2^s}{q} \right) \left( \frac{b}{q} \right) = (-1)^s = -1$.\\
%		2. case: $v_1$ is odd.
%		Now let $c$ be a quadratic non-residue $\mod p_1$ and take an integer $q$ (relatively prime to all primes $\leq n$) such that $q \equiv 1 \mod 8$, $q \equiv c \mod p_1$ and $q \equiv 1 \mod p_i$ for all other $p_i$.
%		Then we get:\\
%		$\left( \frac{n}{q} \right) = \left( \frac{2}{q} \right)^s \left( \frac{p_1}{q} \right)^{v_1} \left( \frac{p_2}{q} \right)^{v_2} ... \left( \frac{p_k}{q} \right)^{v_n} = \left( \frac{q}{p_1} \right)^{v_1} \left( \frac{q}{p_2} \right)^{v_2} ... \left( \frac{q}{p_k} \right)^{v_k}=(-1)^{v_1}=-1$.\\
%		In both cases, $n$ is not a quadratic residue $\mod q$, thus not for at least one of it's prime divisors. This contradicts the assumption of $n$ being a square $\mod$ all primes $>n$.
%	\end{solution}

\begin{problem}
	Let $k$ be an even positive integer and $k\ge 3$. Define $$n=\frac{2^k-1}{3}$$ Find all $k$ such that $(-1)$ is a quadratic residue modulo $n$.
\end{problem}

%	\begin{solution}
%		Let $k=2^lm,l\ge 1$ were $m-$ odd. Then \[\frac{2^k-1}{3}=\frac{2^m+1}{3}(2^m-1)(2^{2m}+1)...(2^{m*2^{l-1}}+1).\]
%		If $m>1$, then $(-1)$ is not quadratic residue mod $2^m-1$ and by mod $n$.\\
%		Therefore $(-1)$ is quadratic residue if and only if $k=2^l, l\ge 2$.
%	\end{solution}

\begin{problem}
	Let $n$ and $k$ be given positive integers. Then prove that
	\begin{itemize}
		\item there are infinitely many prime numbers $p$ such that $\pm 1, \pm 2, \pm 3, \ldots , \pm n$ are quadratic residue of $p$, and
		\item there infinitely many prime numbers $ p > n $ such that $ \pm i/j$ are $k^{th}$-power residue of $p$, where $i$ and $j$ are integers between $1$ and $n$ (inclusive).
	\end{itemize}
\end{problem}

%	\begin{solution}
%		Let $2, p_1,p_2,\cdots,p_l$ be those prime numbers $\le n$. If $-1,2,p_1,\cdots, p_l$ are all quadratic residues, so are $\pm 1,\pm 2,\cdots,\pm n$. We want $\bigg(\frac {-1}{p}\bigg)=1$, $\bigg(\frac {2}{p}\bigg)=1$, so we set $p=1\pmod{8}$. By reciprocity, $\bigg(\frac {p_i}{p}\bigg)=(-1)^{\frac{(p-1)(p_i-1)}{4}}\bigg(\frac {p}{p_i}\bigg)=\bigg(\frac {p}{p_i}\bigg)$. We set $p=1\pmod{p_i}$. Now the question is, are there infinitely many $p$ such that
%		\[p=1\pmod{8}, \qquad p=1\pmod{p_i}, \ \ i=1,2,\cdots,l?\]
%		We need a simple lemma: suppose that $(a_n)$, $(b_n)$ are two arithmetic progressions of positive integers, their common differences $d_1$, $d_2$ being relatively prime. Then their intersection (terms appearing in both progressions) is an arithmetic progression with common difference $d_1d_2$.\\
%		According to this lemma, the intersect of arithmetic progressions $(8n+1),(p_1n+1),\cdots,(p_ln+1)$ is again an arithmetic progression. So it does contain infinitely many primes $p$ by Dirichlet.
%	\end{solution}

\begin{problem}
	Let $p>5$ be a prime number and $$A=\{b_1,b_2,\cdots,b_{\frac{p-1}{2}}\}$$ be the set of all quadratic residues modulo $p$, excluding zero. Prove that there doesn't exist positive integers $a$ and $c$ satisfying $(ac,p)=1$ such that set $$B=\{ab_1+c,ab_2+c,\cdots,ab_{\frac{p-1}{2}}+c\}$$ and $A$ are disjoint modulo $p$.
\end{problem}

%	\begin{solution}
%		We work in $\mathbb{Z}/p\mathbb{Z}.$ First suppose $\left(\tfrac{a}{p}\right) = 1$ so that $\left(\tfrac{ab_i}{p}\right) = \left(\tfrac{a}{p}\right)\left(\tfrac{b_i}{p}\right) = 1.$ It follows that $S := \left\{ab_1, ab_2, \cdots , ab_{\frac{p - 1}{2}}\right\}$ is the set of all quadratic residues.\\
%		Now note that the equation $x^2 = \pm c$ has at most four solutions. Then since $|\mathbb{Z}/p\mathbb{Z}| > 5$, we may choose a nonzero element $r$ such that $r^2 \ne \pm c.$ Let $s = \tfrac{c}{r}$ and observe that $\left(\tfrac{r + s}{2}\right)^2 - \left(\tfrac{r - s}{2}\right)^2 = c.$ In particular, $r \ne \pm s$ so $\left(\tfrac{r + s}{2}\right)^2$ and $\left(\tfrac{r - s}{2}\right)^2$ are both quadratic residues. Thus if we set $\left(\tfrac{r - s}{2}\right)^2 = ab_k$ for some $k \in \left\{1, 2, \cdots , \tfrac{p - 1}{2}\right\}$, the element $ab_k + c$ is common to both $A$ and $B.$\\
%		Next, suppose that $\left(\tfrac{a}{p}\right) = -1$ so that $S$ is the set of all quadratic nonresidues modulo $p.$ Consider the arithmetic progression $0, c, 2c, \cdots , (p - 1)c.$\\
%		Notice that if $\left(\tfrac{kc}{p}\right) = -1$ for $k = 1, 2, \cdots , p - 2$, then $\left(\tfrac{(k + 1)c}{p}\right) = -1$ as well, for otherwise $kc \in S$ and therefore $(k + 1)c \ne 0$ would be an element common to $A$ and $B.$ In particular, if we consider some $k \in \{1, 2, \cdots , p - 2\}$ satisfying, $\left(\tfrac{kc}{p}\right) = -1$ then we may inductively obtain $\left(\tfrac{nc}{p}\right) = -1$ for all $k \le n \le p - 1$, implying that there are at least $p - k$ quadratic nonresidues. Since there are precisely $\tfrac{p - 1}{2}$ quadratic nonresidues, we must have $k \ge \tfrac{p + 1}{2}.$ It follows that $c, 2c, \cdots , \left(\tfrac{p - 1}{2}\right)c$ are all quadratic residues. By the multiplicative nature of the Legendre symbol, we see that $T:= \left\{1, 2, \cdots , \tfrac{p - 1}{2}\right\}$ must either be the set of all quadratic residues or all quadratic nonresidues, according to the value of $\left(\tfrac{c}{p}\right).$ However, note that $1$ is obviously a quadratic residue, implying $T$ is the set of all the quadratic residues. But then since the Legendre symbol is multiplicative, it follows that $2 \cdot \tfrac{p - 1}{2} = p - 1 \not\in T$ is a quadratic residue as well, contradiction.
%	\end{solution}

\begin{problem}
	Let $p$ be a prime number such that $p=a^2+5b^2$, where $a$ and $b$ are positive integers and $a$ is odd. Prove that $a$ is a quadratic residue modulo $p$ if and only if $p \equiv 1 \pmod 5$.
\end{problem}

%	\begin{solution}
%		Because $ a$ is odd, by properties of Jacobi Symbol (generalization of Legendre's symbol), we get:
%		$ \left( \frac{a}{p} \right) \left( \frac{p}{a} \right)=(-1)^{\frac{(a-1)(p-1)}{4}}$
%		But $ a$ is odd and $ p \equiv a^2 \equiv 1 \pmod 4$ (because $ b$ must be even), so $ (-1)^{\frac{(a-1)(p-1)}{4}}=1$.
%		Now we know that $ \left( \frac{a}{p} \right)=\left( \frac{p}{a} \right)=\left( \frac{5b^2}{a} \right)=\left( \frac{5}{a} \right)=\left( \frac{a}{5} \right)$.
%		But $ \left( \frac{a}{5} \right)=1 \leftrightarrow a^{\frac{5-1}{2}} \equiv 1 \pmod 5 \leftrightarrow a^2 \equiv 1 \pmod 5 \leftrightarrow p \equiv 1 \pmod 5$, and the result follows.
%	\end{solution}

\begin{problem} %http://www.artofproblemsolving.com/community/q1h1124685p5192082
	Find all primes $p$ such that $5,7$, and $-7$ are quadratic residues modulo $p$.
\end{problem}

%	\begin{solution}
%		If $p = 2$ it works. If $p = 5$ it fails. If $p = 7$ it fails.\\
%		Now $p$ is odd and not $5$ or $7$. Then $\frac{-7}{7} = -1$ is a square, so $p \equiv 1 \pmod{4}$. Also, $1 = \left(\frac{5}{p}\right) = \left(\frac{p}{5}\right)$ and $1 = \left(\frac{-7}{p}\right) = \left(\frac{p}{7}\right)$ in the Legendre symbol. These conditions are sufficient, so $p \equiv \pm 1 \pmod{5}, p \equiv 1, 2, 4 \pmod{7}$.\\
%		This gives $p \equiv 1, 4, 9, 11, 16, 29 \pmod{35}$ and $p \equiv 1 \pmod{4}$ so $p \equiv 1, 9, 29, 81, 109, 121 \pmod{140}$ or $p = 2$.
%	\end{solution}


\begin{problem}
	Prove that there are no positive integers $k$ such that for the first $k$ odd prime numbers $p_1,p_2,\cdots,p_k$, there are $a,n\in\mathbb{Z}^+$ $(n>1)$ satisfying \[ p_1p_2\cdots p_k=a^n+1\] %http://www.artofproblemsolving.com/community/q2h1173055p5644203
\end{problem}

%	\begin{solution}
%		First it is easy to observe that $n$ must be odd by $\pmod 3$. Now lets consider a prime $p$ dividing $n$. If $p$ is one of $p_1,p_2,...,p_k$ then letting $n= mp$ one gets
%		$$a^{mp} \equiv a^m \equiv -1 \pmod p$$
%		Now by LTE on $(a^m)^p + 1$ we get that $p^2$ must divide it which gives u a contradiction as $p^2$ does not divide $LHS$. Hence $n$ does not have any of the first $k$ primes as factors. Now let $d_i$ be the order of $a \pmod {p_i}$. We then get $d_i | p_i -1 , d_i | 2n$ which gives us $d_i = 2$ as $n$ does not have any prime factors $\leq p_k$. This leads us to conclude that $a \equiv -1 \pmod {p_i}$ for all $p_i$ as $a-1$ cannot be divisible by $p_i$. Hence $p_1p_2...p_k | a + 1$ and $n$ must be equal to $1$.
%	\end{solution}

\begin{problem}
	Find all the pairs of positive integers $(x,p)$ such that $p$ is a prime, $x \leq 2p$, and $x^{p-1}$ is a divisor of $(p-1)^{x}+1$. %http://artofproblemsolving.com/community/c6h19762p131811
\end{problem}

%	\begin{solution}
%		$x^{p-1}\mid\left(p-1\right)^{x}+1\mid\left(p-1\right)^{2x}-1$. Thus if $q$ is the smallest a prime divisor of $x$ (the case $x=1$ has already been dealt with) then we have that $q\mid\left(p-1\right)^{x}+1\Rightarrow q\nmid\left(p-1\right)^{x}-1$ ($x$ is odd means $q\neq2$) and $q\mid\left(p-1\right)^{2x}-1$. Thus $\text{ord}_{q}\left(p-1\right)\nmid x$ and $\text{ord}_{q}\left(p-1\right)\mid2x$ hence, as $x$ is odd, $\text{ord}_{q}\left(p-1\right)$ is twice a factor of $x$ so $\frac{\text{ord}_{q}\left(p-1\right)}{2}$ is a factor of $x$ so is either $1$ or at least $q$.\\
%
%		Now by FLT: $\frac{\text{ord}_{q}\left(p-1\right)}{2}\mid\text{ord}_{q}\left(p-1\right)\mid q-1$ so $\frac{\text{ord}_{q}\left(p-1\right)}{2}\leq q-1\Rightarrow\frac{\text{ord}_{q}\left(p-1\right)}{2}=1\Rightarrow\text{ord}_{q}\left(p-1\right)=2$.\\
%
%		Now $\text{ord}_{q}\left(p-1\right)=2\Rightarrow\left(p-1\right)^{2}\equiv1\mod{q}\Rightarrow\left(p-1\right)^{x-1}\equiv1\mod{q}$ ($x$ is odd means $x-1$ is even). Thus $\left(p-1\right)^{x}\equiv p-1\mod{q}\Rightarrow\left(p-1\right)^{x}+1\equiv p\mod{q}$. But $q\mid\left(p-1\right)^{x}+1$ so $q\mid p\Rightarrow q=p\Rightarrow p\mid x$ which is a case that has already been dealt with.
%	\end{solution}

\begin{problem}
	Determine all positive integers $n$ such that $3^{n}+1$ is divisible by $n^{2}$. %http://artofproblemsolving.com/community/q2h1243268p6361190
\end{problem}

%	\begin{solution}
%		Clearly $n=1$ works. Otherwise, let $p\mid n$ be its smallest prime factor. Then $\text{ord}_p(3) \mid 2n$ and $\text{ord}_p(3)\mid p-1$ obviously. Thus since $p-1$ is less than the smallest prime factor of $2n$ we must have $\text{ord}_p(3)= 1\text{ or }2$. If $\text{ord}_p(3)=1$ then $p=2$. If $\text{ord}_p(3)=2$ then $9\equiv 1\pmod{p}\implies p=2$ impossible already by the previous case. Thus we just have $\text{ord}_p(3)=1$ and $p=2$. But this means $n=2m$ for some integer $m$, so $$4m^2\mid 3^{2m}+1=9^m+1$$which is impossible as $9^m+1\equiv 2\pmod{4}$ so we are done.
%	\end{solution}

\begin{problem}
	Find a condition for $a \in \mathbb{N}$ such that there are infinitely many natural $x$ such that $a^{2x} \equiv a^{2a} \pmod p$ implies $a^{x} \equiv -a^a \pmod p$, where $p$ is any positive integer. %http://artofproblemsolving.com/community/q2h1248121p6416139
\end{problem}

%	\begin{solution}
%		If $p\mid a$ then all is clear.\\
%
%		Else $a^{\varphi(p)}\equiv 1\pmod{p}$. If there exists an $x$ satisfying the property then $x+k\varphi(p)$ will satisfy the property for any positive integer $k$, so it suffices to show the statement if there exists an $x$.\\
%
%		Now if $\text{ord}_p(a)$ is even, choose $x=a+\dfrac{1}{2}\text{ord}_p(a)$. Then $a^{2x}= a^{2a+\text{ord}_p(a)} \equiv a^{2a}\pmod{p}$. Thus, $(a^x-a^a)(a^x+a^a)\equiv 0\pmod{p}$ However, $a^x=a^{a+\frac{1}{2}\text{ord}_p(a)} \not\equiv a^a\pmod{p}$ else $\text{ord}_p(a)\mid \frac{1}{2}\text{ord}_p(a)$ contradiction; thus, we must have $a^x+a^a\equiv 0\pmod{p}\implies a^x\equiv -a^a\pmod{p}$.\\
%
%		If $\text{ord}_p(a)$ is odd, then you're pretty out of luck. For example, take $p=7$ and $a=2$; then $2^{2x}\equiv 16\equiv 2\pmod{7}\implies x=2+3k$ for some integer $k$. However, $2^{2+3k}\equiv 2^2\pmod{7}$ so the second statement is never true.\\
%
%		Overall, the statement is true if and only if $\text{ord}_p(a)$ is even.
%	\end{solution}

\begin{problem}
	Show that $n$ does not divide $2^n-1$ for $n>1$. %http://artofproblemsolving.com/community/q2h538208p3094008
\end{problem}

%	\begin{solution}
%		Take $p$ the smallest prime divisor of $n$, then $2^n \equiv 1 \pmod{p}$ so $\text{ord}_p(2)|n$ and $\text{ord}_p(2)|p - 1$, so $\text{ord}_p(2)|\gcd(n, p - 1) = 1$, so $\text{ord}_p(2) = 1$ and $p|1$, contradiction! Note that $\gcd(n, p - 1) = 1$ since $p$ is the smallest divisor of $n$ greater than one.
%	\end{solution}

\begin{problem}[China 2006]
	Find all positive integer pairs $(a,n)$ such that $$\frac{(a+1)^n-a^n}{n}$$ is an integer. %http://artofproblemsolving.com/community/q2h80728p5983734
\end{problem}

%	\begin{solution}
%		We want $(a+1)^n\equiv a^n\pmod{n}$. Clearly $(a,n)=1$ else $(a,n)\mid a+1$ and $(a,n)\mid a$ impossible since $(a,a+1)=1$. Thus, $a^{-1}$ exists, and we can rewrite $a^{-n}(a+1)^n\equiv (1+a^{-1})^n\equiv 1\pmod{n}$. Let $a^{-1}\equiv b \pmod{n}$; clearly we still have $(b,n)$ and now the equation is $$(b+1)^n\equiv 1\pmod{n}\implies n\mid (b+1)^n-1$$If $n=1$, then clearly $(a,n)=(a,1)$ are solutions. Else, consider the smallest prime factor $p\mid n$. Clearly $p\mid (b+1)^n-1\implies \text{ord}_p(b+1)\mid n$. However, we also clearly have $\text{ord}_p(b+1)\mid p-1$. If $\text{ord}_p(b+1)>1$ then it implies there is a factor that is less than or equal to $p-1$ that divides $n$, contradiction with $p$'s minimality. Thus, $\text{ord}_p(b+1)=1$ and so $b+1\equiv 1\pmod{p}\implies p\mid b$. However, this means $p\mid (b,n)$ contradiction with $(b,n)=1$.\\
%
%		So overall, the solutions are $$\boxed{(a,n)=(a,1)\qquad a\in\mathbb{N}}$$
%	\end{solution}

\begin{problem}
	Prove that for any integer $n \geq 2$ the number $$\frac{3^n-2^n}{n}$$ is not an integer. %http://artofproblemsolving.com/community/q2h1163016p5545885
\end{problem}

%	\begin{solution}
%		$n$ must be odd. Let $p$ be the least prime divisor of $n$. Then $\left(3\cdot 2^{-1}\right)^n\equiv 1\pmod{p}$ and by Fermat's Little theorem $\left(3\cdot 2^{-1}\right)^{p-1}\equiv 1\pmod{p}$, so $\text{ord}_p\left(3\cdot 2^{-1}\right)\mid n,p-1$, so $\text{ord}_p\left(3\cdot 2^{-1}\right)\mid \gcd(n,p-1)$, so $\left(3\cdot 2^{-1}\right)^{\gcd(n,p-1)}\equiv 1\pmod{p}$. But $\gcd(n,p-1)=1$, so $3\cdot 2^{-1}\equiv 1\pmod{p}$, so $3\equiv 2\pmod{p}$, so $p\mid 3-2=1$, contradiction.
%	\end{solution}

\begin{problem}[China 2009]
	Find all the pairs of prime numbers $ (p,q)$ such that $$ pq \mid 5^p+5^q$$ %http://artofproblemsolving.com/community/q2h250256p6448384
\end{problem}

%	\begin{solution}
%		If $p$ or $q$ is $2$, then WLOG $p=2$. Clearly $p=q=2$ does not work, so assume $q$ is odd. Now $2q\mid 25+5^q\implies 5^q+25\equiv 5+25\equiv 0\pmod{q}\implies q\mid 30$ so this case has solutions $$\boxed{(p,q)=(2,3), (2,5), (3,2), (5,2)}$$
%		Otherwise, if $p=q=5$ then it works: $$\boxed{(p,q)=(5,5)}$$If $p=5$ and $q\ne 5$ then $5q\mid 5^5+5^q\implies 5^5+5^q\equiv 5^5+5\equiv 0\pmod{q}\implies q\mid 5^4+1=3\times 313$ so $q=2\text{ or }313$: $$\boxed{(p,q)=(5,2), (5,313), (2,5), (313,5)}$$
%		If $p,q\ne 5$ then $5^p+5^q\equiv 5^p+5\equiv 0\pmod{q}\implies 5^{p-1}\equiv -1\pmod{q}$. Then $\text{ord}_q(5)\mid 2(p-1)$. In addition, since $\text{ord}_q(5)\nmid p-1$ because $q>2$, then $v_2(\text{ord}_q(5)) = v_2(2(p-1))=1+v_2(p-1)$. Similarly, $v_2(\text{ord}_p(5)) = 1+v_2(q-1)$.\\
%
%		If $p=q$ then $p^2 \mid 2\cdot 5^p\implies p=5$ so if $p\ne 5$ then we may assume WLOG $p > q$. Then $pq\mid 5^p+5^q = 5^q(5^{p-q}+1)\implies 5^{p-q}\equiv -1\pmod{pq}$. So $5^{p-q}\equiv -1\pmod{q}$ so $\text{ord}_q(5)\mid 2(p-q)$ and since $\text{ord}_q(5)\nmid p-q$ therefore $v_2(\text{ord}_q(5)) = v_2(2(p-q))=1+v_2(p-q)$. This means $v_2(\text{ord}_q(5))=v_2(\text{ord}_p(5))\implies v_2(q-1)=v_2(p-1) = v_2(p-q)$.\\
%
%		If we let $p-1=k_12^e$ and $q-1=k_22^e$ for $k_1, k_2$ odd, then $p-q=(k_1-k_2)2^e$; however, since $2\mid k_1-k_2$ then $v_2(p-q) > v_2(p-1)$ contradiction. Thus there are no more solutions.
%	\end{solution}

\begin{problem}
	Prove that any two different Fermat numbers are relatively prime with each other.
	%http://artofproblemsolving.com/community/q2h524865p2968616
\end{problem}

\begin{note}
	The $n^{th}$ Fermat number is $F_n = 2^{2^n} + 1$.
\end{note}

%	\begin{solution}
%		Let $p\mid F_k=2^{2^k}+1$, then $p\nmid 2^{2^k}-1$ and $p\mid 2^{2^{k+1}}-1$ therefore $\text{ord}_p(2)=k+1$. Now if there exist $l$ such that $p\mid F_l$ then $\text{ord}_p(2)=l+1$ therefore $k=l$ and hence result.
%	\end{solution}

\begin{problem}
	Prove that for all positive integers $n$, $\gcd(n, F_n) = 1$, where $F_n$ is the $n^{th}$ Fermat number. %http://artofproblemsolving.com/community/q2h597271p3544254
\end{problem}

%	\begin{solution}
%		Consider any prime factor $p$ of $2^{2^n}+1$. We have that \[2^{2^n} \equiv -1 \pmod{p} \implies 2^{2^n+1} \equiv 1 \pmod{p}.\]
%		Let $\text{ord}_p(2)$ be the order of $2$ modulo $p$. We know that $\text{ord}_p(2) \nmid 2^n$, while $\text{ord}_p(2) \mid 2^{n+1}$. As there are no other possible primes other than $2$ present, it follows that $\text{ord}_p(2) = 2^{n+1}$. However, this implies that $2^{n+1} \mid p-1 \implies 2^{n+1} \leq p-1$, and it is easy to see that $n<2^{n+1}$, hence it follows that $n < p$. Therefore, $p$ cannot divide $n$ for any prime divisor of $2^{2^n}+1$, so $\gcd(n, 2^{2^n}+1)=1$.
%	\end{solution}

\begin{problem}
	Let $a$ and $b$ be relatively prime integers and let $d$ be an odd prime that divides $a^{2^{k}}+b^{2^{k}}$. Prove that $d-1$ is divisible by $2^{k+1}$. %http://artofproblemsolving.com/community/q2h1177972p5694239
\end{problem}

%	\begin{solution}
%		First, as usual, I'll denote $\text{ord}_p(a)$ as the least positive integer $t$ such that $a^t\equiv 1\pmod{p}$ and prove the following lemma: If $x^k\equiv 1\pmod{m}$, then $\text{ord}_m(x)\mid k$. Proof: for contradiction, let $k=\text{ord}_m(x)h+r$ with $0<r<\text{ord}_m(x)$; but then $1\equiv x^k\equiv \left(x^{\text{ord}_m(x)}\right)^hx^r\equiv 1^hx^r\equiv x^r\pmod{m}$, contradiction.\\
%
%		If $d\mid b$, then $d\mid a$ and $\gcd(a,b)\ge d>1$, contradiction, so $\gcd(d,b)=1$, so $\left(ab^{-1}\right)^{2^k}\equiv -1\pmod{d}$, so $\text{ord}_d\left(ab^{-1}\right)=2^{k+1}$, so by Fermat's Little theorem $2^{k+1}\mid d-1$.
%	\end{solution}

\begin{problem}
	Prove that if $p$ is a prime, then $p^p-1$ has a prime factor greater than $p$. %http://artofproblemsolving.com/community/q2h1244184p6371381
\end{problem}

%	\begin{solution}
%		If we can prove that there exists prime factor $q$ such that $\text{ord}_{q}(p) = p$ then the result follows. To prove this one, just suppose the contrary. Then for all prime factor $q$ of $p^{p} - 1$, then $\text{ord}_{q}(p) = 1$. Let $q$ be a prime divisor of $\frac{p^{p} - 1}{p - 1}$
%		Since $\text{ord}_{q}(p) = 1$, we conclude that $p \equiv 1\pmod{q}$
%		Thus, $0 \equiv p^{p - 1} + p^{p - 2} + \cdots + 1 \equiv p \pmod{q}$. This implies $q\mid p$ which is a contradition.
%		Hence the result follows.
%	\end{solution}

\begin{problem}
	$ $
	\begin{enumerate}
		\item Show that if $p$ is a prime and $\text{ord}_p(a)=3$, then \[\parenthesis{\sum_{j=0}^{2}a^{j^{2}}}^{2}\equiv{-3}\pmod{p}\]

		\item Show that if $p$ is a prime and $\text{ord}_p(a)=4$, then \[\parenthesis{\sum_{j=0}^{3}a^{j^{2}}}^{2}\equiv{8a}\pmod{p}\]

		\item Show that if $p$ is a prime and $\text{ord}_p(a)=6$, then \[\sum_{j=0}^{5}a^{j^{2}}\equiv{0}\pmod{p}\]
	\end{enumerate}
	%http://artofproblemsolving.com/community/q2h470104p2631974
\end{problem}

%	\begin{solution}
%		\begin{enumerate}
%			\item This can be written as $(1+2a)^{2}=4(1+a+a^{2})-3=-3$ (mod $p$).
%
%			\item This can be written as $(2+2a)^2 = 4(a^2+1) + 8a \mod p$. However, note that since $\text{ord}_p(a) = 4$, that $a^2 \equiv -1 \mod p$, so $4(a^2+1) + 8a \equiv 8a \mod p$, as desired.
%
%			\item This can be written as $(2a^4+a^3+2a+1)^2$. However, note that since $\text{ord}_p(a) = 6$, that $a^3 \equiv -1 \mod p$, so $(2a^4+a^3+2a+1)^2 \equiv (2a^4+2a)^2 \equiv 4a^2(a^3+1)^2 \equiv 0 \mod p$, as desired.
%		\end{enumerate}
%	\end{solution}

\begin{problem}[Poland 2016]
	Let $k$ and $n$ be odd positive integers greater than $1$. Prove that if there a exists positive integer $a$ such that $k \mid 2^a+1$ and $n \mid 2^a-1$, then there is no positive integer $b$ satisfying $k \mid 2^b-1$ and  $n \mid 2^b+1$. %http://artofproblemsolving.com/community/q2h1224675p6150214
\end{problem}

%	\begin{solution}
%		Assume that there exists such $b$. We have $k \mid 2^a+1 \mid 2^{2a}-1$ and $n \mid 2^a-1$ so $\text{ord}_k(2) \mid 2a, \text{ord}_k(2) \mid b$ but $\text{ord}_k(2) \nmid a$. Hence, $v_2 \left( \text{ord}_k(2) \right)= v_2(a)+1$ implies $2^{v_2(a)+1} \mid b$ implies $v_2(a)+1 \le v_2(b)$. Similarly, $2^{v_2(b)+1} \mid a$ implies $v_2(b)+1 \le v_2(a)$. This gives a contradiction. Thus, there doesn't exist such $b$.
%	\end{solution}

\begin{problem}
	Let $n>9$ be a positive integer such that $\gcd(n,2014)=1$. Show that if $n \mid 2^n+1$, then $27 \mid n$. %http://artofproblemsolving.com/community/q2h557704p3242761
\end{problem}

%	\begin{solution}
%		Suppose $\gcd(n, 2014) = 1$ but $27 \nmid n$. Now let $p$ be the smallest prime divisor of $n$ so $p|2^{2n} - 1$ so $\text{ord}_p(2)|2n$ but also $\text{ord}_p(2)|p - 1$ so $\text{ord}_p(2)|\gcd(2n, p - 1) = 2$ so $p|2^2 - 1$ so $p = 3$. Now let $n = 3a$ so $a|8^a + 1$ let $q$ be smallest prime divisor of $a$ then $q|8^{2a} - 1$ so $\text{ord}_q(8)|2a, q - 1$ so $\text{ord}_q(8)|2$ so $q|8^2 - 1 = 63$. But if $q = 7$ then $7|8^a + 1$ so $7|2$, bad. So $q = 3$. Let $a = 3b$ so $b|512^b + 1$. Now let $r$ be the least prime divisor of $b$. So by the same argument $q|512^2 - 1$ or $q|7\cdot 73\cdot 19\cdot 27$. So now if $q = 3$ then $27|n$, contradicting our assumption. If $q = 7$ or $73$ then $q|512^b + 1$ so $q|2$, bad. So $q = 19$ but then contradiction to $\gcd(n, 2014) = 1$. This is a contradiction in all cases! Note that $n, a, b$ have prime divisors since $n > 9$.\\
%
%		So the assumption is false and $27|n$.
%	\end{solution}

\begin{problem}
	Find all primes $p$ and $q$ that satisfy
		\begin{align*}
			p^2+1
				& \mid 2003^q+1\\
			q^2+1
				& \mid 2003^p+1
		\end{align*}
	%http://artofproblemsolving.com/community/q2h1230409p6217580
\end{problem}

\begin{problem}
	Prove that there do not exist non-negative integers $a,b$, and $c$ such that $$(2^a-1)(2^b-1)=2^{2^c}+1$$ %http://artofproblemsolving.com/community/q2h454381p2553548
\end{problem}

%	\begin{solution}
%		There's an easy solutions by taking modulo $8$ I think.\\
%
%		But anyways, clearly if $p|(2^{2^c} + 1)$, then $2^{2^c} \equiv -1 \pmod{p} \implies \text{ord}_p(2) = 2^{c+1}$
%		This means $2^{c+1}|(p-1) \implies p \equiv 1 \pmod{2^{c+1}}$. Now if $c \ge 2$ this means $\left ( \frac{2}{p} \right ) = 1$, so $\sqrt{2}$ exists.
%		However, as $\text{ord}_p(2) = 2^{c+1}$ we can easily show $\text{ord}_p(\sqrt{2}) = 2^{c+2} \implies 2^{c+2}|(p-1) \implies p \equiv 1 \pmod{2^{c+2}}$ and the result follows.
%	\end{solution}

\begin{problem}
	Find all triples $(x,y,z)$ of positive integers which satisfy the equation $$2^x+1=z(2^y-1)$$ %http://artofproblemsolving.com/community/q2h525122p2971275
\end{problem}

%	\begin{solution}
%		$y = 1$ is obvious.\\
%
%		So assume $y > 1$. Then note that there exists a prime $p$ that divides both $2^y - 1$ and $2^x + 1$.\\
%
%		Let $p$ be this prime. We have $2^x \equiv -1 \pmod{p} \implies 2^{2x} \equiv 1 \pmod{p}$, so either $2 \equiv -1 \pmod{p}$ or $\text{ord}_p(2) = 2x$.\\
%
%		We have a similar equation $2^y \equiv 1 \pmod{p} \implies \text{ord}_p(2) \mid y$.\\
%
%		We have $y < x$, so it must be that $2 \equiv -1 \pmod{p} \implies p = 3$.\\
%
%		So, we have $2^y - 1 = 3 \implies y = 2$. Then, for all $2^{x} \equiv (-1)^x \equiv -1 \pmod{3} \implies x$ must be odd.\\
%
%		Thus, $(x, y, z) = (x, 1, z), (2k + 1, 2, (2^x + 1)/3)\forall x, y, z \in \mathbb{Z}_{+}$.
%	\end{solution}


\begin{problem}
	Let $p$ be a prime number of the form $3k+2$ that divides $a^2+ab+b^2$ for two positive integers $a$ and $b$. Prove that $p$ divides both $a$ and $b$. %http://artofproblemsolving.com/community/q2h485827p2721969
\end{problem}

%	\begin{solution}
%		If $a^2+ab+b^2 \equiv 0 \mod p$, then multiplying by $a-b$ implies that $a^3 \equiv b^3 \mod p$. Assume for sake of contradiction that $p\nmid a, b$. Then we have that $\left(ab^{-1}\right)^3 \equiv 1 \mod p$, implying that $\text{ord}_p(ab^{-1}) = 3$. Note that the order must divide $p-1$, but this is a contradiction, as $3 \nmid p-1 = 3k+1$.
%
%		So then $p|a, b$ as desired.
%	\end{solution}


\begin{problem}
	Prove Wilson's theorem using primitive roots.
\end{problem}

%	\begin{solution}
%		If $p=2$, it's clear; if $p\ge 3$, then let $g$ be a primitive root mod $p$. Then $(p-1)!\equiv g^{1+2+\cdots+(p-1)}\equiv \left(g^{\frac{(p-1)}{2}}\right)^p\pmod{p}$. Also $g^{p-1}\equiv 1\pmod{p}\iff g^{\frac{p-1}{2}}\equiv \pm 1\pmod{p}$, but $\text{ord}_p(g)=p-1$, so $g^{\frac{p-1}{2}}\equiv -1\pmod{p}$, so $\left(g^{\frac{p-1}{2}}\right)^p\equiv (-1)^p\equiv -1\pmod{p}$.
%	\end{solution}

\begin{problem}
	If $p$ is a prime, show that the product of the primitive roots of $p$ is congruent to to $(-1)^{\varphi(p-1)}$ modulo $p$. %http://www.artofproblemsolving.com/community/q1h1222327p6119792
\end{problem}

%	\begin{solution}
%		Let $g$ be a primitive root modulo $p$ for an odd prime $p$. Then it is well-known that $g^i$ is a primitive root modulo $p$ if and only if $\gcd(i,p-1)=1$ (not hard to prove). So the product of all primitive roots modulo $p$ is $$\prod_{\gcd(i,p-1)=1}{g^i}=g^{\sum_{\gcd(i,p-1)=1}{i}}=g^{\frac{p-1}{2}\cdot\phi(p-1)}\pmod p.$$The sum in the exponent is $\frac{p-1}{2}\cdot\phi(p-1)$ because if $\gcd(k,n)=1$ then $\gcd(n-k,n)=1$ as well. So we can pair them off so that there are $\frac{\phi(p-1)}{2}$ pairs (by definition of the $\phi$ function and the fact that $p-1$ is even), with each pair summing to $p-1$.\\
%		We get the final result from $g^{\frac{p-1}{2}}\equiv -1\pmod p$, which holds because $g$ is a primitive root. If we let $h=g^{\frac{p-1}{2}}$ then $h^2\equiv 1\pmod p$ so $h\equiv \pm 1\pmod p$. But it cannot be $1$ because that would contradict the definition of a primitive root (no exponent lower than $p-1$ can cause $g$ to go to $1$).
%	\end{solution}

\begin{problem}
	Let $g$ be a primitive root modulo a prime $p$. Find $\text{ord}_{p^r}(g)$. %http://www.artofproblemsolving.com/community/q1h624598p3740432
\end{problem}

%	\begin{solution}
%		Since $\varphi(p^r) = (p-1)p^{r-1}$, the multiplicative order of $g$ modulo $p^r$ must be a divisor of it, thus of the form $mp^k$, for some $m\mid p-1$ and $0\leq k \leq r-1$. Then $p\mid p^r \mid g^{mp^k} - 1$, so $(g^{p^k})^m = g^{mp^k} \equiv 1\pmod{p}$. But $p^k \equiv 1\pmod{p-1}$, and $g^{p-1} \equiv 1 \pmod{p}$, so it follows $g^{m} \equiv 1\pmod{p}$. Now remember $g$ is primitive, which forces $m=p-1$.
%	\end{solution}

\begin{problem}
	Prove that if $r$ is a primitive root modulo $m$, then so is the multiplicative inverse of $r$ modulo $m$. %http://www.artofproblemsolving.com/community/q1h1162723p5543625
\end{problem}

%	\begin{solution}
%		You want to prove that if $\text{ord}_m(r)=\varphi(m)$, then $\text{ord}_m\left(r^{-1}\right)=\varphi(m)$. For contradiction, assume $\text{ord}_m\left(r^{-1}\right)=k<\varphi(m)$. Then $\left(r^{-1}\right)^k\equiv 1\pmod{m}$, so $r^k\equiv\left(\left(r^{-1}\right)^k\right)^{-1}\equiv 1^{-1}\equiv 1\pmod{m}$, contradiction.
%	\end{solution}

\begin{problem}
	Prove that $3$ is a primitive root modulo $p$ for any prime $p$ of the form $2^n+1$. %http://www.artofproblemsolving.com/community/q1h1142186p5370120
\end{problem}

%	\begin{solution}
%		\textit{First Solution.} This is not true for $p=3$ so we will assume $n > 1$. In particular this means $2^n+1 \equiv 1 \pmod{4}$.
%		Furthermore note that since $p$ is a prime we actually have it is of the form $2^{2^r} + 1$ for some positive integer $r$. Therefore $p = 2^{2^{r}} +1 \equiv (-1)^{2^r} +1 \equiv 2 \pmod{3}$ which is not a quadratic residue $\pmod{3}$.
%		This means $\genfrac{(}{)}{}{}{3}{p} $ $\genfrac{(}{)}{}{}{p}{3} = (-1)^{(p-1)/2} =1$ since $p \equiv 1 \pmod{4}$ and from the above we know that $\left( \dfrac{p}{3} \right) = -1$. Therefore $3$ is not a quadratic residue $\pmod{p}$.\\
%		We will now prove that for a prime $p=2^{2^{r}}+1$ every quadratic non residue $\pmod{p}$ is a primitive root $\pmod{p}$. Since $p$ is a prime we know that there exists a primitive root $g$ and all the residues $\pmod{p}$ are given by $g,g^2,g^3,\cdots,g^{p-1}$. It is easily seen that $g^2,g^4,\cdots,g^{p-1}$ are $\dfrac{p-1}{2}$ different nonzero residues $\pmod{p}$ and they are all quadratic residues. Therefore all the quadratic non residues are given by $$g,g^3,g^5,\cdots,g^{p-2}.$$We will now take one of this residues, say $g^{2k+1}$ and show that it is a primitive root $\pmod{p}$. We want to show that $g^{2k+1},g^{2(2k+1)},g^{3(2k+1)},\cdots,g^{(p-1)(2k+1)}$ are all different $\pmod{p}$ which obviously happens if and only if $2k+1,2(2k+1),3(2k+1),\cdots,(p-1)(2k+1)$ are all different $\pmod{p-1}$. This happens if and only if $(2k+1,p-1)=1$ or $(2k+1,2^{2^r})=1$ but this is obvious since $2k+1$ is odd and $2^{2^{r}}$ is a power of $2$. Therefore all quadratic non residues are primitive roots $\pmod{p}$ and as we have shown $3$ is a quadratic non residue $\pmod{p}$ so we are done.\\
%
%		\textit{Second Solution.} Clearly $n=2^m$ for some $m$.
%		Now let $p=2^{2^m}+1$. From quadratic respectively law $\left(\frac{3}{p}\right)\left(\frac{p}{3}\right)=1\Longrightarrow \left(\frac{3}{p}\right)=-1\Longrightarrow 3^{2^{n-1}}\equiv -1\pmod{p}(\bigstar)$. Let $d$ be the order of $3$ modulo $p$ then because $d\mid p-1=2^n\Longrightarrow d=2^{\alpha}$, if $\alpha<n$ then $3^{\alpha}\equiv 1\pmod{p}\Longrightarrow 3^{2^{n-1}}\equiv 1\pmod{p}$ but it's contradiction with $(\bigstar)$ so $d=2^n$. this means that $3$ is primitive root modulo $2^n+1$.
%	\end{solution}

\begin{problem}
	Suppose $q\equiv 1\pmod 4$ is a prime, and that $p=2q+1$ is also prime. Prove that $2$ is a primitive root modulo $p$. %http://www.artofproblemsolving.com/community/c6h598837p3554092
\end{problem}

%	\begin{solution}
%		Notice that, by FLT, we have $2^{p-1} \equiv 1 \pmod{p}$, so $2^q \equiv \pm 1 \pmod{p}$. If it were negative $1$, then we are done. Assume that $2^q \equiv 1 \pmod{p}$. This would mean that there exists an integer $k$ such that $k^2 \equiv 2 \pmod{p}$, which implies that $(-1)^{\frac{p^2 - 1}{8}} = 1$, which can easily be seen as a contradiction, because $\frac{p^2 - 1}{8}$ is odd. Hence, we cannot have $2^q \equiv 1 \pmod{p}$.
%	\end{solution}


\begin{problem}
	Find all Fermat primes $F_n$ such that $7$ is a primitive root modulo $F_n$. %http://www.artofproblemsolving.com/community/q1h553691p3216866
\end{problem}

%	\begin{solution}
%		The idea here is that any non-square residue is a primitive root mod $p=2^n+1$.
%		Because a non-primitive root $r$ satisfies $r^d\equiv 1$ for $d<p-1$ and $d|p-1=2^n$, so $d=2^\alpha$ for $ \alpha<n \Rightarrow r^{\frac{p-1}{2}}=r^{2^{n-1}}\equiv 1$.\\
%
%		But $n\ge 2, (\frac{7}{p})=(\frac{p}{7})$ (by quadratic reciprocity), and $p=2^{2^k}+1\equiv 3$ or $5$ $(mod\,7)$. But $3$ and $5$ are not quadratic residues $mod 7$, so $7$ is not a quadratic residue $mod\, p$ $\Rightarrow 7$ is a primitive root $mod\, p$, except for $p=2^1+1$.
%	\end{solution}

\begin{problem}
	Prove that if $F_{m}=2^{2^{m}}+1$ is a prime with $m\geq{1}$, then $3$ is a primitive root of $F_{m}$. %http://www.artofproblemsolving.com/community/q1h503123p2826526
\end{problem}

%	\begin{solution}
%		Assume $F_{m}=2^{2^{m}}+1$ is a prime, for some $m\geq{1}$.\\
%
%		Then $\left (\mathbb{F}_{F_m}^*, \cdot\right)$ is cyclic; let $g$ be a generator (primitive root) of it. Then all odd powers of $g$ are also primitive roots, while the even powers are not. Assume $3$ is not a primitive root, hence $3$ is an even power of $g$, i.e. $3 = g^{2k} = (g^k)^2$. But, by the quadratic reciprocity law, $3$ is a non-quadratic residue modulo $F_m$. This contradiction shows $3$ is a primitive root for $F_m$.
%	\end{solution}

\begin{problem}
	For a given prime $p > 2$ and a positive integer $k$, let \[ S_k = 1^k + 2^k + \cdots + (p - 1)^k\] Find those values of $k$ for which $p \mid S_k$. %http://www.artofproblemsolving.com/community/c6h295883p1602380
\end{problem}

%	\begin{solution}
%		Let $ g$ one primitive root modulo $ p$.
%		Then, the numbers $ 1,2,...,p-1$ are covered modulo $ p$ by the powers of $ g$: $ g^{0},g^{1}, ..., g^{p-2}$.
%		So, the sum is:
%		$$ S = 1+g^{k}+g^{2k}+...+g^{(p-2)k}.$$
%		If $ p-1|k$, the sum is $p-1$ modulo $p$.\\
%		If $ p-1$ does not divide $k$, then $ S=\frac{g^{(p-1)k} - 1}{g^{k}-1}$.
%		Seeing S modulo p, we have that S is 0 modulo p.
%
%		Hence, $ k$ is not multiple of $ (p-1)$.
%	\end{solution}

\begin{problem}
	Show that for each odd prime $p$, there is an integer $g$ such that $1<g<p$ and $g$ is a primitive root modulo $p^n$ for every positive integer $n$. %http://www.artofproblemsolving.com/community/c146h150489
\end{problem}

%	\begin{solution}
%		For $n = 1$, consider a nonsquare $g$ mod $p$ that is not $-1$. $g^\frac{p-1}{2} = -1$ mod $p$. Assume $g^k = 1$ mod $p$ with $k < p-1$. We know that $k | p-1$ however we can also see that $k \nmid \frac{p-1}{2}$ which is only possible if $k = 2$. But if $k = 2$ then $g = 1, -1$ which is a contradiction. Thus $g$ is a primitive root mod $p$. For $n > 1$. Assume primitive roots exist mod $p^{n-1}$. Then these roots to the power of $p^{n-1}-p^{n-2}$ are in the set 1, $p^{n-1}+1$, $2p^{n-1}+1$,..., $(p-1)p^{n-1}+1$ mod $p^n$. We can see that if $g$ is a primitive root mod $p^{n-1}$ then $(p+g)^{p^{n-1}-p^{n-2}}$ or $g^{p^{n-1}-p^{n-2}}\neq 1$ mod $p^{n}$ by the binomial theorem. So this root raised to the power of $p$ is congruent to 1 as can be seen by the binomial theorem. If for this same root $g$, there existed $k < p^{n}-p^{n-1}$ such that $g^k = 1$ mod $p^n$ then $k | p^{n-1}-p^{n-2}$ which is a contradiction.
%	\end{solution}

\begin{problem}
	Show that if $p=8k+1$ is a prime for some positive integer $k$, then $p\mid x^4+1$ for some integer $x$. %http://www.artofproblemsolving.com/community/q1h588840p3486240
\end{problem}

%	\begin{solution}
%		Let's assume that if p is a prime, there exists at least an element called the generator or the primitive root modulo p which is of order $p - 1$. Here, let's choose an arbitrary primitive root and let's call it $g$. From Fermat's Little Theorem, $g^{p-1} = 1\pmod p$ i.e $g^{8k}=1\pmod p$. Hence, we must have:
%		either $g^{4k}=1[p]$ or $g^{4k}=-1 \pmod p$. But as $k>1$, $8k>4k$ so we can't have $g^{4k}=1 \pmod p$ because of the minimality of the order. Hence, $g^{4k}=-1\pmod p$ and taking $X=g^k$, $p | X^{4}+1$. So we are done!
%	\end{solution}

\begin{problem} %https://artofproblemsolving.com/community/c6h486495
	Let $n$ be a positive integer. Prove that
	\begin{align*}
		n
			& \le 4\lambda(n)\parenthesis{2^{\lambda(n)}-1}
	\end{align*}
	where $\lambda(n)$ denotes the Carmichael function of $n$.
\end{problem}

\begin{problem}
	Find $\lambda(1080)$.
\end{problem}

\begin{problem}[RMO 1990]%http://artofproblemsolving.com/community/c6h56051p347213
	Find the remainder when $2^{1990}$ is divided by $1990$.
\end{problem}

\begin{hint}
	Use Carmichael's function.
\end{hint}

\begin{problem} %https://artofproblemsolving.com/community/c3h1309261
	Given three integers $a,b,c$ satisfying $a\cdot b\cdot c=2015^{2016}$. Find the remainder when we divide $A$ by $24$, knowing that
	\begin{align*}
	A=19a^2+5b^2+1890c^2
	\end{align*}
\end{problem}

\begin{problem}[APMO 2006] %https://artofproblemsolving.com/community/c6h80759
	Let $p\ge5$ be a prime and let $r$ be the number of ways of placing $p$ checkers on a $p\times p$ checkerboard so that not all checkers are in the same row (but they may all be in the same column). Show that $r$ is divisible by $p^5$. Here, we assume that all the checkers are identical.
\end{problem}

\begin{problem}[Putnam 1996] %https://artofproblemsolving.com/community/c6h296158
	%https://artofproblemsolving.com/community/c7h592336p3511046
	Let $p$ be a prime greater than $3$. Prove that
	\begin{align*}
		p^2
			& \mid \sum_{i=1}^{\floor{\frac{2p}{3}}}\binom{p}{i}
	\end{align*}
\end{problem}

\begin{problem} %https://artofproblemsolving.com/community/c6h402354
	Let $a$ and $b$ be two positive integers satisfying $0<b\leq a.$ Let $p$ be any prime number. Show that
	\begin{align*}
		\binom{pa}{pb}
			& \equiv \binom{a}{b} \pmod{p^3}
	\end{align*}
\end{problem}

\begin{problem} %https://artofproblemsolving.com/community/c6h380812
	The sequence $a_n$ is defined as follows: $a_1 = 0$ and
	\begin{align*}
		a_{n+1}=\frac{\ensuremath{(4n+2).n^{3}}}{(n+1)^{4}}a_{n}+\frac{3n+1}{(n+1)^{4}}
	\end{align*}
	for $n\ge1 $. Prove that there are infinitely many positive integers $n$ such that $a_n$ is an integer.
\end{problem}

\begin{hint}
	Find an explicit formula for $a_n$ and then try Wolstenholme's theorem.
\end{hint}

\begin{problem} %https://math.stackexchange.com/q/1946715/6715
	Let $p$ be an odd prime. Define
	\begin{align*}
		H_n
			& = 1 + \dfrac{1}{2}+\ldots+\dfrac{1}{n}
	\end{align*}
	to be the $n^{th}$ \textit{harmonic number} for any positive integer $n$. Prove that $p$ divides the numerator of both $H_{p(p-1)}$ and $H_{p^2-1}$.
\end{problem}

\begin{problem} %https://math.stackexchange.com/q/2256601/6715
	Let $p$ be an odd prime number. Define $q = \frac{3p-5}{2}$ and
	\begin{align*}
		S_q
			& = \dfrac{1}{2 \cdot 3 \cdot 4} + \dfrac {1}{5 \cdot 6 \cdot 7} + \cdots + \dfrac{1}{q(q+1)(q+2)}
	\end{align*}
	If we write $\frac{1}{p} - 2S_q $ as an irreducible fraction, prove that $p$ divides the difference between numerator and denominator of this fraction.
\end{problem}

\begin{problem} %https://math.stackexchange.com/q/2322942/6715
	Find the largest power of a prime $p$ which divides
	\begin{align*}
		S_p
			& =\binom{p^{n+1}}{p^n}-\binom{p^{n}}{p^{n-1}}
	\end{align*}
\end{problem}

\begin{problem} %https://math.stackexchange.com/q/1983903/6715
	Let $p \geq 5$ be a prime. Prove that
		\begin{align*}
			\sum_{k=1}^{p-1}\frac{2^k}{k^2}
				& \equiv-\frac{(2^{p-1}-1)^2}{p^2}\pmod p
		\end{align*}
\end{problem}

\begin{problem} %https://math.stackexchange.com/q/1259840/6715
	Let $p\geq 3$ be a prime number and let
		\begin{align*}
			\sum_{j=1}^{p-1}\frac{(-1)^{j}}{j} \binom{p-1}{j} =\frac{a}{b}
		\end{align*}
	where $a$ and $b$ are relatively prime integers. Prove that $p^2\mid a$.
\end{problem}

\begin{problem} %https://math.stackexchange.com/q/273413/6715
	Prove that $\binom{2^{n}-k}{k-1}$ is even for all positive integers $n$ and $k$ such that $2\le k\le 2^{n-1}$.
\end{problem}

\begin{problem} %https://math.stackexchange.com/q/597334/6715
	How many of the following numbers are divisible by $3$?
	\begin{align*}
	\binom{200}{0}, \binom{200}{1}, \binom{200}{2}, \cdots, \binom{200}{200}
	\end{align*}
\end{problem}

	%Diophantine equations

\begin{problem} %https://artofproblemsolving.com/community/c4h1367798
	Find all pairs $(p,q)$ prime numbers such that
		\begin{align*}
			7 p^3 - q^3 = 64
		\end{align*}
\end{problem}

\begin{problem}[BMO 2009] %https://artofproblemsolving.com/community/c6h274318
	Solve the equation
		\begin{align*}
			3^x - 5^y = z^2
		\end{align*}
	in positive integers.
\end{problem}

\begin{problem} %https://artofproblemsolving.com/community/c2113h1042469
	Solve the equation $7^x=3^y+4$ in integers.
\end{problem}

\begin{problem} %https://artofproblemsolving.com/community/c2113h1042469
 Solve the equation $2^x+3=11^y$ in positive integers.
\end{problem}

\begin{problem} %https://artofproblemsolving.com/community/c6h1300647
	Solve the Diophantine equation $$2^x(1+2^y)=5^z-1$$ in positive integers.
\end{problem}

\begin{hint}
	Take modulo $16$.
\end{hint}

\begin{problem}[Putnam 2001] %https://artofproblemsolving.com/community/c7h466477p2612329
	Prove that there are unique positive integers $a$ and $n$ such that $$a^{n+1}-(a+1)^n=2001$$
\end{problem}