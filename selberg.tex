\documentclass{subfile}

\begin{document}
	\textit{Alte Selberg} and \textit{Paul Erd\fH{o}s} together first proved the \textbf{Prime Number Theorem} in an elementary way. Selberg found an interesting identity in the process of the proof, known as the \textit{Selberg Identity}. As we stated in a footnote before, there was a dispute regarding who proved prime number theorem elementarily. We trust what \textcite{goldfeld_2004} said in this paper and conclude that both Erd\fH{o}s and Selberg had contributions in this proof. To be more precise, we believe that Selberg proved \textit{the fundamental identity}\footnote{The identity we are discussing here can be thought of as a basis of what Selberg calls \textit{the fundamental identity}.} but could not proved PNT at the time. Later Erd\fH{o}s proved PNT with the help of fundamental identity, and probably Selberg proved PNT on his own as well (possibly afterwards)\footnote{We are not making any assertion here, only expressing our thoughts. One should not draw any conclusion from ours. You can read the papers and think for yourself what you want to decide.}. However, as mentioned in paper \textcite[Page $6$]{goldfeld_2004}, one must acknowledge the fact Erd\fH{o}s could immediately understand that the fundamental identity implies $\lim\limits_{n\to\infty}\frac{p_{n+1}}{p_n}=1$. This alone demonstrates the magnitude of Erd\fH{o}s's thinking ability. While on this point, we would also like to clarify the magnitude of proving PNT elementarily. \textit{G. H. Hardy} (see \textcite[Page $3$]{goldfeld_2004}) said this about PNT in $1921$:
		\begin{quote}
			No elementary proof of the prime number theorem is known, and one may ask whether it is reasonable to expect one. Now we know that the theorem is roughly equivalent to a theorem about an analytic function, the theorem that Riemann's zeta function has no roots on a certain line. A proof of such a theorem, not fundamentally dependent on the theory of functions, seems to me extraordinarily unlikely. It is rash to assert that a mathematical theorem cannot be proved in a particular way; but one thing seems quite clear. We have certain views about the logic of the theory; we think that some theorems, as we say `lie deep' and others nearer to the surface. If anyone produces an elementary proof of the prime number theorem, he will show that these views are wrong, that the subject does not hang together in the way we have supposed, and that it is time for the books to be cast aside and for the theory to be rewritten.
		\end{quote}

	We need some definitions before stating the Selberg identity. We will use functions defined in chapter \eqref{ch:arithfunc}. The following theorem is almost trivial.
	\begin{theorem}[Invariance Theorem]\slshape
		Let $f$ be an arithmetic function and $I$ be the identity function. Then
			\begin{align*}
				f\ast I=I\ast f=f.
			\end{align*}
	\end{theorem}

	\begin{theorem}\slshape
		\label{thm:dirichmobi}
		Let $f$ be an arithmetic function and $F$ is its summation function. Then
			\begin{align*}
				f=\mu\ast F.
			\end{align*}
	\end{theorem}

	\begin{proof}
		This is immediately resulted from M\" obius inversion theorem (theorem \eqref{thm:mobinv}).
	\end{proof}

	\begin{definition}[Dirichlet Derivative]
		For an arithmetic function $f$, we define its \textit{Dirichlet derivative} as
			\begin{align*}
				f^\prime(n) & =f(n)\ln n.
			\end{align*}
	\end{definition}

	\begin{example}
		$I^\prime(n)=I(n)\ln n=0$ for all positive integers $n$. Also, $u^\prime(n)=\ln n$ and $u^{\prime\prime}(n)=\ln n\cdot\ln n=\ln^2 n$.
	\end{example}
We can easily check that some usual properties of differentiation hold true for Dirichlet derivative as well. For instance:
	\begin{proposition}
		Let $f$ and $g$ be arithmetic functions. Then\footnote{You can see it follows some properties of the usual derivative (if you are familiar with calculus, you should know what derivative is. However, for this purpose you do not need any calculus.)}
			\begin{align*}
				(f+g)^\prime & =f^\prime+g^\prime,\\
				(f\ast g)^\prime& =f^\prime \ast g + f\ast g^\prime.
			\end{align*}
	\end{proposition}
	\begin{proof}
		The first one is obvious. For the second one, we can write
			\begin{align*}
				(f\ast g)^\prime (n)
					&= \left(\sum_{d\mid n} f(d)g\left(\dfrac{n}{d}\right)\right) \cdot \ln n\\
					&= \left(\sum_{d\mid n} f(d)g\left(\dfrac{n}{d}\right)\right) \cdot \left(\ln d + \ln \dfrac{n}{d}\right)\\
					&= \left(\sum_{d\mid n} f(d)g\left(\dfrac{n}{d}\right)\right) \cdot \ln d + \left(\sum_{d\mid n} f(d)g\left(\dfrac{n}{d}\right)\right) \cdot \ln \dfrac{n}{d}\\
					&= \sum_{d\mid n} f(d)\cdot \ln d \cdot g\left(\dfrac{n}{d}\right)+ \sum_{d\mid n} f(d) g\left(\dfrac{n}{d}\right) \cdot \ln \dfrac{n}{d}\\
					&= \sum_{d\mid n} f^\prime(d) \cdot g\left(\dfrac{n}{d}\right)+ \sum_{d\mid n} f(d) g^\prime\left(\dfrac{n}{d}\right)\\
					&= (f^\prime \ast g)(n) + (f\ast g^\prime)(n).
			\end{align*}

	\end{proof}
	\begin{definition}[Von Mangoldt Function]
		For any positive integer $n$, the von Mangoldt function, denoted by $\Lambda(n)$\footnote{$\Lambda$ is the upper case of the symbol lambda ($\lambda$) in Greek.} is defined as
			\begin{align*}
				\Lambda(n) & =
					\begin{cases}\ln p &\mbox{ if }n=p^m\mbox{ for  some prime }p\mbox{ and positive integer }m,\\0&\mbox{ otherwise}.\end{cases}
			\end{align*}
	\end{definition}

	\begin{theorem}
		\label{thm:vonmangoldt}
		Let $n$ be a positive integer. Then $\ln n=\sum_{d\mid n}\Lambda(d)$.
	\end{theorem}

	\begin{proof}
		Let $n=\prod_{i=1}^kp_i^{e_i}$, where $p_i$ are primes ($1 \leq i \leq k$). Then,
			\begin{align*}
				\ln n  & =\ln\left(\prod_{i=1}^kp_i^{e_i}\right)\\
						&=\sum_{i=1}^k\ln{p_i^{e_i}}\\
						&=\sum_{i=1}^ke_i\ln p_i.
			\end{align*}
		On the other hand, if $p$ is a prime, for $d\neq p^m$ we have $\Lambda(d)=0$ by definition. Therefore, only prime powers $p^e$ contribute a $\ln p$ to the sum $\sum_{d\mid n}\Lambda(d)$. So, if $p_i$ is a prime divisor of $n$, $p_i^1,\ldots,p_i^{e_i}$ contribute $e_i \cdot \ln p_i$ to the sum. Thus,
			\begin{align*}
				\sum_{d\mid n}\Lambda(d) = \sum\limits_{i=1}^ke_i\ln p_i.
			\end{align*}

	\end{proof}
	Now we are ready to state and prove the Selberg's identity.

	\begin{theorem}[Selberg's Identity]
		Let $n$ be a positive integer. Then
			\begin{align*}
				\Lambda(n)\ln n+\sum_{d\mid n}\Lambda(d)\Lambda\left(\frac nd\right)=\sum_{d\mid n}\mu(d)\ln^2\frac nd.
			\end{align*}
	\end{theorem}
	\begin{proof}
		We proved in theorem \eqref{thm:vonmangoldt} that $\ln n=\sum_{d\mid n}\Lambda(d)$. We also found that $u^\prime(n) = \ln n$. This can be written as
			\begin{align*}
				\Lambda\ast u & =u^\prime.
			\end{align*}
		Take derivative of both sides of the above equation to obtain
			\begin{align*}
				\Lambda^\prime\ast u+\Lambda\ast u^\prime & =u^{\prime\prime}.
			\end{align*}
		Using $\Lambda\ast u  =u^\prime$ again,
			\begin{align*}
				\Lambda^\prime\ast u+\Lambda\ast(\Lambda\ast u) & =u^{\prime\prime}.
			\end{align*}
		Now multiply both side by $u^{-1}=\mu$ (as proved in theorem \eqref{thm:mobiusinverse}) to get
			\begin{align*}
				\Lambda^\prime\ast(u\ast u^{-1})+\Lambda\ast(\Lambda\ast(u\ast u^{-1}))
					& =u^{\prime\prime}\ast u^\prime.
			\end{align*}
		Now, since $u \ast u^{-1} = I$ and $f \ast I = f$ for any arithmetic function $f$, we have
			\begin{align*}
				\Lambda^\prime+\Lambda\ast\Lambda & =u^{\prime\prime}\ast\mu.
			\end{align*}
		Replacing the functions with their definitions, one easily finds
			\begin{align*}
				\Lambda(n)\log n+\sum_{d\mid n}\Lambda(d)\Lambda\left(\dfrac nd\right) & =\sum_{d\mid n}\mu(d)\log^2\dfrac nd,
			\end{align*}
		as desired.
	\end{proof}

\end{document}
