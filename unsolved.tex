\documentclass[problems.tex]{subfile}

\begin{document}
\chapter{Practice Challenge Problems}\label{ch:unsolved}

	\numberwithin{theorem}{chapter}
	\begin{problem}
		Show that the ratio $\dfrac{\sigma(n)}{n}$ can be arbitrarily large for infinitely many $n$\watermark.
	\end{problem}

	\begin{problem}
		Prove that for all positive integers $n$, there exists an $n$-digit prime.
	\end{problem}

	\begin{problem}
		There are $n$ points on a circle with $n>10$, and each point is given a number that is equal to the average of the numbers of its two nearest neighbors. Show that all the numbers must be equal.
	\end{problem}

	\begin{problem}
		Let $F(n)$ be the $n$th Fibonacci number, show that for some $n>1$, $F(n)$ ends with $2007$ zeros.
	\end{problem}

	\begin{problem}
		Let $N=.23571113...$ where $N$ consists of all prime numbers concatenated together after the decimal. Determine if $N$ is rational or irrational.
	\end{problem}

	\begin{problem}
		Prove that there exists a positive integer $n$ such that the four leftmost digits of the decimal representation of $2n$ is $2007$.
	\end{problem}

	\begin{problem}
		The only sets of $N-1$ (distinct) integers, with no non-empty subset having its sum of elements divisible by $N$, are those where all integers are congruent to a same residue modulo $N$, coprime with $N$.
	\end{problem}

	\begin{problem}[Austrian Mathematical Olympiad, $2016$]
		Determine all composite positive integers $n$ with the following property: If $1=d_1<d_2<\cdots<d_k$ are the divisors of $n$ then
			\begin{align*}
				d_2-d_1:d_3-d_2:\cdots:d_k-d_{k-1} & = 1:2:\cdots k-1
			\end{align*}
	\end{problem}

	\begin{problem}[Belarus $2009$]
		Find all $m,n\in\mathbb{N}$ such that $m!+n!=m^n$.
	\end{problem}

	\begin{problem}
		Integer $n>2$ is given. Find the biggest integer $d$, for which holds, that from any set $S$ consisting of $n$ integers, we can find three different (but not necessarily disjoint) nonempty subsets, such that sum of elements of each of them is divisible by $d$.
	\end{problem}

	\begin{problem}
		Consider the set$ M = \{1, 2, 3, . . . , 2007\}$. Prove that in any way we choose the subset $X$ with $15$ elements of M there exist two disjoint subsets         $A$ and $B$ in $X$ such that the sum of the members of $A$ is equal to the sum of the members of $B$.
	\end{problem}

	\begin{problem}[India $2014$]
		Let $n\in\mathbb{N}$. Show that,
			\begin{align*}
				\left\lfloor\dfrac{n}{1}\right\rfloor+\left\lfloor\dfrac{n}{2}\right\rfloor+\ldots+\left\lfloor\dfrac{n}{n}\right\rfloor+
				\lfloor\sqrt{n}\rfloor
			\end{align*}
		is even.
	\end{problem}

	\begin{problem}
		Given $101$ distinct non-negative integers less than $5050$ show that one can choose four $a, b, c, d$ such that $a + b - c - d$ is a multiple of $5050$.
	\end{problem}

	\begin{problem}[Bulgarian Mathematical Olympiad, $2016$]
		Find all positive integers $m$ and $n$ such that $\left(2^{2^m}+1\right)\left(2^{2^n}+1\right)$ is divisible by $mn$.
	\end{problem}

	\begin{problem}[Slovenia $2010$]
		Find all prime numbers $p, q$ and $r$ such that $p > q > r$ and the numbers $p-q,p-r$ and $q-r$ are also prime.
	\end{problem}

	\begin{problem}[Croatia Mathematical Olympiad, First Round, $2016$]
		Determine the number of positive integers smaller than $1000000$, that are also perfect squares and give a remainder $4$ when divided by $8$.
	\end{problem}

	\begin{problem}
		Prove that among $81$ natural numbers whose prime divisors are in the set $\{2, 3, 5\}$ there exist four numbers whose product is the fourth power of an integer.
	\end{problem}



	\begin{problem}
		We chose $n+2$ numbers from set $\{1,2,...3n\}$. Prove that there are always two among the chosen numbers whose difference is more than $n$ but less than $2n$.
	\end{problem}

	\begin{problem}[India $2014$]
		Let $a, b$ be natural numbers with $ab > 2$. Suppose that the sum of their greatest common divisor and least common multiple is divisble by $a+ b$. Prove that the quotient is at most $\dfrac{a+b}{4}$. When is this quotient exactly equal to $\dfrac{a+b}{4}$?
	\end{problem}


	\begin{problem}
		The integers $1,...,n$ are arranged in any order. In one step any two neighboring integers may be interchanged. Prove that the initial order can never be reached after an odd number of steps.
	\end{problem}

	\begin{problem}
		A palindrome is a number or word that is the same when read forward and backward, for example, $176671$ and $civic$. Can the number obtained by writing the numbers from $1$ to $n$ in order (for some $n>1$) be a palindrome?
	\end{problem}

	\begin{problem}[IMO Shortlist N2, Proposed by Jorge Tipe, Peru]
		A positive integer $N$ is called balanced, if $N=1$ or if $N$ can be written as a product of an even number of not necessarily distinct primes. Given positive integers $a$ and $b$, consider the polynomial $P$ defined by $P(x)=(x+a)(x+b)$.
			\begin{enumerate}[(a)]
				\item Prove that there exist distinct positive integers $a$ and $b$ such that all the number $P(1)$, $P(2)$,$\ldots$, $P(50)$ are balanced.
				\item Prove that if $P(n)$ is balanced for all positive integers $n$, then $a=b$.
			\end{enumerate}
	\end{problem}

	\begin{problem}
		Let $n$ be an integer. Prove that if the equation $x^2+xy+y^2=n$ has a rational solution, then it also has an integer solution.
	\end{problem}

	\begin{problem}[Iran Olympiad, Thrd Round]
		Let $p$ be a prime number. Prove that, there exists integers $x,y$ such that $p=2x^2+3y^2$ if and only if $p\equiv5,11\pmod{24}$.
	\end{problem}

	\begin{problem}[Polish Math Olympiad]
		Let $S$ be a set of all positive integers which can be represented as $ a^2 + 5b^2$ for some integers $ a,b$ such that $ a\bot b$. Let $p$ be a prime number such that $ p = 4n + 3$ for some integer $ n$. Show that if for some positive integer $ k$ the number $ kp$ is in $S$, then $2p$ is in $ S$ as well.
	\end{problem}

	\begin{problem}
		Prove that the equation $x^3-x+9=5y^2$ has no solution in integers.
	\end{problem}

	\begin{problem}[India TST]
		On the real number line, paint red all points that correspond to integers of the form $81x+100y$, where $x$ and $y$ are positive integers. Paint the remaining integer point blue. Find a point $P$ on the line such that, for every integer point $T$, the reflection of $T$ with respect to $P$ is an integer point of a different colour than $T$.
	\end{problem}

	\begin{problem}
		Prove that, for any positive integer $k$, there are positive integers $a,b>1$ such that
		\begin{align*}
		k  & = \dfrac{a^2+b^2-1}{ab}
		\end{align*}
	\end{problem}

	\begin{problem}
		Let $a, b,$ and $c$ be positive integers such that $0 \leq a^2+b^2-abc \leq c$. Prove that $a^2+b^2-abc$ is a perfect square.
	\end{problem}

	\begin{problem}
		Let the $ n^{\text{th}}$ Lemur set, $ L_n$, be the set composed of all positive integers that are equal to the sum of the squares of their first $ n$ divisors. For example, $ L_1 = \{1\}$, $ L_2 = \{\}$, and $ L_4 = \{130\}$.
			\begin{enumerate}[a]
				\item Find $ L_3$, $ L_5$, and $ L_6$.
				\item Describe all $ n$ for which $ L_n$ is empty.
				\item Describe all $ n$ for which $ L_n$ is infinite.
				\item Provide a method for finding members of non-empty Lemur sets.
			\end{enumerate}
	\end{problem}


	\begin{problem}[IMO $2007$, Problem $5$]
		Let $a$ and $b$ be positive integers so that $4ab-1$ divides $(4a^2-1)^2$. Prove that $a=b$.
	\end{problem}

	\begin{problem}\label{prob:mirzakhani9}
		Let $a_1,a_2,\ldots,a_n$ be positive integers such that $a_1<a_2<\cdots < a_n$. Prove that
		\begin{align*}
			\sum_{i=1}^{n-1} \frac{1}{[a_i,a_{i-1}]} & <1
		\end{align*}
	\end{problem}

	\begin{problem}
		Let $N=2^{p_1\cdots p_n}+1$ where $p_i$ are distinct primes greater than $2$ and $\tau(N)$ is the number of divisors of $N$. Maximize $\tau(N)$.
	\end{problem}

	\begin{problem}[Masum Billal]
		Let $n\geq3,a,d$ be positive integers so that $a,a+d,\ldots,a+(n-1)d$ are all primes. If $\lambda(n)$ is the number of primes \textit{strictly less than }$n$, prove that,
			\begin{align*}
				N & = 2^{\left\lfloor\frac{d}{2}\right\rfloor}+1
			\end{align*}
		has at least $2^{2^{\lambda(n)-2}-1}$ divisors.
	\end{problem}

	\begin{problem}
		For $n\geq2$,
		\begin{align*}
			F_n^{\frac{F_{n+m}-1}{2}} &\equiv1\pmod{F_{n+m}}
		\end{align*}
	\end{problem}

	\begin{problem}
		Let $f(x) = x^3 + 17$. Prove that for each natural number $n\geq 2$, there is a natural number $x$ for which $f(x)$ is divisible by $3^n$ but not $3^{n+1}$.
	\end{problem}

	\begin{problem}[Boylai]
		Show that every Fermat prime is of the form $6k-1$.
	\end{problem}

	\begin{problem}[Iran TST $2015$]
		We are given three natural numbers $a_1,a_2,a_3$. For $n\geq3$,
			\begin{align*}
				a_{n+1} = [a_{n-1},a_n]-[a_{n-1},a_{n-2}]
			\end{align*}
		Prove that there exists an index $k\leq a_3+4$ such that $a_k\leq0$.
	\end{problem}

	\begin{problem}[Bosnia Olympiad $2013$, Second Day]
		Find all primes $p$ and $q$ such that $p \mid  30q-1$ and $q \mid  30p-1$.
	\end{problem}

	\begin{problem}[IMO Shortlist $2004$, N$3$, Proposed by Iran]
		$f$ is a function with $f:\mathbb{N}\to\mathbb{N} $ so that
			\begin{align*}
				f^2(m)+f(n) & \mid  (m^2+n)^2
			\end{align*}
		Show that $f(n)=n$.
	\end{problem}

	\begin{problem}[Columbia $2010$]
		Find all pairs of positive integers $(m,n)$ such that $m^2+n^2=(m+1)(n+1)$.
	\end{problem}

	\begin{problem}[AMOC $2014$, Senior Section]
		For which integers $n\geq2$ is it possible to separate the numbers $1, 2,\ldots, n$ into two sets such that the sum of the numbers in one of the sets is equal to the product of the numbers in the other set?
	\end{problem}

	\begin{problem}[Greece]
		Determine all triples $(p, m, n)$ of positive integers such that $p$ is a prime number and $p^m-8=n^3$.
	\end{problem}

	\begin{problem}[Canadian Students Math Olympiad $2011$]
		For a fixed positive integer $k$, prove that there exist infinitely many primes $p$ such that there is an integer $w$, where $w^2-1$ is not divisible by $p$, and the order of $w$ modulo $p$ is the same as the order of $w$ modulo $p^k$.
	\end{problem}

	\begin{problem}[China TST $2009$]
		Let $a>b>1$ and $b$ be an odd integer, $n\in\mathbb{N}$. If $b^n\mid a^n-1$, then prove that $a^b>\dfrac{3^n}{n}$.
	\end{problem}

	\begin{problem}[Kazakhstan $2015$]
		Solve in positive integers: $x^yy^x = (x+y)^z$.
	\end{problem}

	\begin{problem}[Kazakhstan $2015$]
		$P_k(n)$ is the product of all divisors of $n$ that are divisible by $k$ (in empty case it is $1$). Prove that, $P_1(n)\cdot P_2(n)\cdots P_n(n)$ is a perfect square.
	\end{problem}

	\begin{problem}[Russia $2000$]
		If a perfect number greater than $6$ is divisible by $3$, it is also divisible by $9$. If a perfect number greater than $28$ is divisible by $7$, it is also divisible by $49$.
	\end{problem}

	\begin{problem}[Croatia $2014$]
		For a positive integer $n$ denote by $s(n)$ the sum of all positive divisors of $n$ and by $d(n)$ the number of positive divisors of $n$. Determine all positive integers $n$ such that
		\begin{align*}
			s(n) & = n+d(n)+1
		\end{align*}
	\end{problem}

	\begin{problem}[IMO $2003$, Problem $2$, N$3$]
		Find all pairs of positive integers $(a,b)$ such that
		\begin{align*}
		\dfrac{a^2}{2ab^2-b^3+1}
		\end{align*}
		is a positive integer.
	\end{problem}

	\begin{problem}
		Show that $n=\varphi(n)+\tau(n)$ if and only if $n$ is a prime.
	\end{problem}

	\begin{problem}[IMO Shortlist $2004$, N$2$ Part (c)]
		Find all $n$ for which $f(n)=an$ has a unique solution where,
			\begin{align*}
			f(n)=\sum\limits_{i=1}^n(i,n)
			\end{align*}
	\end{problem}

	\begin{problem}
		Let $k$ be a positive integer. Find all positive integers $n$ such that $3^k \mid 2^n -1.$
	\end{problem}

	\begin{problem}
		Let $a,b$ be distinct real numbers such that the numbers
		\[a-b, \ a^2-b^2 , \ a^3-b^3 ,  \ldots\]
		are all integers. Prove that $a,b$ are both integers.
	\end{problem}

	\begin{problem}[MOSP 2001]
		Find all quadruples of positive integers $(x,r,p,n)$ such that $p$ is a prime number, $n,r>1$ and $x^{r} - 1 = p^{n} .$
	\end{problem}

	\begin{problem}[China TST 2009]
		Let $ a > b > 1$ be positive integers and $b$ be an odd number, let $ n$ be a positive integer. If $ b^n \mid a^n-1,$ then show that $ a^b > \frac {3^n}{n}.$
	\end{problem}

	\begin{problem}[Romanian Junior Balkan TST 2008]
		Let $ p$ be a prime number, $ p\neq 3$, and integers $ a,b$ such that $ p\mid a + b$ and $ p^2 \mid a^3 + b^3$. Prove that $ p^2 \mid a + b$ or $ p^3 \mid a^3 + b^3$.
	\end{problem}

	\begin{problem}
		Let $m$ and $n$ be positive integers. Prove that for each odd positive integer $b$ there are infinitely many primes $p$ such that $p^n \equiv 1 \pmod{b^m}$ implies $b^{m-1} \mid n.$
	\end{problem}

	\begin{problem}[IMO 1990]
		Determine all integers $ n > 1$ such that
		\[ \frac {2^n + 1}{n^2}\]
		is an integer.
	\end{problem}

	\begin{problem}
		Find all positive integers $n$ such that
		\[\frac{2^{n-1}+1}{n}\]
		is an integer.
	\end{problem}

	\begin{problem}
		Find all primes $p,q$ such that $\dfrac{(5^p-2^p)(5^q-2^q)}{pq}$ is an integer.
	\end{problem}

	\begin{problem}
		For some natural number $n$ let $a$ be the greatest natural number for which $5^{n}-3^{n}$ is divisible by $2^{a}$. Also let $b$ be the greatest natural number such that $2^{b} \leq n$. Prove that $a \leq b+3$.
	\end{problem}


	\begin{problem}
		Determine all sets of non-negative integers $ x, y$ and $ z$ which satisfy the equation
		\[
			2^{x}+3^{y}=z^{2}
		\]
	\end{problem}

	\begin{problem}[IMO ShortList 2007]
		Find all surjective functions $ f: \mathbb{N} \to \mathbb{N}$ such that for every $ m,n \in \mathbb{N}$ and every prime $ p,$ the number $ f(m + n)$ is divisible by $ p$ if and only if $ f(m)+ f(n)$ is divisible by $ p.$
	\end{problem}

	\begin{problem}[Romania TST 1994]
		Let $ n$ be an odd positive integer. Prove that $((n-1)^n+1)^2$ divides $ n(n-1)^{(n-1)^n+1}+n$.
	\end{problem}

	\begin{problem}
		Find all positive integers $n$ such that $3^{n}-1$ is divisible by $2^n$.
	\end{problem}

	\begin{problem}[Romania TST 2009]
		Let $ a,n\geq 2$ be two integers, which have the following property: there exists an integer $ k\geq 2,$ such that $ n$ divides $ (a-1)^k.$ Prove that $ n$ also divides $ a^{n-1}+a^{n-2}+\cdots + a + 1.$
	\end{problem}

	\begin{problem}
		Find all the positive integers $a$ such that $\frac{5^a + 1}{3^a}$ is a positive integer.
	\end{problem}

	\begin{problem}
		Find all primes $p,q$ such that $pq\mid 5^p+5^q$.
	\end{problem}

	\begin{problem}
		Find all primes $p,q$ such that $pq\mid 2^p+2^q$.
	\end{problem}
\end{document}