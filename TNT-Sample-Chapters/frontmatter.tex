\documentclass[main.tex]{subfile}


\begin{document}
	
	
	\section*{Foreword}
	\newpage
	
	\section*{Preface}
	
	Number theory is acknowledged as the queen of mathematics. It is probably the most enriched and researched section of mathematical literature.  
	
	This book is the result of five or six years (more or less) of our lives solving Olympiad problems in AoPS, training in various math camps, and reading many books, articles or papers of the literature. But, we want to keep this book as self-contained as possible. We will provide some theorems and lemmas which have become more popular in olympiads over the past few years along with problems for demonstration. But we may mention some theorems without proof just to interest readers. We will hardly use theorems like that. If we use one, it will be only because the theorem is related to the topic and really useful for solving problems but the proof is out of our scope.
	
	Many of the topics have been observed on \texttt{artofproblemsolving.com}, widely known as {AoPS}, and some of them were published 
	as articles. Another purpose of this book is to introduce some newly popularized theorems or lemmas and some advanced techniques as well. We should mention that while writing the book, we didn't care about making the book an organized one or a formal one. Therefore, there is no fixed place for the exercises or extra problems, you can find them all over the book, associated with sections or even subsections. This means the problems are topic based in the books. This is good in a way, since the reader can understand the common situations where they can be invoked. But if we kept all the problems same way, it may pose an issue for many readers - having the perfect intuition to know why this theorem had to be used here. As for the remedy, we have attached a final chapter dedicated to problems only. All problems posed there are totally random and they will not be in ascending or descending order of difficulty. This way, the reader has no idea which theorem to use or how to think on it. Even it is possible that no theorem is required at all. Specially if it is a combinatorial number theory problem. Anyway, we think we have made the point.
	
	This book assumes the reader is a total beginner, though it won't matter if the reader is familiar with number theory beforehand. The prerequisite is only basic definitions taught in $8/9$th grade school math, for example logarithms, induction or functions. The only thing that may not be discussed at that level may be binomial coefficients. If you don't know something related to those, first consult with some elementary books which cover a minimum discussion on the topic. We wanted to keep those in the book too, but that would make the book way too large. Our primary target is to improve problem solving skills through the whole book. Finally, we wanted to include more topics but the contents in this book is already a lot. So we couldn't add more important topics like Diophantine equations. Instead, we plan to write them in the next volumes. But that doesn't mean one can't use these techniques to solve problems of that area. We have a plan to make at least three volumes of this series. We will continue there.
	
	\begin{flushright}
		\sl Masum Billal\\
		Amir Hossein Parvardi
	\end{flushright}
	
	
	\section*{Instruction for Readers}

	\begin{enumerate}[i.] 
		\item You may notice that we started quite slowly in chapter one and two. But that's because it's prerequisite for the rest of the book and then we raised the pace. If you are a beginner, make sure you understand every line of the first two chapters, and solve most of the problems.
		\item If you are already familiar with number theory and know some basic definitions, give chapter \eqref{ch:cnt} a try first. That will probably maximize your output from this book. If you think it's too much, just continue how you wanted to at first.
		\item When it comes to definitions, we may not be quite rigorous sometimes, so that the reader, who may be a beginner, can make proper sense of what he or she is reading. But it's fair to warn that you have to think for yourself through the whole document to understand everything properly. In most cases, you need to use a notebook and pen to judge the arguments made or facts stated.
		\item Another issue you should keep in mind is to make sure that you make proper sense of the definitions. Otherwise you cannot go forward. One may also notice that at first we have started slowly, but as we dig deeper the speed gets higher. One may comment that the topics we selected are somewhat unusual for an elementary number theory book. But that's the whole reason why this book is being written -- to provide knowledge of the things that are not so well known to olympiad problem solvers.
		\item As the last sentence, notice that learning only theorems is not the way to learn number theory (and generally the whole mathematics). In order to deliver this important message, we have included a chapter (second) which uses almost no theorems. And don't just memorize theorems! It may seem advantageous to you at first, but it will destroy your number theory skills. There are a bunch of problems waiting for you at the end of each section, do not disappoint them!
		\item And finally, good luck!
	\end{enumerate}
	
	
\end{document}