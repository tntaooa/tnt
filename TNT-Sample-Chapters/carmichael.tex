\documentclass{subfile}


\begin{document}
	\subsection{Carmichael $\lambda$ Function}
	In this section, we discuss a very important function, first introduced by \textit{Robert Daniel Carmicheal} in  \cite{ch:congruence-carmichael-original}. Consider the following scenario: coprime positive integers $a$ and $n$ are given, and $\ord_n(a)=d$. Now, fix $n$. Consider the case when $a^d\equiv1\pmod n$ holds for any positive integer $a$ coprime to $n$. This brings up some questions.
	\begin{problem}\label{prob:CarmichaelQuestion1}
		Does there exists an $a$ such that $\ord_n(a)=d$?
	\end{problem}
	
	\begin{problem}\label{prob:CarmichaelQuestion2}
		How do we find the minimum $d$ such that $a^d\equiv1\pmod n$ holds for any $a$ coprime to $n$?
	\end{problem}
	Let's proceed slowly. We will develop the theories that can solve these problems. For doing that, we have to use properties of order and primitive roots we discussed in previous sections.
	\begin{definition}[Carmichael Function]
		For a positive integer $n$, $\l(n)$ is the smallest positive integer for which $a^{\l(n)}\equiv1\pmod n$ holds for any positive integer $a$ relatively prime to $n$. This number $\l(n)$ is called the \textit{Carmichael function} of $n$. Sometimes, it is also called the \textit{lambda function} of $n$.
	\end{definition}
	Note that Theorem \eqref{thm:ordDiv} implies the following theorem.
	\begin{theorem}\slshape\label{thm:carDiv}
		If $a^d\equiv1\pmod n$ holds for all $a$ coprime to $n$, then $\l(n)\mid d$.
	\end{theorem}
	
	\begin{corollary}\label{cor:LambdaDividesPhi}
		For any positive integer $n$, $\l(n)\mid \t(n)$.
	\end{corollary}
	The following theorem is self-implicating and solves the first problem, if we can prove that $\l(n)$ exists. For now, let's assume it does.
	\begin{theorem}\slshape
		Let $n$ be a positive integer. There exists a positive integer $a$ coprime to $n$ such that $\ord_n(a)=\l(n)$.
	\end{theorem}
	Let's focus on finding $\l(n)$. First, consider the case $n=2^k$.
	\begin{theorem}\slshape
		If $k>2$ then $\l(2^k)=2^{k-2}$.
	\end{theorem}
	
	\begin{proof}
		The integers coprime to $2^k$ are all odd numbers. We will prove by induction that $x^{2^{k-2}} \equiv1\pmod{2^k}$ holds for all odd positive integers $x$. The base case $k=3$ is obvious. Assume that for some $k\geq 3$, we have
			\begin{align*}
				x^{2^{k-2}} & \equiv1\pmod{2^k},
			\end{align*}
		or equivalently, $x^{2^{k-2}} -1=2^kt$ for some $t$. Using the identity $a^2-b^2=(a-b)(a+b)$, we can write
			\begin{align}
				x^{2^{k-1}} -1 = \left(x^{2^{k-2}} -1\right)\left(x^{2^{k-2}} +1\right) = 2^kt\left(2^kt +2\right) = 2^{k+1}t \left(2^{k-1}t+1\right). \label{eq:x^{2^{k-1}}-1}
			\end{align}
		This gives $x^{2^{k-1}} \equiv 1\pmod{2^{k+1}}$, and the induction is complete.
		
		Now, we should prove that $2^{k-2}$ indeed is the smallest such integer. Again, by induction, the base case is to find an $x$ for which $\ord_8(x)=2$. Obviously, any $x=8j\pm3$ satisfies this condition. Assume that for all numbers $t$ from $1$ up to $k$, we have $\l(2^l)=2^{l-2}$. Let $\l(2^{k+1})=\l$. Since we proved that $x^{2^{k-1}} \equiv 1\pmod{2^{k+1}}$ for all odd $x$, it follows from Theorem \ref{thm:carDiv} that $\l \mid 2^{k-1}$. So $\l$ is a power of $2$. If $\l = 2^{k-1}$, we are done. Otherwise, let $\l=2^\a$, where $1 \leq \a <k-1$. Then for every $x$, one can write
			\begin{align}\label{eq:x^2^k+1}
				x^{2^\a} \equiv 1 \pmod{2^{k+1}}.
			\end{align}
		However, similarly as in \eqref{eq:x^{2^{k-1}}-1}, for some $t$,
			\begin{align}\label{eq:x^2^a}
				x^{2^{\a}} -1 &= 2^{\a+2}t \left(2^{\a}t+1\right).
			\end{align}
		In \eqref{eq:x^2^a}, the highest power of $2$ which divides $x^{2^{\a}} -1$ is $2^{\a+2}$ (since $2^{\a}t+1$ is odd). But $$\a+2 <(k-1)+2=k+1$$, which contradicts \eqref{eq:x^2^k+1}. The induction is complete.
	\end{proof}
	
	\begin{theorem}\slshape
		For any prime $p$ and any positive integer $k$, 
		\[\l(p^k)=\l(2p^k)=\t(p^k).\]
	\end{theorem}
	
	\begin{proof}
		Consider the congruence equation $x^d\equiv1\pmod{p^k}$ and let $d=\l(p^k)$. By Corollary \ref{cor:LambdaDividesPhi}, $d \mid \t(p^k)$. Take $x=g$ where $g$ is a primitive root modulo $p^k$. Then, $\ord_{p^k}(g)=\t(p^k)$ and we immediately have $\t(p^k)|d$. Thus, $d=\t(p^k)$. A very similar proof can be stated to show that $\l (2p^k)=\t (2p^k) = \t (p^k)$.
	\end{proof}
	
	\begin{theorem}\slshape
		Let $a$ and $b$ be coprime positive integers. Then
		 \[\l(ab)=\lcm(\l(a),\l(b)).\]
	\end{theorem}
	
	\begin{proof}
		Suppose that $$\l(a)=d, \quad \l(b)=e, \quad \text{and} \quad \l(ab)=h.$$ Then, $$x^d\equiv1\pmod a, \quad x^e\equiv1\pmod b, \quad \text{and} \quad x^{h}\equiv1\pmod{ab}.$$ We also have $x^h\equiv1\pmod a$ and $x^h\equiv1\pmod b$ as well. Hence, $d \mid h$ and $e \mid h$. This means that $[d,e]=h$ since $[d,e]$ is the smallest positive integer that is divisible by both $d$ and $e$.
	\end{proof}
	Generalization of this theorem is as follows.
	\begin{theorem}\slshape
		For any two positive integers $a$ and $b$,
		\begin{align*}
		\lcm(\l(a),\l(b)) & = \l(\lcm(a,b)).
		\end{align*}
	\end{theorem}
	The next theorem combines the above results and finds $\l(n)$ for all $n$.
	\begin{theorem}\slshape\label{thm:CarmichaelFormula}
		Let $n$ be a positive integer with prime factorization $n=p_1^{e_1}p_2^{e_2}\cdots p_r^{e_r}$. Also, let $p$ be a prime and $k$ be a positive integer. Then
		\begin{align*}
		\l(n) & = 
		\begin{cases}
		\t(n),&\text{ if } n = 2,4,p^k, \text{ or } 2p^k,\\
		\dfrac{\t(n)}{2},& \text{ if }n=2^k\text{ with }k>2,\\
		\lcm(\l(p_1^{e_1}),\ldots,\l(p_r^{e_r})), & \text{ otherwise.}
		\end{cases}
		\end{align*}
	\end{theorem}
	
	\begin{theorem}\slshape
		For positive integers $a$ and $b$, if $a \mid b$, then $\l(a) \mid \l(b)$.
	\end{theorem}
	The proof is left as an exercise for the reader. We are now ready to fully solve Problem \ref{prob:CarmichaelQuestion1}.
	\begin{theorem}\slshape
		For fixed positive integers $n$ and $d$, there exists a positive integer $a$  coprime to $n$ so that $\ord_n(a)=d$ if and only if $d \mid \l(n)$.
	\end{theorem}
	
	\begin{proof}
		The ``if'' part is true by Theorem \ref{thm:ordDiv}. For the ``only if'' part, assume that $g$ is an integer with $\ord_n(g)=\l(n)$ and $\l(n)=de$. Then $\ord_n(g^e)=d$, as desired.
	\end{proof}
	
	We finish this section by proposing a theorem. We will leave the proof for the reader as an exercise.
		\begin{theorem}\slshape
			If $\l(n)$ is coprime to $n$, then $n$ is square-free.
		\end{theorem}
	Recall that $n$ is square-free if it is not divisible by any perfect square other than $1$.
	
	\subsection{Primitive $\l$-roots}
	Carmichael defined a generalization of primitive roots as follows using his function. As you will see, this section generalizes everything related to primitive roots.
	\begin{definition}[Primitive $\l$-root]
		Let $a$ and $n$ be coprime positive integers. If $\ord_n(a)=\l(n)$, then $a$ is a primitive $\l$-root modulo $n$. That is, $a^{\l(n)}$ is the smallest power of $a$ which is congruent to $1$ modulo $n$. 
	\end{definition}

	\begin{definition}
		Let $n$ be a positive integer. Define $$\xi(n) = \frac{\t(n)}{\l(n)}.$$ (Read $\xi$ as ``ksi''). According to Corollary \ref{cor:LambdaDividesPhi}, $\xi(n)$ is an integer.
	\end{definition}

	\begin{proposition}
		 There is a primitive root (defined in the previous section) modulo $n$ if and only if $\xi(n)=1$. Carmichael calls a primitive root a $\varphi$-primitive root, and they are in fact a special case of $\lambda$-primitive roots.
	\end{proposition}

	Now, the existence of primitive root is generalized to the following theorem, cited from Carmichael's paper \cite{ch:congruence-carmichael-original}.
	\begin{theorem}[Carmicahel]\slshape
		For any positive integer $n$, the congruence equation
			\begin{align*}
				x^{\l(n)} & \equiv1\pmod n
			\end{align*}
		has a solution $a$ which is a primitive $\lambda$-root, and for any such $a$, there are $\t(\l(n))$ primitive roots congruent to powers of $a$.
	\end{theorem}
	We can show that this theorem is true in a similar fashion to what we did in last section, and we leave it as an exercise.
	
	As we mentioned earlier in Proposition \ref{prop:phiproperties}, $\varphi(n)$ is always even for $n>2$. As it turns out, $\lambda$ and $\varphi$ share more common properties than we think. The following next two problems, which were taken from \cite{ch:congruence-primitive-lambda}, demonstrate this.
	
	\begin{problem}
		For any integer $n\geq 1$, either $\xi(n)=1$ or $\xi(n)$ is even.
	\end{problem}

	\begin{hint}
		Use the formula for $\lambda(n)$ in Theorem \ref{thm:CarmichaelFormula}.
	\end{hint}

	\begin{problem}
		If $\lambda(n)>2$, the number of primitive $\lambda$-roots modulo $n$ is even.
	\end{problem}

	

	The next theorem generalizes Theorem \ref{thm:genWilson}, which itself was a generalization to Wilson's theorem. 
	\begin{theorem}\slshape
		Let $n$ be a positive integer such that $\l(n)>2$. Also, suppose that $g$ is a primitive $\l$-root modulo $n$. The product of primitive $\l$-roots of $n$ is congtuent to $1$ modulo $n$.
	\end{theorem}
	
	\begin{proof}
		If $g$ is a primitive $\l$-root modulo $n$, all the primitive $\l$-roots are $$\{g^{e_1}, g^{e_2}, \ldots, g^{e_{k}}\},$$ where $e_i$ (for $1 \leq i \leq k$) are all (distinct) positive integers with $(e_i, \lambda(n))=1$. Also, note that we can pair them up since $\l(n)$ is even if $n>2$. In fact, we can pair $g^{e_i}$ with $g^{\lambda(n)-e_i}$ for all $i$. Then,
		\begin{equation*}
		g^{e_1}\cdot g^{e_2}\cdots g^{e_k}  \equiv g^{\l(n)}\cdots g^{\l(n)}  \equiv 1\pmod n. \qedhere
		\end{equation*}
	\end{proof}
	
	\begin{theorem}\slshape
		For any $n$, there are $\t(\l(n))$ primitive $\l$-roots modulo $n$.
	\end{theorem}
	The proof is similar to that of Theorem \ref{thm:npr}, so we skip it.
	

\end{document}