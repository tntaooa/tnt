\documentclass{subfile}


\begin{document}
	\subsection{Carmichael \texorpdfstring{$\lambda$}{Lambda} Function}
	In this section, we discuss a very important function in number theory. \textcite{carmichael_1910} first introduced it for generalizing Euler's totient function. Consider that two relatively prime positive integers $a,n$ are given, and $\ord_n(a)=d$. Now, fix $n$. Consider the case when $a^d\equiv1\pmod n$ holds for any positive integer $a$ relatively prime to $n$. This brings up some questions.
	\begin{problem}\label{prob:CarmichaelQuestion1}
		Does there exists an $a$ such that $\ord_n(a)=d$?
	\end{problem}

	\begin{problem}\label{prob:CarmichaelQuestion2}
		How do we find the minimum $d$ such that $a^d\equiv1\pmod n$ holds for any $a$ relatively prime to $n$?
	\end{problem}
	Let's proceed slowly. We will develop the theories that can solve these problems. For doing that, we have to use properties of order and primitive roots we discussed in previous sections.
	\begin{definition}[Carmichael Function]
		For a positive integer $n$, $\lambda(n)$ is the smallest positive integer for which $a^{\lambda(n)}\equiv1\pmod n$ holds for any positive integer $a$ relatively prime to $n$. This number $\lambda(n)$ is called the \textit{Carmichael function} of $n$. Sometimes, it is also called the \textit{lambda function} of $n$ or the \textit{minimum universal exponent} $\pmod m$ \textcite[Chapter $\S$VI, Section $4$, Page $265$]{sierpinski_schinzel_1988}.
	\end{definition}
	Note that Theorem \eqref{thm:ordDiv} implies the following theorem.
	\begin{theorem}\slshape\label{thm:carDiv}
		If $a^d\equiv1\pmod n$ holds for all $a$ relatively prime to $n$, then $\lambda(n)\mid d$.
	\end{theorem}

	\begin{corollary}\label{cor:LambdaDividesPhi}
		For any positive integer $n$, $\lambda(n)\mid \varphi(n)$.
	\end{corollary}
	The following theorem is self-implicating and solves the first problem, if we can prove that $\lambda(n)$ exists. For now, let's assume it does.
	\begin{theorem}\slshape
		Let $n$ be a positive integer. There exists a positive integer $a$ relatively prime to $n$ such that $\ord_n(a)=\lambda(n)$.
	\end{theorem}
	Let's focus on finding $\lambda(n)$. First, consider the case $n=2^k$.
	\begin{theorem}\slshape
		If $k>2$ then $\lambda(2^k)=2^{k-2}$.
	\end{theorem}

	\begin{proof}
		The integers relatively prime to $2^k$ are all odd numbers. We will prove by induction that $x^{2^{k-2}} \equiv1\pmod{2^k}$ holds for all odd positive integers $x$. The base case $k=3$ is obvious. Assume that for some $k\geq 3$, we have
			\begin{align*}
				x^{2^{k-2}} & \equiv1\pmod{2^k}
			\end{align*}
		or equivalently, $x^{2^{k-2}} -1=2^kt$ for some $t$. Using the identity $a^2-b^2=(a-b)(a+b)$, we can write
			\begin{align}
				x^{2^{k-1}} -1
					& = \left(x^{2^{k-2}} -1\right)\left(x^{2^{k-2}} +1\right)\\
					& = 2^kt\left(2^kt +2\right)\\
					& = 2^{k+1}t \left(2^{k-1}t+1\right) \label{eq:x^{2^{k-1}}-1}
			\end{align}
		This gives $x^{2^{k-1}} \equiv 1\pmod{2^{k+1}}$, and the induction is complete.

		Now, we should prove that $2^{k-2}$ indeed is the smallest such integer. Again, by induction, the base case is to find an $x$ for which $\ord_8(x)=2$. Obviously, any $x=8j\pm3$ satisfies this condition. Assume that for all numbers $t$ from $1$ up to $k$, we have $\lambda(2^l)=2^{l-2}$. Let $\lambda(2^{k+1})=\lambda$. Since we proved that $x^{2^{k-1}} \equiv 1\pmod{2^{k+1}}$ for all odd $x$, it follows from \autoref{thm:carDiv} that $\lambda \mid 2^{k-1}$. So $\lambda$ is a power of $2$. If $\lambda = 2^{k-1}$, we are done. Otherwise, let $\lambda=2^\alpha$, where $1 \leq \alpha <k-1$. Then for every $x$, one can write
			\begin{align}\label{eq:x^2^k+1}
				x^{2^\alpha}
					& \equiv 1 \pmod{2^{k+1}}
			\end{align}
		However, similarly as in \eqref{eq:x^{2^{k-1}}-1}, for some $t$,
			\begin{align}\label{eq:x^2^a}
				x^{2^{\alpha}} -1
					&= 2^{\alpha+2}t \left(2^{\alpha}t+1\right)
			\end{align}
		In \eqref{eq:x^2^a}, the highest power of $2$ which divides $x^{2^{\alpha}} -1$ is $2^{\alpha+2}$ (since $2^{\alpha}t+1$ is odd). But $$\alpha+2 <(k-1)+2=k+1$$, which contradicts \eqref{eq:x^2^k+1}. The induction is complete.
	\end{proof}

	\begin{theorem}\slshape
		For any prime $p$ and any positive integer $k$,
			\begin{align*}
				\lambda(p^k)
					& =\lambda(2p^k)\\
					& =\varphi(p^k)
			\end{align*}
	\end{theorem}

	\begin{proof}
		Consider the congruence equation $x^d\equiv1\pmod{p^k}$ and let $d=\lambda(p^k)$. By Corollary \ref{cor:LambdaDividesPhi}, $d \mid \varphi(p^k)$. Take $x=g$ where $g$ is a primitive root modulo $p^k$. Then, $\ord_{p^k}(g)=\varphi(p^k)$ and we immediately have $\varphi(p^k)|d$. Thus, $d=\varphi(p^k)$. A very similar proof can be stated to show that $\lambda (2p^k)=\varphi (2p^k) = \varphi(p^k)$.
	\end{proof}

	\begin{theorem}\slshape
		Let $a$ and $b$ be relatively prime positive integers. Then
		 \[\lambda(ab)=\lcm(\lambda(a),\lambda(b))\]
	\end{theorem}

	\begin{proof}
		Suppose that
			\begin{align*}
				\lambda(a)
					& =d\\
				\lambda(b)
					& =e\\
				\lambda(ab)
					& =h
			\end{align*}
		Then
			\begin{align*}
				x^d
					& \equiv1\pmod a\\
				x^e
					& \equiv1\pmod b\\
				x^{h}
					& \equiv1\pmod{ab}
			\end{align*}
		We also have $x^h\equiv1\pmod a$ and $x^h\equiv1\pmod b$ as well. Hence, $d \mid h$ and $e \mid h$. This means that $[d,e]=h$ since $[d,e]$ is the smallest positive integer that is divisible by both $d$ and $e$.
	\end{proof}
	Generalization of this theorem is as follows.
	\begin{theorem}\slshape
		For any two positive integers $a$ and $b$,
		\begin{align*}
		\lcm(\lambda(a),\lambda(b)) & = \lambda(\lcm(a,b))
		\end{align*}
	\end{theorem}
	The next theorem combines the above results and finds $\lambda(n)$ for all $n$.
	\begin{theorem}\slshape\label{thm:CarmichaelFormula}
		Let $n$ be a positive integer with prime factorization $n=p_1^{e_1}p_2^{e_2}\cdots p_r^{e_r}$. Also, let $p$ be a prime and $k$ be a positive integer. Then
		\begin{align*}
			\lambda(n) & =
				\begin{cases}
					\varphi(n)& \mbox{ if } n\in\{2,4,p^k,2p^k\}\\
					\dfrac{\varphi(n)}{2}& \mbox{ if }n=2^k\mbox{ with }k>2\\
					\lcm(\lambda(p_1^{e_1}),\cdots,\lambda(p_r^{e_r}))& \mbox{ otherwise}
				\end{cases}
		\end{align*}
	\end{theorem}

	\begin{theorem}\slshape
		For positive integers $a$ and $b$, if $a \mid b$, then $\lambda(a) \mid \lambda(b)$.
	\end{theorem}
	The proof is left as an exercise for the reader. We are now ready to fully solve Problem \ref{prob:CarmichaelQuestion1}.
	\begin{theorem}\slshape
		For fixed positive integers $n$ and $d$, there exists a positive integer $a$  relatively prime to $n$ so that $\ord_n(a)=d$ if and only if $d \mid \lambda(n)$.
	\end{theorem}

	\begin{proof}
		The ``if'' part is true by \autoref{thm:ordDiv}. For the ``only if'' part, assume that $g$ is an integer with $\ord_n(g)=\lambda(n)$ and $\lambda(n)=de$. Then $\ord_n(g^e)=d$, as desired.
	\end{proof}

	We finish this section by proposing a theorem. We will leave the proof for the reader as an exercise.
		\begin{theorem}\slshape
			If $\lambda(n)$ is relatively prime to $n$, then $n$ is square-free.
		\end{theorem}
	Recall that $n$ is square-free if it is not divisible by any perfect square other than $1$.

	\subsection{Primitive \texorpdfstring{$\lambda$}{Lambda}-roots}
	\textcite[Page $232-233$, Result II]{carmichael_1910} defined a generalization of primitive roots as follows using his function. As you will see, this section generalizes everything related to primitive roots.
	\begin{definition}[Primitive $\lambda$-root]
		Let $a$ and $n$ be relatively prime positive integers. If $\ord_n(a)=\lambda(n)$, then $a$ is a primitive $\lambda$-root modulo $n$. That is, $a^{\lambda(n)}$ is the smallest power of $a$ which is congruent to $1$ modulo $n$.
	\end{definition}
\textcite{cameron_preece_2014} shows the following theorem.
	\begin{definition}
		Let $n$ be a positive integer. Define $\xi(n) = \frac{\varphi(n)}{\lambda(n)}$ (read $\xi$ as ``ksi''). According to Corollary \ref{cor:LambdaDividesPhi}, $\xi(n)$ is an integer.
	\end{definition}

	\begin{proposition}
		 There is a primitive root (defined in the previous section) modulo $n$ if and only if $\xi(n)=1$. Carmichael calls a primitive root a $\varphi$-primitive root, and they are, in fact, a special case of $\lambda$-primitive roots.
	\end{proposition}

	Now, the existence of a primitive root is generalized to the following theorem from Carmichael's original paper.
	\begin{theorem}[Carmicahel]\slshape
		For any positive integer $n$, the congruence equation
			\begin{align*}
				x^{\lambda(n)} & \equiv1\pmod n
			\end{align*}
		has a solution $a$ which is a primitive $\lambda$-root, and for any such $a$, there are $\varphi(\lambda(n))$ primitive roots congruent to powers of $a$.
	\end{theorem}
	We can show that this theorem is true in a similar fashion to what we did in last section, and we leave it as an exercise.

	As we mentioned earlier in Proposition \ref{prop:phiproperties}, $\varphi(n)$ is always even for $n>2$. As it turns out, $\lambda$ and $\varphi$ share some common properties.
	\begin{problem}
		For any integer $n\geq 1$, either $\xi(n)=1$ or $\xi(n)$ is even.
	\end{problem}

	\begin{hint}
		Use the formula for $\lambda(n)$ in \autoref{thm:CarmichaelFormula}.
	\end{hint}

	\begin{problem}
		If $\lambda(n)>2$, the number of primitive $\lambda$-roots modulo $n$ is even.
	\end{problem}

	The next theorem generalizes \autoref{thm:genWilson}, which itself was a generalization to Wilson's theorem.
	\begin{theorem}\slshape
		Let $n$ be a positive integer such that $\lambda(n)>2$. Also, suppose that $g$ is a primitive $\lambda$-root modulo $n$. The product of primitive $\lambda$-roots of $n$ is congruent to $1$ modulo $n$.
	\end{theorem}

	\begin{proof}
		Since $\lambda(n)>2$, we can easily argue that it must be even. If $g$ is a primitive $\lambda$-root modulo $n$, all the primitive $\lambda$-roots are
			\begin{align*}
				\{g^{e_1}, g^{e_2}, \cdots, g^{e_{k}}\}
			\end{align*}
		where $e_i$ (for $1 \leq i \leq k$) are all (distinct) positive integers with $(e_i, \lambda(n))=1$. Also, note that we can pair them up since $\lambda(n)$ is even if $n>2$. In fact, we can pair $g^{e_i}$ with $g^{\lambda(n)-e_i}$ for all $i$. Then,
		\begin{align*}
			g^{e_1}\cdot g^{e_2}\cdots g^{e_k}
				& \equiv g^{\lambda(n)}\cdots g^{\lambda(n)}\\
				& \equiv 1\pmod n
		\end{align*}
	\end{proof}

	\begin{corollary}
		For any $n$, there are $\varphi(\lambda(n))$ primitive $\lambda$-roots modulo $n$.
	\end{corollary}


\end{document}