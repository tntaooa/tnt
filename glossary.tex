\newglossaryentry{sophiegermainidentity}{
	name={Sophie Germain Identity},
	description={
		\begin{align*}
			a^4+4b^4=(a^2+2ab+2b^2)(a^2-2ab+2b^2)
		\end{align*}
	},
	sort=sophiegermainidentity,
}

\newglossaryentry{binomial}{
	name={Binomial Theorem},
	description={
		For any positive integer $n$,
			\begin{align*}
				(a+b)^n
					& = a^n+\binom{n}{1}a^{n-1}b+\binom{n}{2}a^{n-2}b^2+\cdots+\binom{n}{1}ab^{n-1}+b^n\\
					& = \sum_{i=0}^{n}\binom{n}{i}a^{n-i}b^i\\
					& = \sum_{i=0}^{n}\binom{n}{i}a^ib^{n-i}
			\end{align*}
	},
	sort=binomialtheorem,
}

\newglossaryentry{binomialidentities}{
	name={Binomial Identities},
	description={
		For positive integers $n$ and $k$ such that $k \leq n$,
			\begin{enumerate}
				\item $\binom{n}{k}=\binom{n}{n-k}$
				\item $\binom{n}{k}=\binom{n-1}k+\binom{n-1}{k-1}$ (Pascal's recurrence)
				\item $\binom{n}{k}=\dfrac{n}{k}\binom{n-1}{k-1}$ (absorption property), \label{id:binomreduction}
				\item $\binom{n}{0}+\binom{n}{1}+\cdots+\binom{n}{n-1}+\binom{n}{n} = 2^n$
				\item $\binom{0}{k} + \binom{1}{k} + \cdots +\binom{n-1}{k} + \binom{n}{k} = \binom{n+1}{k+1}$
				\item $\binom{n}0^2+\binom{n}{1}^2+\cdots+\binom{n}{n-1}^2+\binom{n}{n}^2 = \binom{2n}{n}$ \label{id:binomsquaressum}
				\item If $n$ and $k$ are relatively prime, then $n$ divides $\binom nk$ and $k$ divides $\binom{n-1}{k-1}$.
			\end{enumerate}
	},
	sort=binomialidentities,
}

\newglossaryentry{fibonaccibrahmagupta}{
	name={Fibonacci-Brahmagupta Identity},
	description={
		For any reals $a,b,c,d$, and any integer $n$,
			\begin{align*}
				(a^2+nb^2)(c^2+nd^2)
					&=(ac-nbd)^2+n(ad+bc)^2\\
					&=(ac+nbd)^2+n(ad-bc)^2
			\end{align*}
		In other words, the product of two numbers of the form $a^2+nb^2$ is of the same form. Particularly, for $n=1$,
			\begin{align*}
				(a^2+b^2)(c^2+d^2)
					& =(ac+bd)^2+(ad-bc)^2\\
					& =(ad+bc)^2+(ac-bd)^2
			\end{align*}
	},
	sort=fibonaccibrahmagupta,
}

\newglossaryentry{eulerfoursquare}{
	name={Euler's Four Square Identity},
	description={
		The following identity is a generalization of Fibonacci-Brahmagupta Identity. \textit{Lagrange} used this identity to prove the \textit{Sum of Four Squares Theorem}. Let $a_1,a_2,\ldots,a_4$ and $b_1,b_2,\ldots,b_4$ be reals. Then,
			\begin{align*}
				(a_1^2+a_2^2+a_3^2+a_4^2)(b_1^2+b_2^2+b_3^2+b_4^2)
					& =(a_1 b_1 + a_2 b_2 + a_3 b_3 + a_4 b_4)^2 +(a_1 b_2 - a_2 b_1 + a_3 b_4 - a_4 b_3)^2\\
					& +(a_1 b_3 - a_2 b_4 - a_3 b_1 + a_4 b_2)^2 +(a_1 b_4 + a_2 b_3 - a_3 b_2 - a_4 b_1)^2
			\end{align*}
		An interested reader can see \textit{Degen's eight-square identity} or \textit{Pfister's sixteen-square identity}, but they do not look pretty at all so we do not include them here.
	},
	sort=eulerfoursquare,
}

\newglossaryentry{lebesgue}{
	name={Lebesgue Identity},
	description={
		\begin{align*}
			(a^2+b^2-c^2-d^2)^2 + (2ac+2bd)^2 + (2ad-2bc)^2 & = (a^2+b^2+c^2+d^2)^2
		\end{align*}
	},
	sort=lebesgue,
}

\newglossaryentry{euleraida}{
	name={Euler-Aida Ammei Identity},
	description={
		Let $x_1,x_2,\cdots,x_n$ be reals. Then,
			\begin{align*}
				(x_1^2-x_2^2-\cdots-x_n)^2 + \sum_{i=2}^{n}(2x_1x_i)^2
					& = (x_1^2+x_2^2+\cdots+x_n^2)^2
			\end{align*}
	},
	sort=euleraida,
}

\newglossaryentry{bhaskara}{
	name={Bhaskara's Lemma},
	description={
		Let $m,x,y,n$ and $k$ be integers such that $k \neq 0$. If $y^2-nx^2=k$, then
			\begin{align*}
				\left(\dfrac{mx+ny}{k}\right) ^2-n\left( \dfrac{mx+y}{k}\right)^2
					& = \dfrac{m^2-n}{k}
			\end{align*}
		This identity is used in {\it Chakravala method} to find the solutions to {\it Pell-Fermat equation}.
	},
	sort=bhaskara,
}

\newglossaryentry{sumofdifference}{
	name={Sum of Differences},
	description={
		Let $a_1, a_2, a_3, \cdots$ be an infinite sequence of numbers. Then, for any positive integer $n$,
			\begin{align*}
				a_n
					& = a_1 + \sum_{k=1}^{n-1} \left(a_{k+1} - a_{k}\right)
			\end{align*}
		Expand the sum on the right side to obtain
			\begin{align*}
				\sum_{k=1}^{n-1} \left(a_{k+1} - a_{k}\right)
					&=  \left(a_{n} - a_{n-1}\right) +  \left(a_{n-1} - a_{n-2}\right) + \cdots +  \left(a_{2} - a_{1}\right)\\
					&= a_n - a_1
			\end{align*}
		The conclusion follows.
	},
	sort=sumofdifference,
}