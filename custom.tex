\usepackage{float, enumerate, xlop, tikz, cancel, pgfplots, subcaption, mathdots}
\usepackage{centernot}

\usetikzlibrary{fit,matrix,shapes}

%tikz
\newcommand*{\xMin}{0}%
\newcommand*{\xMax}{6}%
\newcommand*{\yMin}{0}%
\newcommand*{\yMax}{6}%

\newtheorem{theorem}{{\textbf{Theorem}}}[]
\newtheorem{lemma}[]{{\textbf{Lemma}}}
\newtheorem{proposition}[]{{\textbf{Proposition}}}
\newtheorem{corollary}[]{{\textbf{Corollary}}}
\newtheorem{conjecture}{{\textbf{Conjecture}}}
\newtheorem{observation}{{\textbf{Observation}}}
\newtheorem{fact}{{\textbf{Fact}}}
\newtheorem{identity}{}
\pgfplotsset{compat=1.7,}

\theoremstyle{definition}
\newtheorem{definition}{Definition}
\newtheorem*{notation}{\textbf{\textit{Notation}}}
\newtheorem{problem}{\textbf{\textit{Problem}}}
\newtheorem*{question}{\textbf{\textit{Question}}}
\newtheorem*{acknowledgment}{\textbf{\textit{Acknowledgment}}}
\newtheorem*{remark}{\textbf{\textit{Remark}}}
\newtheorem*{note}{\textbf{\textit{Note}}}
\newtheorem*{solution}{\textit{{Solution}}}

\theoremstyle{remark}
\newtheorem*{hint}{Hint}
\newtheorem*{example}{Example}
\newtheorem*{hint2}{Hint}


%\numberwithin{equation}{chapter}
%\numberwithin{theorem}{section}
%\numberwithin{conjecture}{chapter}
%\numberwithin{fact}{chapter}

\DeclareMathOperator{\psp}{psp}
\DeclareMathOperator{\epsp}{epsp}
\DeclareMathOperator{\spsp}{spsp}


%for arithmetic function diagrams
\pgfplotsset{soldot/.style={,only marks,mark=*}} \pgfplotsset{holdot/.style={,fill=white,only marks,mark=*}}


% For showing remainder in the long division in base conversion
\newcommand{\chiffre}[2]{\tikz[remember picture] \node[inner sep=0pt](#1){#2};}

\newcommand{\entoure}[2]{\tikz[remember picture,overlay] \node[preaction={draw=black,ultra thick,opacity=.2,
		transform canvas={xshift=1.5pt,yshift=-1.5pt}},draw,ellipse,ultra thick,inner sep=.5em,fit=(#1.center)(#2.center)]{} ;}

%long division
\input longdiv.tex
\endinput
