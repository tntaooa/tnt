\documentclass[]{article}
\usepackage{amsmath}
\usepackage{amsthm}
\usepackage{amsfonts}
\usepackage{hyperref}
\usepackage{enumerate}
\usepackage{amssymb}
\theoremstyle{definition}
\newtheorem{problem}{Problem}
\newtheorem{solution}{Solution}
%opening
\title{Number Theory Problems of $2015$ Competitions}
\author{Amir Hossein Parvardi}

\begin{document}
	
\maketitle

\section{2015 Contests}


\begin{problem}[Croatia First Round Competition 2015]
	Does there exist a positive integer $n$ such that $n^2+2n+2015$ is a perfect square?
\end{problem}

\begin{problem}[Croatia First Round Competition 2015]
	Determine the largest positive integer $n$ such that
		\begin{align*}
			n+5 \mid n^4 + 1395
		\end{align*}
\end{problem}


\begin{problem}[Croatia First Round Competition 2015]
	Let $n$ be a positive integer and define
		\begin{align*}
			S_n = \sum_{k=1}^{n} k!(k^2+k+1).
		\end{align*}
	Find $\frac{S_n + 1}{(n+1)!}$.
\end{problem}


\begin{problem}[Croatia First Round Competition 2015]
	Let $n$ be a positive integer. Each of the numbers $n, n + 1, n + 2, \dots , 2n -1$ has a largest odd divisor. Determine the sum of these largest odd divisors.
\end{problem}


\begin{problem}[Croatia Second Round Competition 2015]
	Positive integers $a$ and $b$ and prime number $p$ satisfy the equation $a^2 + p^2 = b^2$.	Prove that $2(b + p)$ is a perfect square.
\end{problem}


\begin{problem}[Croatia Second Round Competition 2015]
	Determine all quadruples $(a, b, c, d)$ of positive integers such that
		\begin{align*}
			a^3 = b^2, c^5 = d^4, \text{ and } a - c = 9.
		\end{align*}
\end{problem}


\begin{problem}[Croatia Second Round Competition 2015]
	Let $n$ be a positive integer larger than $1$ such that both $2n - 1$ and $3n - 2$ are perfect squares. Prove that $10n - 7$ is composite.
\end{problem}


\begin{problem}[Croatia Second Round Competition 2015]
	Let $a = \sqrt[2015]{2015}$ and $(a_n)$ be a sequence such that $a_1 = a$ and $a_{n+1} = a^{a_n}$ for $n \geq 1$. Does there exist a positive integer $n$ such that $a_n \geq 2015$?
\end{problem}


\begin{problem}[Croatia Second Round Competition 2015]
	A positive integer is called \textit{wacky} if its decimal representation contains $100$ digits, and if by removing any of those digits one gets a $99$-digit number divisible by $7$. How many wacky positive integers are there?
\end{problem}


\begin{problem}[Croatia Final Round National Competition 2015]
	Prove that there does not exist a positive integer $n$ such that $7^n - 1$ is divisible by $6^n - 1$.
\end{problem}


\begin{problem}[Croatia Final Round National Competition 2015]
	Determine all triples $(p, m, n)$ of positive integers such that $p$ is a prime number and
		\begin{align*}
			p^m - n^3 =8.
		\end{align*}
\end{problem}



\begin{problem}[Croatia Final Round National Competition 2015]
	Determine all triples $(p, m, n)$ of positive integers such that $p$ is a prime number and
		\begin{align*}
			2^mp^2+1 = n^5.
		\end{align*}
\end{problem}


\begin{problem}[Croatia Final Round National Competition 2015]
	Determine all positive integers $n$ for which there exists a divisor $d$ of $n$ such that
		\begin{align*}
			dn + 1 | d^2 + n^2.
		\end{align*}
\end{problem}


\begin{problem}[Croatian Mathematical Olympiad 2015, IMO Shortlist 2014]
	Let $n > 1$ be a given integer. Prove that infinitely many terms of the sequence $(a_k )_{k\ge 1}$, defined by \[a_k=\left\lfloor\frac{n^k}{k}\right\rfloor,\] are odd. (For a real number $x$, $\lfloor x\rfloor$ denotes the largest integer not exceeding $x$.)
\end{problem}



\begin{problem}[Croatian Mathematical Olympiad 2015]
	Let $n > 2$ be a positive integer and $p$ a prime number. If the number $p - 1$ is divisible by $n$, and the number $n^3 - 1$ is divisible by $p$, prove that $4p - 3$ is a square of an integer.
\end{problem}


\begin{problem}[Croatian TST for MEMO 2015]
	Determine all positive integers $x$ and $y$ such that
		\begin{align*}
			x(x^2 + 19) = y(y^2 - 10).
		\end{align*}
\end{problem}

\end{document}