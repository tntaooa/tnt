\begin{problem}[APMO 2016]
	A positive integer is called \textit{fancy} if it can be expressed in the form $$2^{a_1}+2^{a_2}+ \cdots+ 2^{a_{100}}$$where $a_1,a_2, \cdots, a_{100}$ are non-negative integers that are not necessarily distinct. Find the smallest positive integer $n$ such that no multiple of $n$ is a fancy number. %%\flushright \href{http://artofproblemsolving.com/community/c6h1243426p6362865}{Link}
\end{problem}

\begin{problem}[Argentina Intercollegiate Olympiad First Level 2016]
	Find all positive integers $a,b,c,$ and $d$, all less than or equal to $6$, such that
	\begin{align*}
		\frac{a}{b} = \frac{c}{d} + 2
	\end{align*}
\end{problem}

\begin{problem}[Argentina Intercollegiate Olympiad Second Level 2016]
	Find all positive integers $x$ and $y$ which satisfy the following conditions:
	\begin{enumerate}
		\item $x$ is a $4-$digit palindromic number, and
		\item $y=x+312$ is a $5-$digit palindromic number.
	\end{enumerate}
	\textbf{Note.} A palindromic number is a number that remains the same when its digits are reversed. For example, $16461$ is a palindromic number.
\end{problem}

\begin{problem}[Argentina Intercollegiate Olympiad Third Level 2016]
	Find a number with the following conditions:
	\begin{enumerate}
		\item it is a perfect square,
		\item when $100$ is added to the number, it equals a perfect square plus $1$, and
		\item when $100$ is again added to the number, the result is a perfect square.
	\end{enumerate}
\end{problem}

\begin{problem}[Argentina Intercollegiate Olympiad Third Level 2016]
	Let $a_1, a_2, \dots, a_{15}$ be an arithmetic progression. If the sum of all $15$ terms is twice the sum of the first $10$ terms, find $\frac{d}{a_1}$, where $d$ is the common difference of the progression.
\end{problem}

\begin{problem}[Austria Federal Competition for Advanced Students Final Round 2016]
	Determine all composite positive integers $n$ with the following property: If $1 = d_1 < d_2 <
	\dots < d_k = n$ are all the positive divisors of $n$, then
	\begin{align*}
		(d_2-d_1):(d_3-d_2):\dots : (d_k - d_{k-1}) = 1:2\dots :(k-1)
	\end{align*}
\end{problem}

\begin{problem}[Austria National Competition Final Round 2016]
	Let $a,b$, and $c$ be integers such that
	\begin{align*}
		\frac{ab}{c} + \frac{ac}{b} + \frac{bc}{a}
	\end{align*}
	is an integer. Prove that each of the numbers
	\begin{align*}
		\frac{ab}{c} , \frac{ac}{b}, \text{ and } \frac{bc}{a}
	\end{align*}
	is an integer.
\end{problem}

\begin{problem}[Austria Beginners' Competition 2016]
	Determine all non-negative integers $n$ having two distinct positive divisors with the same distance from $n/3$.
\end{problem}

\begin{problem}[Austria Regional Competition 2016]
	Determine all positive integers $k$ and $n$ satisfying the equation
	\begin{align*}
		k^2 - 2016 = 3^n
	\end{align*}
\end{problem}

\begin{problem}[Azerbaijan TST 2016]
	The set $A$ consists of natural numbers such that these numbers can be expressed as $2x^2+3y^2,$ where $x$ and $y$ are integers. $(x^2+y^2\not=0)$
	\begin{enumerate}
		\item Prove that there is no perfect square in the set $A.$
		\item Prove that multiple of odd number of elements of the set $A$ cannot be a perfect square.
	\end{enumerate}
	%%\flushright \href{http://artofproblemsolving.com/community/c6h1201200p5906668}{Link}
\end{problem}

\begin{problem}[Azerbaijan Junior Mathematical Olympiad 2016]
	Given
	\begin{align*}
		34!=295232799039a041408476186096435b0000000
	\end{align*}
	in decimal representation, find the numbers $a$ and $b$. %%\hfill \href{http://artofproblemsolving.com/community/c6h1195518p5852632}{Link}
\end{problem}

\begin{problem}[Azerbaijan Junior Mathematical Olympiad 2016]
	Prove that if for a real number $a $, $a+\frac {1}{a} $ is integer then $a^n+\frac {1}{a^n} $ is also integer for any positive integer $n$. %\hfill \href{http://artofproblemsolving.com/community/c6h1195519p5852637}{Link}
\end{problem}

\begin{problem}[Azerbaijan Junior Mathematical Olympiad 2016]
	A quadruple $(p,a,b,c)$ of positive integers is called a \textit{good quadruple} if
	\begin{enumerate}[(a)]
		\item $p $ is odd prime,
		\item $a,b,c $ are distinct,
		\item $ab+1,bc+1$, and $ca+1$ are divisible by $p$.
	\end{enumerate}
	Prove that for all good quadruple $p+2\le \frac {a+b+c}{3} $, and show the equality case. %\flushright \href{http://artofproblemsolving.com/community/c6h1195523p5852672}{Link}
\end{problem}

\begin{problem}[Balkan 2016]
	Find all monic polynomials $f$ with integer coefficients satisfying the following condition: there exists a positive integer $N$ such that $p$ divides $2(f(p)!)+1$ for every prime $p>N$ for which $f(p)$ is a positive integer.
	\\
	Note: A monic polynomial has a leading coefficient equal to 1. %\hfill \href{http://artofproblemsolving.com/community/c6h1239091p6316984}{Link}
\end{problem}

\begin{problem}[Bay Area Olympiad 2016]
	Let $A=2^k - 2$ and $B= 2^k \cdot A$, where $k$ is an integer ($k \ge 2$).
	Show that, for every integer $k$ greater than or equal to $2$,
	\begin{enumerate}
		\item $A$ and $B$ have the same set of distinct prime factors.
		\item $A+1$ and $B+1$ have the same set of distinct prime factors.
	\end{enumerate}
	%\flushright \href{http://artofproblemsolving.com/community/c6h1203225p5926002}{Link}
\end{problem}

\begin{problem}[Bay Area Olympiad 2016]
	Find a positive integer $N$ and $a_1, a_2, \cdots, a_N$ where $a_k = 1$ or $a_k = -1$, for each $k=1,2,\cdots,N,$ such that $$a_1 \cdot 1^3 + a_2 \cdot 2^3 + a_3 \cdot 3^3 \cdots + a_N \cdot N^3 = 20162016$$or show that this is impossible. %\hfill \href{http://artofproblemsolving.com/community/c6h1203525p5929703}{Link}
\end{problem}

\begin{problem}[Belgium Flanders Math Olympiad Final Round 2016]
	Find the smallest positive integer $n$ which does not divide $2016!$.
\end{problem}

\begin{problem}[Belgium National Olympiad Final Round 2016]
	Solve the equation
	\begin{align*}
		2^{2m+1}+9\cdot2^m+5=n^2
	\end{align*}
	for integers $m$ and $n$.
\end{problem}

\begin{problem}[Benelux 2016]
	Find the greatest positive integer $N$ with the following property: there exist integers $x_1, \dots, x_N$ such that $x^2_i - x_ix_j$ is not divisible by $1111$ for any $i\ne j.$ %\hfill \href{http://artofproblemsolving.com/community/c6h1236281p6284414}{Link}
\end{problem}

\begin{problem}[Benelux 2016]
	Let $n$ be a positive integer. Suppose that its positive divisors can be partitioned into pairs (i.e. can be split in groups of two) in such a way that the sum of each pair is a prime number. Prove that these prime numbers are distinct and that none of these are a divisor of $n.$ %\hfill \href{http://artofproblemsolving.com/community/c6h1236282p6284421}{Link}
\end{problem}

\begin{problem}[Bosnia and Herzegovina TST 2016]
	For an infinite sequence $a_1<a_2<a_3< \dots$ of positive integers we say that it is nice if for every positive integer $n$ holds $a_{2n}=2a_n$. Prove the following statements:
	\begin{enumerate}[(a)]
		\item If there is given a nice sequence and prime number $p>a_1$, there exist some term of the sequence which is divisible by $p$.
		\item For every prime number $p>2$, there exist a nice sequence such that no terms of the sequence are divisible by $p$.
	\end{enumerate}
	%\flushright \href{http://artofproblemsolving.com/community/c6h1243216p6360340}{Link}
\end{problem}

\begin{problem}[Bosnia and Herzegovina TST 2016]
	Determine the largest positive integer $n$ which cannot be written as the sum of three numbers bigger than $1$ which are pairwise relatively prime. %\hfill \href{http://artofproblemsolving.com/community/c6h1243220p6360359}{Link}
\end{problem}

\begin{problem}[Bulgaria National Olympiad 2016]
	Find all positive integers $m$ and $n$ such that $\left(2^{2^m}+1\right)\left(2^{2^n}+1\right)$ is divisible by $mn$.
\end{problem}

\begin{problem}[Bulgaria National Olympiad 2016]
	Determine whether there exists a positive integer $n<10^9$ such that $n$ can be expressed as a sum of three squares of positive integers by more than $1000$ distinct ways.
\end{problem}

\begin{problem}[Canadian Mathematical Olympiad Qualification 2016]
	$ $
	\begin{enumerate}[(a)]
		\item Find all positive integers $n$ such that $11\mid (3^n + 4^n)$.
		\item Find all positive integers $n$ such that $31\mid (4^n + 7^n + 20^n)$.
	\end{enumerate}
	%\flushright \href{http://artofproblemsolving.com/community/c6h1259325p6529513}{Link}
\end{problem}

\begin{problem}[Canadian Mathematical Olympiad Qualification 2016]
	Determine all ordered triples of positive integers $(x, y, z)$ such that $\gcd(x+y, y+z, z+x) > \gcd(x, y, z)$. %\hfill \href{http://artofproblemsolving.com/community/c6h1259332p6529539}{Link}
\end{problem}

\begin{problem}[Canada National Olympiad 2016]
	Find all polynomials $P(x)$ with integer coefficients such that $P(P(n) + n)$ is a prime number for infinitely many integers $n$. %\hfill \href{http://artofproblemsolving.com/community/c6h1225741p6160566}{Link}
\end{problem}

\begin{problem}[CCA Math Bonanza 2016]
	Let $f(x) = x^2 + x + 1$. Determine the ordered pair $(p,q)$ of primes satisfying $f(p) = f(q) + 242$. %\hfill \href{http://artofproblemsolving.com/community/c4h1249117p6423872}{Link}
\end{problem}

\begin{problem}[CCA Math Bonanza 2016]
	Let $f(x) = x^2 + x + 1$. Determine the ordered pair $(p,q)$ of primes satisfying $f(p) = f(q) + 242$. %\hfill \href{http://artofproblemsolving.com/community/c4h1249117p6423872}{Link}
\end{problem}

\begin{problem}[CCA Math Bonanza 2016]
	Compute \[\sum_{k=1}^{420} \gcd(k,420)\]
	%\flushright \href{http://artofproblemsolving.com/community/c4h1249124p6423895}{Link}
\end{problem}

\begin{problem}[CCA Math Bonanza 2016]
	Pluses and minuses are inserted in the expression \[\pm 1 \pm 2 \pm 3 \dots \pm 2016\]such that when evaluated the result is divisible by 2017. Let there be $N$ ways for this to occur. Compute the remainder when $N$ is divided by $503$. %\hfill \href{http://artofproblemsolving.com/community/c4h1249141p6423961}{Link}
\end{problem}

\begin{problem}[CCA Math Bonanza 2016]
	What is the largest integer that must divide $n^5-5n^3+4n$ for all integers $n$? %\hfill \href{http://artofproblemsolving.com/community/c4h1249149p6423999}{Link}
\end{problem}

\begin{problem}[CCA Math Bonanza 2016]
	Determine the remainder when
		\begin{align*}
			2^6\cdot3^{10}\cdot5^{12}-75^4\left(26^2-1\right)^2+3^{10}-50^6+5^{12}
		\end{align*}
	is divided by $1001$. %\href{http://artofproblemsolving.com/community/c4h1249159p6424026}{Link}
\end{problem}

\begin{problem}[CentroAmerican 2016]
	Find all positive integers $n$ that have $4$ digits, all of them perfect squares, and such that $n$ is divisible by $2, 3, 5$, and $7$. %\hfill \href{http://artofproblemsolving.com/community/c6h1259646p6532180}{Link}
\end{problem}

\begin{problem}[CentroAmerican 2016]
	We say a number is \textit{irie} if it can be written in the form $1+\dfrac{1}{k}$ for some positive integer $k$. Prove that every integer $n \geq 2$ can be written as the product of $r$ distinct irie numbers for every integer $r \geq n-1$. %\hfill \href{http://artofproblemsolving.com/community/c6h1260754p6542783}{Link}
\end{problem}

\begin{problem}[Chile 2016]
	Determine all triples of positive integers $(p, n, m)$ with $p$ a prime number, which satisfy the equation:
	\begin{align*}
		p^m - n^3 = 27
	\end{align*}
\end{problem}

\begin{problem}[Chile 2016\footnote{Thanks to Kamal Kamrava and Behnam Sajadi for translating the problem.}]
	Find all prime numbers that do not have a multiple ending in $2015$.
\end{problem}

\begin{problem}[Chile 2016]
	Find the number of different numbers of the form $\floor{\frac{i^2}{2015}}$, where $i=1,2,\dots,2015$.
\end{problem}

\begin{problem}[China Girls Mathematical Olympiad 2016]
	Let $m$ and $n$ are relatively prime integers and $m>1,n>1$. Show that there are positive integers $a,b,c$ such that $m^a=1+n^bc$ , and $n$ and $c$ are relatively prime. %\hfill \href{http://artofproblemsolving.com/community/c6h1288391p6805896}{Link}
\end{problem}

\begin{problem}[China National Olympiad 2016]
	Let $p$ be an odd prime and $a_1, a_2,...,a_p$ be integers. Prove that the following two conditions are equivalent:
	\begin{enumerate}
		\item There exists a polynomial $P(x)$ with degree $\leq \frac{p-1}{2}$ such that $P(i) \equiv a_i \pmod p$ for all $1 \leq i \leq p$.
		\item For any natural $d \leq \frac{p-1}{2}$,
			\begin{align*}
				\sum_{i=1}^p (a_{i+d} - a_i )^2
					& \equiv 0 \pmod p
			\end{align*}
		where indices are taken modulo $p$.
	\end{enumerate}
	%\flushright \href{http://artofproblemsolving.com/community/c6h1174698p5660061}{Link}
\end{problem}

\begin{problem}[China South East Mathematical Olympiad 2016]
	Let $n$ be a positive integer and let $D_n$ be the set of all positive divisors of $n$. Define $f(n)=\sum\limits_{d\in D_n}{\frac{1}{1+d}}$.
	Prove that for any positive integer $m$,
		\begin{align*}
			\sum_{i=1}^{m}{f(i)}
				& <m
		\end{align*}
	%\flushright \href{http://artofproblemsolving.com/community/c6h1281358p6741745}{Link}
\end{problem}

\begin{problem}[China South East Mathematical Olympiad 2016]
	Let $\{ a_n\}$ be a sequence consisting of positive integers such that $n^2 \mid \sum_{i=1}^{n}{a_i}$ and $a_n\leq (n+2016)^2$ for all $n\geq 2016$.
	Define $b_n=a_{n+1}-a_n$. Prove that the sequence $\{ b_n\}$ is eventually constant. %\hfill \href{http://artofproblemsolving.com/community/c6h1281364p6741764}{Link}
\end{problem}

\begin{problem}[China South East Mathematical Olympiad 2016]
	Define the sets
	\begin{align*}
		A &=\{a^3+b^3+c^3-3abc:a,b,c\in\mathbb{N}\}\\
		B &=\{(a+b-c)(b+c-a)(c+a-b):a,b,c\in\mathbb{N}\}\\
		P &=\{n:n\in A\cap B,1\le n\le 2016\}
	\end{align*}
	Find the number of elements of $P$. %\href{http://artofproblemsolving.com/community/c6h1285319p6774603}{Link}
\end{problem}

\begin{problem}[China TST‌2016]
	Let $c,d \geq 2$ be positive integers. Let $\{a_n\}$ be the sequence satisfying $a_1 = c, a_{n+1} = a_n^d + c$ for $n = 1,2,\dots$.
	Prove that for any $n \geq 2$, there exists a prime number $p$ such that $p\mid a_n$ and $p \nmid a_i$ for $i = 1,2,\dots, n-1$. %\hfill \href{http://artofproblemsolving.com/community/c6h1212540p6016827}{Link}
\end{problem}

\begin{problem}[China TST 2016]
	Set positive integer $m=2^k\cdot t$, where $k$ is a non-negative integer, $t$ is an odd number, and let $f(m)=t^{1-k}$. Prove that for any positive integer $n$ and for any positive odd number $a\le n$, $\prod_{m=1}^n f(m)$ is a multiple of $a$. %\hfill \href{http://artofproblemsolving.com/community/c6h1215111p6043322}{Link}
\end{problem}

\begin{problem}[China TST 2016]
	Does there exist two infinite positive integer sets $S,T$, such that any positive integer $n$ can be uniquely expressed in the form
		\begin{align*}
			n
				& =s_1t_1+s_2t_2+\ldots+s_kt_k
		\end{align*}
	where $k$ is a positive integer dependent on $n$, $s_1<s_2<\dots<s_k$ are elements of $S$, $t_1,\dots, t_k$ are elements of $T$? %\hfill \href{http://artofproblemsolving.com/community/c6h1215112p6043327}{Link}
\end{problem}

\begin{problem}[China TST 2016]
	Let $a,b,b',c,m,q$ be positive integers, where $m>1,q>1,|b-b'|\ge a$. It is given that there exist a positive integer $M$ such that
		\begin{align*}
			S_q(an+b)
				& \equiv S_q(an+b')+c\pmod{m}
		\end{align*}
	holds for all integers $n\ge M$. Prove that the above equation is true for all positive integers $n$. (Here $S_q(x)$ is the sum of digits of $x$ taken in base $q$). %\hfill \href{http://artofproblemsolving.com/community/c6h1217570p6069193}{Link}
\end{problem}

\begin{problem}[China Western Mathematical Olympiad 2016]
	For an $n$-tuple of integers, define a transformation to be:
		\begin{align*}
			(a_1,a_2,\dots,a_{n-1},a_n)
				& \rightarrow (a_1+a_2, a_2+a_3, \dots, a_{n-1}+a_n, a_n+a_1)
		\end{align*}
	Find all ordered pairs of integers $(n,k)$ with $n,k\geq 2$, such that for any $n$-tuple of integers $(a_1,a_2,\dots,a_{n-1},a_n)$, after a finite number of transformations, every element in the of the $n$-tuple is a multiple of $k$. %\hfill \href{http://artofproblemsolving.com/community/c6h1292364p6843838}{Link}
\end{problem}

\begin{problem}[China Western Mathematical Olympiad 2016]
	Prove that there exist infinitely many positive integer triples $(a,b,c)$ such that $a ,b,c$ are pairwise relatively prime ,and $ab+c ,bc+a ,ca+b$ are pairwise relatively prime. %\hfill \href{http://artofproblemsolving.com/community/c6h1290756p6827171}{Link}
\end{problem}

\begin{problem}[Croatia First Round Competition 2016]
	Can the sum of squares of three consecutive integers be divisible by $2016$?
\end{problem}

\begin{problem}[Croatia First Round Competition 2016]
	Let $a = 123456789$ and $N = a^3 - 2a^2 - 3a$. Prove that $N$ is a multiple of $540$.
\end{problem}

\begin{problem}[Croatia First Round Competition 2016]
	Find all pairs $(a, b)$ of positive integers such that $1 < a, b \leq 100$ and
	\begin{align*}
		\frac{1}{\log_a 10} + \frac{1}{\log_b 10}
	\end{align*}
	is a positive integer.
\end{problem}

\begin{problem}[Croatia First Round Competition 2016]
	A sequence $(a_n)$ is given: $a_1=a_2=1$, and
	\begin{align*}
		a_{n+1} = \frac{a_2^2}{a_1} + \frac{a_3^2}{a_2} + \dots + \frac{a_n^2}{a_{n-1}}
	\end{align*}
	for $n\geq 2$. Find $a_{2016}$.
\end{problem}

\begin{problem}[Croatia First Round Competition 2016]
	Let $a, b$, and $c$ be integers. If $4a+5b-3c$ is divisible by $19$, prove that $6a-2b+5c$ is also divisible by $19$.
\end{problem}

\begin{problem}[Croatia First Round Competition 2016]
	Determine all pairs of positive integers $(x, y)$ such that $x^2 - y! = 2016$.
\end{problem}

\begin{problem}[Croatia Second Round Competition 2016]
	$ $
	\begin{enumerate}[(a)]
		\item Prove that there are no two positive integers such that the difference of their squares is $987654$.
		\item Prove that there are no two positive integers such that the difference of their cubes is $987654$.
	\end{enumerate}
\end{problem}

\begin{problem}[Croatia Second Round Competition 2016]
	How many ordered pairs $(m, k)$ of positive integers satisfy the following?
	\begin{align*}
		20m = k(m - 15k)
	\end{align*}
\end{problem}

\begin{problem}[Croatia Second Round Competition 2016]
	Determine all pairs $(a, b)$ of positive integers such that
	\begin{align*}
		a^3 - 3b &= 15\\
		b^2 - a  &= 13
	\end{align*}
\end{problem}

\begin{problem}[Croatia Second Round Competition 2016]
	Prove that, for every positive integer $n > 3$, there are $n$ different positive integers whose reciprocals add up to $1$.
\end{problem}

\begin{problem}[Croatia Second Round Competition 2016]
	Determine all pairs $(a, b)$ of integers such that $(7a - b)^2 = 2(a - 1)b^2$.
\end{problem}

\begin{problem}[Croatia Final Round National Competition 2016]
	Determine the sum
	\begin{align*}
		\frac{2^2+1}{2^2-1} + \frac{3^2+1}{3^2-1} + \dots + \frac{100^2+1}{100^2-1}
	\end{align*}
\end{problem}

\begin{problem}[Croatia Final Round National Competition 2016]
	Let $a,b$, and $c$ be positive integers such that
	\begin{align*}
		c &= a + \frac{b}{a} - \frac{1}{b}
	\end{align*}
	Prove that $c$ is the square of an integer.
\end{problem}

\begin{problem}[Croatia Final Round National Competition 2016]
	Determine all pairs $(m, n)$ of positive integers for which exist integers $a, b$, and $c$ that satisfy
	\begin{align*}
		a+b+c
			& =0\\
		a^2+b^2+c^2
			& =2^m \cdot 3^n
	\end{align*}
\end{problem}

\begin{problem}[Croatia Final Round National Competition 2016]
	Prove that there does not exist a positive integer $k$ such that $k + 4$ and $k^2 + 5k + 2$ are cubes of positive integers.
\end{problem}

\begin{problem}[Croatia Final Round National Competition 2016]
	Determine all triples $(m, n, k)$ of positive integers such that $3^m + 7^n = k^2$.
\end{problem}

\begin{problem}[Croatian Mathematical Olympiad 2016]
	Find all pairs $(p,q)$ of prime numbers such that
		\begin{align*}
			p(p^2 - p - 1)
				& = q(2q + 3)
		\end{align*}
	%\flushright \href{http://artofproblemsolving.com/community/c6h1234390p6260615}{Link}
\end{problem}

\begin{problem}[Croatian TST for MEMO 2016, Sweden 2014]
	Find all pairs $(m, n)$ of positive integers such that
	\begin{align*}
		3 \cdot 5^m - 2\cdot 6^n
			& = 3
	\end{align*}
\end{problem}

\begin{problem}[Croatia IMO TST $2016$]
	Prove that for every positive integer $n$ there exist integers $a$ and $b$ such that $n$ divides $4a^2 + 9b^2 - 1$.
\end{problem}

\begin{problem}[Croatia IMO TST 2016, Bulgaria TST 2016]
	Let $p > 10^9$ be a prime number such that $4p + 1$ is also prime.
	Prove that the decimal expansion of $\frac{1}{4p+1}$ contains all the digits $0,1, \ldots, 9$. %\hfill \href{http://artofproblemsolving.com/community/c6h1233196p6246286}{Link}
\end{problem}

\begin{problem}[Denmark Georg Mohr Contest Second Round 2016]
	Find all possible values of the number
	\begin{align*}
		\frac{a+b}{c}+ \frac{a+c}{b}+ \frac{b+c}{a}
	\end{align*}
	where $a, b,$ and $c$ are positive integers, and $\frac{a+b}{c}, \frac{a+c}{b},$ and $\frac{b+c}{a}$ are also positive integers.
\end{problem}

\begin{problem}[ELMO 2016]
	Cookie Monster says a positive integer $n$ is $crunchy$ if there exist $2n$ real numbers $x_1,x_2,\ldots,x_{2n}$, not all equal, such that the sum of any $n$ of the $x_i$'s is equal to the product of the other $n$ of the $x_i$'s. Help Cookie Monster determine all crunchy integers. %\hfill \href{http://artofproblemsolving.com/community/c6h1262189p6556895}{Link}
\end{problem}

\begin{problem}[ELMO $2016$]
	Big Bird has a polynomial $P$ with integer coefficients such that $n$ divides $P(2^n)$ for every positive integer $n$. Prove that Big Bird's polynomial must be the zero polynomial. %\hfill \href{http://artofproblemsolving.com/community/c6h1262192p6556902}{Link}
\end{problem}

\begin{problem}[Estonia IMO TST First Stage 2016]
	Let $p$ be a prime. Find all integers (not necessarily positive) $a,b,$ and $c$ such that
	\begin{align*}
		a^bb^cc^a = p
	\end{align*}
\end{problem}

\begin{problem}[Estonia IMO TST First Stage 2016]
	Prove that for every positive integer $n \geq 3$,
	\begin{align*}
		2 \cdot \sqrt 3 \cdot \sqrt[3]{4} \dots \sqrt[n-1]{n} >n
	\end{align*}
\end{problem}

\begin{problem}[Estonia IMO TST Second Stage 2016]
	Find all positive integers $n$ such that
	\begin{align*}
		\left(n^2+11n-4\right)\cdot n! + 33 \cdot 13^n + 4
	\end{align*}
	is a perfect square.
\end{problem}

\begin{problem}[Estonia National Olympiad Tenth Grade 2016]
	Find all pairs of integers $(a, b)$ which satisfy
	\begin{align*}
		3 (a^2 + b^2 ) - 7 (a + b) = -4
	\end{align*}
\end{problem}

\begin{problem}[Estonia National Olympiad Eleventh Grade 2016]
	Find the greatest positive integer $n$ for which $3^{2016} - 1$ is divisible by $2^n$.
\end{problem}

\begin{problem}[Estonia National Olympiad Eleventh Grade 2016]
	Let $n$ be a positive integer. Let $\delta(n)$ be the number of positive divisors of $n$ and let $\sigma(n)$ be their sum. Prove that
	\begin{align*}
		\sigma(n) > \frac{\left(\delta(n)\right)^2}{2}
	\end{align*}
\end{problem}

\begin{problem}[Estonia Regional Olympiad Tenth Grade 2016]
	Does the equation
	\begin{align*}
		x^2+y^2+z^2+w^2=3 + xy + yz + zw
	\end{align*}
	has a solution in which $x, y, z$, and $w$ are different integers?
\end{problem}

\begin{problem}[Estonia Regional Olympiad Twelfth Grade 2016]
	Determine whether the logarithm of $6$ in base $10$ is larger or smaller than $\displaystyle\frac{7}{9}$.
\end{problem}

\begin{problem}[Estonia Regional Olympiad Twelfth Grade 2016]
	Find the largest positive integer $n$ so that one can select $n$ primes $p_1, p_2, \dots, p_n$ (not necessarily distinct) such that
	\begin{align*}
		p_1, p_1+p_2, \dots, p_1+p_2+\dots+p_n
	\end{align*}
	are all primes.
\end{problem}

\begin{problem}[European Girls' Mathematical Olympiad 2016]
	Let $S$ be the set of all positive integers $n$ such that $n^4$ has a divisor in the range $n^2 +1, n^2 + 2,\dots,n^2 + 2n$. Prove that there are infinitely many elements of $S$ of each of the forms $7m, 7m+1, 7m+2, 7m+5, 7m+6$ and no elements of $S$ of the form $7m+3$ and $7m+4$, where $m$ is an integer. %\hfill \href{http://artofproblemsolving.com/community/c6h1227241p6177824}{Link}
\end{problem}

\begin{problem}[European Mathematical Cup Seniors 2016]
	$A=\{a,b,c\}$ is a set containing three positive integers. Prove that we can find a set $B \subset A$, say $B=\{x,y\}$, such that for all odd positive integers $m$ and $n$,
	\begin{align*}
		10 \mid x^m y^n - x^n y^m
	\end{align*}
\end{problem}

\begin{problem}[European Mathematical Cup Juniors 2016]
	Let $d(n)$ denote the number of positive divisors of $n$. For a positive integer $n$ we define $f(n)$ as
	\begin{align*}
		f(n) & = d(k_1) + d(k_2) + \ldots + d(k_m)
	\end{align*}
	where $ 1=k_1 < k_2 < \ldots < k_m=n$ are all divisors of the number $n$. We call an integer $n>1$ \textit{almost perfect} if $f(n)=n$. Find all almos perfect numbers.
\end{problem}

\begin{problem}[Finland MAOL Competition 2016]
	Let $n$ be a positive integer. Find all pairs $(x,y)$ of positive integers such that
	\begin{align*}
		(4a-b)(4b-a)=1770^n
	\end{align*}
\end{problem}

\begin{problem}[Germany National Olympiad First Round Ninth/Tenth Grade, 2016] $ $
	\begin{enumerate}[(A)]
		\item Prove that there exists an integer $a>1$ such that the number
		\begin{align*}
			82 \cdot \left(a^8 - a^4\right)
		\end{align*}
		is divisible by the product of three consecutive positive integers each of which has at least two digits.
		\item Determine the smallest prime number $a$ with at least two digits such that the number
		\begin{align*}
			82 \cdot \left(a^8 - a^4\right)
		\end{align*}
		is divisible by the product of three consecutive positive integers each of which has at least two digits.
		\item Determine the smallest integer $a>1$ such that the number
		\begin{align*}
			82 \cdot \left(a^8 - a^2\right)
		\end{align*}
		is divisible by the product of three consecutive positive integers each of which has at least two digits.
	\end{enumerate}
\end{problem}

\begin{problem}[Germany National Olympiad First Round Eleventh/Twelfth Grade, 2016]
	Consider the following system of equations:
	\begin{align*}
		2(z-1) - x &= 55\\
		4xy - 8z   &= 12\\
		a(y+z)     &= 11
	\end{align*}
	Find two largest real values for $a$ for which there are positive integers $x, y$, and $z$ that satisfy the system of equations. In each of these solutions, determine $xyz$.
\end{problem}

\begin{problem}[Germany National Olympiad First Round Eleventh/Twelfth Grade, 2016]
	Find all pairs $(a, b)$ of positive integers for which $(a + 1) (b + 1)$ is divisible by $ab$.
\end{problem}

\begin{problem}[Germany National Olympiad Second Round Tenth Grade, 2016]
	For each of the following cases, determine whether there exist prime numbers $x,y$, and $z$ such that the given equality holds
	\begin{enumerate}[(a)]
		\item $y=z^2-x^2$.
		\item $x^2+y=z^4$.
		\item $x^2 +y^3 = z^4$.
	\end{enumerate}
\end{problem}

\begin{problem}[Germany National Olympiad Second Round Eleventh/Twelfth Grade, 2016]
	The sequence $x_1, x_2, x_3, \dots$ is defined as $x_1 = 1$ and
	\begin{align*}
		x_{k+1} = x_k + y_k
	\end{align*}
	where $y_k$ is the last digit of decimal representation of $x_k$. Prove that the sequence $x_1, x_2, x_3, \dots$ contains all powers of $4$. That is, for every positive integer $n$, there exists some natural $k$ for which $x_k=4^n$.
\end{problem}

\begin{problem}[Germany National Olympiad Third Round Eleventh/Twelfth Grade, 2016]
	Find all positive integers $a$ and $b$ which satisfy
	\begin{align*}
		\binom{ab+1}{2} = 2ab(a+b)
	\end{align*}
\end{problem}

\begin{problem}[Germany National Olympiad Third Round Eleventh/Twelfth Grade, 2016]
	Let $m$ and $n$ be two positive integers. Prove that for every positive integer $k$, the following statements are equivalent:
	\begin{enumerate}
		\item $n+m$ is a divisor of $n^2+km^2$.
		\item $n+m$ is a divisor of $k+1$.
	\end{enumerate}
\end{problem}

\begin{problem}[Germany National Olympiad Fourth Round Ninth Grade, 2016]
	Find all triples $(a, b, c)$ of integers which satisfy
	\begin{align*}
		a^3 + b^3 &= c^3 + 1\\
		b^2 - a^2 &= a + b\\
		2a^3-6a   &= c^3 - 4a^2
	\end{align*}
\end{problem}

\begin{problem}[Germany National Olympiad Fourth Round Tenth Grade, 2016\footnote{Thanks to Arian Saffarzadeh for translating the problem.}]
	A sequence of positive integers $a_1, a_2, a_3, \dots$ is defined as follows: $a_1$ is a $3$ digit number and $a_{k+1}$ (for $k \geq 1$) is obtained by
	\begin{align*}
		a_{k+1} = a_k + 2 \cdot Q(a_k)
	\end{align*}
	where $Q(a_k)$ is the sum of digits of $a_k$ when represented in decimal system. For instance, if one takes $a_1 = 358$ as the initial term, the sequence would be
	\begin{align*}
		a_1 &= 358,\\
		a_2 &= 358 + 2 \cdot 16 = 390\\
		a_3 &= 390+ 2 \cdot 12 = 414\\
		a_4 &= 414 + 2 \cdot 9 = 432\\
		&\phantom{=}\vdots
	\end{align*}
	Prove that no matter what we choose as the starting number of the sequence,
	\begin{enumerate}[(a)]
		\item the sequence will not contain $2015$.
		\item the sequence will not contain $2016$.
	\end{enumerate}
\end{problem}

\begin{problem}[Germany National Olympiad Fourth Round Eleventh Grade, 2016]
	Find all positive integers $m$ and $n$ with $m \leq 2n$ which satisfy
	\begin{align*}
		m \cdot \binom{2n}{n}
			& = \binom{m^2}{2}
	\end{align*}
\end{problem}

\begin{problem}[Germany TST 2016]
	The positive integers $a_1,a_2, \dots, a_n$ are aligned clockwise in a circular line with $n \geq 5$. Let $a_0=a_n$ and $a_{n+1}=a_1$. For each $i \in \{1,2,\dots,n \}$ the quotient
		\begin{align*}
			q_i
				& =\frac{a_{i-1}+a_{i+1}}{a_1}
		\end{align*}
	is an integer. Prove that
		\begin{align*}
			2n
				& \leq q_1+q_2+\dots+q_n\\
				& < 3n
		\end{align*}
	%\flushright \href{http://artofproblemsolving.com/community/c6h1269699p6629923}{Link}
\end{problem}

\begin{problem}[Germany TST 2016, Taiwan TST First Round 2016]
	Determine all positive integers $M$ such that the sequence $a_0, a_1, a_2, \cdots$ defined by
		\begin{align*}
			a_0
				& = M + \frac{1}{2}\\
			a_{k+1}
				& = a_k\lfloor a_k \rfloor\\
				& \vdots
		\end{align*}
	contains at least one integer term. %\hfill \href{http://artofproblemsolving.com/community/c6h1268859p6622268}{Link}
\end{problem}

\begin{problem}[Greece 2016]
	Find all triplets of non-negative integers $(x,y,z)$ and $x\leq y$ such that
		\begin{align*}
			x^2+y^2
				& =3 \cdot 2016^z+77
		\end{align*}
	%\flushright \href{http://artofproblemsolving.com/community/c6h1205284p5947020}{Link}
\end{problem}

\begin{problem}[Greece TST 2016]
	Given is the sequence $(a_n)_{n\geq 0}$ which is defined as follows:$a_0=3$ and $a_{n+1}-a_n=n(a_n-1) \ , \ \forall n\geq 0$. Determine all positive integers $m$ such that $\gcd (m,a_n)=1$ for all $n\geq 0$. %\hfill \href{http://artofproblemsolving.com/community/c6h1269148p6624948}{Link}
\end{problem}

\begin{problem}[Harvard-MIT Math Tournament 2016]
	Denote by $\mathbb{N}$ the positive integers. Let $f:\mathbb{N} \rightarrow \mathbb{N}$ be a function such that, for any $w,x,y,z \in \mathbb{N}$,
		\begin{align*}
			f(f(f(z)))f(wxf(yf(z)))
				& =z^{2}f(xf(y))f(w)
		\end{align*}
	Show that $f(n!) \ge n!$ for every positive integer $n$. %\hfill \href{http://artofproblemsolving.com/community/c6h1231859p6230953}{Link}
\end{problem}

\begin{problem}[Hong Kong (China) Mathematical Olympiad 2016]
	Find all integral ordered triples $(x,y,z)$ such that
		\begin{align*}
			\sqrt{\frac{2015}{x+y}}+\sqrt{\frac{2015}{y+z}}+\sqrt{\frac{2015}{x+z}}
		\end{align*}
	are positive integers. %\hfill \href{http://www.artofproblemsolving.com/community/c6h37234p233034}{Link}
\end{problem}

\begin{problem}[Hong Kong Preliminary Selection Contest 2016]
	Find the remainder when $$19^{17^{15^{\iddots^{3^{1}}}}}$$ is divided by $100$.
\end{problem}

\begin{problem}[Hong Kong Preliminary Selection Contest 2016]
	Let $k$ be an integer. If the equation
		\begin{align*}
			kx^2 + (4k - 2)x + (4k - 7)
				& = 0
		\end{align*}
	has an integral root, find the sum of all possible values of $k$.
\end{problem}

\begin{problem}[Hong Kong Preliminary Selection Contest 2016]
	Let $n$ be a positive integer. If the two numbers $(n + 1)(2n + 15)$ and $n(n + 5)$ have exactly the same prime factors, find the greatest possible value of $n$.
\end{problem}

\begin{problem}[Hong Kong Preliminary Selection Contest 2016]
	An arithmetic sequence with $10$ terms has common difference $d > 0$. If the absolute value of each term is a prime number, find the smallest possible value of $d$.
\end{problem}

\begin{problem}[Hong Kong Preliminary Selection Contest 2016]
	Let $a_1 = \frac{2}{3}$ and
		\begin{align*}
			a_{n+1}
				&= \frac{a_n}{4} + \sqrt{\frac{24a_n+9}{256}} - \frac{9}{48}
		\end{align*}
	for all integers $n \geq 1$. Find the value of
		\begin{align*}
			a_1+a_2+a_3+\dots
		\end{align*}
\end{problem}

\begin{problem}[Hong Kong TST 2016]
	Find all natural numbers $n$ such that $n$, $n^2+10$, $n^2-2$, $n^3+6$, and $n^5+36$ are all prime numbers. %\hfill \href{http://artofproblemsolving.com/community/c6h1155577p5481499}{Link}
\end{problem}

\begin{problem}[Hong Kong TST 2016]
	Find all triples $(m,p,q)$ such that
		\begin{align*}
			2^mp^2 +1
				& = q^7
		\end{align*}
	where $p$ and $q$ are primes and $m$ is a positive integer. %\hfill \href{http://artofproblemsolving.com/community/c6h1150516p5442572}{Link}
\end{problem}

\begin{problem}[Hong Kong TST 2016]
	Find all prime numbers $p$ and $q$ such that $p^2\mid q^3+1$ and $q^2\mid p^6-1$. %\hfill \href{http://artofproblemsolving.com/community/c6h1232696p6240989}{Link}
\end{problem}

\begin{problem}[Hong Kong TST 2016]
	Let $p$ be a prime number greater than $5$. Suppose there is an integer $k$ satisfying that $k^2+5$ is divisible by $p$. Prove that there are positive integers $m$ and $n$ such that $p^2=m^2+5n^2$. %\hfill \href{http://artofproblemsolving.com/community/c6h1235235p6270723}{Link}
\end{problem}

\begin{problem}[IberoAmerican 2016]
	Find all prime numbers $p,q,r,k$ such that $pq+qr+rp = 12k+1$. %\hfill \href{http://artofproblemsolving.com/community/c6h1311821p7030145}{Link}
\end{problem}

\begin{problem}[IberoAmerican 2016]
	Let $k$ be a positive integer and $a_1, a_2,\dots, a_k$ digits. Prove that there exists a positive integer $n$ such that the last $2k$ digits of $2^n$ are, in the following order, $a_1, a_2,\dots, a_k , b_1, b_2, \dots, b_k$, for certain digits $b_1, b_2, \dots, b_k$. %\hfill \href{http://artofproblemsolving.com/community/c6h1312214p7032871}{Link}
\end{problem}

\begin{problem}[IMO Shortlist 2015, India TST 2016, Taiwan TST Second Round 2016, Croatian Mathematical Olympiad 2016, Switzerland TST 2016]
	Let $m$ and $n$ be positive integers such that $m>n$. Define $$x_k=\frac{m+k}{n+k}$$ for $k=1,2,\ldots,n+1$. Prove that if all the numbers $x_1,x_2,\ldots,x_{n+1}$ are integers, then $x_1x_2\ldots x_{n+1}-1$ is divisible by an odd prime. %\flushright \href{http://artofproblemsolving.com/community/c6h1268874p6622379}{Link}
\end{problem}

\begin{problem}[IMO 2016]
	A set of postive integers is called fragrant if it contains at least two elements and each of its elements has a prime factor in common with at least one of the other elements. Let $P(n)=n^2+n+1$. What is the least possible positive integer value of $b$ such that there exists a non-negative integer $a$ for which the set
		\begin{align*}
			\{P(a+1),P(a+2),\ldots,P(a+b)\}
		\end{align*}
	is fragrant? %\hfill \href{http://artofproblemsolving.com/community/c6h1270992p6642559}{Link}
\end{problem}

\begin{problem}[India IMO Training Camp 2016]
	Given that $n$ is a natural number such that the leftmost digits in the decimal representations of $2^n$ and $3^n$ are the same, find all possible values of the leftmost digit. %\hfill \href{http://artofproblemsolving.com/community/c6h1276412p6696449}{Link}
\end{problem}

\begin{problem}[India IMO Practice Test 2016]
	We say a natural number $n$ is perfect if the sum of all the positive divisors of $n$ is equal to $2n$. For example, $6$ is perfect since its positive divisors $1,2,3,6$ add up to $12=2\times 6$. Show that an odd perfect number has at least $3$ distinct prime divisors. %\hfill \href{http://artofproblemsolving.com/community/c6h1276406p6696419}{Link}
\end{problem}

\begin{problem}[India TST 2016]
	Let $n$ be a natural number. We define sequences $\langle a_i\rangle$ and $\langle b_i\rangle$ of integers as follows. We let $a_0=1$ and $b_0=n$. For $i>0$, we let
		\begin{align*}
			(a_i,b_i)
				& =
				\begin{cases}
					\left(2a_{i-1}+1,b_{i-1}-a_{i-1}-1\right) & \text{if } a_{i-1}<b_{i-1}\\
					\left( a_{i-1}-b_{i-1}-1,2b_{i-1}+1\right) & \text{if } a_{i-1}>b_{i-1}\\
					\left(a_{i-1},b_{i-1}\right) & \text{if } a_{i-1}=b_{i-1}
				\end{cases}
		\end{align*}
	Given that $a_k=b_k$ for some natural number $k$, prove that $n+3$ is a power of two. %\hfill \href{http://artofproblemsolving.com/community/c6h1276387p6696361}{Link}
\end{problem}

\begin{problem}[India TST 2016]
	Let $\mathbb N$ denote the set of all natural numbers. Show that there exists two nonempty subsets $A$ and $B$ of $\mathbb N$ such that
	\begin{enumerate}
		\item $A\cap B=\{1\};$
		\item every number in $\mathbb N$ can be expressed as the product of a number in $A$ and a number in $B$;
		\item each prime number is a divisor of some number in $A$ and also some number in $B$;
		\item one of the sets $A$ and $B$ has the following property: if the numbers in this set are written as $x_1<x_2<x_3<\cdots$, then for any given positive integer $M$ there exists $k\in \mathbb N$ such that $x_{k+1}-x_k\ge M$;
		\item Each set has infinitely many composite numbers.
	\end{enumerate}
	%\flushright \href{http://artofproblemsolving.com/community/c6h1276405p6696414}{Link}
\end{problem}

\begin{problem}[India National Olympiad 2016]
	Let $\mathbb{N}$ denote the set of natural numbers. Define a function $T:\mathbb{N}\rightarrow\mathbb{N}$ by $T(2k)=k$ and $T(2k+1)=2k+2$. We write $T^2(n)=T(T(n))$ and in general $T^k(n)=T^{k-1}(T(n))$ for any $k>1$.
	\begin{enumerate}[(i)]
		\item Show that for each $n\in\mathbb{N}$, there exists $k$ such that $T^k(n)=1$.
		\item For $k\in\mathbb{N}$, let $c_k$ denote the number of elements in the set $\{n: T^k(n)=1\}$. Prove that $c_{k+2}=c_{k+1}+c_k$, for $k\ge 1$.
	\end{enumerate}
	%\flushright \href{http://artofproblemsolving.com/community/c6h1186108p5763326}{Link}
\end{problem}

\begin{problem}[India National Olympiad 2016]
	Consider a non-constant arithmetic progression $a_1, a_2,\cdots, a_n,\cdots$. Suppose there exist relatively prime positive integers $p>1$ and $q>1$ such that $a_1^2, a_{p+1}^2$ and $a_{q+1}^2$ are also the terms of the same arithmetic progression. Prove that the terms of the arithmetic progression are all integers. %\hfill \href{http://artofproblemsolving.com/community/c6h1186114p5763347}{Link}
\end{problem}

\begin{problem}[Iran Third Round National Olympiad 2016]
	Let $F$ be a subset of the set of positive integers with at least two elements and $P$ be a polynomial with integer coefficients such that for any two elements of $F$ like $a$ and $b$, the following two conditions hold
	\begin{enumerate}[(i)]
		\item $a+b \in F$, and
		\item $\gcd(P(a),P(b))=1$.
	\end{enumerate}
	Prove that $P(x)$ is a constant polynomial. %\hfill \href{http://artofproblemsolving.com/community/c6h1301958p6940711}{Link}
\end{problem}

\begin{problem}[Iran Third Round National Olympiad 2016]
	Let $P$ be a polynomial with integer coefficients. We say $P$ is \textit{good} if there exist infinitely many prime numbers $q$ such that the set $$X=\left\{P(n) \mod q : \quad n\in \mathbb N\right\}$$
	has at least $\frac{q+1}{2}$ members. Prove that the polynomial $x^3+x$ is good. %\hfill \href{http://artofproblemsolving.com/community/c6h1290176p6821252}{Link}
\end{problem}

\begin{problem}[Iran Third Round National Olympiad 2016]
	Let $m$ be a positive integer. The positive integer $a$ is called a \textit{golden residue} modulo $m$ if $\gcd(a,m)=1$ and $x^x \equiv a \pmod m$ has a solution for $x$. Given a positive integer $n$, suppose that $a$ is a golden residue modulo $n^n$. Show that $a$ is also a golden residue modulo $n^{n^n}$. %\hfill \href{http://artofproblemsolving.com/community/c6h1301963p6940755}{Link}
\end{problem}

\begin{problem}[Iran Third Round National Olympiad 2016]
	Let $p,q$ be prime numbers ($q$ is odd). Prove that there exists an integer $x$ such that
		\begin{align*}
			q
				& \mid (x+1)^p-x^p
		\end{align*}
	if and only if
		\begin{align*}
			q
				& \equiv 1 \pmod p
		\end{align*}
	%\flushright \href{http://artofproblemsolving.com/community/c6h1300984p6930188}{Link}
\end{problem}

\begin{problem}[Iran Third Round National Olympiad 2016]
	We call a function $g$ \textit{special} if $g(x)=a^{f(x)}$ (for all $x$) where $a$ is a positive integer and $f$ is polynomial with integer coefficients such that $f(n)>0$ for all positive integers $n$.

	A function is called an \textit{exponential polynomial} if it is obtained from the product or sum of special functions. For instance, $2^{x}3^{x^{2}+x-1}+5^{2x}$ is an exponential polynomial.

	Prove that there does not exist a non-zero exponential polynomial $f(x)$ and a non-constant polynomial $P(x)$ with integer coefficients such that
	$$P(n)\mid f(n)$$for all positive integers $n$. %\hfill \href{http://artofproblemsolving.com/community/c6h1300987p6930211}{Link}
\end{problem}

\begin{problem}[Iran Third Round National Olympiad 2016]
	A sequence $P=\left \{ a_{n} \right \}_{n=1}^{\infty}$ is called a \textit{permutation} of natural numbers if for any natural number $m$, there exists a unique natural number $n$ such that $a_n=m.$

	We also define $S_k(P)$ as
	$S_k(P)=a_{1}+a_{2}+\dots +a_{k}$ (the sum of the first $k$ elements of the sequence).

	Prove that there exists infinitely many distinct permutations of natural numbers like $P_1,P_2, \dots$ such that for all $k$ and $i<j$,
		\begin{align*}
			S_k(P_i)\mid S_k(P_j)
		\end{align*}
	%\flushright \href{http://artofproblemsolving.com/community/c6h1300990p6930248}{Link}
\end{problem}

\begin{problem}[Iran TST 2016]
	Let $p \neq 13$ be a prime number of the form $8k+5$ such that $39$ is a quadratic non-residue modulo $p$. Prove that the equation $$x_1^4+x_2^4+x_3^4+x_4^4 \equiv 0 \pmod p$$ has a solution in integers such that $p\nmid x_1x_2x_3x_4$. %\hfill \href{http://artofproblemsolving.com/community/c6h1272976p6661198}{Link}
\end{problem}

\begin{problem}[Italy National Olympiad 2016]
	Determine all pairs of positive integers $(a,n)$ with $a\ge n\ge 2$ for which $(a+1)^n+a-1$ is a power of $2$. %\hfill \href{http://artofproblemsolving.com/community/c6h1241079p6338542}{Link}
\end{problem}

\begin{problem}[Japan Mathematical Olympiad Preliminary 2016]
	For $1\leq n\leq 2016$, how many integers $n$ satisfying the condition: the reminder divided by $20$ is smaller than the one divided by $16$. %\hfill \href{http://artofproblemsolving.com/community/c6h1195499p5852489}{Link}
\end{problem}

\begin{problem}[Japan Mathematical Olympiad Preliminary 2016]
	Determine the number of pairs $(a, b)$ of integers such that $1 \leq a, b \leq 2015$, $a$ is divisible by $b + 1$, and $2016 - a$ is divisible by $b$.%\href{http://artofproblemsolving.com/community/c6h1195508p5852517}{Link}
\end{problem}

\begin{problem}[Japan Mathematical Olympiad Finals 2016]
	Let $p$ be an odd prime number. For positive integer $k$ satisfying $1\le k\le p-1$, the number of divisors of $kp+1$ between $k$ and $p$ exclusive is $a_k$. Find the value of $a_1+a_2+\ldots + a_{p-1}$. %\hfill \href{http://artofproblemsolving.com/community/c6h1225510p6158692}{Link}
\end{problem}

\begin{problem}[Junior Balkan Mathematical Olympiad 2016]
	Find all triplets of integers $(a,b,c)$ such that the number
		\begin{align*}
			N = \frac{(a-b)(b-c)(c-a)}{2} + 2
		\end{align*}
	is a power of $2016$. %\hfill \href{http://artofproblemsolving.com/community/c6h1263182p6565545}{Link}
\end{problem}

\begin{problem}[Korea Summer Program Practice Test 2016]
	A infinite sequence $\{ a_n \}_{n \ge 0}$ of real numbers satisfy $a_n \ge n^2$. Suppose that for each $i, j \ge 0$ there exist $k, l$ with $(i,j) \neq (k,l)$, $l - k = j - i$, and $a_l - a_k = a_j - a_i$. Prove that $a_n \ge (n + 2016)^2$ for some $n$. %\hfill \href{http://artofproblemsolving.com/community/c6h1291457p6833569}{Link}
\end{problem}

\begin{problem}[Korea Summer Program Practice Test 2016]
	A finite set $S$ of positive integers is given. Show that there is a positive integer $N$ dependent only on $S$, such that any $x_1, \dots, x_m \in S$ whose sum is a multiple of $N$, can be partitioned into groups each of whose sum is exactly $N$. (The numbers $x_1, \dots, x_m$ need not be distinct.) %\hfill \href{http://artofproblemsolving.com/community/c6h1291452p6833555}{Link}
\end{problem}

\begin{problem}[Korea Winter Program Practice Test 2016]
	$p(x)$ is an irreducible polynomial with integer coefficients, and $q$ is a fixed prime number. Let $a_n$ be a number of solutions of the equation $p(x)\equiv 0\mod q^n$. Prove that we can find $M$ such that $\{a_n\}_{n\ge M}$ is constant. %\hfill \href{http://artofproblemsolving.com/community/c6h1190175p5802127}{Link}
\end{problem}

\begin{problem}[Korea Winter Program Practice Test 2016]
	Find all $\{a_n\}_{n\ge 0}$ that satisfies the following conditions.
	\begin{enumerate}
		\item $a_n\in \mathbb{Z}$,
		\item $a_0=0, a_1=1$,
		\item For infinitely many $m$, $a_m=m$, and
		\item For every $n\ge2$,
			\begin{align*}
				\{2a_i-a_{i-1} | i=1, 2, 3, \cdots , n\}\equiv \{0, 1, 2, \cdots , n-1\}\pmod n
			\end{align*}
	\end{enumerate}
	%\hfill \href{http://artofproblemsolving.com/community/c6h1189496p5795891}{Link}
\end{problem}

\begin{problem}[Korea Winter Program Practice Test 2016]
	Find all positive integers $a, b, m$, and $n$ such that
	\begin{align*}
		a^2+b^2 &=m^2-n^2\\
		ab &=2mn
	\end{align*}
	%\hfill \href{http://artofproblemsolving.com/community/c6h1189489p5795855}{Link}
\end{problem}

\begin{problem}[Korea Winter Program Practice Test 2016]
	Find all pairs of positive integers $(n,t)$ such that $6^n+1=n^2t$, and $(n,29 \times 197)=1$. %\hfill \href{http://artofproblemsolving.com/community/c6h1189502p5795910}{Link}
\end{problem}

\begin{problem}[Korea National Olympiad Final Round 2016]
	Prove that for all rationals $x,y$, $x-\frac{1}{x}+y-\frac{1}{y}=4$ is not true. %\hfill \href{http://artofproblemsolving.com/community/c6h1214200p6032947}{Link}
\end{problem}

\begin{problem}[Kosovo TST 2016]
	Show that for any $n\geq 2$, the number $2^{2^n}+1$ ends with $7$. %\hfill \href{http://artofproblemsolving.com/community/c6h1222334p6119900}{Link}
\end{problem}

\begin{problem}[Latvia National Olympiad 2016]
	$ $
	\begin{enumerate}
		\item Given positive integers $x$ and $y$ such that $xy^2$ is a perfect cube, prove that $x^2y$ is also a perfect cube. %\hfill \href{http://artofproblemsolving.com/community/c6h1276625p6698532}{Link}
		\item Given that $x$ and $y$ are positive integers such that $xy^{10}$ is perfect 33rd power of a positive integer, prove that $x^{10}y$ is also a perfect 33rd power. %\hfill \href{http://artofproblemsolving.com/community/c6h1276645p6698675}{Link}
		\item Given that $x$ and $y$ are positive integers such that $xy^{433}$ is a perfect 2016-power of a positive integer, prove that $x^{433}y$ is also a perfect 2016-power. %\hfill \href{http://artofproblemsolving.com/community/c6h1276664p6698829}{Link}
		\item Given that $x$, $y$ and $z$ are positive integers such that $x^3y^5z^6$ is a perfect 7th power of a positive integer, show that also $x^5y^6z^3$ is a perfect 7th power. %\hfill \href{http://artofproblemsolving.com/community/c6h1276676p6699066}{Link}
	\end{enumerate}
\end{problem}

\begin{problem}[Latvia National Olympiad 2016]
	Prove that among any 18 consecutive positive $3$ digit numbers, there is at least one that is divisible by the sum of its digits. %\hfill \href{http://artofproblemsolving.com/community/c6h1276687p6699218}{Link}
\end{problem}

\begin{problem}[Latvia National Olympiad 2016]
	Two functions are defined by equations: $f(a) = a^2 + 3a + 2$ and $g(b, c) = b^2 - b + 3c^2 + 3c$. Prove that for any positive integer $a$ there exist positive integers $b$ and $c$ such that $f(a) = g(b, c)$. %\flushright \href{http://artofproblemsolving.com/community/c6h1276691p6699257}{Link}
\end{problem}

\begin{problem}[Macedonian National Olympiad 2016]
	Solve the equation in the set of natural numbers $1+x^z + y^z = \text{lcm}(x^z,y^z)$. %\hfill \href{http://artofproblemsolving.com/community/c6h1225106p6154392}{Link}
\end{problem}

\begin{problem}[Macedonian National Olympiad 2016]
	Solve the equation in the set of natural numbers $xyz+yzt+xzt+xyt = xyzt + 3$. %\hfill \href{http://artofproblemsolving.com/community/c6h1225109p6154402}{Link}
\end{problem}

\begin{problem}[Macedonian Junior Mathematical Olympiad 2016]
	Solve the equation
	\begin{align*}
		x_1^4 + x_2^4 + \dots + x_{14}^4 &= 2016^3 -1
	\end{align*}
	in the set of integers.
\end{problem}

\begin{problem}[Macedonian Junior Mathematical Olympiad 2016]
	Solve the equation
	\begin{align*}
		x + y^2 + \left(\gcd(x,y)\right)^2 &= xy \cdot \gcd(x,y)
	\end{align*}
	in the set of positive integers.
\end{problem}

\begin{problem}[Mediterranean Mathematics Olympiad 2016]
	Determine all integers $n\ge1$ for which the number $n^8+n^6+n^4+4$ is prime. %\hfill \href{http://artofproblemsolving.com/community/c6h1252007p6455074}{Link}
\end{problem}

\begin{problem}[Middle European Mathematical Olympiad 2016]
	Find all $f : \mathbb{N} \to \mathbb{N} $ such that $f(a) + f(b)$ divides $2(a + b - 1)$ for all $a, b \in \mathbb{N}$. %\hfill \href{http://artofproblemsolving.com/community/c6h1295282p6869698}{Link}
\end{problem}

\begin{problem}[Middle European Mathematical Olympiad 2016]
	A positive integer $n$ is Mozart if the decimal representation of the sequence $1, 2, \ldots, n$ contains each digit an even number of times.
	Prove that:
	\begin{enumerate}
		\item All Mozart numbers are even.
		\item There are infinitely many Mozart numbers.
	\end{enumerate}
	%\flushright \href{http://artofproblemsolving.com/community/c6h1295945p6876321}{Link}
\end{problem}

\begin{problem}[Middle European Mathematical Olympiad 2016]
	For a positive integer $n$, the equation $a^2 + b^2 + c^2 + n = abc$ is given in the positive integers.
	Prove that:
	\begin{enumerate}
		\item There does not exist a solution $(a, b, c)$ for $n = 2017$.
		\item For $n = 2016$, $a$ is divisible by $3$ for all solutions $(a, b, c)$.
		\item There are infinitely many solutions $(a, b, c)$ for $n = 2016$.
	\end{enumerate}
	%\flushright \href{http://artofproblemsolving.com/community/c6h1295948p6876344}{Link}
\end{problem}

\begin{problem}[Netherlands TST 2016]
	Find all positive integers $k$ for which the equation: $$ \text{lcm}(m,n)-\text{gcd}(m,n)=k(m-n)$$has no solution in integers positive $(m,n)$ with $m\neq n$. %\hfill \href{http://artofproblemsolving.com/community/c6h1309257p7009221}{Link}
\end{problem}

\begin{problem}[Nordic Mathematical Competition 2016]
	Determine all sequences $(a_n)_{n=1}^{2016}$ of non-negative integers such that all sequence elements are less than or equal to $2016$ and
	\begin{align*}
		i+j \mid ia_i + ja_j
	\end{align*}
	for all $i,j \in \{1, 2, \dots, 2016\}$.
\end{problem}

\begin{problem}[Norway Niels Henrik Abel Mathematics Competition Final Round 2016] $ $
	\begin{enumerate}[(a)]
		\item Find all positive integers $a, b, c,$ and $d$ with $a \leq b$ and $c \leq d$ such that
			\begin{align*}
				a + b &= cd\\
				c + d &= ab
			\end{align*}
		\item Find all non-negative integers $x, y$, and $z$ such that
			\begin{align*}
				x^3 + 2y^3 + 4z^3 = 9!
			\end{align*}
	\end{enumerate}
\end{problem}

\begin{problem}[Pan-African Mathematical Olympiad 2016]
	For any positive integer $n$, we define the integer $P(n)$ by
		\begin{align*}
			P(n)
				& = n(n+1)(2n+1)(3n+1)\dots(16n+1)
		\end{align*}
	Find the greatest common divisor of the integers $P(1), P(2), P(3),\dots,P(2016)$. %\hfill \href{http://artofproblemsolving.com/community/c6h1235286p6271340}{Link}
\end{problem}

\begin{problem}[Philippine Mathematical Olympiad Area Stage 2016]
	Let $a, b$, and $c$ be three consecutive even numbers such that $a > b > c$. What is the value of $a^2 + b^2 + c^2 - ab - bc - ac$?
\end{problem}

\begin{problem}[Philippine Mathematical Olympiad Area Stage 2016]
	Find the sum of all the prime factors of 27,000,001.
\end{problem}

\begin{problem}[Philippine Mathematical Olympiad Area Stage 2016]
	Find the largest number $N$ so that
	\begin{align*}
		\sum_{n=5}^{N} \frac{1}{n(n-2)} < \frac{1}{4}
	\end{align*}
\end{problem}

\begin{problem}[Philippine Mathematical Olympiad Area Stage 2016]
	Let $s_n$ be the sum of the digits of a natural number $n$. Find the smallest value of $\frac{n}{s_n}$ if $n$ is a four-digit number.
\end{problem}

\begin{problem}[Philippine Mathematical Olympiad Area Stage 2016]
	The $6$ digit number $\overline{739ABC}$ is divisible by $7, 8$, and $9$. What values can $A, B$, and $C$ take?
\end{problem}

\begin{problem}[Polish Mathematical Olympiad 2016]
	Let $p$ be a certain prime number. Find all non-negative integers $n$ for which polynomial $P(x)=x^4-2(n+p)x^2+(n-p)^2$ may be rewritten as product of two quadratic polynomials $P_1, \ P_2 \in \mathbb{Z}[X]$. %\hfill \href{http://artofproblemsolving.com/community/c6h1224662p6149472}{Link}
\end{problem}

\begin{problem}[Polish Mathematical Olympiad 2016]
	Let $k, n$ be odd positive integers greater than $1$. Prove that if there a exists natural number $a$ such that $k\mid 2^a+1, \ n\mid 2^a-1$, then there is no natural number $b$ satisfying $k\mid 2^b-1, \ n\mid 2^b+1$. %\hfill \href{http://artofproblemsolving.com/community/c6h1224675p6149643}{Link}
\end{problem}

\begin{problem}[Polish Mathematical Olympiad 2016]
	There are given two positive real number $a<b$. Show that there exist positive integers $p, \ q, \ r, \ s$ satisfying following conditions:
	\begin{enumerate}
		\item $a< \frac{p}{q} < \frac{r}{s} < b$.
		\item $p^2+q^2=r^2+s^2$.
	\end{enumerate}
	%\flushright \href{http://artofproblemsolving.com/community/c6h1224679p6149678}{Link}
\end{problem}

\begin{problem}[Romania Danube Mathematical Competition 2016]
	Determine all positive integers $n$ such that all positive integers less than or equal to $n$ and prime to $n$ are pairwise relatively prime.
\end{problem}

\begin{problem}[Romania Danube Mathematical Competition 2016]
	Given an integer $n \geq 2$, determine the numbers that can be written in
	the form
		\begin{align*}
			\sum_{i=2}^{k} a_{i-1}a_i
		\end{align*}
	where $k$ is an integer greater than or equal to $2$, and $a_1,a_2,\dots, a_k$ are positive integers that add up to $n$.
\end{problem}


\begin{problem}[Romania Imar Mathematical Competition 2016]
	Determine all positive integers expressible, for every integer $n \geq 3$,
	in the form
		\begin{align*}
			\frac{(a_1 + 1)(a_2 + 1) \dots (a_n + 1) - 1}{a_1a_2\dots a_n}
		\end{align*}
	where $a_1, a_2, \dots, a_n$ are pairwise distinct positive integers.
\end{problem}



\begin{problem}[Romanian Masters in Mathematics 2016]
	A \textit{cubic sequence} is a sequence of integers given by $a_n =n^3 + bn^2 + cn + d$, where $b, c$ and $d$ are integer constants and $n$ ranges over all integers, including negative integers.
	\begin{enumerate}[(a)]
		\item Show that there exists a cubic sequence such that the only terms
		of the sequence which are squares of integers are $a_{2015}$ and $a_{2016}$.
		\item Determine the possible values of $a_{2015} \cdot a_{2016}$ for a cubic sequence
		satisfying the condition in part $(a)$.
	\end{enumerate}
	%\flushright \href{http://artofproblemsolving.com/community/c6h1204358p5938852}{Link}
\end{problem}


\begin{problem}[Romanian Mathematical Olympiad District Round Grade 5, 2016]
	Find all three-digit numbers which decrease 13 times when the tens'	digit is suppressed.
\end{problem}


\begin{problem}[Romanian Mathematical Olympiad District Round Grade 5, 2016]
	If $A$ and $B$ are positive integers, then $\overline{AB}$ will denote the number obtained by writing, in order, the digits of $B$ after the digits of $A$. For instance, if $A = 193$ and $B = 2016$, then $\overline{AB} = 1932016$.
	Prove that there are infinitely many perfect squares of the form $\overline{AB}$ in each of the following situations:
	\begin{enumerate}[(a)]
		\item $A$ and $B$ are perfect squares;
		\item $A$ and $B$ are perfect cubes;
		\item $A$ is a perfect cube and $B$ is a perfect square;
		\item $A$ is a perfect square and $B$ is a perfect cube.
	\end{enumerate}
\end{problem}

\begin{problem}[Romanian Mathematical Olympiad District Round Grade 6, 2016]
	The positive integers $m$ and $n$ are such that $m^{2016}+m+n^2$ is divisible
	by $mn$.
	\begin{enumerate}[(a)]
		\item Give an example of such $m$ and $n$, with $m > n$.
		\item Prove that $m$ is a perfect square.
	\end{enumerate}
\end{problem}

\begin{problem}[Romanian Mathematical Olympiad District Round Grade 7, 2016]
	Find all pairs of positive integers $(x,y)$ such that
	\begin{align*}
		x + y = \sqrt x + \sqrt y + \sqrt{xy}
	\end{align*}
\end{problem}

\begin{problem}[Romanian Mathematical Olympiad District Round Grade 7, 2016]
	Let
	\begin{align*}
		M &= \left\{x_1 + 2x_2 + 3x_3 + \dots + 2015x_{2015} : x_1, x_2, \dots, 	x_{2015} \in \{-2, 3\}\right\}
	\end{align*}
	Prove that $2015 \in M$ but $2016 \not \in M$.
\end{problem}

\begin{problem}[Romanian Mathematical Olympiad District Round Grade 8, 2016]
	For each positive integer $n$ denote $x_n$ the number of the positive
	integers with $n$ digits, divisible by $4$, formed with digits $2, 0, 1$, or $6$.
	\begin{enumerate}[(a)]
		\item Compute $x_1, x_2, x_3$, and $x_4$;
		\item Find $n$ so that
		\begin{align*}
			1 + \floor{\frac{x_2}{x_1}}+ \floor{\frac{x_3}{x_2}}+ \ldots +\floor{\frac{x_{n+1}}{x_n}}
				& = 2016
		\end{align*}
	\end{enumerate}
\end{problem}

\begin{problem}[Romanian Mathematical Olympiad District Round Grade 8, 2016]
	$ $
	\begin{enumerate}[(a)]
		\item Prove that, for every integer $k$, the equation $x^3 - 24x + k = 0$ has at most one integer solution.
		\item Prove that the equation $x^3 + 24x - 2016 = 0$ has exactly one integer solution.
	\end{enumerate}
\end{problem}

\begin{problem}[Romanian Mathematical Olympiad District Round Grade 9, 2016]
	Let $a$ and $n$ be positive integers such that
	\begin{align*}
		\left\{ \sqrt{n + \sqrt{n}}\right\} = \big\{\sqrt a\big\}
	\end{align*}
	where $\{ \cdot \}$ denotes the fractional part. Prove that $4a+1$ is a perfect square.
\end{problem}

\begin{problem}[Romanian Mathematical Olympiad District Round Grade 9, 2016]
	Let $a \geq 2$ be an integer. Prove that the following statements are equivalent:
	\begin{enumerate}[(a)]
		\item One can find positive integers $b$ and $c$ such that $a^2 = b^2 + c^2$.
		\item One can find a positive integer $d$ such that the equations $x^2 - ax + d = 0$ and $x^2 - ax - d = 0$ have integer roots.
	\end{enumerate}
\end{problem}

\begin{problem}[Romanian Mathematical Olympiad Final Round Grade 5, 2016]
	Two positive integers $x$ and $y$ are such that
	\begin{align*}
		\frac{2010}{2011}
			& < \frac{x}{y}\\
			& < \frac{2011}{2012}
	\end{align*}
	Find the smallest possible value of the sum $x + y$.
\end{problem}

\begin{problem}[Romanian Mathematical Olympiad Final Round Grade 5, 2016]
	Find all the positive integers $a, b$, and $c$ with the property $a + b + c = abc$.
\end{problem}

\begin{problem}[Romanian Mathematical Olympiad Final Round Grade 6, 2016]
	We will call a positive integer \textit{exquisite} if it is a multiple of the number of its divisors (for instance, $12$ is exquisite because it has $6$ divisors and $12$ is a multiple	of $6$).
	\begin{enumerate}[(a)]
		\item Find the largest exquisite two digit number.
		\item Prove that no exquisite number has its last digit $3$.
	\end{enumerate}
\end{problem}

\begin{problem}[Romanian Mathematical Olympiad Final Round Grade 6, 2016]
	Find all positive integers $a$ and $b$ so that $\frac{a + 1}{b}$ and $\frac{b+2}{a}$ are simultaneously positive integers.
\end{problem}

\begin{problem}[Romanian Mathematical Olympiad Final Round Grade 6, 2016]
	Let $a$ and $b$ be positive integers so that there exists a prime number $p$ with the property $[a, a + p] = [b, b + p]$. Prove that $a = b$. Here, $[x, y]$ denotes the least common multiple of $x$ and $y$.
\end{problem}

\begin{problem}[Romanian Mathematical Olympiad Final Round Grade 7, 2016]
	Find all non-negative integers $n$ such that
	\begin{align*}
		\sqrt{n+3} + \sqrt{n+\sqrt{n+3}}
	\end{align*}
	is an integer.
\end{problem}

\begin{problem}[Romanian Mathematical Olympiad Final Round Grade 7, 2016]
	Find all the positive integers $p$ with the property that the sum of the first $p$ positive integers is a four-digit positive integer whose decomposition into prime factors is of the form $2^m3^n(m + n)$, where $m$ and $n$ are non-negative integers.
\end{problem}

\begin{problem}[Romanian Mathematical Olympiad Final Round Grade 8, 2016]
	Let $n$ be a non-negative integer. We will say that the non-negative integers $x_1, x_2, \dots, x_n$ have property $(P)$ if
		\begin{align*}
			x_1x_2 \cdots x_n
				& = x_1 + 2x_2  + \ldots + nx_n
		\end{align*}
	\begin{enumerate}[(a)]
		\item  Show that for every non-negative integer $n$, there exists $n$ positive integers with property $(P)$.
		\item Find all integers $n \geq 2$ so that there exists $n$ positive integers $x_1, x_2, \dots, x_n$	with $x_< x_2< \dots< x_n$, having property $(P)$.
	\end{enumerate}
\end{problem}

\begin{problem}[Romanian Mathematical Olympiad Final Round Grade 9, 2016]
	$ $
	\begin{enumerate}[(a)]
		\item Prove that $7$ cannot be written as a sum of squares of three rational numbers.
		\item Let $a$ be a rational number that can be written as a sum of squares of three rational numbers. Prove that $a^m$ can be written as a sum of squares of three rational numbers, for any positive integer $m$.
	\end{enumerate}
\end{problem}

\begin{problem}[Romanian National Mathematical Olympiad Small Juniors Shortlist, 2016]
	Find all non-negative integers $n$ so that $n^2 - 4n + 2$, $n^2 - 3n + 13$ and $n^2 - 6n + 19$ are simultaneously primes.
\end{problem}

\begin{problem}[Romanian National Mathematical Olympiad Small Juniors Shortlist, 2016]
	We will call a number good if it is a positive integer with at least two digits and by removing one of its digits we get a number which is equal to the sum of its initial digits (for instance, $109$ is good: remove $9$ to get $10 = 1 + 0 + 9$).
	\begin{enumerate}[(a)]
		\item Find the smallest good number.
		\item Find how many numbers are good.
	\end{enumerate}
\end{problem}

\begin{problem}[Romanian National Mathematical Olympiad Small Juniors Shortlist, 2016]
	Four positive integers $a, b, c$, and $d$ are not divisible by $5$ and the sum of their squares is divisible by $5$. Prove that
	\begin{align*}
		N = (a^2 + b^2)(a^2 + c^2)(a^2 + d^2)(b^2 + c^2)(b^2 + d^2)(c^2 + d^2)
	\end{align*}
	is divisible by $625$.
\end{problem}

\begin{problem}[Romanian National Mathematical Olympiad Small Juniors Shortlist, 2016]
	For a positive integer $n$ denote $d(n)$ the number of its positive divisors and $s(n)$ their sum. It is known that $n + d(n) = s(n) + 1$, $m + d(m) = s(m) + 1$, and $nm + d(nm) + 2016 = s(nm)$. Find $n$ and $m$.
\end{problem}

\begin{problem}[Romanian National Mathematical Olympiad Small Juniors Shortlist, 2016]
	Prove that there are no positive integers of the form
		\begin{align*}
			n = \underbrace{\overline{aa\dots a}}_{k\text{ times}} + 5a
		\end{align*}
	divisible by $2016$ where $k>1$.
\end{problem}

\begin{problem}[Romanian National Mathematical Olympiad Small Juniors Shortlist, 2016]
	Find the smallest positive integer of the form
		\begin{align*}
			n = \underbrace{\overline{aa\dots a}}_{k\text{ times}} + a(a-2)^2
		\end{align*}
	divisible by $2016$ where $k>1$.
\end{problem}

\begin{problem}[Romanian National Mathematical Olympiad Small Juniors Shortlist, 2016]
	A positive integer $k$ will be called of \textit{type $n$} ($n \neq k$) if $n$ can be obtained by adding to $k$ the sum or the product of the digits of $k$.
	\begin{enumerate}[(a)]
		\item Show that there are at least two numbers of type 2016.
		\item  Find all numbers of type 216.
	\end{enumerate}
\end{problem}

\begin{problem}[Romanian National Mathematical Olympiad Juniors Shortlist, 2016]
	$ $
	\begin{enumerate}[(a)]
		\item Prove that $2^n+3^n+5^n+8^n$ is not a perfect square for any positive integer $n$.
		\item Find all positive integers $n$ so that
			\begin{align*}
				1^n + 4^n + 6^n + 7^n
					& = 2^n + 3^n + 5^n + 8^n
			\end{align*}
	\end{enumerate}
\end{problem}

\begin{problem}[Romanian National Mathematical Olympiad Juniors Shortlist, 2016]
	$ $
	\begin{enumerate}[(a)]
		\item Find all perfect squares of the form $\overline{aabcc}$.
		\item Let $n$ be a given positive integer. Prove that there exists a perfect square of the form
			\begin{align*}
				\overline{aab\underbrace{cc\cdots c}_{2n\text{ times}}}
			\end{align*}
	\end{enumerate}
\end{problem}

\begin{problem}[Romanian National Mathematical Olympiad Juniors Shortlist, 2016]
	Prove that $2n^2 + 27n + 91$ is a perfect square for infinitely many positive integers $n$.
\end{problem}

\begin{problem}[Romanian National Mathematical Olympiad Seniors Shortlist, 2016]
	Let $p$ be a prime number and $n_1, n_2, \dots, n_k \in \{1, 2, \dots, p -1\}$ be positive integers. Show that the equation
		\begin{align*}
			x_1^{n_1} + x_2^{n_2} + \dots + x_k^{n_k} = x_{k+1}^p
		\end{align*}
	has infinitely many positive integer solutions.
\end{problem}

\begin{problem}[Romanian National Mathematical Olympiad Seniors Shortlist, 2016]
	Let $n \geq 4$ be a positive integer and define $A_n = \{1, 2, \dots, n -1\}$. Find the number of solutions in the set $A_n \times  A_n \times A_n \times A_n$ of the system
	\begin{align*}
		\begin{cases}
			x+z &= 2y\\
			y+t &= 2z
		\end{cases}
	\end{align*}
\end{problem}

\begin{problem}[Romanian Stars of Mathematics Junior Level 2016]
	Show that there are positive odd integers $m_1 < m_2 < \dots  $ and positive integers $n_1 < n_2 < \dots$ such that $m_k$ and $n_k$ are relatively prime, and $m_k^k - 2n_k^4$ is a perfect square for each index $k$.
\end{problem}


\begin{problem}[Romanian Stars of Mathematics Junior Level 2016]
	Given an integer $n \geq 3$ and a permutation $a_1, a_2, \dots, a_n$ of the first $n$ positive integers, show that at least $\sqrt n$ distinct residue classes modulo $n$ occur in the list
		\begin{align*}
			a_1, a_1 + a_2, \dots, a_1 + a_2 + \dots + a_n
		\end{align*}
\end{problem}


\begin{problem}[Romanian Stars of Mathematics Senior Level 2016]
	Let $n$ be a positive integer and let $a_1, a_2, \dots, a_n$ be $n$ positive integers. Show that
	\begin{align*}
		\sum_{k=1}^{n} \frac{\sqrt{a_k}}{1+a_1+a_2+\dots +a_k} &< \sum_{k=1}^{n^2} \frac{1}{k}
	\end{align*}
\end{problem}

\begin{problem}[Romania TST for Junior Balkan Mathematical Olympiad 2016]
	Let $M$ be the set of natural numbers $k$ for which there exists a natural number $n$ such that
		\begin{align*}
			3^n
				& \equiv k\pmod n
		\end{align*}
	Prove that $M$ has infinitely many elements. %\hfill \href{http://artofproblemsolving.com/community/c6h1256385p6499304}{Link}
\end{problem}

\begin{problem}[Romania TST for Junior Balkan Mathematical Olympiad 2016]
	Let $n$ be an integer greater than $2$ and consider the set
	\begin{align*}
		A = \{2^n-1,3^n-1,\dots,(n-1)^n-1\}
	\end{align*}
	Given that $n$ does not divide any element of $A$, prove that $n$ is a square-free number. Does it necessarily follow that $n$ is a prime? %\hfill \href{http://artofproblemsolving.com/community/c6h1257385p6509508}{Link}
\end{problem}

\begin{problem}[Romania TST for Junior Balkan Mathematical Olympiad 2016]
	Let $n$ be a positive integer and consider the system
	\begin{align*}
		S(n):\begin{cases}
			x^2+ny^2=z^2\\
			nx^2+y^2=t^2
		\end{cases}
	\end{align*}
	where $x,y,z$, and $t$ are naturals. If
	\begin{itemize}
		\item $M_1=\{n\in\mathbb N:$ system $S(n)$ has infinitely many solutions$\}$, and
		\item $M_1=\{n\in\mathbb N:$ system $S(n)$ has no solutions$\}$,
	\end{itemize}
	prove that
	\begin{enumerate}[(a)]
		\item $7 \in M_1$ and $10 \in M_2$.
		\item sets $M_1$ and $M_2$ are infinite.
	\end{enumerate}
	%\hfill \href{http://artofproblemsolving.com/community/c6h1257390p6509523}{Link}
\end{problem}

\begin{problem}[Romania TST 2016]
	Let $n$ be a positive integer and let $a_1, a_2, \dots, a_n$ be pairwise distinct positive integers. Show that
	\begin{align*}
		\sum_{k=1}^{n} \frac{1}{[a_1, a_2, \dots, a_k]}
			& < 4
	\end{align*}
	where $[a_1, a_2, \dots, a_k]$ is the least common multiple of the integers $a_1, a_2, \dots, a_k$.
\end{problem}

\begin{problem}[Romania TST 2016]
	Determine the integers $k \geq 2$ for which the sequence
		\begin{align*}
			\binom{2n}{n} \pmod k
		\end{align*}
	is eventually periodic where $0\leq k\leq 2n$.
\end{problem}

\begin{problem}[Romania TST 2016]
	Given positive integers $k$ and $m$, show that $m$ and $\binom{n}{k}$ are relatively prime for infinitely many integers $n \geq k$.
\end{problem}

\begin{problem}[Romania TST 2016]
	Prove that:
	\begin{enumerate}[(a)]
		\item If $(a_n)_{n\geq 1}$ is a strictly increasing sequence of positive integers such that $(a_{2n-1}+a_{2n})/a_n$ is constant as n runs through all positive integers, then this constant is an integer greater than or equal to $4$; and
		\item Given an integer $N \geq 4$, there exists a strictly increasing sequence $(a_n)_{n\geq 1}$ of positive integers such that $(a_{2n-1}+a_{2n})/a_n=N$ for all indices $n$.
	\end{enumerate}
\end{problem}

\begin{problem}[Romaina TST 2016]
	Given a positive integer $k$ and an integer $a \equiv 3 \pmod 8$, show that $a^m + a + 2$ is divisible by $2^k$ for some positive integer $m$.
\end{problem}

\begin{problem}[Romaina TST 2016]
	Given a positive integer $n$, show that for no set of integers modulo $n$, whose size exceeds $1 + \sqrt{n + 4}$, is it possible that the pairwise sums of unordered pairs be all distinct.
\end{problem}

\begin{problem}[Romania TST 2016]
	Given a prime $p$, prove that the sum
		\begin{align*}
			\sum\limits_{k=1}^{\floor{q/p}}{k^{p-1}}
		\end{align*}
	is not divisible by $q$ for all but finitely many primes $q$. %\hfill \href{http://artofproblemsolving.com/community/c6h1251136p6445449}{Link}
\end{problem}

\begin{problem}[Romania TST 2016]
	Determine the positive integers expressible in the form $\frac{x^2+y}{xy+1}$, for at least two pairs $(x,y)$ of positive integers. %\hfill \href{http://artofproblemsolving.com/community/c6h1251122p6445252}{Link}
\end{problem}

\begin{problem}[All-Russian Olympiads 2016, Grade 11]
	Let $n$ be a positive integer and let $k_0,k_1, \dots,k_{2n}$ be nonzero integers such that
		\begin{align*}
			k_0+k_1 +\dots+k_{2n}
				& \neq 0
		\end{align*}
	Is it always possible to a permutation $(a_0,a_1,\dots,a_{2n})$ of $(k_0,k_1,\dots,k_{2n})$ so that the equation
		\begin{align*}
			a_{2n}x^{2n}+a_{2n-1}x^{2n-1}+\dots+a_0
				& =0
		\end{align*}
	has not integer roots? %\hfill \href{http://artofproblemsolving.com/community/c6h1238101p6307084}{Link}
\end{problem}

\begin{problem}[San Diego Math Olympiad 2016]
	Let $a$, $b$, $c$, $d$ be four integers. Prove that
		\begin{align*}
			\left(b-a\right)\left(c-a\right)\left(d-a\right)\left(d-c\right)\left(d-b\right)\left(c-b\right)
		\end{align*}
	is divisible by $12$. %\hfill \href{http://artofproblemsolving.com/community/c6h1217917p6073408}{Link}
\end{problem}

\begin{problem}[San Diego Math Olympiad 2016]
	Quadratic equation $ x^2+ax+b+1=0$ have 2 positive integer roots, for integers $ a,b$. Show that $ a^2+b^2$ is not a prime. %\hfill \href{http://artofproblemsolving.com/community/c6h307428p1660713}{Link}
\end{problem}

\begin{problem}[Saudi Arabia Pre-selection Test 2016]
	Let $p$ be a given prime. For each prime $r$, we define
	\begin{align*}
		F(r) &= \frac{(p^{rp} -1)(p-1)}{(p^r-1)(p^p-1)}
	\end{align*}
	\begin{enumerate}
		\item Show that $F(r)$ is a positive integer for any prime $r \neq p$.
		\item Show that $F(r)$ and $F(s)$ are relatively prime for any primes $r$ and $s$ such that $r \neq p, s\neq p$ and $r \neq s$.
		\item Fix a prime $r \neq p$. Show that there is a prime divisor $q$ of $F(r)$ such that $p \mid q - 1$ but $p^2 \nmid q - 1$.
	\end{enumerate}
\end{problem}

\begin{problem}[Saudi Arabia Pre-selection Test 2016]
	Let $u$ and $v$ be positive rational numbers with $u \neq v$. Assume that there are infinitely many positive integers $n$ with the property that $u^n - v^n$ is an integer. Prove that $u$ and $v$ are integers.
\end{problem}

\begin{problem}[Saudi Arabia Pre-selection Test 2016]
	Let $a$ and $b$ be two positive integers such that
	\begin{align*}
		b + 1
			& \mid a^2 + 1\\
		a + 1
			& \mid b^2 + 1
	\end{align*}
	Prove that both $a$ and $b$ are odd.
\end{problem}

\begin{problem}[Saudi Arabia Pre-selection Test 2016]
	$ $
	\begin{enumerate}
		\item Prove that there are infinitely many positive integers $n$ such that there exists a permutation of $1, 2,3, \dots, n$ with the property that the difference between any two adjacent numbers is equal to either $2015$ or $2016$.
		\item Let $k$ be a positive integer. Is the statement in part $1$ still true if we replace the numbers $2015$ and $2016$ by $k$ and $k + 2016$, respectively?
	\end{enumerate}
\end{problem}

\begin{problem}[Saudi Arabia Pre-selection Test 2016]
	Let $n$ be a given positive integer. Prove that there are infinitely many pairs of positive integers $(a, b)$ with $a, b > n$ such that
	\begin{align*}
		\prod_{i=1}^{2015} (a+i) &\mid b(b+2016)\\
		\prod_{i=1}^{2015} (a+i) &\nmid b\\
		\prod_{i=1}^{2015} (a+i) &\nmid (b+2016)
	\end{align*}
\end{problem}

\begin{problem}[Saudi Arabia TST for Gulf Mathematical Olympiad 2016]
	Find all positive integer $n$ such that there exists a permutation $(a_1, a_2,\dots , a_n)$ of $(1, 2,3, \dots, n)$ satisfying the condition:
		\begin{align*}
			k
				& \mid a_1 + a_2 + \ldots+ a_k
		\end{align*}
	for $1\leq k\leq n$.
\end{problem}

\begin{problem}[Saudi Arabia TST for Balkan Mathematical Olympiad 2016]
	Show that there are infinitely many positive integers $n$ such that $n$ has at least two prime divisors and $20^n + 16^n$ is divisible by $n^2$.
\end{problem}

\begin{problem}[Saudi Arabia TST for Balkan Mathematical Olympiad 2016]
	Let $m$ and $n$ be odd integers such that $(n^2 - 1)$ is divisible by $m^2 + 1 - n^2$. Prove that $|m^2 + 1 - n^2|$ is a perfect square.
\end{problem}

\begin{problem}[Saudi Arabia TST for Balkan Mathematical Olympiad 2016]
	Let $a > b > c > d$ be positive integers such that
		\begin{align*}
			a^2 + ac - c^2
				& = b^2 + bd - d^2
		\end{align*}
	Prove that $ab + cd$ is a composite number.
\end{problem}

\begin{problem}[Saudi Arabia TST for Balkan Mathematical Olympiad 2016]
	For any positive integer $n$, show that there exists a positive integer $m$ such that $n$ divides $2016^m + m$.
\end{problem}

\begin{problem}[Saudi Arabia TST for Balkan Mathematical Olympiad 2016]
	Let $d$ be a positive integer. Show that for every integer $S$, there exist a positive integer $n$ and a sequence $a_1, a_2,\dots , a_n \in \{-1, 1\}$ such that
		\begin{align*}
			S = a_1(1 + d)^2 + a_2(1 + 2d)^2 + \dots + a_n(1 + nd)^2
		\end{align*}
\end{problem}

\begin{problem}[Saudi Arabia TST for Balkan Mathematical Olympiad 2016]
	Let $p$ and $q$ be given primes and the sequence $(p_n)_{n\geq 1}$ defined recursively as follows: $p_1 = p$, $p_2 = q$, and $p_{n+2}$ is the largest prime divisor of the number $(p_n + p_{n+1} + 2016)$ for all $n \geq 1$. Prove that this sequence is bounded. That is, there exists a positive real number $M$ such that $a_n < M$ for all positive integers $n$.
\end{problem}

\begin{problem}[Saudi Arabia IMO TST 2016]
	Let $n \geq 3$ be an integer and let $x_1, x_2,\dots , x_n$ be $n$ distinct integers. Prove that
		\begin{align*}
			(x_1 - x_2)^2 +(x_2 - x_3)^2 + \dots+(x_n - x_1)^2
				& \geq 4n - 6
		\end{align*}
\end{problem}

\begin{problem}[Saudi Arabia IMO TST 2016]
	Let $k$ be a positive integer. Prove that there exist integers $x$ and $y$, neither of which divisible by $7$, such that
		\begin{align*}
			x^2 + 6y^2
				& = 7^k
		\end{align*}
\end{problem}

\begin{problem}[Saudi Arabia IMO TST 2016]
	Define the sequence $a_1, a_2,\dots a_s$ follows: $a_1 = 1$, and for every $n \geq 2$, $a_n = n - 2$ if $a_{n-1} = 0$ and $a_n = a_{n-1} - 1$, otherwise. Find the number of $1 \leq k \leq 2016$ such that there are non-negative integers $r$ and $s$ and a positive integer $n$ satisfying $k = r + s$ and $a_{n+r} = a_n + s.$
\end{problem}

\begin{problem}[Saudi Arabia IMO TST 2016]
	Let a be a positive integer. Find all prime numbers $p$ with the following property: there exist exactly $p$ ordered pairs of integers $(x, y)$, with $0 \leq x, y \leq p - 1$, such that $p$ divides $y^2 - x^3 - a^2x$.
\end{problem}

\begin{problem}[Saudi Arabia IMO TST 2016]
	Find the number of permutations $(a_1, a_2, \dots, a_{2016})$ of the first $2016$ positive integers satisfying the following two conditions:
	\begin{enumerate}
		\item $a_{i+1} - a_i \leq 1$ for all $i =1 , 2, \dots, 2015$, and
		\item There are exactly two indices $i < j$ with $1 \leq i < j \leq 2016$ such that $a_i = i$ and $a_j = j$.
	\end{enumerate}
\end{problem}

\begin{problem}[Saudi Arabia IMO TST 2016]
	Call a positive integer $N \geq 2$ \textit{special} if for every $k$ such that $2 \leq k \leq N$, $N$ can be expressed as a sum of $k$ positive integers that are relatively prime to $N$ (although not necessarily relatively prime to each other). Find all special positive integers.
\end{problem}

\begin{problem}[Serbia Additional TST 2016]
	Let $w(x)$ be largest odd divisor of $x$. Let $a,b$ be natural numbers such that $(a,b)=1$ and $a+w(b+1)$ and $b+w(a+1)$ are powers of two. Prove that $a+1$ and $b+1$ are powers of two. %\hfill \href{http://artofproblemsolving.com/community/c6h1222842p6126065}{Link}
\end{problem}

\begin{problem}[Serbia National Olympiad 2016]
	Let $n>1$ be an integer. Prove that there exist $m>n^n $ such that $$\frac {n^m-m^n}{m+n}$$ is a positive integer. %\hfill \href{http://artofproblemsolving.com/community/c6h1220643p6101949}{Link}
\end{problem}

\begin{problem}[Serbia National Olympiad 2016]
	Let $a_1, a_2, \dots, a_{2^{2016}}$ be positive integers not bigger than $2016$. We know that for each $n \leq 2^{2016}$, $a_1a_2 \dots a_{n} +1 $ is a perfect square. Prove that for some $i $ , $a_i=1$. %\hfill \href{http://artofproblemsolving.com/community/c6h1221134p6106775}{Link}
\end{problem}

\begin{problem}[Serbia TST for Junior Balkan Mathematical Olympiad 2016]
	Find minimal number of divisors that can number $|2016^m-36^n|$ have,where $m$ and $n$ are natural numbers. %\hfill \href{http://artofproblemsolving.com/community/c6h1245773p6388606}{Link}
\end{problem}

%\begin{problem}[Slovakia Category A Mathematical Olympiad 2017]
%	Find all prime numbers $p$ for which there exists a natural number $n$ such that $p^n+1$ is a perfect cube.
%\end{problem}
%
%
%\begin{problem}[Slovakia Category B Mathematical Olympiad 2017]
%	Find all eight digit numbers
%		\begin{align*}
%			\overline{\ast 2 \ast 0 \ast 1 \ast 6},
%		\end{align*}
%	which are divisible by $2016$, where each asterisk is an odd digit.
%\end{problem}
%
%
%
%\begin{problem}[Slovakia Category C Mathematical Olympiad 2017]
%	Find the largest integer $d$ such that for any positive integer $n$, the number
%		\begin{align*}
%			V(n) = n^4 + 11n^2 - 12
%		\end{align*}
%	is divisible by $d$.
%\end{problem}

\begin{problem}[Slovakia Domestic Category B Mathematical Olympiad 2016]
	Let $k, l$, and $m$ be positive integers such that
	\begin{align*}
		\frac{k+m+klm}{lm+1} &= \frac{2051}{44}
	\end{align*}
	Find all possible values for $klm$.
\end{problem}

\begin{problem}[Slovakia Domestic Category B Mathematical Olympiad 2016]
	A positive integer has the property that the number of its even divisors is $3$ more than the number of its odd divisors. What is the ratio of sum of all even divisors over the sum of all odd divisors of this number? Find all possible answers.
\end{problem}

\begin{problem}[Slovakia Domestic Category C Mathematical Olympiad 2016]
	Find all possible values for the product $pqr$, where $p, q$, and $r$ are primes satisfying
	\begin{align*}
		p^2 - (q+r)^2 = 637
	\end{align*}
\end{problem}

\begin{problem}[Slovakia School Round Category C Mathematical Olympiad 2016]
	Find all four digit numbers $\overline{abcd}$ such that
	\begin{align*}
		\overline{abcd} = 20 \cdot \overline{ab} + 16 \cdot \overline{cd}
	\end{align*}
\end{problem}

\begin{problem}[Slovakia Regional Round Category B Mathematical Olympiad 2016]
	Determine all positive integers $k,l$, and $m$ such that
		\begin{align*}
			\frac{3l+1}{3kl+k+3} = \frac{lm + 1}{5lm+m+5}
		\end{align*}
\end{problem}

\begin{problem}[Slovakia Regional Round Category C Mathematical Olympiad 2016]
	Find the least possible value of
		\begin{align*}
			3x^2 - 12xy + y^4
		\end{align*}
	where $x$ and $y$ are non-negative integers.
\end{problem}

\begin{problem}[Slovakia National Round Category A Mathematical Olympiad 2016]
	Let $p>3$ be a prime. Determine the number of all $6-$tuples $(a,b,c,d,e,f)$ of positive integers with sum $3p$ such that
		\begin{align*}
			\frac{a+b}{c+d}, \frac{b+c}{d+e}, \frac{c+d}{e+f}, \frac{d+e}{f+a}, \frac{e+f}{a+b}
		\end{align*}
	are all integers.
\end{problem}

\begin{problem}[Slovakia TST 2016]
	Let $n$ be a positive integer and let $S_n$ be the set of all positive divisors of $n$ (including $1$ and $n$). Prove that the rightmost digit of more than half of the elements of $S_n$ is $3$.
\end{problem}

\begin{problem}[Slovakia TST 2016]
	Find all odd integers $M$ for which the sequence $a_0,a_1,a_2,\dots$ defined by $a_0=\frac{1}{2}(2M+1)$ and $a_{k+1} = a_k \lfloor a_k \rfloor$ for $k=0,1,2,\dots$ contains at least one integer.
\end{problem}

\begin{problem}[South Africa National Olympiad 2016]
	Let $k$ and $m$ be integers with $1 < k < m$. For a positive integer $i$, let $L_i$ be the least common multiple of $1,2,\ldots,i$.
	Prove that $k$ is a divisor of
		\begin{align*}
			L_i \cdot \squarebracket{\binom{m}{i} - \binom{m-k}{i}}
		\end{align*}
	for all $i \geq 1$. %\hfill \href{http://artofproblemsolving.com/community/c6h1310094p7016426}{Link}
\end{problem}

\begin{problem}[Slovenia National Math Olympiad First Grade 2016]
	Find all relatively prime integers $x$ and $y$ that solve the equation
	\begin{align*}
		4x^3 + y^3 = 3xy^2
	\end{align*}
\end{problem}

\begin{problem}[Slovenia National Math Olympiad Fourth Grade 2016]
	Find all integers $a,b,c,$ and $d$ that solve the equation
	\begin{align*}
		a^2 + b^2 + c^2 &= d + 13\\
		a + 2b + 3c &= \frac{d}{2}+ 13
	\end{align*}
\end{problem}

\begin{problem}[Slovenia IMO TST 2016]
	Let
	\begin{align*}
		N &= 2^{15} \cdot 2015
	\end{align*}
	How many divisors of $N^2$ are strictly smaller than $N$ and do not divide $N$?
\end{problem}

\begin{problem}[Slovenia IMO TST 2016, Philippine 2015]
	Prove that for all positive integers $n \geq 2$,
	\begin{align*}
		\frac{1}{2} + \sqrt{\frac{1}{2}}+ \sqrt[3]{\frac{2}{3}}+ \dots + \sqrt[n]{\frac{n-1}{n}} < \frac{n^2}{n+1}
	\end{align*}
\end{problem}

\begin{problem}[Slovenia IMO TST 2016, Romania JBMO TST 2015]
	Find all positive integers $a, b, c$, and $d$ such that
	\begin{align*}
		4^a \cdot 5^b - 3^c \cdot 11^d = 1
	\end{align*}
\end{problem}

\begin{problem}[Spain National Olympiad 2016]
	Two real number sequences are guiven, one arithmetic $\left(a_n\right)_{n\in \mathbb {N}}$ and another geometric sequence $\left(g_n\right)_{n\in \mathbb {N}}$ none of them constant. Those sequences verifies $a_1=g_1\neq 0$, $a_2=g_2$ and $a_{10}=g_3$. Find with proof that, for every positive integer $p$, there is a positive integer $m$, such that $g_p=a_m$. %\hfill \href{http://artofproblemsolving.com/community/c6h1221152p6106939}{Link}
\end{problem}

\begin{problem}[Spain National Olympiad 2016]
	Given a positive prime number $p$. Prove that there exist a positive integer $\alpha$ such that $p\mid \alpha(\alpha-1)+3$, if and only if there exist a positive integer $\beta$ such that $p\mid \beta(\beta-1)+25$. %\hfill \href{http://artofproblemsolving.com/community/c6h1221267p6107917}{Link}
\end{problem}

\begin{problem}[Spain National Olympiad 2016]
	Let $m$ be a positive integer and $a$ and $b$ be distinct positive integers strictly greater than $m^2$ and strictly less than $m^2+m$. Find all integers $d$ such that $m^2 < d < m^2+m$ and $d$ divides $ab$. %\hfill \href{http://artofproblemsolving.com/community/c6h1290418p6823262}{Link}
\end{problem}

\begin{problem}[Switzerland Preliminary Round 2016]
	Determine all natural numbers $n$ such that for all positive divisors $d$ of $n$,
	\begin{align*}
		d + 1 \mid n + 1
	\end{align*}
\end{problem}

\begin{problem}[Switzerland Final Round 2016]
	Find all positive integers $n$ for which primes $p$ and $q$ exist such that
	\begin{align*}
		p(p+1) + q(q+1) = n(n+1)
	\end{align*}
\end{problem}

\begin{problem}[Switzerland Final Round 2016]
	Let $a_n$ be a sequence of positive integers defined by $a_1 = m$ and $a_{n} = a_{n-1}^2 - 1$ for $n= 2, 3, 4, \dots$. A pair $(a_k, a_l)$ is called \textit{interesting} if
	\begin{enumerate}[(i)]
		\item $0 < l - k < 2016$, and
		\item $a_k$ divides $a_l$.
	\end{enumerate}
	Prove that there exists a positive integer $m$ such that the sequence $a_n$ contains no interesting pair.
\end{problem}

\begin{problem}[Switzerland TST 2016]
	Let $n$ be a positive integer. We call a pair of natural numbers \textit{incompatible} if their greatest common divisor is equal to $1$. Find the minimum value of incompatible pairs when one divides the set $\{1,2,\dots,2n\}$ into $n$ pairs.
\end{problem}

\begin{problem}[Switzerland TST 2016]
	Let $n$ be a positive integer. Show that $7^{7^n} + 1$ has at least $2n + 3$ prime divisors (not necessarily distinct).
\end{problem}

\begin{problem}[Switzerland TST 2016]
	Find all positive integers $n$ such that
	\begin{align*}
		\sum_{\substack{d \mid n \\ 1 \leq d \leq n}} d^2 = 5(n+1)
	\end{align*}
\end{problem}

\begin{problem}[Syria Central Round First Stage 2016]
	A positive integer $n \geq 2$ is called \textit{special} if $n^2$ can be written as sum of $n$ consecutive positive integers (for instance, $3$ is special since $3^2 = 2 + 3 + 4$).
	\begin{enumerate}[(i)]
		\item Prove that $2016$ is not special.
		\item Prove that the product of two special numbers is also special.
	\end{enumerate}
\end{problem}

\begin{problem}[Syria Central Round Second Stage 2016]
	Find all integers $a$ and $b$ such that $a^3 - b^2 = 2$.
\end{problem}

\begin{problem}[Syria TST 2016]
	Find all positive integers $m$ and $n$ such that
	\begin{align*}
		\frac{1}{m} + \frac{1}{n} = \frac{3}{2014}
	\end{align*}
\end{problem}

\begin{problem}[Taiwan TST First Round 2016]
	Find all ordered pairs $(a,b)$ of positive integers that satisfy $a>b$ and the equation $(a-b)^{ab}=a^bb^a$. %\hfill \href{http://artofproblemsolving.com/community/c6h1269733p6630169}{Link}
\end{problem}

\begin{problem}[Taiwan TST Second Round 2016]
	Let $a$ and $b$ be positive integers such that $a! + b!$ divides $a!b!$. Prove that $3a \ge 2b + 2$. %\hfill \href{http://artofproblemsolving.com/community/c6h1268852p6622214}{Link}
\end{problem}

\begin{problem}[Taiwan TST Second Round 2016]
	Let $\left< F_n\right>$ be the Fibonacci sequence, that is, $F_0=0$, $F_1=1$, and $F_{n+2}=F_{n+1}+F_{n}$ holds for all non-negative integers $n$.
	Find all pairs $(a,b)$ of positive integers with $a < b$ such that $F_n-2na^n$ is divisible by $b$ for all positive integers $n$. %\hfill \href{http://artofproblemsolving.com/community/c6h1274121p6672411}{Link}
\end{problem}

\begin{problem}[Taiwan TST Third Round 2016]
	Let $n$ be a positive integer. Find the number of odd coefficients of the polynomial $(x^2-x+1)^n$. %\hfill \href{http://artofproblemsolving.com/community/c6h1275931p6691592}{Link}
\end{problem}

\begin{problem}[Taiwan TST Third Round 2016]
	Let $k$ be a positive integer. A sequence $a_0, a_1, \dots, a_n$ $(n>0)$ of positive integers satisfies the following conditions:
	\begin{enumerate}[(i)]
		\item $a_0=a_n=1$;
		\item $2\leq a_i\leq k$ for each $k=1,2,\dots,n-1$
		\item For each $j=2,3,\dots,k$, the number j appears $\varphi (j)$ times in the sequence $a_0,a_1,\dots,a_n$ ($\varphi (j)$ is the number of positive integers that do not exceed $j$ and are relatively prime to $j$);
		\item For any $i=1,2,\dots,n-1$, $\gcd(a_{i-1},a_i)=1=\gcd(a_i,a_{i+1})$, and $a_i$ divides $a_{i-1}+a_{i+1}$
	\end{enumerate}
	There is another sequence $b_0,b_1,\dots,b_n$ of integers such that $$\frac{b_{i+1}}{a_{i+1}}>\frac{b_i}{a_i}$$ for all $i=0,1,\dots,n-1$. Find the minimum value for $b_n-b_0$. %\hfill \href{http://www.artofproblemsolving.com/community/c6h1232316p6236640}{Link}
\end{problem}

\begin{problem}[Taiwan TST Third Round 2016]
	Let $f(x)$ be the polynomial with integer coefficients ($f(x)$ is not constant) such that
	\[(x^3+4x^2+4x+3)f(x)=(x^3-2x^2+2x-1)f(x+1)\]Prove that for each positive integer $n\geq8$, $f(n)$ has at least five distinct prime divisors. %\hfill \href{http://artofproblemsolving.com/community/c6h1278533p6716098}{Link}
\end{problem}

\begin{problem}[Turkey TST for European Girls' Mathematical Olympiad 2016]
	Prove that for every square-free integer $n>1$, there exists a prime number $p$ and an integer $m$ satisfying $p \mid n$ and $n \mid p^2+p\cdot m^p$.
	%\flushright \href{http://artofproblemsolving.com/community/c6h1248658p6420003}{Link}
\end{problem}

\begin{problem}[Turkey TST for Junior Balkan Mathematical Olympiad 2016]
	Let $n$ be a positive integer, $p$ and $q$ be prime numbers such that
		\begin{align*}
			pq
				& \mid n^p+2\\
			n+2
				& \mid n^p+q^p
		\end{align*}
	Prove that there exists a positive integer $m$ satisfying $q \mid 4^m \cdot n +2$. %\hfill \href{http://artofproblemsolving.com/community/c6h1246256p6393813}{Link}
\end{problem}

\begin{problem}[Turkey TST for Junior Balkan Mathematical Olympiad 2016]
	Find all pairs $(p, q)$ of prime numbers satisfying
		\begin{align*}
			p^3+7q
				& = q^9+5p^2+18p
		\end{align*}
	%\flushright \href{http://artofproblemsolving.com/community/c6h1246266p6393971}{Link}
\end{problem}

\begin{problem}[Turkey TST 2016]
	$p$ is a prime. Let $K_p$ be the set of all polynomials with coefficients from the set $\{0,1,\dots ,p-1\}$ and degree less than $p$. Assume that for all pairs of polynomials $P,Q\in K_p$ such that $P(Q(n))\equiv n\pmod p$ for all integers $n$, the degrees of $P$ and $Q$ are equal. Determine all primes $p$ with this condition. %\hfill \href{http://artofproblemsolving.com/community/c6h1225651p6159696}{Link}
\end{problem}

\begin{problem}[Turkmenistan Regional Olympiad 2016]
	Find all distinct prime numbers $p,q,r,s$ such that
		\begin{align*}
			1-\frac{1}{p} - \frac{1}{q} -\frac{1}{r} - \frac{1}{s}
				& =\frac{1}{pqrs}
		\end{align*}
	%\flushright \href{http://artofproblemsolving.com/community/c6h1202195p5915235}{Link}
\end{problem}

\begin{problem}[Tuymaada Senior League 2016]
	For each positive integer $k$ determine the number of solutions of the equation
		\begin{align*}
			8^k
				& = x^3 + y^3 + z^3 - 3xyz
		\end{align*}
	in non-negative integers $x,y$, and $z$ such that $0 \leq x \leq y \leq z$.
\end{problem}

\begin{problem}[Tuymaada Senior League 2016]
	The ratio of prime numbers $p$ and $q$ does not exceed $2$ ($p \neq q$). Prove
	that there are two consecutive positive integers such that the largest
	prime divisor of one of them is $p$ and that of the other is $q$.
\end{problem}

\begin{problem}[Tuymaada Junior League 2016]
	Is there a positive integer $N > 10^{20}$ such that all its decimal digits
	are odd, the numbers of digits $1, 3, 5, 7, 9$ in its decimal representation
	are equal, and it is divisible by each $20-$digit number obtained from it by
	deleting digits? (Neither deleted nor remaining digits must be consecutive.)
\end{problem}

\begin{problem}[Ukraine TST for UMO 2016]
	Find all numbers $n$ such, that in $[1;1000]$ there exists exactly $10$ numbers with digit sum equal to $n$. %\hfill \href{http://artofproblemsolving.com/community/c6h1199874p5893842}{Link}
\end{problem}

\begin{problem}[Ukraine TST for UMO 2016]
	Number $125$ is written as the sum of several pairwise distinct and relatively prime numbers, greater than $1$. What is the maximal possible number of terms in this sum? %\hfill \href{http://artofproblemsolving.com/community/c6h1200058p5895479}{Link}
\end{problem}

\begin{problem}[Ukraine TST for UMO 2016]
	Given prime number $p$ and different natural numbers $m, n$ such that $p^2=\frac{m^2+n^2}{2}$. Prove that $2p-m-n$ is either square or doubled square of an integer number. %\hfill \href{http://artofproblemsolving.com/community/c6h1199895p5893899}{Link}
\end{problem}

\begin{problem}[Ukraine TST for UMO 2016]
	Solve the equation $n(n^2+19)=m(m^2-10)$ in positive integers. %\hfill \href{http://artofproblemsolving.com/community/c6h1202161p5914918}{Link}
\end{problem}

\begin{problem}[USA AIME 2016]
	For $-1 < r < 1$, let $S(r)$ denote the sum of the geometric series
		\begin{align*}
			12 + 12r + 12r^2 + 12r^3 + \cdots
		\end{align*}
	Let $a$ between $-1$ and $1$ satisfy $S(a)S(-a)=2016$. Find $S(a) + S(-a)$. %\hfill \href{http://artofproblemsolving.com/community/c5h1207186p5966121}{Link}
\end{problem}

\begin{problem}[USA AIME 2016]
	For a permutation $p = (a_1,a_2,\ldots,a_9)$ of the digits $1,2,\ldots,9$, let $s(p)$ denote the sum of the three $3$-digit numbers $a_1a_2a_3$, $a_4a_5a_6$, and $a_7a_8a_9$. Let $m$ be the minimum value of $s(p)$ subject to the condition that the units digit of $s(p)$ is $0$. Let $n$ denote the number of permutations $p$ with $s(p) = m$. Find $|m - n|$. %\hfill \href{http://artofproblemsolving.com/community/c5h1207194p5966150}{Link}
\end{problem}

\begin{problem}[USA AIME 2016]
	A strictly increasing sequence of positive integers $a_1, a_2, a_3, \ldots$ has the property that for every positive integer $k$, the subsequence $a_{2k-1}, a_{2k}, a_{2k+1}$ is geometric and the subsequence $a_{2k}, a_{2k+1}, a_{2k+2}$ is arithmetic. Suppose that $a_{13} = 2016$. Find $a_1$. %\hfill \href{http://artofproblemsolving.com/community/c5h1207204p5966186}{Link}
\end{problem}

\begin{problem}[USA AIME 2016]
	Find the least positive integer $m$ such that $m^2 - m + 11$ is a product of at least four not necessarily distinct primes. %\hfill \href{http://artofproblemsolving.com/community/c5h1207207p5966196}{Link}
\end{problem}

\begin{problem}[USA AIME 2016]
	Let $x,y$ and $z$ be real numbers satisfying the system
		\begin{align*}
			\log_2(xyz-3+\log_5 x) &= 5\\
			\log_3(xyz-3+\log_5 y) &= 4 \\
			\log_4(xyz-3+\log_5 z) &= 4
		\end{align*}
	Find the value of $|\log_5 x|+|\log_5 y|+|\log_5 z|$. %\hfill \href{http://artofproblemsolving.com/community/c5h1213121p6023524}{Link}
\end{problem}

\begin{problem}[USA AIME 2016]
	For polynomial $P(x)=1-\frac{1}{3}x+\frac{1}{6}x^2$, define
		\begin{align*}
			Q(x)
				& = P(x)P(x^3)P(x^5)P(x^7)P(x^9)\\
				& = \sum\limits_{i=0}^{50}a_ix^i
		\end{align*}
	Then,
		\begin{align*}
			\sum\limits_{i=0}^{50}|a_i|=\frac{m}{n}
		\end{align*}
	where $m$ and $n$ are relatively prime positive integers. Find $m+n$. %\hfill \href{http://artofproblemsolving.com/community/c5h1213127p6023562}{Link}
\end{problem}

\begin{problem}[USA AIME 2016]
	Find the number of sets $\{a,b,c\}$ of three distinct positive integers with the property that the product of $a,b,$ and $c$ is equal to the product of $11,21,31,41,51,$ and $61$. %\hfill \href{http://artofproblemsolving.com/community/c5h1213129p6023597}{Link}
\end{problem}

\begin{problem}[USA AIME 2016]
	The sequences of positive integers $1,a_2,a_3,\ldots$ and $1,b_2,b_3,\ldots$ are an increasing arithmetic sequence and an increasing geometric sequence, respectively. Let $c_n=a_n+b_n$. There is an integer $k$ such that $c_{k-1}=100$ and $c_{k+1}=1000$. Find $c_k$. %\hfill \href{http://artofproblemsolving.com/community/c5h1213132p6023608}{Link}
\end{problem}

\begin{problem}[USA AIME 2016]
	For positive integers $N$ and $k$, define $N$ to be $k$-nice if there exists a positive integer $a$ such that $a^k$ has exactly $N$ positive divisors. Find the number of positive integers less than $1000$ that are neither $7$-nice nor $8$-nice. %\flushright \href{http://artofproblemsolving.com/community/c5h1213134p6023621}{Link}
\end{problem}

\begin{problem}[USAJMO 2016]
	Prove that there exists a positive integer $n < 10^6$ such that $5^n$ has six consecutive zeros in its decimal representation. %\hfill \href{http://artofproblemsolving.com/community/c5h1230489p6213569}{Link}
\end{problem}

\begin{problem}[USAMO 2016]
	Prove that for any positive integer $k$,
		\begin{align*}
			(k^2)!\cdot\displaystyle\prod_{j=0}^{k-1}\frac{j!}{(j+k)!}
		\end{align*}
	is an integer. %\hfill \href{http://artofproblemsolving.com/community/c5h1230499p6213627}{Link}
\end{problem}

\begin{problem}[USAMO 2016]
	$ $
	\begin{enumerate}[(a)]
		\item Prove that if $n$ is an odd perfect number then $n$ has the following form $n=p^sm^2$ where $p$ is prime has form $4k+1$, $s$ is positive integers has form $4h+1$, and $m\in\mathbb{Z}^+$, $m$ is not divisible by $p$.
		\item Find all $n\in\mathbb{Z}^+$, $n>1$ such that $n-1$ and $\frac{n(n+1)}{2}$ is perfect number.
	\end{enumerate}
	%\flushright \href{http://artofproblemsolving.com/community/c6h1182397p5729489}{Link}
\end{problem}

\begin{problem}[USA TSTST 2016]
	Decide whether or not there exists a nonconstant polynomial $Q(x)$ with integer coefficients with the following property: for every positive integer $n > 2$, the numbers \[ Q(0), \; Q(1), Q(2), \; \dots, \; Q(n-1) \]produce at most $0.499n$ distinct residues when taken modulo $n$. %\hfill \href{http://artofproblemsolving.com/community/c6h1264175p6575217}{Link}
\end{problem}

\begin{problem}[USA TSTST 2016]
	Suppose that $n$ and $k$ are positive integers such that
		\begin{align*}
			1
				& = \underbrace{\varphi( \varphi( \dots \varphi(}_{k\ \text{times}} n) \dots ))
		\end{align*}
	Prove that $n \le 3^k$. %\hfill \href{http://artofproblemsolving.com/community/c6h1264726p6580534}{Link}
\end{problem}

\begin{problem}[USA TST 2016]
	Let $\sqrt 3 = 1.b_1b_2b_3 \dots _{(2)}$ be the binary representation of $\sqrt 3$. Prove that for any positive integer $n$, at least one of the digits $b_n$, $b_{n+1}$, $\dots$, $b_{2n}$ equals $1$. %\hfill \href{http://artofproblemsolving.com/community/c6h1243905p6368201}{Link}
\end{problem}

\begin{problem}[Venezuela Final Round Fourth Year 2106]
	Find all pairs of prime numbers $(p, q)$, with $p <q$, such that the numbers $p + 2q, 2p + q$ and $p + q - 22$ are also primes.
\end{problem}

\begin{problem}[Zhautykov Olympiad 2016]
	$a_1,a_2,...,a_{100}$ are permutation of $1,2,...,100$. $S_1=a_1, S_2=a_1+a_2,...,S_{100}=a_1+a_2+...+a_{100}$Find the maximum number of perfect squares from $S_i$. %\hfill \href{http://artofproblemsolving.com/community/c6h1185316p5756328}{Link}
\end{problem}

\begin{problem}[Zhautykov Olympiad 2016]
	We call a positive integer $q$ a \textit{convenient denominator} for a real number $\alpha$ if
		\begin{align*}
			\scaleleftright[.5ex]{|}{\alpha - \dfrac{p}{q}}{|}
				& <\dfrac{1}{10q}
		\end{align*}
	for some integer $p$. Prove that if two irrational numbers $\alpha$ and
	$\beta$ have the same set of convenient denominators then either $\alpha+\beta$ or $\alpha- \beta$ is an integer\watermark. %\hfill \href{http://artofproblemsolving.com/community/c6h1185684p5759537}{Link}
\end{problem}