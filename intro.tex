\documentclass{subfile}


\begin{document}
	%\glossary{Notations}
%	\nomenclature[]{$\mathbb{Z}$}{The set of integers i.e. \{\ldots,-2,-1,0,1,2,\ldots\}}
%	\nomenclature[]{$\mathbb{N}$}{}
%	\printnomenclature
	%\section*{Notations}
	%Unless we state explicitly, a number means an integer and we let $p,q$ denote only primes.
		\begin{itemize}
			\item $\mathbb{N,N}_0$, $\mathbb{Z, Q, R, P},$ and $\mathbb{C}$ are the sets of positive integers, non-negative integers, integers, rational numbers, real numbers, primes and complex numbers, respectively.
			\item $|a|$ denotes the absolute value of $a$ for any real number $a$.
			\item $\min(a,b)$ is the minimum of $a$ and $b$ and $\max(a,b)$ is the maximum of $a$ and $b$.
			\item $(a,b)$ and $[a,b]$ are greatest common divisor and least common multiple of $a$ and $b$ respectively.
			\item $a|b$ means $b$ is divisible by $a$ without any remainder.
			\item $a\perp b$ means $(a,b)=1$.
			\item $p_n$ is the $n^{\text{th}}$ prime.
			\item $\pi(x)$ is the number of primes less than or equal to the real number $x$.		
			\item $\varphi(n)$ is the number of positive integers less than $n$ which are coprime to $n$.
			\item $d(n)$ is the number of divisors of $n$.
			\item $\sigma(n)$ is the sum of divisors of $n$.
			\item $\varphi(n)$ is the {\it Euler function}.
			\item $\mu(n)$ is the {\it M\H{o}bius function}.
			\item $\l(n)$ is the {\it Carmichael function}
			\item $v_n(a)$ is the largest non-negative integer $\alpha$ so that $n^\alpha|a$ but $n^{\alpha+1}\nmid a$.
			\item $n!=1\cdot2\cdots n$.
			\item $\displaystyle \binom{n}{k}$ is the \textit{binomial coefficient} indexed by non-negative integers $n$ and $k$.
			\item $\left(\dfrac{a}{p}\right)$ is the \textit{Legendre symbol} for integer $a$ and prime $p$.
			\item $\lfloor x \rfloor$ is the  largest integer not greater than $x$.
			\item $\lceil x \rceil$ is the smallest integer integer not less than $x$.
		\end{itemize}
	
		\newpage
\end{document}