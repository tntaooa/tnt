\documentclass[main.tex]{subfile}

\begin{document}
	Let $n$ be a fixed positive integer. There are many cases when we are interested in integers $a$ coprime to $n$ for which there exists another integer $x$ such that $a \equiv x^2 \pmod n$. As an example, assume that we want to solve the quadratic congruence relation
	\begin{align*}
		ax^2 + bx + c \equiv 0 \pmod n
	\end{align*}
	for $x$. Multiply both sides of the above relation by $4a$ to obtain
	\begin{align*}
		4a^2x^2+4abx+c \equiv 0 \pmod n.
	\end{align*}
	Rewriting the left side of the last relation as $(2ax+b)^2 -b^2+c$, we have
	\begin{align*}
		(2ax+b)^2 \equiv b^2-c \pmod n,
	\end{align*}
	which is of the form $y^2 \equiv z \pmod n$. Therefore, solving any quadratic congruence relation is equivalent to solving $x^2 \equiv a \pmod n$ for some $a$. We call such an $a$ a \textbf{quadratic residue} modulo $n$. Quadratic residues play an important role in cryptography. They are even used in acoustical engineering.
	
	In this section, we will discuss different aspects of quadratic residues in number theory.
	
	\begin{definition}[Quadratic Residue]
		Let $m>1$ be a positive integer and let $a$ be an integer such that $(a,n)=1$. Then $a$ is a \textit{quadratic residue} of $n$ if there exists an integer $x$ such that
		\[x^2\equiv a\pmod n.\]
		If there is no such $x$, then $a$ is a \textit{quadratic non-residue} of $n$.
	\end{definition}
	
	\begin{example}
		$2$ is a quadratic residue modulo $7$ because $3^2 \equiv 2 \pmod 7$. However, $3$ is a non-residue modulo $7$. In fact, modulo $7$,
		\begin{align*}
			1^2 \equiv 1, \quad 2^2 \equiv 4,\\
			3^2 \equiv 2, \quad 4^2 \equiv 2,\\
			5^2 \equiv 4, \quad 6^2 \equiv 1.
		\end{align*}
		This means that the only quadratic residues modulo $7$ are $1, 2$, and $4$.
	\end{example}
	
	\begin{corollary}\label{cor:qrequiv}
		If $a \equiv b \pmod n$, then $a$ is a quadratic residue (non-residue) modulo $n$, then so is $b$.
	\end{corollary}
	
	\begin{note}
		Whenever we say that an integer $a$ is a quadratic residue (non-residue) modulo $n$, it is clear that $a+kn$ is also a quadratic residue(non-residue) modulo $n$. Therefore, in order to find which numbers are quadratic residues modulo $n$, we only need to check the numbers $1, 2, \ldots, n-1$. Obviously, $a=0$ is a quadratic residue modulo any $n$, and we omit this case in our calculations.
	\end{note}
	
	\begin{theorem}\slshape\label{thm:primeresidue}
		Let $p$ be an odd prime number. There are exactly $\displaystyle \frac{p-1}{2}$ quadratic residues modulo $p$ (excluding zero). Furthermore, the residues come from the numbers $1^2, 2^2, \ldots, \displaystyle \left(\frac{p-1}{2}\right)^2$.
	\end{theorem}
	
	\begin{proof}
		Clearly, the quadratic residues modulo $p$ are
		\begin{align*}
			1^2, 2^2, \ldots, (p-1)^2 \pmod p.
		\end{align*}
		Note that $x^2 \equiv (p-x)^2 \pmod p$ for $x=1,2,\ldots,p-1$. So we only need to go halfway, i.e., we should only consider the numbers
		\begin{align*}
			1^2, 2^2, \ldots, \displaystyle \left(\frac{p-1}{2}\right)^2 \pmod p.
		\end{align*}
		These numbers are distinct modulo $p$, because otherwise if $x^2 \equiv y^2 \pmod p$ for some $x,y \in \{1,2,\ldots,\frac{p-1}{2}\}$, then
		\begin{align*}
			p\mid x^2 -y^2\\
			\implies p\mid (x-y)(x+y).
		\end{align*}
		But note that $x+y< \frac{p-1}{2}+\frac{p-1}{2}=p-1$, and so $p \nmid x+y$, which means $p\mid x-y$. Finally, since $x$ and $y$ are less than $p$, we should have $x=y$.
		So we have proved that there are exactly $(p-1)/2$ quadratic residues and they are
		\begin{equation*}
			1^2, 2^2, \ldots, \displaystyle \left(\frac{p-1}{2}\right)^2. 
		\end{equation*}
	\end{proof}
	
	\begin{theorem}\slshape\label{thm:qrnr}
		Let $p$ be an odd prime. Then,
		\begin{enumerate}[(i)]
			\item the product of two quadratic residues is also a quadratic residue,
			\item the product of two quadratic non-residues is also a quadratic residue, and
			\item the product of a quadratic residue and a quadratic non-residue is a quadratic non-residue.
		\end{enumerate}
	\end{theorem}
	
	\begin{proof}
		The first one is obvious. If $a \equiv x^2 \pmod p$ and $b\equiv y^2 \pmod p$, then $ab \equiv (xy)^2 \pmod p$. 
		Let's prove $(iii)$ now. Assume that $a \equiv x^2$ is a residue and $b$ is a non-residue modulo $p$ and suppose to the contrary that $ab$ is a residue and $ab \equiv y^2 \pmod p$. Then
		\begin{align*}
			ab \equiv x^2b \equiv y^2 \pmod p.
		\end{align*}
		Note that since $(x^2,p)=(a,p)=1$, the multiplicative inverse of $x^2$ exists. Therefore
		\begin{align*}
			b \equiv (x^2)^{-1} \cdot y^2 \equiv (x^{-1})^2 \cdot y^2 \equiv (x^{-1} \cdot y)^2 \pmod p,
		\end{align*}
		which contradicts the assumption that $b$ is a non-residue. So $ab$ is a non-residue.
		In order to prove $(ii)$, we use the fact that if $a$ is an integer coprime to $p$, then
		\begin{align*}
			\{a,2a,\ldots,(p-1)a\} = \{ 1,2,\ldots,p-1\}.
		\end{align*}
		(The proof is easy, try it yourself). From Theorem \ref{thm:primeresidue}, we see that there are exactly   $(p-1)/2$ quadratic residues among $\{a,2a,\ldots,(p-1)a\}$. Assume that $a$ is a fixed quadratic non-residue modulo $p$. From the proof of $(iii)$, we can say that whenever $a$ is multiplied by one of $\displaystyle \frac{p-1}{2}$ quadratic residues of the set $\{ 1,2,\ldots,p-1\}$, the result is a non-residue. Therefore, each non-residue element of $\{a,2a,\ldots,(p-1)a\}$ is multiplication of $a$ by a residue in the same set. This means that the multiplication of $a$ by any non-residue element of the set $\{a,2a,\ldots,(p-1)a\}$ is a residue, and we are done.
	\end{proof}
	
	Theorem \ref{thm:qrnr} gives us a nice result. Quadratic residues and quadratic non-residues act just like $1$ and $-1$. How? Notice that $1 \times 1 =1, (-1) \times 1 =-1,$ and $(-1) \times (-1)=1$, and this guides us to a point that quadratic residues behave like $1$, and quadratic non-residues behave like $-1$. %Euler noticed this behavior and found a criterion to see whether a number is residue or non-residue. We will explain Euler's criterion
	We can represent this result using Legendre's notation.
	
	\begin{definition}[Legendre Symbol]
		We call $\left(\frac{a}{p}\right)$ the {\it Legendre symbol} for a prime $p$. It is defined by:
		\begin{align*}
			\left(\dfrac{a}{p}\right) 
			& = 
			\begin{cases}
				0 & \mbox{if }p\mid a,\\
				1 &\mbox{if }a\mbox{ is a quadratic residue of }p,\\
				-1 &\mbox{otherwise}.
			\end{cases}
		\end{align*}
		
	\end{definition}
	
	Using this notation, Theorem \ref{thm:qrnr} becomes
	\begin{theorem}\label{thm:qrproduct}
		Let $p$ be an odd prime and let $a,b$ be two integers. Then
		\begin{align*}
			\left(\dfrac{ab}{p}\right) = \left(\dfrac{a}{p}\right) \left(\dfrac{b}{p}\right).
		\end{align*}
	\end{theorem}
	
	\begin{remark}
		Clearly, the same relation holds for the product of any $n$ integers, that is, 
		\begin{align*}
			\left(\dfrac{a_1a_2\cdots a_n}{p}\right) = \left(\dfrac{a_1}{p}\right) \left(\dfrac{a_2}{p}\right) \cdots \left(\dfrac{a_n}{p}\right).
		\end{align*}
	\end{remark}
	
	\begin{example}
		Theorem \eqref{thm:qrproduct} is helpful specially when dealing with big numbers. For instance, let's see if $18$ is a quadratic residue modulo $73$. According to the theorem,
		\begin{align*}
			\left(\dfrac{18}{73}\right) = \left(\dfrac{3}{73}\right) \left(\dfrac{3}{73}\right) \left(\dfrac{2}{73}\right).
		\end{align*}
		Note that we don't need to calculate $ \left(\frac{3}{73}\right)$ because whatever it is ($-1$ or $1$), it has appeared twice and $1^2=(-1)^2=1$. So
		\begin{align*}
			\left(\dfrac{18}{73}\right) = \left(\dfrac{2}{73}\right).
		\end{align*}
		Now, how can we find $ \left(\frac{2}{73}\right)$? It would be a pain to check all the values $1,2,\ldots, \frac{73-1}{2}$ to see if square of any of them is equal to $2$ modulo $73$. There are two ways to trick this. One is to use the general formula for $ \left(\frac{2}{p}\right)$ which will be discussed next. The second (and better) idea is to \textit{construct} the solution. Assume that you have a prime $p$ and an integer $a$ coprime to $p$. We know from Corollary \ref{cor:qrequiv} that if $a$ is a residue (non-residue), then $a+kp$ is also a residue (non-residue). So we will add multiples of $p$ to $a$ and check if we have reached a perfect square. If yes, then $a$ is a residue. Otherwise, we factorize the new number into perfect squares times some other number $b$. Continue this process until you reach either a non-residue $b$ (which means $a$ was a non-residue) or a perfect square factorization (which means $a$ was a residue). This whole process might seem a little confusing to you, but applying it to our case ($a=2, p=73$) will make it clear:
		\begin{align*}
			2 &\equiv 75 \equiv 148\equiv 2^2 \cdot 37\\
			& \equiv 2^2\cdot (37+73) \equiv 2^2 \cdot 110\\
			&\equiv 2^2\cdot (110+73) \equiv 2^2 \cdot 183\\
			&\equiv 2^2\cdot (183+73) \equiv 2^2 \cdot 256\\
			&\equiv 2^2 \cdot 16^2\\
			&\equiv 32^2.
		\end{align*}
		So $2$ is a quadratic residue modulo $73$ and finally
		\begin{align*}
			\left(\dfrac{18}{73}\right) = \left(\dfrac{2}{73}\right)=1.
		\end{align*}
		
	\end{example}
	
	
	\subsection{Euler's Criterion}
	In the last example of previous section, we provided a method for computing $ \left(\frac{a}{p}\right)$. However, this method only works when $p$ and $a$ are small enough to make the calculations. Fortunately, Euler developed a criteria  to find out whether an integer is a quadratic residue or a quadratic non-residue. After we explain and prove Euler's criterion, we will explore some special cases, e.g., we will find the value of $ \left(\frac{2}{p}\right)$ and $ \left(\frac{-1}{p}\right)$ for all primes $p$.
	
	\begin{theorem}[Euler's Criterion]\slshape
		\label{thm:eulerscriterion}
		Let $p$ be an odd prime and let $a$ be an integer coprime to $p$. Then
		\begin{align*}
			a^{\frac{p-1}{2}} \equiv \left(\dfrac{a}{p}\right) \pmod p.
		\end{align*}
	\end{theorem}
	
	\begin{proof}
		First notice that from Fermat's theorem, $a^{p-1} \equiv 1 \pmod p$. Since $p-1$ is even, we can write this as
		\begin{align*}
			a^{p-1} - 1 = \left(a^{\frac{p-1}{2}} - 1\right)\left(a^{\frac{p-1}{2}} + 1\right) \equiv 0 \pmod p.
		\end{align*}
		So either $a^{\frac{p-1}{2}} \equiv 1$ or $a^{\frac{p-1}{2}} \equiv -1 \pmod p$.
		Assume that $a$ is a quadratic residue modulo $p$. That is, $ \left(\frac{a}{p}\right)=1$. We should prove that $a^{\frac{p-1}{2}} \equiv 1$. Since $a$ is a residue, there exists some integer $x$ for which $a \equiv x^2 \pmod p$. So, from Fermat's theorem,
		\begin{align*}
			a^{\frac{p-1}{2}} \equiv x^{p-1} \equiv 1 \pmod p.
		\end{align*}
		Now assume the case where $a$ is a quadratic non-residue modulo $p$. Then we should prove that $a^{\frac{p-1}{2}} \equiv -1$. We will use an interesting approach here. Let $b \in \{1,2,\ldots,p-1\}$. Since $(b,p)=1$, the congruence equation $bx \equiv a$ has a unique solution $x \equiv a \cdot b^{-1} \pmod p$. Also, $x \not \equiv b \pmod p$ because otherwise $b^2 \equiv a \pmod p$ which is in contradiction with $a$ being a non-residue. This means that the set $\{1,2,\ldots,p-1\}$ can be divided into $\displaystyle \frac{p-1}{2}$ pairs $(b,x)$ such that $bx \equiv a \pmod p$. So
		\begin{align*}
			(p-1)! = 1 \times 2 \times \cdots \times (p-1) \equiv \underbrace{a \times a \times \cdots \times a}_{\frac{p-1}{2}\text{ times}} \equiv a^{\frac{p-1}{2}} \pmod p.
		\end{align*}
		By Wilson's theorem, $(p-1)! \equiv -1 \pmod p$ and therefore $a^{\frac{p-1}{2}} \equiv -1 \pmod p$, as desired. The proof is complete.
	\end{proof}
Let's find $\left(\frac{a}{p}\right)$ for some small values of $a$.
	\begin{problem}
		What is $\left(\frac{-1}{p}\right)$ for a prime $p$?
	\end{problem}
	
	\begin{solution}
		This is a simple case. By Euler's criterion, we get
		\begin{align*}
			(-1)^{\frac{p-1}{2}} \equiv \left(\dfrac{-1}{p}\right) \pmod p.
		\end{align*}
		If $p \equiv 1 \pmod 4$, that is, if $p=4k+1$, then $\frac{p-1}{2}$ is even and so $(-1)^{\frac{p-1}{2}} =1$. On the other side, we know that $\left(\frac{-1}{p}\right)$ is either $1$ or $-1$. So in this case it must equal one. This means that
		\begin{align*}
			p \equiv 1 \pmod 4 \iff \left(\dfrac{-1}{p}\right)=1.
		\end{align*}
		Note that the \textit{only if} part of the above statement is true because if $(-1)^{\frac{p-1}{2}}=1$, then $\frac{p-1}{2} = 2k$ for some integer $k$. So $p=4k+1$, or $p \equiv 1 \pmod 4$. You can easily check that the following statement is also true
		\begin{align*}
			p \equiv 3 \pmod 4 \iff \left(\dfrac{-1}{p}\right)=-1.
		\end{align*}
		Each prime has either the form $p \equiv 1 \pmod 4$ or $p \equiv 3 \pmod 4$, so these primes together make all primes. All in all,
	\end{solution}
	
	\begin{theorem}\slshape\label{thm:-1qr} For all primes $p$,
		\begin{align*}
		\left(\dfrac{-1}{p}\right) 
		& = 
		\begin{cases}
		1,&\mbox{ if } p \equiv 1 \pmod 4\text{ or }p=2,\\
		-1, &\mbox{ if } p \equiv 3 \pmod 4.
		\end{cases}
		\end{align*}
	\end{theorem}
This infers the following theorem and the next.
	\begin{theorem}\slshape
		$-1$ is a quadratic residue of a prime $p$ if and only if $p\equiv1\pmod4$.
	\end{theorem}
	
	\begin{theorem}\slshape \label{thm:a^2+b^2}
		Let $a$ and $b$ be coprime positive integers. Then every prime divisor of $a^2+b^2$ is either $2$ or of the form $4k+1$.
	\end{theorem}

	\begin{proof}
		Let $p$ be a prime divisor of $a^2+b^2$. If $a$ and $b$ both are odd then $p$ can be $2$. Now assume $p$ is larger than $2$. Then
			\begin{align*}
				a^2  \equiv-b^2 \pmod p\implies \left(a^2\right)^{\frac{p-1}{2}} \equiv \left(-b^2\right)^{\frac{p-1}{2}}\pmod p.
			\end{align*}
		If $p$ is of the form $4k+3$, then $\frac{p-1}{2}=2k+1$ is odd, hence,
			\begin{align*}
			\left(a^2\right)^{\frac{p-1}{2}} \equiv \left(-b^2\right)^{\frac{p-1}{2}}\pmod p \implies a^{p-1} \equiv -b^{p-1} \pmod p.
			\end{align*}
		Clearly, since $p\mid a^2+b^2$, if $p$ divides one of $a$ or $b$, it should divide the other one. But this is impossible because $a \bot b$. So $p \bot a$ and $p \bot b$. By Fermat's little theorem, $a^{p-1} \equiv b^{p-1} \equiv 1 \pmod p$, which is in contradiction with the above equation since $p$ is odd.
		So, $p$ cannot be of the form $4k+3$ and therefore every odd prime divisor of $a^2+b^2$ is of the form $4k+1$.
	\end{proof}
	
	\begin{note}
	We could just say,
			\begin{align*}
				a^2  \equiv-b^2\pmod p \implies (ab^{-1})^2 \equiv-1\pmod p,
			\end{align*}
		which means that $-1$ is a quadratic residue of $p$, and therefore $p\equiv1\pmod4$.
	\end{note}
We get the following corollary, which can directly solve an IMO problem.
	\begin{corollary}\label{cor:4n+1}
		Let $k$ be a positive integer. Every divisor of $4k^2+1$ is of the form $4n+1$ for some integer $n$.
	\end{corollary}
	
	\begin{proof}
		Thanks to the previous theorem, we know that all the prime divisors of $4k^2+1$ are of the form $4t+1$. Every divisor of $4k^2+1$ is a multiplication of its prime divisors. And if we multiply two numbers of the form $4t+1$, then the number is again of the same form (multiply $4x+1$ and $4y+1$ and see the result yourself). Hence, conclusion.
	\end{proof}
	
	\begin{problem}[Iran, Third Round Olympiad, 2007]
		Can $4xy-x-y$ be a square for integers $x$ and $y$?
	\end{problem}
	
	\begin{solution}
		Assume that $4xy-x-y=t^2$. We can rearrange it as $x(4y-1)  = t^2+y$, so
			\begin{align*}
				&4y-1  \mid t^2+y\\
				\implies &4y-1  \mid 4t^2+4y\\
				\implies &4y-1  \mid 4t^2+4y-(4y-1)\\
				\implies &4y-1  \mid 4t^2+1
			\end{align*}
		Since $4y-1$ is of the form $4k+3$, it must have at least one prime factor of the form $4j+3$. Then we have a prime factor of $4t^2+1$ which is of this form, a contradiction. Thus, it can't be a square.
	\end{solution}
Back to quadratic residues, the next step is to determine if $2$ is a quadratic residue modulo a prime. Unfortunately, we cannot apply Euler's criterion directly in this case because we do not know what $2^\frac{p-1}{2} \pmod p$ would be for different values of $p$. So our challenge is to find $2^\frac{p-1}{2}$ modulo $p$.
	
	\begin{problem}\label{pr:qr2p}
		What is $ \left(\frac{2}{p}\right)$ for a prime $p$?
	\end{problem}
	
	\begin{solution}
		As explained just above, we only need to find a way to calculate $2^\frac{p-1}{2} \pmod p$. The idea is similar to what we did in the proof of Fermat's little theorem. Remember that in Fermat's theorem, we needed to somehow construct $a^{p-1}$ and we did that by multiplying elements of the set $\{a,2a,\ldots,(p-1)a\}$. Now how can we construct $2^\frac{p-1}{2}$? The idea is to find a set with  $\frac{p-1}{2}$ elements such that the product of all elements has the factor $2^\frac{p-1}{2}$. One possibility is to consider the set $A=\{2, 4, \ldots, p-1\}$. Then the product of elements of $A$ is
		\begin{align}\label{eq:factorialqr1}
			2 \times 4 \times \cdots \times (p-1) &= 2^\frac{p-1}{2} \times 1 \times 2 \times \cdots \times \frac{p-1}{2} \nonumber\\
			& = 2^\frac{p-1}{2} \times \left(\frac{p-1}{2}\right)!.
		\end{align} 
		In order to get rid of the term $ \left(\frac{p-1}{2}\right)!$, we have to compute the product of elements of $A$ in some other way. Notice that we are looking for $2 \times 4 \times \cdots \times (p-1)$ and we want to make it as close as possible to $ \left(\frac{p-1}{2}\right)!$ so that we can cancel out this term and find the value of $2^\frac{p-1}{2} \pmod p$. To construct this factorial, we need all the numbers in the set $$B=\displaystyle \left\{1, 2, \ldots, \frac{p-1}{2}\right\}.$$ However, we only have even integers in $A$. For example, take $p=11$. Then $A=\{2,4,6,8,10\}$ and $B=\{1,2,3,4,5\}$. Now, we want to construct $5!$ using the product of elements of $A$. Clearly, the elements $2$ and $4$ are directly chosen from $A$. Now it remains to somehow construct the product $1\times 3 \times 5$ with the elements $6,8,10$. The trick is pretty simple: just notice that $10 \equiv -1, 8 \equiv -3$, and $6 \equiv -5 \pmod{11}$. This means that $ 6 \times 8 \times 10 \equiv (-1)^3  \cdot 1 \times 3 \times 5$, and so
		\begin{align}\label{eq:factorialqr2}
		2 \times 4 \times 6 \times 8 \times 10 \equiv (-1)^3 \cdot 5!.
		\end{align}
		Comparing equations \eqref{eq:factorialqr1} (for $p=11$) and \eqref{eq:factorialqr2}, we find that
		\begin{align*}
		2^5 \cdot 5! \equiv (-1)^3 \cdot 5! \pmod{11}\implies 2^5 \equiv (-1)^3 \equiv -1 \pmod{11}.
		\end{align*}
		Let's go back to the solution of the general problem. As in the example of $p=11$, we are searching for the power of $(-1)$ appeared in the congruence relation. In fact, this power equals the number of even elements bigger than $ \frac{p-1}{2}$ and less than or equal to $p-1$. Depending on the remainder of $p$ modulo $8$, this power of $(-1)$ can be even or odd. Consider the case when $p \equiv 1 \pmod 8$. Then $p-1 = 8k$ for some positive integer $k$ and so the even numbers less than or equal to $\frac{p-1}{2} = 4k$ in the set $A=\{2,4,\ldots,8k\}$ are $2,4,\ldots,4k$, which are $2k+1$ numbers. Therefore
			\begin{align*}
				2 \times 4 \times \cdots \times (p-1)
					&= \overbrace{\left(2 \times 4 \times \cdots \times 4k\right)}^{2k+1 \text{ items}} \cdot \overbrace{\left((4k+2) \times (4k+4) \times \cdots \times 8k\right)}^{2k \text{ items}}\\
					& \equiv \left(2 \times 4 \times \cdots \times 4k\right) \cdot \left((-(4k-1)) \times (-(4k-3)) \times \cdots \times (-1)\right) \\
					& \equiv (-1)^{2k} \cdot (4k)!\\
					& \equiv (-1)^{2k} \cdot \left(\frac{p-1}{2}\right)! \pmod p.
			\end{align*} 
		Compare this result with \eqref{eq:factorialqr1}, you see that
			\begin{align*}
				2^\frac{p-1}{2} \times \left(\frac{p-1}{2}\right)!
					& \equiv (-1)^{2k} \cdot \left(\frac{p-1}{2}\right)! \pmod p\\
				\implies 2^\frac{p-1}{2}
					& \equiv (-1)^{2k} \equiv 1 \pmod p.
			\end{align*}
		So if $p \equiv 1 \pmod 8$, then $ \left(\frac{2}{p}\right)=1$.
		
		The process is similar for $p \equiv 3, 5, 7 \pmod 8$ and we put it as an exercise for the reader. 
		
		After all this work, we are finally done computing $ \left(\frac{2}{p}\right)$. The final result is stated in the following theorem.
		
		\begin{theorem}\slshape\label{thm:2qr}
			\begin{align*}
			\left(\dfrac{2}{p}\right) 
			& = 
			\begin{cases}
			1,&\mbox{ if } p \equiv 1 \mbox{ or } 7\pmod 8,\\
			-1,&\mbox{ if } p \equiv 3 \mbox{ or } 5\pmod 8.
			\end{cases}
			\end{align*}
		\end{theorem}
		
	\end{solution}
We can generalize the method used in the solution of Problem \ref{pr:qr2p} to find $ \left(\frac{a}{p}\right)$ for all integers $a$ and primes $p$.
	\begin{theorem}[Gauss' Criterion]\slshape\label{thm:gausscriterion}
		Let $p$ be a prime number and let $a$ be an integer coprime to $p$. Let $\mu(a,p)$ denote the number of integers $x$ among
		\begin{align*}
		a, 2a, \ldots, \frac{p-1}{2}a
		\end{align*}
		such that $\displaystyle x > p/2 \pmod p$. Then
		\begin{align*}
		\left(\dfrac{a}{p}\right) = (-1)^{\mu(a,p)}.
		\end{align*}
	\end{theorem}
	
The proof is left as an exercise for the reader.
		\begin{theorem}\slshape
			The smallest quadratic non-residue of a prime $p$ is a prime less than $\sqrt{p}+1$.
		\end{theorem}
		
		\begin{proof}
			Let $r$ be the smallest quadratic non-residue of prime $p$. Then, any $i<r$ is a quadratic residue of $p$. If $r=kl$ for $k,l>1$, using Legendre's symbol, we have
			\begin{align*}
				\left(\dfrac{r}{p}\right) & = \left(\dfrac{k}{p}\right)\cdot\left(\dfrac{l}{p}\right),
			\end{align*}
			which gives $-1=1\cdot 1$ (since $r$ is a quadratic non-residue modulo $p$) and we get a contradiction. So, $r$ can't be a prime. Let's move on to the next part of the theorem.
			
			We have that $r,\cdots,(r-1)r$ are quadratic non-residues mosulo $p$. If any of them is greater than $p$, say $ri$, we have $r(i-1)<p<ri$ for some $i$. But then,
			\begin{align*}
				ri & < p+r,
			\end{align*}
			so that $ri=p+s$ with $s<r$. Thus,
			\begin{align*}
				ri & = p+s\\
				& \equiv s\pmod p.
			\end{align*}
			Since $s<r$, $s$ is a quadratic residue of $p$, which in turn means $ri$ is a quadratic residue of $p$ as well. Another contradiction, so $r(r-1)<p$. If $r\geq\sqrt{p}+1$, we have $r(r-1)\geq\sqrt{p}(\sqrt{p}+1)=p+\sqrt{p}>p$, yet another contradiction. The claim is therefore true.
		\end{proof}
	
	\subsection{Quadratic Reciprocity}
	Assume we want to compute $ \left(\frac{a}{p}\right)$ for some integer $a$ and a prime $p$. The remark after Theorem \ref{thm:qrproduct} says that it's enough to find $$ \left(\dfrac{a_1}{p}\right), \left(\dfrac{a_2}{p}\right), \ldots, \left(\dfrac{a_n}{p}\right),$$ where $a_1,a_2,\ldots,a_n$ are divisors of $a$. This means that if we know the value of $\left(\frac{q}{p}\right)$ for a prime $q$, we can find the values of $\left(\frac{a}{p}\right)$ for any $a$.
	
	So let's discuss on the value of $\left(\frac{q}{p}\right)$. If $q$ is a big prime number, then by Corollary \ref{cor:qrequiv}, we can reduce $q$ modulo $p$ until we reach some $c<p$ and find $\left(\dfrac{q}{p}\right)$ by some method (Euler's or Gauss's criteria). So we can handle the case when $q$ is a big prime.
	
	Now, what about the case when $p$ is a big prime? In this case, finding $\left(\frac{q}{p}\right)$ would be very hard with theorems and methods stated by now. There is a very nice property of prime numbers which helps us to handle this case. This property is called the \textit{Law of Quadratic Reciprocity} which relates $\left(\frac{q}{p}\right)$ and $\left(\frac{p}{q}\right)$. So, in case $q$ is big, we can first calculate $\left(\frac{q}{p}\right)$ and then use this law to find $\left(\frac{p}{q}\right)$.
	
	\begin{theorem}[Law of Quadratic Reciprocity]\label{thm:lawofqr}
		Let $p$ and $q$ be different odd primes. Then
		\begin{align*}
		\left(\dfrac{p}{q}\right)\left(\dfrac{q}{p}\right)=(-1)^{\frac{p-1}{2}\cdot \frac{q-1}{2}}.
		\end{align*}
	\end{theorem}
	
	\begin{example}
		Let's find $\left(\frac{11}{6661}\right)$. We have
		\begin{align}\label{eq:qrex}
		\left(\dfrac{11}{6661}\right) \left(\dfrac{6661}{11}\right)=(-1)^{5\cdot 3330}=-1.
		\end{align}
		Now, since $6661 \equiv 6 \pmod{11}$, we obtain
		\begin{align*}
		\left(\dfrac{6661}{11}\right) = \left(\dfrac{6}{11}\right) = \left(\dfrac{3}{11}\right)\left(\dfrac{2}{11}\right)=1 \times (-1) = -1.
		\end{align*}
		Replacing in equation \eqref{eq:qrex}, we finally find
		\begin{align*}
		\left(\dfrac{11}{6661}\right) = 1.
		\end{align*}
	\end{example}
	
	The proof of this theorem is a bit complicated and it would make you lose the continuity of the context. For this reason, we will provide the proof in section \eqref{sec:qrlawproof}.
	
	\subsection{Jacobi Symbol}
	
	In previous sections, whenever we used the Legendre symbol $\left(\frac{p}{q}\right)$, we needed $p$ to be a prime number. We are now interested in cases where $p$ can be a composite number. In $1837$, Jacobi generalized the symbol used by Legendre in this way:
	
	\begin{definition}[Jacobi Symbol]\label{def:jacobi}
		Let $a$ be an integer and $n=p_1^{\alpha_1}p_2^{\alpha_2}\cdots p_k^{\alpha_k}$, where $p_i$ are odd primes and $\alpha_i$ are non-negative integers ($1 \leq i \leq k$). The \textit{Jacobi symbol} is defined as the product of the Legendre symbols corresponding to the prime factors of $n$:
		\begin{align*}
		\left(\frac{a}{n}\right) = \left(\frac{a}{p_1}\right)^{\alpha_1}\left(\frac{a}{p_2}\right)^{\alpha_2}\cdots \left(\frac{a}{p_k}\right)^{\alpha_k}.
		\end{align*}
		Also, we define $\left(\frac{a}{1}\right)$ to be $1$.
	\end{definition}
	
	\begin{remark}
			The immediate result of the above definition is that if $\gcd(a,n) = 1$, then $\left(\frac{a}{n}\right)$ is either $+1$ or $-1$. Otherwise, it equals zero.
	\end{remark}
	
	\begin{example}
		\begin{align*}
		\left(\frac{14}{2535}\right) &= \left(\frac{14}{3}\right)\left(\frac{14}{5}\right) \left(\frac{14}{13}\right)^{2}\\
		&= \left(\frac{2}{3}\right)\left(\frac{4}{5}\right) \left(\frac{1}{13}\right)^{2}\\
		&= (-1) \cdot 1 \cdot 1\\
		&= -1.
		\end{align*}
	\end{example}
	
	\begin{example}
		\begin{align*}
			\left(\frac{2}{15}\right) =\left(\frac{2}{3}\right)\left(\frac{2}{5}\right) = (-1)\cdot (-1)=1.
		\end{align*}

	\end{example}
	
	\begin{note}
		If $\left(\frac{a}{n}\right)=-1$ for some $a$ and $n$, then $a$ is a quadratic non-residue modulo $n$. However, the converse is not necessarily true. The above example shows you a simple case when $a$ is a quadratic non-residue modulo $n$ but $\left(\frac{a}{n}\right)=1$.
	\end{note}
	
	\begin{theorem}
		Let $a$ be an integer and $n=p_1^{\alpha_1}p_2^{\alpha_2}\cdots p_k^{\alpha_k}$, where $p_i$ are odd primes and $\alpha_i$ are non-negative integers ($1 \leq i \leq k$). Then $a$ is a quadratic residue modulo $n$ if and only if $a$ is a quadratic residue modulo every $p_i^{\alpha_i}$  ($1 \leq i \leq k$).
	\end{theorem}
	
	\begin{proof}
		The \textit{if} part is easy to prove. Assume that $a$ is a quadratic residue modulo $n$ and $a \equiv x^2 \pmod n$ for some integer $x$. Then $a \equiv x^2 \pmod{p_i^{\alpha_i}}$ since $p_i$s are relatively prime to each other. Now the \textit{only if} part: assume that $a \equiv x_i^2 \pmod{p_i^{\alpha_i}}$ for all $i$, where $x_i$ are integers. According to Chinese Remainder Theorem, since the numbers ${p_i^{\alpha_i}}$ are pairwise coprime, the system of congruence equations
		\begin{align*}
			x & \equiv x_1\pmod{p_1^{\alpha_1}}\\
			x & \equiv x_2\pmod{p_2^{\alpha_2}}\\
			& \vdots\\
			x & \equiv x_k\pmod{p_k^{\alpha_k}}
		\end{align*}
		has a solution for $x$. Now $x^2 \equiv x_i^2 \equiv a \pmod{p_i^{\alpha_i}}$, and therefore $x^2 \equiv a \pmod n$, which means that $a$ is a quadratic residue modulo $n$.
	\end{proof}
	
	The following theorem sums up almost everything explained in quadratic residues. The proofs are simple and straightforward, so we leave them as exercises for the reader.
		\begin{theorem}
			Let $a$ and $b$ be any two integers. Then for every two odd integers $m$ and $n$, we have
			\begin{enumerate}[i.]
				\item $\displaystyle \left(\frac{ab}{n}\right) = \left(\frac{a}{n}\right) \left(\frac{b}{n}\right)$,
				\item $\displaystyle \left(\frac{a+bn}{n}\right) = \left(\frac{a}{n}\right)$,
				\item $\displaystyle \left(\dfrac{m}{n}\right)\left(\dfrac{n}{m}\right)=(-1)^{\frac{m-1}{2}\cdot \frac{n-1}{2}}$,
				\item $\displaystyle \left(\frac{a}{mn}\right) = \left(\frac{a}{n}\right) \left(\frac{a}{m}\right)$,
				\item $\displaystyle \left(\frac{-1}{n}\right) = (-1)^{\frac{n-1}{2}}$,
				\item $\displaystyle \left(\frac{2}{n}\right) = (-1)^{\frac{n^2-1}{8}}$.
			\end{enumerate}
		\end{theorem}
	
		\begin{theorem}\slshape
			If a positive integer is a quadratic residue modulo every prime, then it is a perfect square. 
		\end{theorem}
	This is a really cool theorem. However, proving this might be challenging!\footnote{A popular proof for this uses Dirichlet's theorem: For two co-prime positive integers $a$ and $b$, there are infinitely many primes in the sequence $\{an+b\}_{n\geq1}$. This theorem is very famous for being difficult to prove and it is well beyond our scope.} Give it a try.
		\begin{theorem}\slshape
			Let $b,n > 1$ be integers. Suppose that for each $k > 1$ there exists an integer $a_k$ such that $b - a^n_k$ is divisible by $k$. Prove that $b = A^n$ for some integer $A$.
		\end{theorem}
		
		\begin{proof}
			Assume that $ b$ has a prime factor, $ p$, so that $ p^{xn + r}\|b$ with $ 0 < r < n$. Then, we can let $ k = p^{xn + n}$. It follows that $ b\equiv a_k^n\bmod p^{xn + n}$. Since $ p^{xn + r}\|b$, we see that $ p^{xn + r}\|a_k^n$. Then, $ p^{x + \frac {r}{n}}\|a_k^n$, which is a contradiction since $ \frac {r}{n}$ is not an integer. Thus, we have a contradiction, so $ r = 0$, which means that only $ n$th powers of primes fully divide $ b$, so $ b$ is an $ n$th power.
		\end{proof}
		

	
\end{document}