\documentclass{subfile}

\begin{document}
	\textit{Diophantine equations} are an especial kind of equations which allow solutions only in integers. They have been studied for a really long time. The name is taken after the mathematician \textit{Diophantus of Alexandria}. We have avoided discussing such equations in this book because this area is too huge for us to include right now, and for the same reason we had to ax a lot of topics. However, it is compulsory that we discuss how to use modular arithmetic to solve some particular Diophantine equations. And even if the whole equation can not be solved, we can say a lot about the solutions using modular properties.

		\subsection{Some Useful Properties}
			There are some  modular arithmetic properties that usually come handy. But before showing them, we intend to pose a question.
			\begin{question}
				Find two positive integer whose sum of squares is $123$.
			\end{question}
			Since there does not exist many squares below $123$, you may try to do it by hand. And after exhausting all possible cases, you must conclude there are no such integers. But if you are clever, you don't have to go through trial and error. Let's write $a^2+b^2=123$ and notice the following. Exactly one of $a$ or $b$ must be odd since $123$ is odd. Without loss of generality, assume $a$ is even (you can take $b$ if you want). Then $b$ is odd, and we know $b^2\equiv1\pmod4$. Thus, $a^2+b^2\equiv1\pmod4$, whereas $123\equiv3\pmod4$. This is a straight contradiction implying there are no such positive integers $a$ and $b$. The idea seems simple enough, yet powerful to be of great use.

			For reaching such a contradiction (it is often the case, Diophantine equations usually do not have any solutions), we use some common facts. The main idea is the same: find a proper $n$ so that the two sides of the equation leave different remainders modulo $n$.

			You might ask what happens if the equation actually \textit{does} have a solution in integers? Let us explain this with an example. Suppose that you are given the simple linear Diophantine equation $6x+5y = 82$ and you want to solve it over non-negative integers. Let's solve this problem by trial and error. First, notice that $x\leq 13$ (otherwise $6x$ would exceed $82$). We can draw a table to find the solutions.

\begin{table}[h]
\centering
\begin{tabular}{|c|c|c|c|c|c|c|c|}
\hline
$x$ & 0 & 1 & 2 & 3 & 4 & 5 & 6 \\
\hline
$y$ & none & none & 14 & none & none & none & none \\
\hline
$x$ & 7 & 8 & 9 & 10 & 11 & 12 & 13 \\
\hline
$y$ & 8 & none & none & none & none & 2 & none \\
\hline
\end{tabular}
\caption{Solving $6x+5y=82$ by trial and error.}
\label{table:diophantine}
\end{table}

			As seen in Table \ref{table:diophantine}, we need to do $13$ calculations to find the solutions $$(x, y)=(2,14), (7,8), (12, 2)$$

			Now, consider the same linear equation $6x+5y = 82$ again. We are going to solve it using modular arithmetic this time. Take modulo $5$ from both sides of the equation. The left side would be $x$ while the right side is $2$, giving us the relation $x \equiv 2 \pmod 5$. Although this does not give us the solution directly, it helps us find the solutions much faster. Just notice that we already know $x$ must be less than or equal to $13$, and it must have a remainder of $2$ when divided by $5$. The only choices for $x$ then are $2, 7$, and $12$. We can now plug these values of $x$ into the equation and find the solutions with only three calculations (instead of thirteen).


			Sometimes we need to use some theorems such as Fermat's little theorem or Wilson's theorem and pair them up with some modular arithmetic. Here are some highly useful congruences:
				\begin{theorem}\label{thm:diophModulo}
					Let $x$ be an integer (not necessarily positive). Then
					\begin{align*}
					x^2 & \equiv 0,1 \pmod3\\
					x^2 & \equiv 0,1 \pmod4\\
					x^2 & \equiv 0,1,4\pmod8\\
					x^2 & \equiv 0,1,4,9 \pmod{16}\\
					x^3 & \equiv 0, \pm1 \pmod7\\
					x^3 & \equiv 0, \pm1 \pmod9\\
					x^4 & \equiv 0,1 \pmod{16}\\
					x^4 & \equiv 0, \pm 1, \pm 4 \pmod{17}\\
					x^5 & \equiv 0, \pm1 \pmod{11}\\
					x^6 & \equiv 0,1,4 \pmod{13}
					\end{align*}
				\end{theorem}
			Most of congruences above can be proved easily. Some are direct consequence of Fermat's or Euler's theorem. Or you can just consider the complete set of residue of the modulus and then investigate their powers. Whatever the case, we will leave the proofs as exercises. Sometimes you may notice that Fermat's little theorem or Euler's theorem is disguised in the equation.

				\begin{problem}
					The sum of two squares is divisible by $3$. Prove that both of them are divisible by $3$.
				\end{problem}

				\begin{solution}
					Assume that $a^2+b^2$ is divisible by $3$. If $a$ is divisible by $3$, so must be $b$. So, take $a$ not divisible by $3$. Then, from the properties above, we have $a^2\equiv1\pmod3$ and $b^2\equiv1\pmod3$. And this immediately gives us a contradiction that $a^2+b^2\equiv1+1\equiv2\pmod3$.
				\end{solution}

				\begin{remark}
					We could just use \autoref{thm:a^2+b^2} which shows that every prime factor of $a^2+b^2$ is of the form $4k+1$ if $a$ and $b$ are coprime.
				\end{remark}

				\begin{problem}
					Show that there are no integers $a,b,c$ for which $a^2+b^2-8c=6$.
				\end{problem}

				\begin{solution}
					The term $-8c$ guides us to choose the right modulo. Consider the equation modulo $8$. We have $a^2+b^2\equiv 6\pmod{8}$. By \autoref{thm:diophModulo}, $a^2\equiv 0, 1$, or  $4\pmod{8}$. Now you may check the possible combinations to see that $a^2+b^2\equiv 6\pmod{8}$ is impossible.
				\end{solution}

				\begin{problem}
					Solve the Diophantine equation $x^4-6x^2+1=7 \cdot 2^y$ in integers.
				\end{problem}

				\begin{solution}
					There are no solutions for $y<0$. So assume $y\geq 0$. Add $8$ to both sides of the equation to get
						\begin{align*}
							 (x^2-3)^2=7\cdot 2^y+8
						\end{align*}
					Note that if $y \geq 3$, the right hand side of above equation is divisible by $8$. So taking modulo $8$ may seem reasonable. However, it leads to $(x^2-3)^2 \equiv 0 \pmod 8$ and no further results are included. We should look for another modulo. If $y \geq 4$, then the right hand side is congruent to $8$ modulo $16$. However, the left hand side, $(x^2-3)^2$ is a square and so it's $0, 1, 4$, or $9$ modulo $16$. The only left cases are $y=0,1,2$, and $3$ which imply no solutions. Hence, no solutions at all.
				\end{solution}


			Let's see another problem in which we will also see an application of \textit{Fermat's method of infinite descent}. This is a technique for solving Diophantine equations but we briefly use the idea here.
				\begin{problem}
					Find all integer solutions to the equation
						\begin{align*}
							x^2+y^2 & = 7(z^2+t^2)
						\end{align*}
				\end{problem}

				\begin{solution}
					First of all, using the same approach as in previous problem, we can prove that $7$ divides both $x$ and $y$. Let $x=7a$ and $y=7b$ and substitute them in the equation. After dividing by $7$,
						\begin{align*}
							z^2+t^2 & = 7(a^2+b^2)
						\end{align*}
					Note that this equation looks like the original one. However, $z$ and $t$ in the latter equation are strictly smaller than $x$ and $y$ in the original equation. We can continue this process by noting the fact that $z$ and $t$ are divisible by $7$. So, assume that $z=7u$ and $t=7v$ and rewrite the equation as
						\begin{align*}
						u^2+v^2 & = 7(a^2+b^2)
						\end{align*}
					This process can be done infinitely many times. Thus, we get that $x$ and $y$ are divisible by $7^i$ for all positive integers $i$, which is not possible. So, the equation does not have any solutions. The process of finding new equations similar to the original one is called the method of \textit{infinite descent}.
				\end{solution}


				\begin{problem}
					Show that the following equation does not have any solutions in positive integers:
						\begin{align*}
							5^n-7^m = 1374
						\end{align*}
				\end{problem}

				\begin{solution}
					The most important thing in solving a Diophantine equation is to take the right modulo. In this case, it's obvious that the easiest mods to take are $5$ and $7$. Let's take modulo $5$ from both sides of the equation. Since $7^m \equiv 2^m \pmod 5$,
						\begin{align*}
							-2^m
								& \equiv -1 \pmod 5\\
							\implies 2^m
								& \equiv 1 \pmod 5
						\end{align*}
					Since $\ord_5(2)=4$, we have $4 \mid m$. Let $m=4k$ for some integer $k$. So $7^m = 7^{4k}=\left(7^k\right)^4$. This reminds us of the fact that $x^2$ (and thus $x^4$) is either $0$ or $1$ modulo $4$. So, $7^m \equiv 1 \pmod 4$. Taking modulo $4$ from the original equation, we get
						\begin{align*}
							5^n - 7^{4k}
								& \equiv 2 \pmod 4\\
							\implies 1-1
								& \equiv 2 \pmod 4
						\end{align*}
					which is a contradiction. Thus, there are no solutions.
				\end{solution}

				\begin{problem}[Kazakhstan 2016]
					Solve in positive integers the equation
					\[n!+10^{2014}=m^4\]
				\end{problem}

				\begin{solution}
					You can usually use modular arithmetic to solve the problem when there is a factorial term in the given equation. The interesting property of $n!$ is that it is divisible by all integers less than or equal to $n$. In this problem, if we find the right modulo $k$, we can assume $n\geq k$ and take modulo $k$ from the equation (we will check the cases when $n < k$ later). It will be $10^{2014} \equiv m^4 \pmod k$. As said before, we guess the equation does not have any solutions. So, we are searching for a modulo $k$ for which $m^4$ cannot be congruent to $10^{2014}$. We should first try the simplest values for $k$, i.e., values of $k$ for which $m^4$ can have a few values. For $k=16$, we have $m^4 \equiv 0 \pmod{16}$, no contradiction. For $k=17$, we have $m^4 \equiv 8 \pmod{17}$, which is impossible because $m^4$ can only have the values $0, \pm 1$, or $\pm 4$ modulo $17$. We have found our desired contradiction, and we just have to check the values of $n < 17$. This is easy. Obviously, $n! + 10^{2014}$ is bigger than $10^{2014}$. However, the smallest perfect square bigger than $10^{2014}$ is
						\begin{align*}
							\left(10^{1007}+1\right)^2 = 10^{2014} + 2 \cdot 10^{1007} + 1
						\end{align*}
					which is way bigger than $10^{2014} + 16!$. So, no solutions in this case as well.
				\end{solution}

				\begin{problem}
					Prove that the equation $x^2+5=y^3$ has no integer solutions.
				\end{problem}

				\begin{solution}
					Taking modulo $4$, since $x^2+5$ is congruent to either $1$ or $2$ modulo $5$, but $y^3$ is never congruent to $2$ modulo $4$, we have $x^2 + 5 \equiv y^3 \equiv 1 \pmod 4$, and so $x$ is even, $y \equiv 1 \pmod 4$. Rewrite the equation as
						\begin{align*}
							x^2+4 = (y-1)(y^2+y+1)
						\end{align*}
					Note that since $y \equiv 1 \pmod 4$, we have $y^2+y+1 \equiv 3 \pmod 4$. According to theorem \eqref{thm:4k+3prime}, we know that every number congruent to $3$ modulo $4$ has a prime divisor also congruent to $3$ modulo $4$. Let $p \equiv 3 \pmod 4$ be that prime divisor of $y^2+y+1$. Then
						\begin{align*}
							x^2 + 4 \equiv 0 \pmod p
						\end{align*}
					If we raise both sides of the congruence equation $x^2 \equiv -4 \pmod p$ to the power of $\frac{p-1}{2}$ (which is an odd integer since $p \equiv 3 \pmod 4$), we have
						\begin{align*}
							\left(x^2\right)^{\frac{p-1}{2}}
								& \equiv -\left(4\right)^{\frac{p-1}{2}} \pmod p
						\end{align*}
					or, by Fermat's little theorem,
						\begin{align*}
							1
								& \equiv x^{p-1}\\
								& \equiv -4^{p-1}\\
								& \equiv -1 \pmod p
						\end{align*}
					This is the contradiction we were looking for and the equation does not have integer solutions.
				\end{solution}

				\begin{note}
					The idea of taking a prime $p \equiv 3 \pmod 4$ of a number $n \equiv 3 \pmod 4$ comes handy in solving Diophantine equations pretty a lot. Keep it in mind.
				\end{note}

				\begin{problem}[Romania JBMO TST 2015]
					Solve in nonnegative integers the equation
					\[21^x+4^y=z^2\]
				\end{problem}

				\begin{solution}
					First, let us consider the case $x=0$. Then $z^2-1 =(z-1)(z+1)=4^y$. If $z-1=1$ and $z+1=4^y$, we have no solutions. Otherwise, both $z-1$ and $z+1$ should be perfect powers of $4$, which is impossible.

					We can show, in a similar way, that the case $y=0$ gives no solutions as well. So, suppose that $x$ and $y$ are positive integers.

					Rewrite the original equation as
						\begin{align*}
							3^x \cdot 7^x = (z-2^y)(z+2^y)
						\end{align*}
					There are a few cases to check:
						\begin{itemize}
							\item If $z-2^y=1$, then $z+2^y = z-2^y+2^{y+1}=1+2^{y+1}=21^x$. This implies $2^{y+1}=21^x-1$. But the right hand side of the latter equation is divisible by $20$, contradiction. So no solutions in this case.

							\item If both $z-2^y$ and $z+2^y$ are divisible by $21$, then $21|(z-2^y, z+2^y)$. This is impossible because if $d=(z-2^y, z+2^y)$, then $d\mid (z+2^y) -(z-2^y)=2^{y+1}$, which means $d$ is a power of $2$.

							\item If $z-2^y=3^x$ and $z+2^y=7^x$, then
								\begin{align}\label{eq:romaniajbmotst2015}
									7^x-3^x=2^{y+1}
								\end{align}
							 $y=1$ gives the solution $(x,y,z)=(1,1,5)$. Assume $y \geq 2$. Take modulo $8$ from equation \eqref{eq:romaniajbmotst2015}. $2^{y+1}$ is divisible by $8$ and so $7^x-3^x \equiv 0 \pmod 8$. But this does not happen for any $x$ (just consider two cases when $x$ is even or odd). So the only solution in this case is $(x,y,z)=(1,1,5)$.
						\end{itemize}
					Note that we have used the fact that $z+2^y > z-2^y$ to omit some cases (like when $z-2^y=21^x$ and $z+2^y = 1$). So, $(x,y,z)=(1,1,5)$ is the only solution to the given equation.
				\end{solution}


\end{document}