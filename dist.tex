Distribution of prime numbers is the topic which encouraged number theorist to start a new branch called \textit{Analytic Number Theory}. We have in fact discussed a little bit about distribution of prime numbers already when we proved Bertrand's theorem. Let us focus on it a bit more.

There are $4$ primes less than $10$, $25$ primes less than $100$, $168$ less than $1000$ and so on. And finding a formula for the number of primes less than $n$ has always fascinated mathematicians. Well, Gauss did not exactly provide a formula for the number of primes, but he noticed that the value of $\dfrac{n}{\ln n}$ and the number primes less than $n$, $\pi(n)$, gets closer as $n$ tends to infinity. This gave birth to the \textit{Prime Number Theorem or PNT}, conjectured by Gauss. It was unproven for about $100$ years. Then Chebyshev provided a partial proof using his functions (which were later known as, Chebyshev function of type $1$ and $2$), and that was only the start of analytical number theory. There is a huge underlying significance here. Gauss did not conjecture any exact formula for $\pi(n)$. But since he was unable to provide one, he estimated instead. Analytical number theory does not provide exact formulas like elementary number theory, rather it shows some estimation, and mathematicians tend to prove the estimates or improve them. This is because, most of the times providing exact formulas for the functions are either very hard or not so pretty. For example, one can find an exact formula for finding the $n$th prime number, but one will not like it.

We will start with some functions and analyzing their properties. The obvious question is, why do mathematicians define such functions? In this case, why are these functions and their properties important? The reason is simple. If you can not understand primes directly, understand some functions that can characterize them or tell us something about them, some function that we can analyze. At first, they can be intimidating. So, we will try to show examples in order to make sense why these functions have something to do with primes.
%\begin{definition}[Primorial]The primorial of $n$, $n\#$ is the product of all primes less than or equal to $n$. That is,
%\[n\#=\prod_{p\leq n}p.\]
%Clearly, for the $n^{th}$ prime number, primorial $p_n\#$ is the product of the first $n$ primes
%	For a prime $p$, \[p\#=\prod_{q\leq p}q\]
%	Some authors define it as,
%	\[P(n)=p_1p_2\cdots p_n=\prod_{i=1}^{n}p_i\]
%	You can use whichever you like.
%\end{definition}

%\begin{example}
%	$p_7$ is the seventh prime, so $p_7=17$. The primorial of $17$ is
%	\begin{align*}
%	p_7\# = 2 \cdot 3 \cdot 5 \cdot 7 \cdot 11 \cdot 13 \cdot 17 = 510,510.
%	\end{align*}
%\end{example}

\subsection{Chebyshev Functions}
	\begin{definition}
		Let $x>0$ be a real number. We define \textit{Chebyshev's $\vartheta$-function} as
			\begin{align*}
				\vartheta(x)= \sum_{p \leq x} \ln p,
			\end{align*}
		where the sum extends over all primes $p$ less than or equal to $x$.
	\end{definition}
	
	\begin{example}
		Take $x=142.61$. The primes less than $x$ are $2, 3, 5, \ldots, 137, 139$. So
			\begin{align*}
				\vartheta(142.61)&=\sum_{p \leq 142} \ln p = \ln 2 + \ln 3 + \cdots + \ln 137 + \ln 139
			\end{align*}
	\end{example}
	
	\begin{corollary}
		If $p_1, p_2, \ldots, p_k$ are primes less than or equal to $x$, then
			\begin{align*}
			\vartheta(x)= \ln \left( p_1p_2\cdots p_k\right).
			\end{align*}			
	\end{corollary}
	
	\begin{lemma}
		\label{lem:binomialprimeinequality}
		Let $k$ and $n$ be positive integers such that $k<n < 2k+1 $. Then
			\begin{align*}
				\binom{n}{k} \geq \prod_{k < p \leq n} p,
			\end{align*}
		where the product extends over all primes $p$ between $k$ and $n$.
	\end{lemma}
	
	\begin{proof}
		Write $\binom{n}{k}$ as
			\begin{align}
				\label{eqn:binomialprimeinequality1}
				\binom{n}{k} = \frac{n(n-1)(n-2) \cdots (k+2)(k+1)}{(n-k)!}.
			\end{align}
		Let $p_1, p_2, \ldots, p_m$ be the primes between $k+1$ and $n$ (including). Since $n \leq 2k+1$ can be represented as $n-k < k+1$, we have
			\begin{align*}
				n-k < k+1 \leq p_i \leq n
			\end{align*}
		for $i=1,2,\ldots,m$. So $n-k < p_i$, which means that $p_i$ is relatively prime to all positive integers less than or equal to $n-k$. In other words, $(p_i, (n-k)!)=1$ for all $i$. The rest is easy: the numerator of \eqref{eqn:binomialprimeinequality1} can be regarded as the product of $p_1p_2\cdots p_m$ and another integer, say, $s$. Since $\binom{n}{k}$ is an integer and also $(p_i, (n-k)!)=1$ for all $i$, we conclude that $s$ must be divisible by $(n-k)!$. Thus,
			\begin{align*}
			\binom{n}{k} & = \frac{p_1p_2\cdots p_m \cdot s}{(n-k)!}= p_1p_2\cdots p_m \cdot \frac{s}{(n-k)!}  \geq p_1p_2\cdots p_m  = \prod_{k < p \leq n} p,
			\end{align*}
		as claimed.
	\end{proof}
	
	\begin{proposition}
		\label{prop:chebyshevthetainequality}
		Let $x >0$ be a real number. Then
			\begin{align}
			\label{eqn:chebyshevthetainequality1}
				\vartheta(x) \leq 2x \ln 2.
			\end{align}
	\end{proposition}
	
	\begin{proof}
		We induct on $\lfloor x \rfloor$. For our base cases, we note that for $0 \le x < 2$, we have $\vartheta(x) = 0 \leq 2x \ln 2$.
		
		Now suppose that $x \geq 2$. Let $n = \lfloor x \rfloor$ and suppose that the inequality holds for all reals $y$ such that $\lfloor y \rfloor <n$. Note that
			\begin{align*}
				2^x \geq 2^n &= (1+1)^n = \binom{n}{0} + \binom{n}{1} + \cdots + \binom{n}{\lfloor n/2 \rfloor} + \cdots + \binom{n}{n-1}+\binom{n}{n}\\
					&\geq  \binom{n}{\lfloor n/2 \rfloor} \geq \prod_{\lfloor n/2 \rfloor < p \leq n} p,
			\end{align*} 
		where we have used lemma \eqref{lem:binomialprimeinequality} to write the last line. Taking logarithms from the above inequality, we find
			\begin{align*}
				x \ln 2 &\geq \sum_{\lfloor n/2 \rfloor < p \le n} \ln p \\
						 &= \vartheta(x) - \vartheta(\lfloor n/2 \rfloor) \\
						 &\geq \vartheta(x) - 2\lfloor n/2 \rfloor \ln 2 \\
						 &\geq \vartheta(x) - x \ln 2,
			\end{align*}
		by the inductive hypothesis. Therefore 
			\begin{align*}
				2x \ln 2 \ge \vartheta(x),
			\end{align*}
		as desired.
	\end{proof}
	
	\begin{definition}
		Let $x>0$ be a real number. We define \textit{Chebyshev's $\psi$-function} as
			\begin{align*}
				\psi(x) = \sum_{p^a \leq x} \ln p,
			\end{align*}
		where $p^a$ ranges over all the powers of primes $p_1, p_2, \ldots, p_k$ which do not exceed $x$. In other words, $\ln p$ appears in the sum each time a power of $p$ is less than or equal to $x$.
	\end{definition}
	
	\begin{example}
		Let's find $\psi(10.5)$. The primes which do not exceed $10.5$ are $2,3,5$, and $7$. The powers of these primes which do not exceed $10.5$ are $2, 2^2, 2^3, 3, 3^2, 5$, and $7$. Therefore
			\begin{align*}
				\psi(10.5) &= \ln 2 + \ln 2 + \ln 2 +\ln 3 + \ln 3+ \ln 5 + \ln 7\\
						   &= \ln(2^3 \times  3^2 \times 5 \times 7)\\
						   &=\ln(2520)\\
						   &\approx 7.83.
			\end{align*}
	\end{example}
	
	\begin{corollary}
		Let $p_1, p_2, \ldots, p_k$ be primes not exceeding a positive real number $x$. Also, assume that $p_1^{a_1}, p_2^{a_2}, \ldots, p_k^{a_k}$ are the largest powers of these primes which do not exceed $x$. Then
			\begin{align*}
				\psi(x)=\ln \left( p_1^{a_1} p_2^{a_2} \cdots p_k^{a_k} \right).
			\end{align*}
	\end{corollary}
	
	\begin{corollary}
		Let $x$ be a positive real number. Then
			\begin{align*}
				\psi(x) = \lcm([1, 2, \ldots, \lfloor x \rfloor]).
			\end{align*}
	\end{corollary}
	
	\begin{proof}
		Let $p_1, p_2, \ldots, p_k$ be primes which do not exceed $x$. Let $p_1^{a_1}, p_2^{a_2}, \ldots , p_k^{a_k}$ be the largest powers of these primes which do not exceed $x$. Each number in the set $A=\{1, 2, \ldots, \lfloor x \rfloor\}$ is of the form $p_1^{b_1} p_2^{b_2} \cdots p_k^{b_k}$, where $b_i$ is an integer and $0 \leq b_i \leq a_i$ (for $i=1,2,\ldots,k$). It is easy to check (see proposition \eqref{prop:gcdfactorization}) that the least common multiple of all such integers is $p_1^{a_1} p_2^{a_2} \cdots p_k^{a_k}$. The previous corollary proves the claim now.
	\end{proof}
	
	\begin{proposition}
		\label{prop:chebyshevpsiinequality}
		For any real $x>0$, we have
			\begin{align*}
				\psi(x) = \sum_{p \leq x} \Big\lfloor\frac{\ln x}{\ln p} \Big\rfloor \ln p.
			\end{align*}
	\end{proposition}
	
	\begin{proof}
		Let $p$ be a prime not exceeding $x$. We just need to show that the power of $p$ appearing in $\psi(x)$ equals $\lfloor \ln x/\ln p \rfloor$. This is rather obvious. Let $a$ be the power of $p$ we are searching for. Then $p^a \leq x <p^{a+1}$. Taking logarithms and dividing by $\ln p$, we find the desired result.
	\end{proof}
	
Chebyshev attempted to prove the Prime Number Theorem, and he succeeded in proving a slightly weaker version of the theorem. In fact, he proved that if the limit $\pi(x)  \ln(x)/x$ as $x$ goes to infinity exists at all, then it is equal to one. He showed that this ratio is bounded above and below by two explicitly given constants near 1, for all sufficiently large $x$. Although Chebyshev was unable to prove PNT completely, his estimates for $\pi(x), \vartheta(x)$, and $\psi(x)$ were strong enough to prove Bertrand's postulate at his time. We will state these estimations but hesitate to provide the proofs as they need some calculus background.

	\begin{theorem}[Chebyshev Estimates]
		\label{thm:chebyshevestimates}
		If the following limits exist, they are all equal to $1$.
		\begin{align}
			&\lim_{x \to \infty} \frac{\vartheta(x)}{x}, \quad \lim_{x \to \infty} \frac{\psi(x)}{x}, \quad \text{and} \quad 
			\lim_{x \to \infty} \frac{\pi(x) \ln(x)}{x}. \label{eq:chebyshevestimates:eq1}					
		\end{align}
	\end{theorem}


The inequalities in the next theorem show that $n/\ln(n)$ is the correct order of magnitude for $\pi(n)$. In fact, these inequalities are pretty weak and better inequalities can be obtained with greater effort but the following theorem is of our interest because of its elementary proof.

	\begin{theorem}\slshape
	\label{thm:pi(n)approximation}
		For all integers $m\geq2$, 
		\begin{align}
		\label{eq:pi(n)approximation:0}
		\dfrac{1}{6}\dfrac{m}{\ln m}<\pi(m)<4\dfrac{m}{\ln m}.
		\end{align}
	\end{theorem}
	
	\begin{proof}
		Let's prove the leftmost inequality first. Assume that $n \geq 1$ is an intger. One can easily show by induction that
		\begin{align}
		\label{eq:pi(n)approximation:1}
		2^n \leq \binom{2n}{n} < 4^n.
		\end{align}
		Using the fact that $\binom{2n}{n} = (2n)!/(n!)^2$, we can take logarithms from \eqref{eq:pi(n)approximation:1} to obtain
		\begin{align}
		\label{eq:pi(n)approximation:2}
		n \ln 2 \leq \ln (2n)! - 2 \ln n! < n \ln 4.
		\end{align}
		We must now find a way to compute $\ln(2n)!$ and $\ln n!$. Let $k$ be a positive integer. By theorem \eqref{thm:legendre}, we have
		\begin{align}
		\label{eq:pi(n)approximation:3}
		v_p(k!) = \sum_{i=1}^{\a}\left\lfloor\dfrac{k}{p^i}\right\rfloor,
		\end{align} 
		where $\a$ is some positive integer. Here, we need to find $\a$. See proof of theorem \eqref{thm:legendre} to realize that $\a+1$ is actually the number of digits of $k$ in base $p$. On the other hand, we know that the number of digits of a positive integer $x$ in base $y$ is $\lfloor\log_y x\rfloor+1$ (prove this as an exercise). So, in our case, $\a+1=\lfloor\log_p k\rfloor+1$, or simply $\a = \lfloor\log_p k\rfloor$. Since we are working with natural logarithms (i.e., logarithms in base $e$), it would be better to write $a = \lfloor \frac{\ln k}{\ln p}\rfloor$. Finally, substituting $n$ and $2n$ for $k$ in equation \eqref{eq:pi(n)approximation:3}, we get
			\begin{align*}
				v_p(n!) = \sum_{i=1}^{\left\lfloor \frac{\ln n}{\ln p}\right\rfloor}\left\lfloor\dfrac{n}{p^i}\right\rfloor, \quad v_p\left((2n)!\right) = \sum_{i=1}^{\left\lfloor \frac{\ln 2n}{\ln p}\right\rfloor}\left\lfloor\dfrac{2n}{p^i}\right\rfloor.
			\end{align*}
		It is clear that $n! = \prod\limits_{p\leq n} p^{v_p(n!)}$, where the product is extended over all primes $p$ less than or equal to $n$. After taking logarithms in the latter equation, the product turns into a sum:
			\begin{align*}
			\ln n! = \sum_{p \leq n} v_p(n!) \ln p = \sum_{p \leq n}  \sum_{i=1}^{\left\lfloor \frac{\ln n}{\ln p}\right\rfloor}\left\lfloor\dfrac{n}{p^i}\right\rfloor \ln p.
			\end{align*}
		Similarly,
			\begin{align*}
				\ln (2n)! = \sum_{p \leq n} v_p\left((2n)!\right) \ln p = \sum_{p \leq 2n}  \sum_{i=1}^{\left\lfloor \frac{\ln 2n}{\ln p}\right\rfloor}\left\lfloor\dfrac{2n}{p^i}\right\rfloor \ln p.
			\end{align*}
		Hence, the left hand side of inequality \eqref{eq:pi(n)approximation:2} becomes
			\begin{align}
			n \ln 2 \leq \ln (2n)! - 2 \ln n! &= \sum_{p \leq 2n}  \sum_{i=1}^{\left\lfloor \frac{\ln 2n}{\ln p}\right\rfloor}\left\lfloor\dfrac{2n}{p^i}\right\rfloor \ln p - 2  \sum_{p \leq n} \sum_{i=1}^{\left\lfloor \frac{\ln n}{\ln p}\right\rfloor} \left\lfloor\dfrac{n}{p^i}\right\rfloor \ln p \nonumber\\
											  &= \sum_{p \leq 2n}  \sum_{i=1}^{\left\lfloor \frac{\ln 2n}{\ln p}\right\rfloor}\left\lfloor\dfrac{2n}{p^i}\right\rfloor \ln p -  \sum_{p \leq 2n} \sum_{i=1}^{\left\lfloor \frac{\ln 2n}{\ln p}\right\rfloor} 2 \left\lfloor\dfrac{n}{p^i}\right\rfloor \ln p \nonumber\\
										   	  &= \sum_{p \leq 2n}  \sum_{i=1}^{\left\lfloor \frac{\ln 2n}{\ln p}\right\rfloor} \Bigg(\left\lfloor\dfrac{2n}{p^i}\right\rfloor - 2 \left\lfloor\dfrac{n}{p^i}\right\rfloor\Bigg) \ln p. 		\label{eq:pi(n)approximation:4}
			\end{align}
		Since for all rationals $x$, $\lfloor2x\rfloor-2\lfloor x\rfloor$ is either $0$ or $1$, we can write
			\begin{align*}
			n \ln 2 &\leq \sum_{p \leq 2n}  \Bigg(\sum_{i=1}^{\left\lfloor \frac{\ln 2n}{\ln p}\right\rfloor} 1\Bigg) \ln p\\
					&\leq \sum_{p \leq 2n} \ln 2n\\
					&=\pi(2n)\ln 2n.			
			\end{align*}
		The proof is almost finished. Note that $\ln 2 \approx 0.6931 > 1/2$ and therefore
			\begin{align*}
				\pi(2n) \geq \frac{n \ln 2}{\ln 2n} > \frac{1}{2}\frac{n}{\ln 2n} = \frac{1}{4}\frac{2n}{\ln 2n} > \frac{1}{6}\frac{2n}{\ln 2n}.
			\end{align*}
		So the left side inequality of \eqref{eq:pi(n)approximation:0} is proved for even positive integers $m=2n$. We now prove it for $m=2n+1$. Since $2n/(2n+1)\geq 2/3$, we get
			\begin{align*}
				\pi(2n+1) \geq \pi(2n) > \frac{1}{4}\frac{2n}{\ln 2n} > \frac{1}{4} \frac{2n}{2n+1}\frac{2n+1}{\ln (2n+1)} \geq \frac{1}{6}\frac{2n+1}{\ln (2n+1)},
			\end{align*}
		and this proves the left hand inequality of \eqref{eq:pi(n)approximation:0} for all $m \geq 2$.
		
		We will now prove the other inequality of \eqref{eq:pi(n)approximation:0}. We shall make use of proposition \eqref{prop:chebyshevthetainequality}. 
%the right side inequality of \eqref{eq:pi(n)approximation:2}. Turn back to \eqref{eq:pi(n)approximation:4} and consider only the term $i=1$ in the inner sum to obtain 
%			\begin{align*}
%			n \ln 4 > \ln (2n)! - 2 \ln n! &= \sum_{p \leq 2n}  \sum_{i=1}^{\left\lfloor \frac{\ln 2n}{\ln p}\right\rfloor} \Bigg(\left\lfloor\dfrac{2n}{p^i}\right\rfloor - 2 \left\lfloor\dfrac{n}{p^i}\right\rfloor\Bigg) \ln p\\
%								  &\geq \sum_{p \leq 2n} \Bigg(\left\lfloor\dfrac{2n}{p}\right\rfloor - 2 \left\lfloor\dfrac{n}{p}\right\rfloor\Bigg) \ln p\\
%								  &\geq \sum_{p \leq 2n} \ln p\\
%								  &= \vartheta(2n) - \vartheta(n).
%			\end{align*}
%		So $n \ln 4 > \vartheta(2n) - \vartheta(n)$ for all positive integers $n$. Write down this last equation for $n=2^s, 2^{s-1}, \ldots, 2^1, 2^0$, where $s$ is some non-negative integer. Then
%			\begin{align*}
%				\vartheta(2^{s+1})-\vartheta(2^s) &< 2^{s+1}\ln 2\\
%				\vartheta(2^{s})-\vartheta(2^{s-1}) &< 2^s \ln 2\\
%				&\phantom{<} \vdots\\
%				\vartheta(2)-\vartheta(1) &< 2^s \ln 2\\			
%			\end{align*}
%		Summing up, we get
%			\begin{align*}
%				\vartheta(2^{s+1}) < \sum_{i=0}^{s+1} 2^i \ln 2 = 2^{s+2} \ln 2.
%			\end{align*}
%		Now, assume that the positive integer $m \geq 2$ is given. Choose $s$ such that $2^s \leq m < 2^{s+1}$. According to above inequality,
%			\begin{align*}
%				\vartheta(n) \leq \vartheta(2^{s+1}) <2^{s+2} \ln 2 \leq 4n \ln 2,
%			\end{align*}
%		and so
%			\begin{align}
%			\label{eq:pi(n)approximation:5}
%				\vartheta(n) <4n \ln 2.
%			\end{align}
		Let $\a$ be an arbitrary real number such that $0 < \a < 1$. Then $n> n^{\a}$ and so $\pi(n) \geq \pi(n^{\a})$. Using equation \eqref{eqn:chebyshevthetainequality1}, one can write
			\begin{align*}
				\Big(\pi(n) - \pi(n^{\a})\Big) \ln n^{\a} &= \Big(\sum_{p \leq n} 1 - \sum_{p \leq n^{
						\a}} 1\Big) \ln n^{\a}\\
														  &=\sum_{n^{\a} \leq p \leq n} \ln n^{\a}\\
														  &\leq \sum_{n^{\a} \leq p \leq n} p\\
														  &\leq \vartheta(n)\\
														  &< 2n \ln 2.
			\end{align*}
		This already means that
			\begin{align*}
				\pi(n) &< \frac{2n \ln 2}{\a \ln n} + \pi(n^{\a})\\
					   &< \frac{2n \ln 2}{\a \ln n} + n^{\a}\\
					   &= \frac{n}{\ln n} \Big(\frac{2 \ln 2}{\a} + \frac{\ln n}{n^{1-\a}} \Big).
			\end{align*}
		We use a bit calculus to finish the proof. Let $ f(x)=\frac{\ln x}{x^{1-\a}}$. You can easily calculate the derivative $f'(x)$ of $f$ and find that it equals zero for $x=e^{1/1-\a}$. Putting this value into $f$, you will see that $\frac{\ln n}{n^{1-\a}} \leq 1/e(1-\a)$. Since $\a$ is an arbitrary number, choosing $\a = 2/3$ helps us finish the proof:
			\begin{align*}
				\pi(n) < \frac{n}{\ln n} \Big(3 \ln 2 + \frac{3}{e} \Big)< 4 \frac{n}{\ln n}.
			\end{align*}
	\end{proof}


Here is an interesting problem which combines several concepts. This problem appeared in Theorem 137 of \cite{ch:primes-hardy}, and we are going to provide an elegant solution by Professor Peyman Nasehpour \cite{ch:primes-nasehpour}.
\begin{problem}\label{prob:nasehpour}
	Let $r$ be a real number whose decimal representation is
	\begin{align*}
	r & = 0.r_1r_2\ldots r_n \ldots = 0.011010100010\ldots,
	\end{align*}
	where $r_n=1$ if when $n$ is prime and $r_n=0$ otherwise. Show that $r$ is irrational.
\end{problem}

\begin{solution}
	We will prove a more general statement: such a number $r$ in any base $b>1$ is irrational. First, we define the \textit{average of digits of $r$ in base $b$}, denoted by $\text{Av}_b(r)$, as
	\begin{align*}
	\text{Av}_b(r) &= \lim_{n\to \infty} \frac{r_1+r_2+\cdots+r_n}{n}.
	\end{align*}
	It is clear that $\text{Av}_b(r)$ is well-defined if the above limit exists. It is a good exercise for you to prove that if $\text{Av}_b(r)=0$, then $r$ is irrational. The latter result is true because if $r$ is rational, $\text{Av}_b(r)$ exists and is positive. If you have doubts about it, see \cite{ch:primes-nasehpour}). Now, with the definition of $r$ and by Prime Number Theorem,
	\begin{align*}
	\text{Av}_b(r) &= \lim_{n\to \infty} \frac{\pi(n)}{n} =  \lim_{n\to \infty} \frac{n/\log n}{n} =  \lim_{n\to \infty} \frac{1}{\log n} = 0,
	\end{align*}
	and so $r$ is irrational!
\end{solution}
	\begin{theorem}[Euler]\sl
		The sum $S=\dfrac{1}{2}+\dfrac{1}{3}+\dfrac{1}{5}+\ldots=\sum\limits_{p\in\P }\dfrac{1}{p}$ diverges i.e. does not have a finite summation.
	\end{theorem}
The proof is due to Dustin J. Mixon, which appeared at American Mathematical Monthly \cite{dustin}.
	\begin{proof}
		Let $p_i$ be the $i$th prime number and the sum does not diverge. Then there must be a $k$ such that
			\begin{align*}
				\sum_{i=k+1}^{\infty}\dfrac{1}{p_i} & < 1
			\end{align*}
		We let $A$ be the set of positive integers which has all prime factors less than or equal to $p_k$, and $B$ be the set of positive integers with all prime factors greater than or equal to $p_{k+1}$. From the fundamental theorem of arithmetic, each positive integer can be uniquely expressed as a product $ab$ where $a\in A$ and $b\in B$. We have
			\begin{align*}
				\sum_{a\in A}\dfrac{1}{a}
					& = 
					\sum_{x_1=0}^{\infty}\cdots\sum_{x_k=0}^{\infty}\dfrac{1}{p_1^{x_1}\cdots p_k^{x_k}}\\
					& =
					\left(\sum_{x_1=0}^{\infty}\dfrac{1}{p_1^{x_1}}\right)\cdots\sum_{x_k=0}^{\infty}\dfrac{1}{p_k^{x_k}}\\
					& < \infty
			\end{align*}
		Moreover, assume that $B_i$ is the set of positive integers with exactly $i$ distinct prime factors. This yields,
			\begin{align*}
				\sum_{b\in B}\dfrac{1}{b}
					& =
					\sum_{i=0}^{\infty}\sum_{b\in B_i}\dfrac{1}{b}\\
					& \leq \sum_{i=1}^{\infty}\left(\sum_{j=k+1}^{\infty}\dfrac{1}{p_{j}}\right)^j<\infty
			\end{align*}
		Since every positive integer greater than $1$ belongs to exactly one of $A$ or $B$, we have
			\begin{align*}
				\dfrac{1}{2}+\cdots+\dfrac{1}{n}+\cdots
					& = \sum_{n=2}^{\infty}\dfrac{1}{n}\\
					& = \sum_{a\in A}\sum_{b\in B}\dfrac{1}{ab}\\
					& = \sum_{a\in A}\dfrac{1}{a}\sum_{b\in B}\dfrac{1}{b}\\
					& < \infty.
			\end{align*}
		The claim follows from this (how?).
		
	\end{proof}
	
\section{The Selberg Identity}
	\subfile{selberg.tex}
