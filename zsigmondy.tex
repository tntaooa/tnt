\documentclass{subfile}

\begin{document}
	\textit{Zsigmondy's theorem} is one of the tactics that can easily tackle a good number of \textit{hard problems} in recent years. This is indeed a mighty theorem to be used in an olympiad. But it seems everyone has accepted it as a tool for solving problems because the proof of its theorem is quite elementary\footnote{according to the first author, there may be a difference in opinion} (still hard). At the IMO, often some problems appear that can be solved very easily using some heavy theorems. But those theorems are usually not accepted by everyone for not being as elementary as needed. But in this case, everyone seems to like this theorem. The first author was really keen to provide the proof of this theorem along with some other similar theorems\footnote{such as Carmichael Theorem}. But that really is beyond the scope of our book. Since it is a well established theorem, for now, we will just assume it is true. Rather we focus on how to implement this theorem in solving problems.
	
		\begin{definition}[Primitive Divisor]
			For a sequence of integers $a_1,a_2,\ldots,a_n,\ldots$ a prime number $p$ is a \textit{primitive} divisor of $a_n$ if $p$ divides $a_n$ but $p$ doesn't divide $a_k$ for any $k<n$. \textit{R. D. Carmichael} called such a prime an \textit{intrinsic} divisor.
		\end{definition}
		
		\begin{example}
			Consider the sequence $a_k=2^k-1$. $a_1=1,a_2=3,a_3=7,a_4=15$. Note that, $a_3$ has primitive divisor $7$ and $a_4$ has the primitive divisor $5$.
		\end{example}
		
		\begin{theorem}[Zsigmondy's Theorem, $1882$]\slshape
			Let $a,b$ be co-prime integers and $n\geq1$ be an integer. 
				\begin{itemize}
					\item $a^n-b^n$ has a primitive divisor except when:
						\begin{enumerate}[(a)]
							\item $a-b=1$, $n=1$.
							\item $a=2,b=1$ and $n=6$.
							\item $a+b$ is a power of $2$ and $n=2$.
						\end{enumerate}
					\item $a^n+b^n$ has a primitive divisor for $n\geq2$ except for the case $2^3+1^3$.
				\end{itemize}\label{thm:zsigmondy}
		\end{theorem}
	This theorem can be extended even further.
		\begin{theorem}[First Extension]\slshape
			Let $p$ be a primitive divisor of $a^n+b^n$. Then $p$ does not divide $a^k+b^k$ for $n+1\leq k\leq2n$.
		\end{theorem}
		
		\begin{proof}
			Since $n+1\leq k\leq2n$, for $k=n+l$, we get $1\leq l\leq n$. $p$ does not divide any of $a$ or $b$. For the sake of contradiction, let's assume, $p$ divides $a^k+b^k$.
				\begin{align*}
					p | a^l(a^n+b^n) &= a^k+a^lb^n\text {, and}\\
					p |b^l(a^n+b^n) &= a^nb^l+b^k.
				\end{align*}
			Therefore
			\begin{align*}
				p |a^k+a^lb^n+a^nb^l+b^k.
			\end{align*}
			We already know $p|a^k+b^k$, so if $n=l+m$ (since $l\leq n$), then
				\begin{align*}
					p|a^lb^n+a^nb^l & = a^lb^l(a^m+b^m).
				\end{align*}
			Since $p\bot ab$, we have $p\bot a^lb^l$. So $p|a^m+b^m$ where $m< n$, which is a contradiction.
		\end{proof}
	In a similar fashion, we can prove the following theorem.
		\begin{theorem}[Second Extension]\slshape
			Let $p$ be a primitive divisor of $a^n+b^n$. Then $p$ does not divide $a^k-b^k$ for $1\leq k<\dfrac{n}{2}$.
		\end{theorem}
		
	In this section, we will see some demonstration of its power in solving problems and then develop a theorem that generalizes a problem from the IMO Shortlist. The main idea is to find some contradictions using the fact that $a^n-b^n$ will have a prime factor that won't divide something else.
		\begin{problem}[Japanese Math Olympiad, 2011]
			Find all $5-$tuples $(a,n,x,y,z)$ of positive integers so that
				\begin{align*}
					a^n-1 & = (a^x-1)(a^y-1)(a^z-1).
				\end{align*}
		\end{problem}
		
		\begin{solution}
			If $a,n\geq3$ and $n>x,y,z$, we already know from the theorem that $a^n-1$ has a prime divisor that none of $a^x-1,a^y-1$ or $a^z-1$ has. Therefore, two sides can never be equal. We are left with cases $n\leq3$. Note that, $n\notin\{x,y,z\}$. But $a^x-1$ divides $a^n-1$, so $x$ divides $n$. Thus, $n>x,y,z$ and hence $a,n\leq3$, like we said before.
			
			Now, either $a<3$ or $n<3$. If $a<3$, then $a=2$ and
				\begin{align*}
					2^n-1 & = (2^x-1)(2^y-1)(2^z-1).
				\end{align*}
			Here, the only exception is $n=6$ and $2^6-1=63=3\cdot3\cdot7=(2^2-1)(2^2-1)(2^3-1)$. So, $\{x,y,z\}=\{2,2,3\}$. Only $n<3$ is left to deal with and it is easy to check that there are no solutions in this case.
		\end{solution}
		
		\begin{problem}[Polish Math Olympiad]
			If $p$ and $q$ are distinct odd primes, show that $2^{pq}-1$ has at least three distinct prime divisors.
		\end{problem}
		
		\begin{solution}
			Without loss of generality, consider that $2<q<p<pq$. Then $2^q-1$ has at least one prime factor, $2^p-1$ has a prime factor that is not in $2^q-1$ and $2^{pq}-1$ has a prime factor that is not in any of $2^p-1$ or $2^q-1$. Since $2^p-1|2^{pq}-1$ and $2^{q}-1|2^{pq}-1$, we have three distinct prime factors.
		\end{solution}
		
		
		\begin{problem}[Hungary 2000, Problem 1]
			Find all $4-$tuples $(a,b,p,n)$ of positive integers with $p$ a prime number such that
				\begin{align*}
					a^3+b^3=p^n.
				\end{align*}
		\end{problem}
		
		\begin{solution}
			To apply the theorem, first we need to make $a$ and $b$ co-prime. If $q$ is a prime divisor of $(a,b)=g$, then $q|p$. Therefore, $g=p^r$ for some $r$. Let, $a=p^rx,b=p^ry$ with $x\bot y$. Then,
				\begin{align*}
					x^3+y^3 & = p^{n-3r}.
				\end{align*}
			Assume that $m=n-3r$. Since the power is three, we need to consider the exceptional case first. The case $x=2$ and $y=1$ when $p=3$ and $n-3r=2$ produces infinitely many solutions. Otherwise, $x^3+y^3$ has a prime divisor that does not divide $x+y$. Obviously $x+y$ is divisible by $p$ since $x+y>1$. This is a contradiction. Therefore, the only families of solutions are
				\begin{align*}
					(a,b,p,n)=(2\cdot 3^r,3^r,3,3r+2) \text{ and }(3^r,2\cdot 3^r,3,3r+2),
				\end{align*}
			for any positive integer $r$.
		\end{solution}
		
	The next problem is from the IMO Shortlist.
		\begin{problem}[IMO Shortlist 2002, Problem $4$]
			If $p_1,p_2,\dots,p_n$ are distinct primes greater than $3$, prove that, $2^{p_1p_2\dots p_n}+1$ has at least $4^n$ divisors.
		\end{problem}
	Here, we will prove a much more generalized form of this problem.  You can certainly see how much the problem can be improved with this theorem.
		\begin{theorem}\slshape
			If $p_1,p_2,\dots,p_n$ are all primes greater than $3$, then $2^{p_1p_2\dots p_n}+1$ has at least $2^{2^n}$ divisors.
		\end{theorem}
	In order to prove this theorem, let's first prove the following lemma.
		\begin{lemma}\slshape
			Let $N=2^{p_1\cdots p_n}+1$ where $p_i>3$ is a prime. Then $N$ has at least $2^n$ distinct prime divisors.
		\end{lemma}
		
		\begin{proof}
			The number $M=p_1p_2\dots p_n$ has $\underbrace{(1+1)(1+1)\dots(1+1)}_{n\text{ times}}=2^n$ divisors. Say the divisors are
				\begin{align*}
					1=d_1<d_2<\ldots<d_{2^n}=p_1\cdots p_n.
				\end{align*}
			Then first $2^{d_1}+1$ has the prime divisor $3$. $2^{d_2}+1$ has a divisor that is not $3$. Generally, each $d_i>d_{i-1}$ gives us a new primitive divisor that was not in $2^{d_{i-1}}+1$. Therefore, we have at least $2^n$ distinct prime divisors.
		\end{proof}
	Now we prove the theorem.
		\begin{proof}
			Let's assume that these $2^n$ primes are $q_1, q_2, \dots, q_{2^n}$. Then,
				\begin{align*}
					N & = q_1q_2\dots q_{2^n}K,
				\end{align*}
			for some integer $K\geq1$. Thus, every divisor of $D=q_1q_2\dots q_{2^n}$ is a divisor of $N$ and so, $N$ has at least $2^{2^n}$ divisors.
		\end{proof}
	This theorem can be generalized even more.
		\begin{theorem}\slshape
			Let $a,b,n$ be positive integers with $a\bot b$.
				\begin{enumerate}[i.]
					\item If $3\nmid n$, $a^n+b^n$ has at least $2^{\tau(n)}$ divisors. That is,
						\begin{align*}
							\tau(a^n+b^n) &\geq2^{\tau(n)}.
						\end{align*}
					\item If $n$ is odd and $a-b>1$, $a^n-b^n$ has at least $2^{\tau(n)}$ divisors. That is,
						\begin{align*}
							\tau(a^n-b^n) & \geq2^{\tau(n)}.
						\end{align*}
				\end{enumerate}
		\end{theorem}
		
		\begin{problem}[Romanian TST 1994]
			Prove that the sequence $a_n = 3^n - 2^n$ contains no three numbers in geometric progression.
		\end{problem}
		
		\begin{solution}
			Assume to the contrary, $a_n^2=a_ma_k$, since they are in geometric progression. So
				\begin{align*}
					(3^n-2^n)^2 & = (3^k-2^k)(3^m-2^m).
				\end{align*}
			Since $k,m,n$ are distinct, we must have $k<n<m$. If not, we can not have $n<k<m$ or $n>m>k$ because that would make one side larger. But due to the fact $m>n$, we get that, $3^m-2^m$ has a prime divisor that does not divide $3^n-2^n$.
		\end{solution}
		
			The next problem is taken from the IMO Shortlist.
		\begin{problem}[IMO Shortlist 2000]
			Find all positive integers $a,m$, and $n$ such that 
				\begin{align*}
					a^m+1 & |(a+1)^n.
				\end{align*}
		\end{problem}
		
		\begin{solution}
			Note that $(a,m,n)=(1,m,n)$ is a solution for all $m,n$. $(a,m,n)=(2,3,n)$ is a solution for $n>1$. If $m\neq3$ and $a,m\geq2$, then $a^m+1$ has a prime factor that is not a prime factor of $a+1$. Therefore, in such cases there are no solutions.
		\end{solution}
		
\end{document}