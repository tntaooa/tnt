Wilson's theorem says that $(p-1)! \equiv -1 \pmod p$ for all primes $p$. Clearly, for any integer $n$ larger than $p$, we have $n! \equiv 0 \pmod p$. Now, if we remove the multiples of $p$ from $n!$ and then calculate the result modulo $p$, what would it be? We will state this as a generalization for Wilson's theorem. But first, some definitions and lemmas.

	\begin{definition}
		Let $n$ be a positive integer and $p$ a prime number. The \textit{$p$-reduced factorial of $n$} is the product of all positive integers less than or equal to $n$ which are not divisible by $p$. We denote this by $(n!)_p$.
	\end{definition}

	\begin{example}
		The $5$-reduced factorial of $10$, $(10!)_5$, is
			\begin{align*}
				(10!)_5
					& = 9 \times 8 \times 7 \times 6 \times 4 \times 3 \times 2 \times 1\\
					& = 72,576
			\end{align*}
	\end{example}


	\begin{theorem}\label{thm:reducedfactorialmodp}
		Let $p$ be a prime number and let $(n_k n_{k-1}\cdots n_1 n_0)_p$ be a positive integer. Then
			\begin{align*}
				(n!)_p
					& \equiv (-1)^{\floor{\frac{n}{p}}} \cdot n_0!\pmod p
			\end{align*}
	\end{theorem}

	\begin{proof}
		The numbers not divisible by $p$ among $1, 2, \cdots, n$ are
			\[\begin{array}{*{20}{c}}
				1&2& \cdots &{{n_0}}& \cdots &{p - 1}\\
				{p + 1}&{p + 2}& \cdots &{p + {n_0}}& \cdots &{2p - 1}\\
				\vdots & \vdots & \ddots & \vdots & \ddots & \vdots \\
				{( {[\frac{n}{p}] - 1} )p + 1}&{( {[\frac{n}{p}] - 1} )p + 2}& \cdots &{( {[\frac{n}{p}]p - 1} )p + {n_0}}& \cdots &{[\frac{n}{p}]p - 1}\\
				{[\frac{n}{p}]p + 1}&{[\frac{n}{p}]p + 2}& \cdots &{[\frac{n}{p}]p + {n_0}}&{}&{}
			\end{array}\]

	Product of these numbers, $(n!)_p$ is
		\begin{align*}
			\left(\prod_{k=0}^{[\frac np]-1}\!\!\left((kp+1)\cdot(kp+2)\cdots(kp+p-1)\right)\right)  \cdot
			\left([\tfrac np]p+1\right)\left([\tfrac np]p+2\right)\dots\left([\tfrac np]p+n_0\right)
		\end{align*}
	which is equal to
		\begin{align*}
			\left(\prod_{k=0}^{[\frac np]-1}\!\!\left(1\cdot2\cdots(p-1)\right)\right)  \cdot
			\left([\tfrac np]p+1\right)\left([\tfrac np]p+2\right)\dots\left([\tfrac np]p+n_0\right)
				& \equiv \left(\prod_{k=0}^{[\frac np]-1}\!\!\left(-1\right) \right) \cdot \left(
				1 \cdot 2 \cdots n_0\right)\\
				& \equiv(-1)^{[\frac np]} n_0!\pmod p
		\end{align*}
	\end{proof}


	\begin{proposition}
		Let $p\geq 3$ be a prime and $n$ be a positive integer. Then
			\begin{align*}
				(p^n!)_p \equiv -1 \pmod{p^n}
			\end{align*}
	\end{proposition}

	\begin{proof}
		This is exactly the same as the proof of Wilson's theorem. All numbers in the product $(p^n!)_p$ have a multiplicative inverse modulo $p^n$. If the inverse of a number $a$ among these numbers is $b \neq a$, then $a b \equiv 1 \pmod{p^n}$ and we can remove $a$ and $b$ from the product $(p^n!)_p$. Our only concern is when the inverse of $a$ equals $a$ itself. But if that's the case, we have
			\begin{align*}
				a^2
					& \equiv 1 \pmod{p^n}\\
				\implies p^n
					& \mid (a-1)(a+1)
			\end{align*}
		But since $(a-1,a+1)=2$, we should have $a \equiv \pm 1 \pmod{p^n}$, which means $a$ is either $1$ or $p^n-1$. All in all, we see that the product of all numbers in $(p^n!)_p$ except $p^n-1$ equals $1$ modulo $p^n$, and if we multiply this number by $p^n-1$, the result will be $-1$ modulo $p^n$.
	\end{proof}

	\begin{problem}
		Prove that $(2^n!)_2 \equiv 1 \pmod{2^n}$.
	\end{problem}

We are ready to prove the following theorem.

	\begin{theorem}[Generalization of Wilson's Theorem]\label{thm:wilsongeneral}
		Let $p$ be a prime number and let $(n_k n_{k-1}\cdots n_1 n_0)_p$ be the representation of a positive integer $n$ in base $p$. Then
			\begin{align}\label{eq:wilsongeneral}
				\dfrac{n!}{p^{v_p(n!)}}\equiv (-1)^{v_p(n!)} n_0!n_1!\dots n_k!\pmod p
			\end{align}
	\end{theorem}

	\begin{proof}
		According to \autoref{thm:reducedfactorialmodp}, one can write
			\begin{align*}
				n!
					& = (n!)_p \cdot p^{\floor{n/p}}\func{}{\floor{\frac{n}{p}}}!\\
					& \equiv (-1)^{\lfloor n/p\rfloor} n_0! \cdot p^{\floor{n/p}}\func{}{\floor{\frac{n}{p}}}!
			\end{align*}
		Now write $\func{}{\floor{\frac np}}!$ in the same way and continue this process. The result is concluded.
	\end{proof}

	\begin{note}
		If you are interested, you can find a (different) generalization of Wilson's theorem in Problem \ref{thm:genWilson}.
	\end{note}
	\begin{theorem}[Generalization of Lucas' Theorem]
		Let $p$ be a prime number and $m, n$, and $r$ be non-negative integers such that $r=m-n$ and
			\begin{align*}
				m&=m_kp^k+m_{k-1}p^{k-1}+\cdots +m_1p+m_0\\
				n&=k_kp^k+n_{k-1}p^{k-1}+\cdots +n_1p+n_0\\
				r&=r_kp^k+r_{k-1}p^{k-1}+\cdots +r_1p+r_0
			\end{align*}
		Also, let $\ell = v_{p}\func{}{\binom{m}{n}}$. Then
			\begin{align*}
				\frac1{p^\ell}\binom{m}{n}\equiv (-1)^\ell
				\func{}{\frac{m_0!}{n_0!r_0!}}\cdots \func{}{\frac{m_d!}{n_d!r_d!}}\pmod p
			\end{align*}
	\end{theorem}

	\begin{proof}
		Note that
			\begin{align*}
				\ell
					&= v_{p}\func{}{\binom{m}{n}}\\
					& = v_{p}\func{}{\frac{m!}{n!r!}}\\
					& = v_p(n!)-v_p(n!)-v_p(r!)\\
					& =\sum_{i=1}^{k}\floor{\dfrac{m}{p^i}}- \sum_{i=1}^{k}\floor{\dfrac{n}{p^i}}-\sum_{i=1}^{k}\floor{\dfrac{r}{p^i}}\\
					&=\sum_{i=1}^{k}\func{}{\floor{\dfrac{m}{p^i}}- \floor{\dfrac{n}{p^i}}- \floor{\dfrac{r}{p^i}}}
			\end{align*}
		Just like the proof of \autoref{thm:wilsongeneral}, we can write
			\begin{align*}
				\binom {m}{n}
					& =\frac{(m!)_p}{(n!)_p (r!)_p}\cdot
				\frac{p^{\floor{m/p}}}{p^{\floor{n/p}} \cdot p^{\floor{r/p}}}\cdot
				\frac{\floor{\dfrac{m}{p}}!}{\floor{\dfrac{n}{p}}! \cdot \floor{\dfrac{r}{p}}!}
			\end{align*}
		Use induction and generalization of Wilson's theorem to finish the proof.
	\end{proof}