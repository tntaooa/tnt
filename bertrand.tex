\documentclass{subfile}

\begin{document}
	\label{sec:bertrandspostulate}
	\textit{Bertrand's postulate} is a very nice and influential theorem in number theory. \textit{Joseph Bertrand} first conjectured it, but he couldn't prove it entirely. Later, \textit{Chebyshev} proved it, using analytic number theory tools. \textit{S. Ramanujan} \textcite{meher_murty_2013} proved it using properties of \textit{Gamma function}, which is beyond the scope of this book. The first elementary proof of this theorem was given by \textcite{erdos_1932}. It was the first paper he published! We will show that proof here.

	There are many formulations of this theorem. All of them are equivalent.

	\begin{theorem}[Bertrand's Postulate]\label{thm:bertrand}\slshape
		For all integers $n>1$, there is a prime $p$ so that $n<p<2n$. The following are equivalent formulation.
		\begin{itemize}
			\item Let $p_n$ denote the $n^{th}$ prime number, starting from $p_1=2$. Then
			\begin{eqnarray*}
				p_{n+1} <2p_n.
			\end{eqnarray*}
			\item For any integer $n >1$, we have
			\begin{eqnarray*}
				\pi(n)-\pi\left(\dfrac{n}{2}\right)\geq1.
			\end{eqnarray*}
		\end{itemize}
	\end{theorem}

	\begin{lemma}\label{lem:bertrand1}
		For any positive integer $n$,
		\[\binom{2n}{n}\geq\dfrac{4^n}{2n+1}.\]
	\end{lemma}

	\begin{proof}
		From binomial the theorem, we already know that
		\begin{eqnarray*}
			(1+1)^{2n}  & = & 1+\binom{2n}1+\cdots+\binom{2n}{n}+\cdots+\binom{2n}{2n}.
		\end{eqnarray*}
		Since the binomial coefficients exhibit a symmetry , i.e., since $\binom{2n}{k}=\binom{2n}{2n-k}$, all other terms in the above sum are smaller than $\binom{2n}{n}$. Therefore
		\[2^{2n}\leq(2n+1)\binom{2n}n,\]
		which is what we wanted.

	\end{proof}
	In section \eqref{sec:powerofprimes} of previous chapter, we defined $v_p(n)$ to be the highest power of a prime $p$ which divides $n$.
	\begin{lemma}\label{lem:bertrand2}
		Let $n$ be a positive integer and let $2n/3<p\leq n$ be a prime. Then $p \nmid \binom{2n}n$.
	\end{lemma}

	\begin{proof}
		We have
		\begin{align*}
			v_p \left(\binom{2n}n\right) =v_p \left(\frac{(2n)!}{(n!)^2}\right) = v_p((2n)!) - v_p((n!)^2) = v_p((2n)!) - 2v_p((n!)).
		\end{align*}
		Note that $2n/3<p$ means $2n<3p$, and so the only multiples of $p$ which appear in $(2n)!$ are $p$ and $2p$. Hence $v_p((2n)!) = 2$. Also, $p<n$ immediately gives $v_p((n!))=1$. Therefore
		\begin{align*}
			v_p \left(\binom{2n}n\right) = v_p((2n)!) - 2v_p((n!)) =  2 - 2 \cdot 1 = 0.
		\end{align*}
	\end{proof}

	\begin{lemma}\label{lem:bertrand3}
		Let $n$ be a positive integer. Let $p$ be any prime divisor of $N=\binom{2n}n$. Then $p^{v_p(N)}\leq2n$.
	\end{lemma}

	\begin{proof}
		Let $\alpha$ be the positive integer for which $p^\alpha \leq 2n<p^{\alpha +1}$. Then using theorem \autoref{thm:legendre} of chapter \eqref{ch:arithfunc},
		\begin{align*}
			v_p(N) & =  v_p((2n)!)-2v_p(n!)= \sum_{i=1}^{\alpha}\left\lfloor\dfrac{2n}{p^i}\right\rfloor-2\sum_{i=1}^{\alpha }\left\lfloor\dfrac{n}{p^i}\right\rfloor\\
			& =  \sum_{i=1}^{\alpha}\left(\left\lfloor\dfrac{2n}{p^i}\right\rfloor-2\left\lfloor\dfrac{n}{p^i}\right\rfloor\right) \leq \sum_{i=1}^{\alpha}1 =\alpha
		\end{align*}
		The last line is true because for a rational $x$, $\lfloor2x\rfloor-2\lfloor x\rfloor\in\{0,1\}$. Therefore, $p^{v_p(N)} \leq p^{\alpha} \leq 2n$.
	\end{proof}

	\begin{lemma}\label{lem:bertrand4}
		Let $n$ be a positive integer. Any prime $p$ with $n+2\leq p\leq 2n+1$ divides $\binom{2n+1}{n}$.
	\end{lemma}

	\begin{proof}
		Since $p>n+1$,
		\begin{align*}
			\nu_p\left(\binom{2n+1}{n}\right)  & =  v_p \left(\frac{(2n+1)!}{(n!)(n+1)!}\right) = v_p((2n+1)!)-v_p(n!)-v_p((n+1)!)=1.
		\end{align*}
	\end{proof}

	\begin{lemma}\label{lem:bertrand5}
		For any positive integer $n$,
		\[\binom{2n+1}{n}\leq2^{2n}.\]
	\end{lemma}

	\begin{proof}
		From binomial theorem and the fact that $\binom{2n+1}{n}=\binom{2n+1}{n+1}$,
		\begin{align*}
			(1+1)^{2n+1}  & =     1+\binom{2n+1}1+\cdots+\binom{2n+1}{n}+\binom{2n+1}{n+1}+\cdots+\binom{2n+1}{2n+1}\\
			& \geq  \binom{2n+1}{n}+\binom{2n+1}{n+1} =  2\binom{2n+1}{n}.
		\end{align*}
		This finishes the proof.
	\end{proof}

	The following lemma is really a nice one, and the proof requires a good insight.

	\begin{lemma}\label{lem:bertrand6}
		The product of all primes less than or equal to $n$ is less than or equal to $4^n$.
	\end{lemma}

	\begin{proof}
		We will use induction. The proof is trivial for $n=1$ and $n=2$. Assume it is true for all positive integers up to $n-1$. We will show that it is also true for $n$.

		If $n$ is even and greater than $2$, $n$ is definitely not a prime. Thus,
		\begin{align*}
			\prod_{p\leq n}p =  \prod_{p\leq n-1}p\ \leq  4^{n-1}<  4^n.
		\end{align*}
		Now, assume that $n$ is odd. Take $n=2m+1$. We have
		\begin{align*}
			\prod_{p\leq n}p =	\prod_{p\leq 2m+1} p  = \left(\prod_{p\leq m+1}p \right) \left( \prod_{m+2\leq p\leq2m+1}p\right).
		\end{align*}
		By induction hypothesis, the first product, $\prod\limits_{p\leq m+1}p$, is less than or equal to $4^{m+1}$. From lemma \autoref{lem:bertrand4}, we know that any prime $p$ such that $m+2\leq p\leq2m+1$ divides $\binom{2m+1}{m}$. Therefore, the second product, $\prod\limits_{m+2\leq p\leq2m+1}p$, is less than or equal to $\binom{2m+1}{m}$. Combining these results with lemma \autoref{lem:bertrand5}, we get
		\begin{align*}
			\prod_{p\leq n}p &\leq 4^{m+1}\binom{2m+1}{m}\leq 4^{m+1}2^{2m} =  4^{2m+1} = 4^n.
		\end{align*}
		So, the lemma is also true for $n$ and, we are done.
	\end{proof}

	We are ready to prove Bertrand's postulate.

	\begin{proof}[Proof of Bertrand's postulate]
		We want to show that for any positive integer $n$, there exists a prime $p$ such that $n <p \leq 2n$. Assume the converse, i.e., suppose that there exists some $n$ for which there is no prime $p$ with $n<p \leq 2n$. We will find an upper bound for $N = \binom{2n}{n}$ and seek for a contradiction. Let us divide the prime divisors of $N$ into two groups:
		\begin{itemize}
			\item Consider all prime divisors of $N$, say $p$, such that $p \leq \sqrt{2n}$. Let $p_1, p_2, \ldots, p_k$ be such primes. Clearly, $k \leq \sqrt{2n}$. According to lemma \autoref{lem:bertrand3}, $p_i^{v_{p_i}(N)} \leq 2n$ (for $1 \leq i \leq k$). Therefore,
			\begin{eqnarray*}
				\prod_{i=1}^{k} p_i^{v_{p_i}(N)} \leq (2n)^{\sqrt{2n}}.
			\end{eqnarray*}

			\item Consider all prime divisors of $N$ which are larger than $\sqrt{2n}$. Let $q_1, q_2, \ldots, q_m$ be such primes. Again, by lemma \autoref{lem:bertrand3}, we must have $q_i^{v_{q_i}(N)} \leq 2n$ (where $1 \leq i \leq m$). However, since $q_i > \sqrt{2n}$, we find that $v_{q_i}(N)=1$ for all $i$.

			Now, by our hypothesis, there are no primes $p$ such that $n < p \leq 2n$. On the other hand, lemma \autoref{lem:bertrand2} says that there are no prime divisors of $N$ such that $2n/3 < p \leq n$. Altogether, we find that $ \sqrt{2n} < q_i \leq 2n/3$ for $1 \leq i \leq m$. Hence,
			\begin{eqnarray*}
				\prod_{i=1}^{m} q_i^{v_{q_i}(N)} =  \prod_{i=1}^{m} q_i =  \prod\limits_{\substack{\sqrt{2n}<p\leq2n/3\\p|N}} p.
			\end{eqnarray*}
		\end{itemize}
		We now use the fact that $N = 	\prod\limits_{i=1}^{k} p_i^{v_{p_i}(N)}\cdot \prod\limits_{i=1}^{m} q_i^{v_{q_i}(N)}$, where $p_i$ and $q_i$ are as defined above. According to what we have found,
		\begin{eqnarray*}
			N =\prod\limits_{i=1}^{k} p_i^{v_{p_i}(N)}\cdot \prod\limits_{i=1}^{m} q_i^{v_{q_i}(N)} & \leq & (2n)^{\sqrt{2n}} \cdot \prod\limits_{\substack{\sqrt{2n}<p\leq2n/3\\p|N}} p \\
			& \leq & (2n)^{\sqrt{2n}} \cdot \prod_{p\leq2n/3} p\\
			& \leq & (2n)^{\sqrt{2n}} \cdot 4^{2n/3}.
		\end{eqnarray*}
		Note that we have used lemma \autoref{lem:bertrand6} for writing the last line.

		Combining this with the result of lemma \autoref{lem:bertrand1}, we see that
		\begin{eqnarray}
		\frac{4^n}{2n+1} \leq (2n)^{\sqrt{2n}} \cdot 4^{2n/3}.
		\end{eqnarray}
		However, this inequality can hold only for small values of $n$. Actually, one can check that the inequality fails for $n \geq 468$. For $n < 468$, one can check that
		\begin{align*}
		2, 3, 5, 7, 13, 23, 43, 83, 163, 317, 631
		\end{align*}
		is a sequence of primes, each term of which is less than twice the term preceding it. Therefore, any interval $\{n+1, n+2, \ldots, 2n\}$ with $n<468$ contains one of the primes in this sequence.

		Hence, we have reached the contradiction we were looking for. This means that there always exist a prime $p$ such that $n<p\leq 2n$ for any positive integer $n$. The proof is complete.
	\end{proof}


	\begin{theorem}\slshape
		For any positive integer $n$, the set $S=\{1,2,\ldots,2n\}$ can be partitioned into $n$ pairs $(a_i,b_i)$ so that $a_i+b_i$ is a prime.
	\end{theorem}

	Before we show the proof, readers are highly encouraged to prove it themselves. This is the kind of theorem that shows how good human thinking can be.

	\begin{proof}
		We will proceed by induction. The theorem is clearly true for $n=1$ since $1+2=3$, a prime. Assume that the theorem is true for all $k<n$ and we can split the set $\{1,2,\ldots,2k\}$ into pairs with a prime sum. By Bertrand's postulate, there is a prime $p$ with $2n<p<4n$. Let $p=2n+m$, where $m$ must be odd since $p$ is odd. Consider the set $\{m,m+1, \ldots , 2n\}$. It has an even number of elements. Also, we can make pairs of $(m,2n), (m+1,2n-1), \ldots$ with sum $p$, which is a prime. Now we only have to prove that the set $\{1,2,\ldots,m-1\}$ can be paired into elements with a prime sum. This is true by induction hypothesis because $m-1<2n$. The proof is complete.
	\end{proof}

	\begin{problem}
		Let $n > 5$ be an integer and let $p_1, p_2, \ldots, p_k$ be all the primes smaller than $n$. Show that $p_1 + p_2 + \cdots + p_k >n$.
	\end{problem}

	\begin{solution}
		We first show by induction that $\sum_{i=1}^{k} p_i > p_{k+1}$ for $k \geq 3$. The base case, $k=3$ is true because $2+3+5>7$. Assume that $\sum_{i=1}^{k} p_i > p_{k+1}$, then by the first alternative form of Bertrand's postulate stated in theorem \autoref{thm:bertrand},
			\begin{align*}
				\sum_{i=1}^{k+1} p_i = p_{k+1} + \sum_{i=1}^k p_i > 2p_{k+1} > p_{k+2},
			\end{align*}
		and the induction is complete. Now, since $p_k<n\leq p_{k+1}$, we have
			\begin{align*}
				\sum_{i=1}^k p_i > p_{k+1} \geq n.
			\end{align*}
	\end{solution}

	\begin{problem}[China 2015]\label{prob:china2015-bertrand}
		Determine all integers $k$ such that there exists infinitely many positive integers $n$ satisfying
		\begin{align*}
		n+k
			& \nmid \binom{2n}{n}.
		\end{align*}
	\end{problem}

	\begin{solution}
		We will show that the problem statement holds for all integers $k \neq 1$. Note that for $k=1$, we have
			\begin{align*}
				\binom{2n}{n} - \binom{2n}{n+1} = \frac{1}{n+1}\binom{2n}{n},
			\end{align*}
		and therefore $n+1 | \binom{2n}{n}$. Assume that $k>1$. By Bertrand's postulate, there exists an odd prime $p$ such that $k<p<2k$. Choose $n=(p-k)+p^m$ for any positive integer $m$. From theorem \autoref{thm:legendrealternative}, we can write
			\begin{align*}
				v_p \left(\binom{2n}n\right)  &=  v_p((2n)!) - 2v_p((n!))\\
				&=  \frac{2n - s_p(2n)}{p-1} - 2 \cdot \frac{n - s_p(n)}{p-1}\\
				&=   \frac{2s_p(n) - s_p(2n)}{p - 1}.
			\end{align*}
		Since $2n = 2(p - k) + 2p^m$ and $2(p - k) < p$, it follows that $s_p(2n) = 2(p - k) + 2 = 2s_p(n)$ (try to write the base $p$ representation of $n$ and $2n$ to see why). Consequently, $v_p\left(\tbinom{2n}{n}\right) = 0$. However, $p \mid n + k$, so we have $n + k \nmid \tbinom{2n}{n}$ for infinitely many $n$, as desired.

		For negative $k$, one can choose $n = -k + p^m$ for an odd prime $p > |2k|$ (which exists by Bertrand's postulate) and any positive integer $m$. In a similar manner as above, one obtains $v_p\left(\tbinom{2n}{n}\right) = 0$, but $p \mid n + k$. Consequently, $n + k \nmid \tbinom{2n}{n}$. The proof is complete.
	\end{solution}
After the theorem was proved, number theorists tried to tighten the interval. Also, a question was raised regarding the general case.
	\begin{problem}
		Let $c$ be a real number. What is the minimum value of $c$ such that, there is always a prime between $n$ and $n+cn$ for positive integers $n>1$?
	\end{problem}
\textcite{nagura_1952} proved the case for $c=1/5$.
	\begin{theorem}[Nagura]\slshape
		For $x\geq25$, there is always a prime number between $x$ and $6x/5$.
	\end{theorem}
The proof uses a property of gamma function (a function involving the gamma function turns out to be a prime counting function). We will not be proving the improvements or generalizations, but they are worth mentioning. The general case of this theorem would be like this:
	\begin{problem}
		Let $k$ be a positive integer. Does there always exist a prime between $kn$ and $(k+1)n$?
	\end{problem}
\textcite{bachraoui_2006} proved the case $k=2$. The idea is an extension of Erd\H{o}s's proof.
	\begin{theorem}[Bachraoui]\slshape
		For a positive integer $n>1$, there is always a prime in the interval $[2n,3n]$.
	\end{theorem}
\textcite{loo_2011} proved the case for $k=3$ without using prime number theorem or any deep analytical method.
	\begin{theorem}[Loo]\slshape
		For a positive integer $n\geq2$, there is always a prime in the interval $(3n,4n)$.
	\end{theorem}
\textcite{moses_shevelev_greathouse_2013} proves the following theorem.
	\begin{theorem}
		The list of integers $k$ for which every interval $(kn,(k+1)n)$ contains a prime for $n>1$ is $\{1,2,3,5,9,14\}$ and no others, at least for $k\leq10^9$.
	\end{theorem}
There are some nice conjectures involving this theorem.
	\begin{conjecture}[Legendre's conjecture]
		There always exists a prime in the interval $[n^2,(n+1)^2]$.
	\end{conjecture}

	\begin{theorem}[Mitra's conjecture]
		Assume the general Bertrand's postulate. There exists at least two primes in the interval $[n^2,(n+1)^2]$.
	\end{theorem}

	\begin{theorem}[Brocard's conjecture]
		Assume the general Bertrand's postulate. For each $n>1$, there are at least $4$ primes in the interval $[p_n^2,p_{n+1}^2]$.
	\end{theorem}

	\begin{theorem}[Andrica's conjecture]
		Assume the general Bertrand's postulate holds true. For any positive integer $n$, $\sqrt{p_{n+1}}-\sqrt{p_n}<1$.
	\end{theorem}
\end{document}